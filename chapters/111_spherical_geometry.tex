%!TEX root=../GaugeCNNTheory.tex


\subsection%
    [هندسه‌ی کره ۲-بعدی \texorpdfstring{$S^2$}{S2}]%
    {هندسه‌ی کره ۲-بعدی $S^2$}
\label{sec:sphere_geometry}

به عنوان پایه‌ای برای بحث ما در مورد \lr{CNN}های کروی، این بخش به بحث در مورد هندسه دیفرانسیل کره (واحد)~$M = S^2$ می‌پردازد.
این کره معمولاً به عنوان زیرمجموعه‌ای از نقاط در فضای اقلیدسی ۳-بعدی $\Euc_3$ تعریف می‌شود که فاصله واحد از مبدأ دارند:
\begin{align}
    S^2 \,:=\, \big\{ p\in\Euc_3 \,\big|\, \lVert p\rVert = 1 \big\}
\end{align}
به عنوان یک سطح جایگذاری شده، این کره یک متریک ریمانی (فرم بنیادی اول) را از فضای جایگذاری $\Euc_3$ به ارث می‌برد.
در ادامه، برای سادگی، ما $\Euc_3$ را با فضای برداری $\R^3$ مدل‌سازی می‌کنیم.
هنگامی که فضاهای مماس $\TpM$ را به معنای واقعی کلمه به عنوان آن دسته از زیرفضاهای دوبعدی $\R^3$ تفسیر کنیم
که شامل تمام بردارهای مماس در $p \in S^2$ هستند، آنگاه متریک، نگاشت‌های نمایی، انتقال‌دهنده‌های موازی, چارچوب‌ها و پیمانه‌ها همگی می‌توانند بر حسب عملیات معمول فضای برداری در $\R^3$ بیان شوند.
قبل از پرداختن به این عبارات مشخص، که هنگام پیاده‌سازی \lr{CNN}های کروی مفید هستند،
ما برخی از ویژگی‌های کره را از دیدگاهی انتزاعی‌تر مورد بحث قرار می‌دهیم.


گروه ایزومتری کره با
\begin{align}
    \Isom(S^2) = \OO3 \,,
\end{align}
داده می‌شود، یعنی دوران‌ها و بازتاب‌های سه‌بعدی، که در شکل~\ref{fig:isometries_sphere} به تصویر کشیده شده‌اند.
عمل هر ایزومتری $\phi\in\OO3$ با عمل معمول آن روی $\R^3$ از طریق ضرب ماتریسی، که به کره جایگذاری شده $S^2 \subset \R^3$ محدود شده است، منطبق است.
توجه داشته باشید که این در واقع یک عمل خوش‌تعریف روی $S^2$ به دست می‌دهد زیرا $\OO3$ بنا به تعریف از تمام نگاشت‌های خطی حافظ فاصله و زاویه تشکیل شده است و بنابراین کره را حفظ می‌کند.
از آنجا که کره جهت‌پذیر است، دارای یک زیرگروه از ایزومتری‌های حافظ جهت است
\begin{align}
    \Isom_+(S^2) = \SO3 \,,
\end{align}
که از تمام دوران‌های سه‌بعدی تشکیل شده است.
زیرگروه‌های دیگری که در زمینه یادگیری عمیق مرتبط هستند، عبارتند از:
هر انتخاب از یک محور دوران، یک زیرگروه از دوران‌های دوبعدی را تعیین می‌کند که با $\SO2$ ایزومورف است و تمام این زیرگروه‌ها با یکدیگر مزدوج هستند.
به طور مشابه، هر انتخاب از یک زیرفضای دوبعدی از $\R^3$ متناظر با یک زیرگروه از بازتاب‌ها نسبت به این صفحه است که با $\Flip$ ایزومورف است.
زیرگروه‌های دوران‌های دوبعدی حول دو محور دوران غیرموازی، $\SO3$ را تولید می‌کنند، که به پارامترسازی زاویه اویلر از $\SO3$ مربوط می‌شود.
یک انتخاب از صفحه بازتاب و هر محور دوران در داخل این صفحه، زیرگروه حاصلضرب نیمه‌مستقیم $\OO2 = \SO2 \rtimes \Flip$ را تولید می‌کند.
اگر محور دوران به جای آن عمود بر صفحه بازتاب انتخاب شود، دوران‌ها و بازتاب‌های دوبعدی جابجا می‌شوند و بنابراین زیرگروه‌هایی ایزومورف با حاصلضرب مستقیم $\SO2 \times \Flip$ را تولید می‌کنند.
$\OO3$ علاوه بر این دارای زیرگروه‌های گسسته است که مرتبط‌ترین آنها از نظر عملی، گروه‌های تقارن اجسام افلاطونی هستند، به عنوان مثال بیست‌وجهی، که در شکل~\ref{fig:ico_neighborhoods} نشان داده شده است.%
\footnote{
    یک لیست جامع از تمام زیرگروه‌های متناهی $\SO3$ را می‌توان در \href{https://ncatlab.org/nlab/show/SO\%283\%29\#finite_subgroups}{nLab} یافت.
}


$\OO3$ به صورت متعدی بر روی کره عمل می‌کند، یعنی برای هر دو نقطه $p$ و $q$ از $S^2$ حداقل یک ایزومتری $\phi\in\OO3$ وجود دارد به طوری که $q = \phi(p)$.
عمل‌های $\OO3$ روی $S^2$ بدون نقطه ثابت نیستند:
هر نقطه $p\in S^2$ توسط زیرگروه $\Stab{p} \cong \OO2 < \OO3$ پایدار می‌شود، که از دوران‌ها و بازتاب‌ها حول محور گذرنده از $p$ در $\R^3$ تشکیل شده است.
در مجموع، این دو ویژگی دلالت بر این دارند که کره یک فضای همگن از $\OO3$ است و به صورت جبری به عنوان فضای خارج‌قسمتی
\begin{align}
    \OO3/\OO2 \,\cong\, S^2 \,,
\end{align}
تحقق می‌یابد، که از هم‌دسته‌هایی به شکل ${\phi\mkern1mu.\mkern-4mu\OO2}$ تشکیل شده است.
یک گزاره مشابه برای $\SO3$ برقرار است، که دارای زیرگروه‌های پایدارساز $\Stab{p} \cong \SO2 < \SO3$ است و بنابراین
\begin{align}
    \SO3/\SO2 \,\cong\, S^2 \,.
\end{align}
با این روابط، قضیه~\ref{thm:GM_conv_homogeneous_equivalence} اثبات می‌کند که هر تبدیل میدان کرنل هموردای $\OO3$ یا $\SO3$ روی $S^2$ معادل یک کانولوشن $\GM$ با $G$ به ترتیب برابر با $\OO2$ یا $\SO2$ است.
این نتیجه با دیدگاه کلاسیک \lr{CNN}های هموردای گروهی روی فضاهای همگن مطابقت دارد~\cite{Cohen2019-generaltheory} -- رابطه دقیق بین این دو در قضیه~\ref{thm:spherical_conv_GM_conv} در ادامه روشن می‌شود.
به یاد بیاورید که ایزومتری‌ها بنا به تعریف، متریک ریمانی را حفظ می‌کنند.
اینکه $\OO3$ به صورت متعدی روی $S^2$ با پایدارساز $\OO2$ عمل می‌کند، بنابراین دلالت بر این دارد که هندسه ریمانی $S^2$ در هر نقطه و در هر جهت و جهت‌گیری «شبیه به هم» به نظر می‌رسد -- $S^2$ یک فضای بیشینه متقارن است.


به عنوان یک منیفلد ریمانی، $S^2$ بنا به طراحی دارای یک $\OO2$-ساختار است.
یک تحدید به چارچوب‌های راست‌گرد، که ممکن است زیرا کره جهت‌پذیر است، $\SO2$-ساختار را در شکل~\ref{fig:G_structure_S2_1} به دست می‌دهد، که توسط دوران‌ها در $\SO3$ حفظ می‌شود.
می‌توان نشان داد که این دو $G$-ساختار $\OOM$ و $\SOM$ به عنوان کلاف‌های اصلی به ترتیب با $\OO3$ و $\SO3$ ایزومورف هستند.
ایزومورفیسم خاص در اینجا با انتخاب یک چارچوب از $G$-ساختار داده می‌شود، که باید با عضو همانی گروه یکی گرفته شود.

قضیه گوی مودار بیان می‌کند که هیچ میدان برداری پیوسته‌ای روی $S^2$ وجود ندارد، که به طور خاص دلالت بر این دارد که هیچ $\{e\}$-ساختار (پیوسته‌ای) نمی‌تواند وجود داشته باشد.
بنابراین، یک کاهش گروه ساختاری فراتر از $\SO2$ نیازمند تغییر در توپولوژی منیفلد است.
برای مثال، سوراخ کردن کره در یک نقطه دلخواه $p\in S$ یک سطح را به دست می‌دهد که با صفحه اقلیدسی همسان‌ریخت است و بنابراین موازی‌پذیر است.%
\footnote{
    این فرآیند به عنوان مثال با تصویر استریوگرافیک کره مطابقت دارد.
}
سوراخ کردن کره در دو نقطه متقابل دلخواه، همانطور که در شکل~\ref{fig:G_structure_S2_2} نشان داده شده است، توپولوژی کره را به توپولوژی یک استوانه تبدیل می‌کند و بنابراین اجازه وجود $\{e\}$-ساختارها را می‌دهد.
رایج‌ترین انتخاب از $\{e\}$-ساختار روی کره سوراخ‌دار $S^2 \backslash \{n,s\}$، $\{e\}$-ساختار $\SO2$-ناوردا در شکل‌های~\ref{fig:G_structure_S2_2} و~\ref{fig:spherical_equirectangular_1} است.
چارچوب‌های آن
\begin{align}\label{eq:spherical_e_structure_frames}
    \left[ \frac{\partial}{\partial\theta} ,\; \frac{1}{\cos(\theta)} \frac{\partial}{\partial\varphi} \right]
\end{align}
با شبکه مختصاتی معمول کروی تراز شده‌اند، که در قراردادهای فیزیک
(یعنی با $\varphi$ و $\theta$ که به ترتیب زاویه سمتی و انحراف نسبت به صفحه $xy$ را نشان می‌دهند)
توسط نگاشت پوشای $2\pi$-متناوب زیر داده می‌شوند:
\begin{align}\label{eq:spherical_coords}
    \chi:\, \big( {\textstyle \minus\frac{\pi}{2}, \frac{\pi}{2} }\big) \times \R \:\to\: S^2 \backslash \{n,s\}
    \,, \quad
    (\theta, \varphi) \,\mapsto
    \begin{pmatrix}
        \cos(\theta) \cos{\varphi} \\
        \cos(\theta) \sin{\varphi} \\
        \sin(\theta)
    \end{pmatrix}
\end{align}
برخی از \lr{CNN}های $\{e\}$-راهبری‌پذیر با نمایش میدان‌های ویژگی روی $S^2 \backslash \{n,s\}$ در مختصات کروی پیاده‌سازی می‌شوند؛ به بخش~\ref{sec:spherical_CNNs_azimuthal_equivariant} در ادامه مراجعه کنید.
از آنجا که نگاشت مختصاتی $\chi$ ایزومتریک \emph{نیست}، این روش‌ها به یک متریک جایگزین (یا $\{e\}$-ساختار) روی مختصات ${(\minus \pi/2 ,\; \pi/2) \times \R \subset \R^2}$ نیاز دارند؛ به چارچوب‌های کشیده شده در شکل~\ref{fig:spherical_equirectangular_1} (راست) مراجعه کنید.


از آنجا که $S^2$ فشرده است، از نظر ژئودزیکی کامل است.
ژئودزیک‌ها توسط دوایر عظیمه کره داده می‌شوند، یعنی آن دایره‌هایی که متناظر با تقاطع کره با یک صفحه گذرنده از مبدأ $\R^3$ هستند.
نگاشت‌های نمایی $\exp_p(v)$ این دوایر عظیمه را از طریق $p$ در جهت $v$ برای فاصله‌ای برابر با $\lVert v\rVert$ دنبال می‌کنند.
بنابراین نگاشت‌های لگاریتمی $\log_p(q)$ برای تمام نقاط $q \in S^2 \backslash \mkern-1mu\minus\mkern1mu p$ که متقابل $p$ نیستند، با بردار یکتایی در جهت کوتاه‌تر در امتداد دایره عظیمه گذرنده از $p$ و $q$ داده می‌شوند، با $\lVert\log_p(q)\rVert$ که با طول قوس در امتداد این مسیر داده می‌شود.
ژئودزیک‌ها بین نقاط متقابل $p$ و $-p$ یکتا نیستند، به طوری که نگاشت لگاریتمی وجود ندارد.


\subsubsection*{هندسه‌ی صریح $S^2$ به عنوان یک سطح جایگذاری شده در $\mathds{R}^3$}
همانطور که در بالا گفته شد، فضاهای مماس $S^2 \subset \R^3$ در هندسه دیفرانسیل کلاسیک سطوح به عنوان زیرفضاهای دوبعدی از فضای جایگذاری~$\R^3$ تعریف می‌شوند.
یک فضای مماس خاص $\TpM$ در $p\in S^2$ در این تفسیر با
\begin{align}
    \TpM \,=\, \big\{ v\in\R^3 \,\big|\, \langle p,v \rangle = 0 \big\} \ \subset\ \R^3 \,,
\end{align}
داده می‌شود، یعنی فضای تمام بردارهایی که بر بردار نرمال سطح در $p$ عمود هستند، که برای کره با خود $p$ منطبق است.
توجه داشته باشید که، علی‌رغم بیان شدن نسبت به چارچوب استاندارد $\R^3$، این بردارهای مماس اشیاء مستقل از مختصات هستند به این معنا که با زوج‌های مرتبی از ضرایب $v^A \in \R^2$ نسبت به یک پیمانه $\psiTMp^A$ از~$\TpM$ توصیف نمی‌شوند.
یکی گرفتن فضاهای مماس با زیرفضاهای فضای جایگذاری اجازه می‌دهد تا بسیاری از روابط جبری انتزاعی بر حسب عملیات فضای برداری روی~$\R^3$ بیان شوند.
در باقیمانده این بخش، ما چنین عباراتی را برای متریک، نگاشت‌های نمایی و لگاریتمی، چارچوب‌ها، پیمانه‌ها، انتقال‌دهنده‌های لوی-چیویتا در امتداد ژئودزیک‌ها و تبدیلات پیمانه القا شده بیان خواهیم کرد.


بنا به تعریف، $S^2$ متریک ریمانی خود را از فضای جایگذاری به ارث می‌برد.
این متریک القا شده برای هر $v,w \in \TpM \subset \R^3$ با
\begin{align}\label{eq:spherical_embedding_metric_explicit}
    \eta_p(v,w) \,:=\, \langle v,w \rangle_{\R^3} \,,
\end{align}
داده می‌شود، یعنی حاصلضرب داخلی استاندارد $\R^3$، که به $\TpM$ محدود شده است.
برای کاهش شلوغی، ما زیرنویس $\R^3$ را در نماد $\langle \cdot,\cdot \rangle_{\R^3}$ در باقیمانده این بخش حذف می‌کنیم.


نگاشت نمایی $\exp_p$ بردارهای $v\in \TpM$ را به نقاط $q = \exp_p(v) \in S^2$ در فاصله‌ای برابر با $\lVert v\rVert$ در امتداد دایره عظیمه در جهت~$v$ نگاشت می‌دهد.
با قرار گرفتن روی همان دایره عظیمه، $p$ و $q$ از طریق یک دوران با زاویه $\alpha = \lVert v\rVert / r = \lVert v\rVert$ حول محور دوران $a = \frac{p\times v}{\lVert p \times v\rVert} = \frac{p\times v}{\lVert v\rVert}$ به هم مرتبط می‌شوند،
که در آن معادلات ساده می‌شوند زیرا کره دارای شعاع واحد $r = \lVert p\rVert = 1$ است و بردارهای $p$ و $v$ در $\R^3$ بر هم عمود هستند.
با استفاده از فرمول دوران رودریگز، $q = p \cos(\alpha) + (a\times p) \sin(\alpha) + a\langle a,p\rangle \big(1- \cos(\alpha) \big)$،
به همراه عمود بودن $\langle a,p\rangle = 0$ و
$a\times p
 = \frac{1}{\lVert v\rVert} (p\times v) \times p
 = \frac{1}{\lVert v\rVert} \big( \langle p,p\rangle v + \langle p,v\rangle p \big)
 = \frac{v}{\lVert v\rVert}$،
این منجر به عبارت صریح
\begin{align}\label{eq:sphere_expmap_explicit}
    \exp_p:\, \R^3 \supset \TpM \to S^2 \subset \R^3, \quad v \mapsto \exp_p(v) = p\mkern1mu \cos \big(\lVert v\rVert\big) + \frac{v}{\lVert v\rVert} \sin \big(\lVert v\rVert\big)
\end{align}
برای نگاشت نمایی می‌شود.

یک عبارت صریح از نگاشت لگاریتمی در امتداد همین خط استدلال یافت می‌شود:
نرم $\log_p(q)$، که در آن $q\in {S^2 \backslash \mkern-1mu\minus\mkern1mu p}$، با زاویه دوران $\alpha = \arccos\!\big( \langle p,q\rangle \big)$ منطبق است.
جهت آن با جهت مماس بر دایره عظیمه داده می‌شود، که ممکن است بر حسب تصویر نرمال شده
$\frac{v}{\lVert v\rVert} = \frac{q - \langle p,q\rangle p}{\lVert q - \langle p,q\rangle p\rVert}$
از~$q$ روی~$\TpM$ بیان شود.
در کل، نگاشت لگاریتمی بنابراین به صورت زیر نمونه‌سازی می‌شود:
\begin{align}\label{eq:sphere_logmap_explicit}
    \log_p:\, S^2\backslash \mkern-1mu\minus\mkern1mu p \to B_{\TpM}(0,\pi), \quad
    q \mapsto \log_p(q) = \arccos\!\big( \mkern-1mu\langle p,q\rangle \mkern-1mu\big) \, \frac{q - \langle p,q\rangle p}{\lVert q - \langle p,q\rangle p\rVert} \,,
\end{align}
که در آن $B_{\TpM}(0,\pi) \subset \TpM \subset \R^3$ گوی باز با شعاع انژکتیویته $\pi$ حول مبدأ $\TpM$ را نشان می‌دهد.


چارچوب‌های مرجع روی $S^2$ بنا به تعریف فقط زوج‌های مرتبی از بردارهای مماس خطی مستقل هستند.
هنگامی که محورهای یک چارچوب مرجع به صراحت به عنوان بردارهایی در فضای جایگذاری $\R^3$ بیان شوند، این چارچوب را می‌توان با ماتریس $3\times 2$ با رتبه ۲
\begin{align}\label{eq:embedding_space_R3_frame}
    \big[ e_1^A,\, e_2^A \big]\ =\ 
    \left[\! \begin{array}{cc}
        e^A_{1,1} & e^A_{2,1} \\
        e^A_{1,2} & e^A_{2,2} \\
        e^A_{1,3} & e^A_{2,3}
    \end{array} \!\right]
    \; =: E^A_p
    \ \ \in\, \R^{3\times2} .
\end{align}
شناسایی کرد. این ایزومورفیسم فضای برداری
\begin{align}
    E^A_p = \big[ e_1^A, e_2^A \big]:\, \R^2 \to \TpM,\ \ \ v^A \mapsto E^A_p v^A = v^A_1 e^A_1 + v^A_2 e^A_2
\end{align}
را از ضرایب برداری به بردارهای مماس مستقل از مختصات تعریف می‌کند.
بنابراین فضاهای مماس $\TpM$ دقیقاً تصویر $E^A_p$ هستند.


پیمانه‌های متناظر $\psiTMp^A: \TpM \to \R^2$ از نظر فنی فقط وارون چارچوب‌ها هستند، هنگامی که به عنوان نگاشت‌های $E^A_p: \R^2 \to \TpM$ تفسیر شوند.
در مقابل، هنگامی که به عنوان ماتریس‌های $3\times 2$ که $\R^2$ را به صورت غیرپوشا به $\R^3$ نگاشت می‌دهند، تفسیر شوند، $E^A_p$ وارون‌پذیر نیست اما فقط یک شبه-وارون را می‌پذیرد
\begin{align}
    \big(E^A_p \big)^+ \,:=\, \big( (E^A_p)^\top E^A_p \big)^{-1} (E^A_p)^\top \ \in\, \R^{2\times 3} .
\end{align}
از نظر هندسی، این ماتریس با
۱) تصویر کردن بردارهای در $\R^3$ به تصویر $E_p^A$ که همان $E_p^A(\R^2) = \TpM \subset \R^3$ است، و
۲) اعمال وارون ایزومورفیسم $E_p^A: \R^2 \to \TpM$ روی این زیرفضا عمل می‌کند.
این بدان معناست که شبه-وارون در واقع وارون $E_p^A$ روی فضای مماس است، که دلالت بر این دارد که نگاشت پیمانه با
\begin{align}
    \psiTMp^A: \TpM \to \R^2, \ \ v \mapsto \big(E_p^A \big)^+ v \,.
\end{align}
داده می‌شود. به صورت باز شده، نگاشت پیمانه مطابق با
\begin{align}
    \psiTMp^A(v)\ =&\ 
    \Bigg( \mkern-9mu
    \begin{array}{cc}
        \langle e_1^A, e_1^A \rangle   & \langle e_1^A, e_2^A \rangle \\[4pt]
        \langle e_2^A, e_1^A \rangle   & \langle e_2^A, e_2^A \rangle
    \end{array}
    \mkern-9mu \Bigg)^{\!-1}
    \Bigg( \mkern-9mu
    \begin{array}{cc}
        \langle e_1^A, v \rangle \\[4pt]
        \langle e_2^A, v \rangle
    \end{array}
    \mkern-9mu \Bigg)
    \notag \\
    \ =&\ 
    \frac{1}{
          \langle e_1^A, e_1^A \rangle \langle e_2^A, e_2^A \rangle
        - \langle e_1^A, e_2^A \rangle \langle e_2^A, e_1^A \rangle
    }
    \Bigg( \mkern-9mu
    \begin{array}{cc}
        \phantom{\minus}\langle e_2^A, e_2^A \rangle   &         \minus \langle e_1^A, e_2^A \rangle \\[4pt]
                 \minus \langle e_2^A, e_1^A \rangle   & \phantom{\minus}\langle e_1^A, e_1^A \rangle
    \end{array}
    \mkern-9mu \Bigg)
    \Bigg( \mkern-9mu
    \begin{array}{cc}
        \langle e_1^A, v \rangle \\[4pt]
        \langle e_2^A, v \rangle
    \end{array}
    \mkern-9mu \Bigg) \,.
\end{align}
عمل می‌کند. توجه داشته باشید که، به طور کلی، $\langle e_i^A, v \rangle \neq v^A_i$.
با این حال، اگر (و تنها اگر) $E^A_p$ یک چارچوب راست‌هنجار باشد، یعنی برای $G\leq\OO2$، نگاشت پیمانه به سادگی با تصویر بردار مماس روی محورهای چارچوب داده می‌شود:
\begin{align}\label{eq:embedding_gauge_map_orthonormal_frame}
    \psiTMp^A(v)
    \ =\ 
    \big(E^A_p \big)^{\!\top} v
    \ =\ 
    \Bigg( \mkern-9mu
    \begin{array}{cc}
        \langle e_1^A, v \rangle \\[4pt]
        \langle e_2^A, v \rangle
    \end{array}
    \mkern-9mu \Bigg)
    \qquad \textup{برای \emph{چارچوب‌های راست‌هنجار}، یعنی}\ \langle e_i^A, e_j^A \rangle = \delta_{ij}
\end{align}


عبارت صریح برای انتقال‌دهنده‌های لوی-چیویتا مستقل از مختصات \emph{در امتداد ژئودزیک‌ها} مشابه عبارت نگاشت نمایی است، با این تفاوت که فرمول دوران رودریگز برای دوران دادن بردارهای مماس بین مبدأ و مقصد به جای دوران نقطه مبدأ به مقصد اعمال می‌شود.
فرض کنید $\gamma$ کوتاه‌ترین ژئودزیک بین $p\in S^2$ و $q\in {S^2 \backslash \mkern-1mu\minus\mkern1mu p}$ باشد.
دوران از $p$ به $q$ در امتداد این ژئودزیک سپس با محور $a = p\times q$ و زاویه $\alpha = \arccos\!\big( \langle p,q\rangle \big)$ داده می‌شود.
بر حسب این کمیت‌ها، انتقال لوی-چیویتا از یک بردار مماس جایگذاری شده $v\in \TpM \subset \R^3$ در امتداد ژئودزیک $\gamma$ با بردار دوران یافته
\begin{align}\label{eq:sphere_transport_embedded}
    \PTMgamma(v)\ =\ v \cos(\alpha) + (a\times v) \sin(\alpha) + \big(a\langle a,v\rangle\big) \big(1- \cos(\alpha) \big)
\end{align}
در $\TqM \subset \R^3$ داده می‌شود.
نسبت به پیمانه‌های $\psiTMp^A$ و $\psiTMq^{\widetilde{A}}$ در نقطه شروع $p$ و نقطه پایان $q$ ژئودزیک، این انتقال‌دهنده با عضو گروه
\begin{align}\label{eq:sphere_transporter_explicit_in_coords}
    g_\gamma^{A\widetilde{A}}
    \ =\ \psiTMp^A \circ \PTMgamma \circ \big(\psiTMq^{\widetilde{A}}\big)^{-1}
    \ =\ \big(E_p^A\big)^+ \circ \PTMgamma \circ E_q^{\widetilde{A}} \,.
\end{align}
بیان می‌شود.


تبدیلات پیمانه القا شده توسط ایزومتری به طور مشابه نسبت به چارچوب‌های مرجع صریح با ضرب ماتریسی زیر داده می‌شوند:
\begin{align}\label{eq:embedded_sphere_isom_induced_gauge_trafo}
    g_\phi^{A\widetilde{A}}(p)
    \ =\ \psiTMphip^A \circ \phi \circ \big(\psiTMp^{\widetilde{A}}\big)^{-1}
    \ =\ \big(E_{\phi(p)}^A \big)^{\!+} \phi\: E_p^{\widetilde{A}}
\end{align}