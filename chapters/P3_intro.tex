%!TEX root=../GaugeCNNTheory.tex

\mypart{مروری بر مقالات مرتبط با شبکه‌های کانولوشنی مستقل از مختصات}
\label{part:literature_review}

فرمول‌بندی شبکه‌های کانولوشنی (\CNN ها) مستقل از مختصات بر حسب کلاف‌های $G$-همبسته روی منیفلدهای ریمانی بسیار کلی است و طیف وسیعی از نمونه‌سازی‌های ممکن برای مدل‌ها را پوشش می‌دهد.
برای اثبات این ادعا، ما مجموعه بزرگی از مدل‌های کانولوشنی را از مقالات علمی مرور کرده و آنها را از دیدگاه یکپارچه \CNN های مستقل از مختصات توضیح می‌دهیم.
اکثر مقالات موجود، مدل‌های خود را به صراحت بر حسب $G$-ساختارها و کلاف‌های $G$-همبسته فرمول‌بندی نمی‌کنند.
بنابراین، $G$-ساختارها و نمایش‌های گروهی که \emph{به طور ضمنی فرض شده‌اند}، از الگوهای اشتراک وزن، تقارن‌های کرنل و ویژگی‌های هموردایی مدل‌ها \emph{استنتاج} می‌شوند؛ به عنوان مثال شکل~\ref{fig:G_structure_R3_no_origin} را ببینید.
جدول~\ref{tab:network_instantiations} خلاصه‌ای از طبقه‌بندی حاصل از \CNN های مستقل از مختصات را ارائه می‌دهد.
بخش‌های بعدی، مدل‌های پوشش داده شده و ویژگی‌های آنها را با جزئیات مورد بحث قرار می‌دهند.

\etocsettocdepth{2}
\etocsettocstyle{}{} % from now on only local tocs
\localtableofcontents

بخش~\ref{sec:instantiations_euclidean} کانولوشن‌های هموردا نسبت به گروه آفين ($\Aff(G)$) را روی فضاهای اقلیدسی $\Euc_d$ پوشش می‌دهد.
این مدل‌ها بر $G$-ساختارهای ناوردا نسبت به $\Aff(G)$ تکیه دارند، همانطور که در شکل~\ref{fig:G_structures_R2_main} نشان داده شده است.
مدل‌های بخش~\ref{sec:instantiations_euclidean_polar} روی فضاهای اقلیدسی سوراخ‌دار $\Euc_d\backslash\{0\}$ عمل می‌کنند؛ شکل‌های~\ref{fig:G_structures_R2_no_origin}، \ref{fig:G_structure_R2_no_origin_logpolar} یا~\ref{fig:G_structure_R3_no_origin} را ببینید.
این مدل‌ها نسبت به دوران حول مبدأ انتخاب شده $\{0\}$ هموردا هستند اما هموردایی نسبت به انتقال ندارند.
\CNN های کروی و بیست‌وجهی در بخش~\ref{sec:instantiations_spherical} مورد بحث قرار می‌گیرند.
اکثر این مدل‌ها $G$-ساختارهایی را فرض می‌کنند که در شکل‌های~\ref{fig:G_structure_S2_1} و~\ref{fig:G_structure_S2_2} به تصویر کشیده شده‌اند و بنابراین به ترتیب نسبت به $\SO3$ یا $\SO2$ هموردا هستند.
بخش~\ref{sec:instantiations_mesh} کانولوشن‌های $\GM$ را روی سطوح عمومی که اغلب به صورت مِش گسسته‌سازی می‌شوند، مرور می‌کند.

چند صفحه آینده به بحث در مورد انتخاب‌های طراحی مختلف برای \CNN های مستقل از مختصات می‌پردازد و یک نمای کلی اولیه از مدل‌های جدول~\ref{tab:network_instantiations} ارائه می‌دهد.

%%%%%%%%%%%%%%%%%%%%%%%%%%%%%%%%%%%%%%%%%%%%%%%%%%%%%%%%%%%%%%%%%%%%%%%%%%%%%%%%%%%%%%%%%%%%%%%%%%%%%%%%%%
\afterpage{ % execute argument of this command *after* end of current page
	\clearpage % clear any pending floats
	%%%%%%%%%%%%%%%%%%%%%%%%%%%%%%%%%%%%%%%%%%%%%%%%%%%%%%%%%%%%%%%%%%%%%%%%%%%%%%%%%%%%%%%%%%%%%%%%%%%%%%%%%%
	\begin{table}
		\begin{center}
			%%%%%%%%%%%%%%%%%%%%%%%%%%%%%%%%%%%%%%%%%%%%%%%%%%%%%%%%%%%%%%%%%%%%%%%%%%%%%%%%%%%%%%%%%%%%%%%%%%
			% center overwide table
			\addtolength{\leftskip} {-2cm} % increase (absolute) value if needed
			\addtolength{\rightskip}{-2cm}
			%%%%%%%%%%%%%%%%%%%%%%%%%%%%%%%%%%%%%%%%%%%%%%%%%%%%%%%%%%%%%%%%%%%%%%%%%%%%%%%%%%%%%%%%%%%%%%%%%%
			\vspace*{-6.5ex}
			\def\arraystretch{1.25} % 1 is the default
\setlength\tabcolsep{2.8ex}
\small
\rowcolors{2}{gray!12.5}{white}
\begin{tabular}{>{\tiny\color{gray}}rccclc}
	\toprule
	\\[-8.0ex]
	& منیفلد & گروه ساختار & تقارن سراسری & نمایش & مرجع \\
	& $M$ & $G$ & $\AffGM$ or $\IsomGM$ & $\rho$ &  \\
	\bottomrule
	\rownumber&
	$\Euc_d$ & $\{e\}$ & $\Trans_d$ & \lr{trivial} & \cite{LeCun1990CNNs,
		zhang2019CNNsShiftInvariant}
	\\
	% 1-dim
	\cmidrule(lr){2-6}
	\cmidrule(lr){2-6}
	\rownumber&
	$\Euc_1$ & $\Scale$ & $\Trans_1 \rtimes \Scale$ & \lr{regular} & \cite{romero2020wavelet} \\
	% 2-dim
	\cmidrule(lr){2-6}
	\cmidrule(lr){2-6}
	\rownumber&
	& $\Flip$ & $\Trans_2 \rtimes \Flip$ & \lr{regular} & \cite{Weiler2019_E2CNN} \\
	\cmidrule(lr){3-6}
	\cmidrule(lr){3-6}
	\rownumber&
	& & & \lr{irreps} & \cite{Worrall2017-HNET,
		Weiler2019_E2CNN,
		walters2020trajectory} \\
	\rownumber&
	& & & \lr{regular} & 
	\makecell{
		\cite{Dieleman2016-CYC,
			Cohen2016-GCNN,
			zhou2017oriented,
			Cohen2017-STEER,
			Weiler2018SFCNN,
			bekkers2018roto,
			Hoogeboom2018-HEX,
			scaife2021RadioGalaxy}
		\\
		\cite{Weiler2019_E2CNN,
			graham2020dense,
			lafarge2020rototranslation,
			smets2020pde,
			wang2020incorporating,
			romero2020attentive,
			mohamed2020data}
		\\
		\cite{shen2020PDOeConvs,
			bekkers2020bspline,
			finzi2020generalizing,
			vanderPol2020MDP2,
			gupta2020rotation,
			mondal2020group,
			walters2020trajectory,
			holderrieth2020steerableCNP}
		\\
		\cite{dey2020groupGANs,
			sifre2012combined,
			bruna2013invariant,
			Sifre2013-GSCAT,
			sifre2014rigid,
			oyallon2015scattering,
			chavan2021rescaling,
			han2021ReDet}
	} \\
	\rownumber&
	& & & \lr{quotients} & \cite{Cohen2017-STEER,
		Weiler2019_E2CNN} \\
	\rownumber&
	& & & \lr{regular}$\xrightarrow{\textup{\lr{pool}}}$\lr{trivial}
	& \cite{Cohen2016-GCNN,
		marcos2016learning,
		Weiler2019_E2CNN} \\
	\rownumber&
	& \multirow{-7.5}{*}{$\SO2$} & \multirow{-7.5}{*}{$\SE2$} & \lr{regular}$\xrightarrow{\textup{\lr{pool}}}$\lr{vector}
	& \cite{Marcos2017-VFN,
		Weiler2019_E2CNN} \\
	\cmidrule(lr){3-6}
	\cmidrule(lr){3-6}
	\rownumber&
	& & & \lr{trivial} & \cite{khasanova2018isometric,
		Weiler2019_E2CNN} \\
	\rownumber&
	& & & \lr{irreps} & \cite{Weiler2019_E2CNN} \\
	\rownumber&
	& & & \lr{regular} & 
	\makecell{
		\cite{Dieleman2016-CYC,
			Cohen2016-GCNN,
			Hoogeboom2018-HEX,
			Cohen2017-STEER,
			Weiler2019_E2CNN}
		\\
		\cite{mondal2020group,
			graham2020dense,
			shen2020PDOeConvs}
	} \\
	\rownumber&
	& & & \lr{quotients} & \cite{Cohen2017-STEER} \\
	\rownumber&
	& & & \lr{regular}$\xrightarrow{\textup{\lr{pool}}}$\lr{trivial}   & \cite{Weiler2019_E2CNN} \\
	\rownumber&
	& \multirow{-6.2}{*}{$\OO2$} & \multirow{-6.2}{*}{$\E2$} & \lr{induced} $\SO2$-\lr{irreps} \hspace*{-2.ex} & \cite{Weiler2019_E2CNN} \\
	\cmidrule(lr){3-6}
	\cmidrule(lr){3-6}
	\rownumber&
	\multirow{-15.35}{*}{$\Euc_2$}
	& & & \lr{regular} & \cite{Worrall2019DeepScaleSpaces,
		Sosnovik2020scale,
		bekkers2020bspline,
		zhu2019scale} \\
	\rownumber&
	& \multirow{-2}{*}{$\Scale$}& \multirow{-2}{*}{$\Trans_2\rtimes\Scale$} & \lr{regular}$\xrightarrow{\textup{\lr{pool}}}$\lr{trivial} & \cite{ghosh2019scale} \\
	% 3-dim
	\cmidrule(lr){2-6}
	\cmidrule(lr){2-6}
	\rownumber&
	& & & \lr{irreps} & \cite{3d_steerableCNNs,
		Thomas2018-TFN,
		miller2020relevance,
		Kondor2018-NBN,
		anderson2019cormorant,
		batzner2021se3equivariant} \\
	\rownumber&
	& & & \lr{quaternion} & \cite{zhang2019quaternion} \\
	\rownumber&
	& & & \lr{regular} & \cite{finzi2020generalizing,
		winkels3DGCNNsPulmonary2018,
		Worrall2018-CUBENET} \\
	\rownumber&
	& \multirow{-4}{*}{$\SO3$} & \multirow{-4}{*}{$\SE3$} & \lr{regular}$\xrightarrow{\textup{\lr{pool}}}$\lr{trivial}
	& \cite{andrearczyk2019exploring} \\
	\cmidrule(lr){3-6}
	\cmidrule(lr){3-6}
	\rownumber&
	& & & \lr{regular} & \cite{winkels3DGCNNsPulmonary2018} \\
	\rownumber&
	& & & \lr{quotient} $\OO3/\OO2$ \hspace*{-2ex}
	& \cite{janssen2018design} \\
	\rownumber&
	& \multirow{-3}{*}{$\OO3$} & \multirow{-3}{*}{$\E3$} & \lr{irrep}$\xrightarrow{\textup{\lr{norm}}}$\lr{trivial} \hspace*{-2ex}
	& \cite{poulenard2019effective} \\
	\cmidrule(lr){3-6}
	\cmidrule(lr){3-6}
	\rownumber&
	& $\operatorname{C}_4$ & $\Trans_3 \rtimes \operatorname{C}_4$ & \lr{regular} & \cite{su2020dv} \\
	\cmidrule(lr){3-6}
	\cmidrule(lr){3-6}
	\rownumber&
	\multirow{-7}{*}{$\Euc_3$}
	& $\operatorname{D}_4$ & $\Trans_3 \rtimes \operatorname{D}_4$ & \lr{regular} & \cite{su2020dv} \\
	% Minkowski
	\cmidrule(lr){2-6}
	\cmidrule(lr){2-6}
	\rownumber&
	$\Euc_{d-1,1}$& $\SO{d\minus1,1}$ & $\Trans_d\rtimes\SO{d\minus1,1}$& \lr{irreps} & \cite{shutty2020learning} \\
	\bottomrule
\end{tabular}
			\vspace*{10pt}
			\captionsetup{width=1.06\columnwidth}
			\caption{
				طبقه‌بندی شبکه‌های کانولوشنی در مقالات از دیدگاه \CNN های مستقل از مختصات.
				خطوط پررنگ هندسه‌های مختلف را از هم جدا می‌کنند.
				کانولوشن‌های هموردا نسبت به گروه آفين روی فضاهای اقلیدسی $\Euc_d$
				(ردیف‌های ۱-۲۶)
				در بخش~\ref{sec:instantiations_euclidean} مرور شده‌اند.
				بخش~\ref{sec:instantiations_euclidean_polar} کانولوشن‌های $\GM$ را روی فضاهای اقلیدسی سوراخ‌دار
				${\Euc_d \backslash \{0\}} \cong {S^{d-1} \mkern-4mu\times\! \R^+}$
				(ردیف‌های ۲۷-۳۰) مورد بحث قرار می‌دهد.
				جزئیات مربوط به \CNN های کروی
				(ردیف‌های ۳۱-۳۶)
				در بخش~\ref{sec:instantiations_spherical} یافت می‌شود.
				مدل‌های موجود در
				ردیف‌های (۳۷-۴۱)
				روی سطوح عمومی، که عمدتاً با مش‌های مثلثی نمایش داده می‌شوند، عمل می‌کنند؛ به بخش~\ref{sec:instantiations_mesh} مراجعه کنید.
				دو خط آخر، کانولوشن‌های موبیوس ما از بخش~\ref{sec:mobius_conv} را فهرست می‌کنند.
				$\Trans_d$،~$\Flip$ و $\Scale$ به ترتیب نمایانگر گروه‌های انتقال، بازتاب و پیمانه هستند، در حالی که $\CN$ و $\DN$ گروه‌های دوری و دووجهی هستند.
				نمایش‌های با ابعاد بی‌نهایت در پیاده‌سازی‌ها گسسته‌سازی یا نمونه‌برداری می‌شوند.
				به عنوان مثال، نمایش‌های منظم $\SO2$ یا $\OO2$ معمولاً با نمایش‌های منظم گروه‌های دوری یا دووجهی $\CN$ یا $\DN$ تقریب زده می‌شوند.
			}
			\label{tab:network_instantiations}
		\end{center}
	\end{table}
	%%%%%%%%%%%%%%%%%%%%%%%%%%%%%%%%%%%%%%%%%%%%%%%%%%%%%%%%%%%%%%%%%%%%%%%%%%%%%%%%%%%%%%%%%%%%%%%%%%%%%%%%%%
	\thispagestyle{empty}
	\clearpage % force a page break
}
%%%%%%%%%%%%%%%%%%%%%%%%%%%%%%%%%%%%%%%%%%%%%%%%%%%%%%%%%%%%%%%%%%%%%%%%%%%%%%%%%%%%%%%%%%%%%%%%%%%%%%%%%%

\subsection*{انتخاب‌های طراحی و نمای کلی}
\label{sec:instantiations_taxonomy}

یک \CNN مستقل از مختصات \emph{در تئوری} به طور کامل با موارد زیر مشخص می‌شود:
\begin{itemize}
	\item[1)] انتخاب یک \emph{منیفلد ریمانی} $(M,\eta)$
	\item[2)] \emph{G-ساختار} آن، $\GM$،
	\item[3)] یک \emph{اتصال G-سازگار} که انتقال‌دهنده‌های ویژگی $\PAgamma$ را مشخص می‌کند،
	\item[4)] \emph{انواع میدان} یا نمایش‌های $G$ برای هر فضای ویژگی، $\rho$، و
	\item[5)] انتخاب \emph{غیرخطی‌های} $G$-هموردا.
\end{itemize}
\emph{ژئودزیک‌ها، نگاشت‌های نمایی} و \emph{لگاریتمی} از اتصال کانونی لوی-چیویتا روی~$M$ به دست می‌آیند.%
\footnote{
	ممکن است عجیب به نظر برسد که ژئودزیک‌ها و انتقال‌دهنده‌های ویژگی بر اساس اتصال‌های بالقوه متفاوتی محاسبه شوند.
	زمانی که اتصال انتقال‌دهنده با لوی-چیویتا متفاوت است، معمولاً به این دلیل است که اتصال لوی-چیویتا با $G$-ساختار انتخاب شده، زمانی که $G<\OO{d}$ باشد، $G$-سازگار نیست.
	چند مثال در پاراگراف مربوط به اتصالات $G$-سازگار در ادامه ارائه شده است.
}
\emph{گروه ایزومتری} $\IsomGM$ که شبکه نسبت به آن هموردا است، از متریک و $G$-ساختار نتیجه می‌شود.
تمام \emph{فضاهای کرنل} $\KG$ توسط نمایش‌های گروهی فضاهای ویژگی که بین آن‌ها نگاشت انجام می‌دهند، تعیین می‌شوند.
\emph{اشتراک وزن} با قرار دادن یک کرنل الگوی $G$-راهبری‌پذیر نسبت به یک چارچوب $G$ دلخواه در $\GpM$ برای هر نقطه $p\in M$ انجام می‌شود.

در عمل، کاربر با پرسش‌های طراحی اضافی روبرو است، به عنوان مثال در مورد گسسته‌سازی هندسه، کدگذاری میدان‌های ویژگی، الگوریتم‌های عددی برای محاسبه ژئودزیک‌ها و انتقال‌دهنده‌ها و غیره.
این بخش یک نمای کلی سطح بالا از تمام انتخاب‌های طراحی مرتبط ارائه می‌دهد.
جزئیات بیشتر در بخش‌های بعدی \ref{sec:instantiations_euclidean}، \ref{sec:instantiations_euclidean_polar}، \ref{sec:instantiations_spherical} و \ref{sec:instantiations_mesh} یافت می‌شود.

\paragraph{گسسته‌سازی منیفلدها و میدان‌های ویژگی:}
پیاده‌سازی‌ها در نمایش منیفلدها و نمونه‌برداری از میدان‌های ویژگی متفاوت هستند.

فضاهای اقلیدسی $\Euc_d$ شبکه‌های پیکسلی منظمی مانند $\Z^d$ یا شبکه شش‌ضلعی را می‌پذیرند~\cite{Hoogeboom2018-HEX}.
به طور کلی‌تر، شبکه‌های محلی منظم برای منیفلدهای محلی تخت مانند نوار موبیوس و بیست‌وجهی مناسب هستند؛ به شکل‌های~\ref{fig:mobius_conv_numerical} و~\ref{fig:ico_cutting} مراجعه کنید.
میدان‌های ویژگی در فضاهای اقلیدسی همچنین می‌توانند روی یک ابر نقطه نامنظم نمونه‌برداری شوند.
این امر به عنوان مثال هنگام پردازش محیط‌های اتمی، که در آن موقعیت اتم‌ها به عنوان مکان‌های نمونه‌برداری عمل می‌کنند، مفید است~\cite{Thomas2018-TFN}.

یک تفاوت مهم بین این دو رویکرد این است که شبکه‌های پیکسلی منظم نسبت به انتقال‌های پیوسته در $\Trans_d = (\R^d,+)$ هموردا نیستند، بلکه تنها نسبت به زیرگروه انتقال‌های گسسته که شبکه را حفظ می‌کنند، به عنوان مثال $(\Z^d,+)$، هموردا هستند. \CNN های روی شبکه‌های منظم علاوه بر این معمولاً عملیات تجمیع فضایی اعمال می‌کنند که هموردایی مدل‌ها را حتی بیشتر کاهش می‌دهد. به طور خاص، با در نظر گیری اینکه عملیات تجمیع دارای گام $n$ پیکسل است، نسبت به انتقال‌ها در $(n\Z^d,+)$ هموردا است.پس از $L$ لایه تجمیع در یک شبکه کانولوشنی، این امر به این معنا است که مدل به طور کلی تنها نسبت به انتقال‌ها در $(n^L\Z^d,+)$ هموردا است ـ این مسئله به صورت تجربی در \cite{azulay2018shift} بررسی شده است.\cite{zhang2019CNNsShiftInvariant} ژانگ پیشنهاد می‌کنند که این مشکل با جایگزینی لایه‌های تجمیع با گام $n$ با لایه‌های تجمیع با گام ۱ (با همان اندازه پنجره تجمیع)، یک فیلتر پایین‌گذر و یک زیرنمونه‌برداری $n$-پیکسلی برطرف شود.

فیلترینگ پایین‌گذر اضافی بین عملیات تجمیع و زیرنمونه‌برداری از اثرات نام‌گذاری جلوگیری می‌کند، که نشان داده شده است شبکه‌ها را نسبت به انتقال‌هایی که عضو $(n\Z^d,+)$ نیستند، به اندازه کافی پایدارتر می‌کند.

فضاهای خمیده مانند کره ۲-بعدی $S^2$ به طور کلی شبکه‌های نمونه‌برداری منظم را نمی‌پذیرند.
یک گسسته‌سازی به ظاهر واضح، بر حسب یک شبکه نمونه‌برداری منظم در مختصات کروی است (معادله~\eqref{eq:spherical_coords} و شکل~\ref{fig:spherical_equirectangular_1})، اما از آنجا که این مختصات ایزومتریک نیستند، سیگنال را در نزدیکی قطب‌ها بیش‌نمونه‌برداری می‌کنند~\cite{zhao2018distortion,tateno2018distortion}.
شبکه‌های نمونه‌برداری تقریباً یکنواخت روی $S^2$ شامل «مجموعه مارپیچ تعمیم‌یافته»~\cite{coors2018spherenet} یا شبکه ایکوسکروی (\lr{icospherical})~\cite{jiang2019spherical,kicanaoglu2019gaugeSphere} هستند.
به طور جایگزین، میدان‌های ویژگی ممکن است در حوزه طیفی گسسته‌سازی شوند.
برای کره، این کار از طریق بسط بر حسب هماهنگ‌های کروی برای میدان‌های اسکالر، هماهنگ‌های کروی با وزن اسپین برای میدان‌های نمایش تحویل‌ناپذیر یا ماتریس‌های \lr{D} ویگنر برای میدان‌های ویژگی عمومی انجام می‌شود~\cite{esteves2018zonalSpherical,esteves2020spinweighted,Cohen2018-S2CNN,kondor2018ClebschGordan}.

سطوح عمومی معمولاً به صورت مش‌های مثلثی نمایش داده می‌شوند؛ به بخش~\ref{sec:surfaces_geom_mesh} مراجعه کنید.
میدان‌های ویژگی سپس می‌توانند روی رئوس، یال‌ها یا وجوه مش نمونه‌برداری شوند~\cite{deGoes2016VectorFieldProcessing}.
رزولوشن بالاتر میدان‌های ویژگی را می‌توان با کدگذاری آنها از طریق نقشه‌های بافت (\lr{texture maps}) به دست آورد~\cite{li2019crossAtlas,huang2019texturenet}.
به طور جایگزین، سطوح ممکن است به صورت ابرهای نقطه نمایش داده شوند~\cite{tatarchenko2018tangent,jin2019NPTCnet}.

\paragraph{\textit{\lr{G}}-ساختارها \textit{\lr{GM}} و گروه‌های ساختاری \textit{\lr{G}}:}
انتخاب خاص $G$-ساختار که باید توسط شبکه رعایت شود، به وظیفه یادگیری و توپولوژی $M$ بستگی دارد (اگر پیوستگی یا همواری کانولوشن مورد نیاز باشد).
به طور کلی، $M$ مجهز به یک $\OO{d}$-ساختار است، یعنی یک کلاف از چارچوب‌های مرجع راست‌هنجار نسبت به متریک ریمانی داده شده.
یک \emph{ارتقا} به گروه‌های ساختاری $G$ با ${\OO{d} < G \leq \GL{d}}$ به طور یکتا توسط تبدیلات پیمانه‌ای با مقادیر $G$ از چارچوب‌های راست‌هنجار تعیین می‌شود.
\emph{تقلیل‌های} گروه ساختاری به $G < \OO{d}$، در مقابل، لزوماً یکتا نیستند و اطلاعات هندسی اضافی را کدگذاری می‌کنند.
به عنوان مثال، یک تقلیل به $G=\SO{d}$ نیازمند یک جهت‌گیری روی منیفلد است.%
\footnote{
	برای یک منیفلد منفرد و همبند، این انتخاب تا زمانی که مقداردهی اولیه کرنل نسبت به هر دو جهت‌گیری متقارن باشد، دلخواه است.
	در این حالت شبکه به سادگی کرنل‌های بازتابیده را برای جهت‌گیری‌های مختلف یاد می‌گیرد.
	هنگام بررسی یک مجموعه داده متشکل از چندین منیفلد، جهت‌گیری نسبی آنها برای تعمیم صحیح مرتبط است.
}
بخش‌های بعدی به بحث در مورد انتخاب‌های بیشتر (عمدتاً به طور ضمنی انجام شده) از $G$-ساختارها که در مقالات یافت می‌شوند، می‌پردازند؛ به عنوان مثال، شکل‌های
\ref{fig:G_structures_R2_main}،
\ref{fig:G_structures_R2_no_origin}،
\ref{fig:G_structure_R2_no_origin_O2}،
\ref{fig:G_structure_R2_no_origin_logpolar}،
\ref{fig:G_structure_R3_no_origin}
\ref{fig:G_structures_S2_main}،
یا~\ref{fig:G_structures_ico} را ببینید.
این ساختارها یا توسط تقاضا برای هموردایی تحت گروه ایزومتری $\IsomGM$ تعیین می‌شوند، یا به طور کانونی روی منیفلد داده شده‌اند، یا به طور خاص برای $\{e\}$-ساختارها، به صورت الگوریتمی از طریق برخی روش‌های ابتکاری ثابت شده‌اند.
به یاد بیاورید که $\{e\}$-ساختارها روی منیفلدهای غیرموازی‌پذیر (طبق تعریف) لزوماً ناپیوسته هستند.

گروه‌های ساختاری که بیشتر در مقالات با آنها مواجه می‌شویم، عبارتند از:
\begin{itemize}
	\item[{\rule[2.0pt]{2pt}{2pt}}]
	گروه \emph{بدیهی} $\{e\}$، متناظر با \CNN های غیرمستقل از مختصات با کرنل‌های نامحدود
	\item[{\rule[2.0pt]{2pt}{2pt}}]
	گروه \emph{بازتاب} $\Flip \cong \Z/2\Z$، که اولین محور چارچوب را برمی‌گرداند
	\item[{\rule[2.0pt]{2pt}{2pt}}]
	گروه‌های \emph{متعامد خاص} $\SO{d}$
	\ \ (دوران‌های پیوسته)
	\item[{\rule[2.0pt]{2pt}{2pt}}]
	گروه‌های \emph{متعامد} $\OO{d}$
	\ \ (دوران‌ها و بازتاب‌های پیوسته)
	\item[{\rule[2.0pt]{2pt}{2pt}}]
	گروه \emph{پیمانه} $\Scale \cong (\R^+,*)$،
\end{itemize}
از آنجا که سه گروه آخر گروه‌های لی پیوسته هستند، در پیاده‌سازی‌های عددی گاهی توسط زیرگروه‌های متناهی تقریب زده می‌شوند.
به عنوان مثال، $\SO2$ و $\OO2$ اغلب با گروه‌های \emph{دوری} $\CN$ یا گروه‌های \emph{دووجهی}~$\DN$ مدل‌سازی می‌شوند، در حالی که دوران‌ها و بازتاب‌های سه‌بعدی در $\OO3$ را می‌توان با گروه‌های \emph{چندوجهی} (گروه‌های تقارن اجسام افلاطونی، به عنوان مثال بیست‌وجهی) تقریب زد.
برای کاهش پیچیدگی طبقه‌بندی مدل‌ها در جدول~\ref{tab:network_instantiations}، ما تصمیم گرفتیم بین تقارن‌های پیوسته و تقریب‌های آنها توسط زیرگروه‌های متناهی تمایز قائل نشویم.
با این حال، ما چنین تقریب‌هایی را در بحث دقیق خود در مورد مدل‌ها در بخش‌های بعدی بیان خواهیم کرد.

\paragraph{اتصالات \textit{\lr{G}}-سازگار:}
تمام مدل‌ها یا اتصال کانونی \emph{لوی-چیویتا} روی $M$ یا اتصال یکتای \emph{بدیهی} را که توسط یک $\{e\}$-ساختار القا می‌شود، در نظر می‌گیرند.
انتخاب اتصال برای شبکه‌هایی که صرفاً روی \emph{میدان‌های اسکالر} عمل می‌کنند، که انتقال آنها همیشه بدیهی است، بی‌اهمیت (و بنابراین نامشخص) می‌شود.

به طور خاص، تمام \CNN های اقلیدسی از بخش~\ref{sec:instantiations_euclidean} از انتقال‌دهنده‌های لوی-چیویتا استفاده می‌کنند، که بردارها را به گونه‌ای منتقل می‌کنند که در معنای معمول در فضاهای اقلیدسی $\Euc_d$ موازی باقی بمانند؛ به شکل~\ref{fig:transport_flat} مراجعه کنید.
این امکان‌پذیر است زیرا اتصال لوی-چیویتا با $G$-ساختارهای مدل‌ها (تعریف شده در معادله~\eqref{eq:G_lifted_G_structure_Rd} و در شکل~\ref{fig:G_structures_R2_main} به تصویر کشیده شده) $G$-سازگار است.%
\footnote{
	در مقابل، $\{e\}$-ساختار اقلیدسی در شکل~\ref{fig:frame_field_automorphism_2} با اتصال لوی-چیویتا روی~$\Euc_2$ ناسازگار خواهد بود.
}

مدل‌های روی فضاهای اقلیدسی سوراخ‌دار $\Euc_d \backslash \{0\}$ از بخش~\ref{sec:instantiations_euclidean_polar} یا بر اساس $\{e\}$-ساختارها هستند و/یا میدان‌های اسکالر را در نظر می‌گیرند.
بنابراین، آنها از اتصالات بدیهی استفاده می‌کنند که با اتصال کانونی لوی-چیویتا روی~$\Euc_d \backslash \{0\}$ متفاوت است.

تمام \CNN های کروی که بر $\SO2$-ساختار در شکل~\ref{fig:G_structure_S2_1} تکیه دارند (در بخش~\ref{sec:spherical_CNNs_fully_equivariant} مرور شده‌اند)، ویژگی‌ها را مطابق با اتصال لوی-چیویتا روی $S^2$ منتقل می‌کنند (شکل~\ref{fig:transport_sphere}).
آنهایی که روی $\{e\}$-ساختار در شکل~\ref{fig:G_structure_S2_2} عمل می‌کنند (در بخش~\ref{sec:spherical_CNNs_azimuthal_equivariant} مرور شده‌اند) دوباره یک اتصال بدیهی را در نظر می‌گیرند زیرا اتصال لوی-چیویتا کروی با این $\{e\}$-ساختار ناسازگار است.
\CNN بیست‌وجهی با $\mathbb{C}_{6}$-ساختار، شکل~\ref{fig:G_structure_ico_3}، ویژگی‌ها را مطابق با اتصال لوی-چیویتا بیست‌وجهی $\mathbb{C}_{6}$-سازگار منتقل می‌کند.%
\footnote{
	اتصالات لوی-چیویتا گسسته روی مش‌ها در بخش~\ref{sec:surfaces_geom_mesh} و~\cite{craneDiscreteDifferentialGeometry2014,craneTrivialConnectionsDiscrete2010} مورد بحث قرار گرفته‌اند.
}

تمام \CNN ها روی سطوح عمومی که در ردیف‌های (۳۷-۳۹) جدول~\ref{tab:network_instantiations} فهرست شده‌اند،
سطوح جهت‌داری را فرض می‌کنند که مجهز به یک $\SO2$-ساختار هستند.
آنها ویژگی‌ها را با اتصال لوی-چیویتا $\SO2$-سازگار سطوح منتقل می‌کنند.
سایر \CNN های سطوح بر اساس $\{e\}$-ساختارها هستند و/یا روی میدان‌های اسکالر عمل می‌کنند -- بنابراین انتقال ویژگی آنها بدیهی است.

کانولوشن‌های نوار موبیوس ما ویژگی‌ها را از طریق اتصال لوی-چیویتا منتقل می‌کنند، که با $\Flip$-ساختار فرض شده سازگار است.

به یاد بیاورید که اتصال لوی-چیویتا به طور یکتا توسط متریک تعیین می‌شود و بنابراین به طور کلی ناوردای ایزومتری است؛ پاورقی~\ref{footnote:LeviCivita_isometry_invariance} در بخش~\ref{sec:isom_expmap_transport} را مقایسه کنید.
از آنجا که اتصالات بدیهی $\{e\}$-سازگار به طور یکتا توسط $\{e\}$-ساختار مشخص می‌شوند، تقارن‌های آن را به اشتراک می‌گذارند، یعنی تحت عمل~$\IsomeM$ ناوردا هستند.
این موضوع با قضیه~\ref{thm:isom_equiv_GM_conv} دلالت بر این دارد که کانولوشن‌های $\GM$ که بر اساس این اتصالات هستند، $\IsomGM$-هموردا می‌باشند.

\paragraph{پول‌بک‌های انتقال‌دهنده و تصویرهای جایگزین به $\boldsymbol{\TpM}$:}
پول‌بک انتقال‌دهنده $\Expspf$ که در تعریف~\ref{dfn:Expf_pullback_field} و معادله~\eqref{eq:transporter_pullback_in_coords} تعریف شده است، یک میدان ویژگی $f$ را در یک پارامترسازی ژئودزیک روی فضای مماس~$\TpM$ نمایش می‌دهد.
بخش انتقال عملیات توسط اتصال $G$-سازگار تعیین می‌شود.
ژئودزیک‌ها ـ و بنابراین نگاشت‌های نمایی $\exp_p: \TpM\to M$ -- در فضاهای اقلیدسی $\Euc_d$ و کره~$S^2$ عبارات فرم بسته دارند.
به طور خاص، نگاشت‌های نمایی روی $\Euc_d$ در مختصات دکارتی به جمع برداری در معادله~\eqref{eq:exp_map_euclidean} کاهش می‌یابند، به طوری که کانولوشن‌های $\GM$ اقلیدسی به کانولوشن‌های متعارف روی~$\R^d$ تبدیل می‌شوند؛ به قضیه~\ref{thm:Euclidean_GM_conv_is_conventional_conv} مراجعه کنید.
ژئودزیک‌ها روی $S^2$ به خوبی شناخته شده‌اند که توسط دوایر عظیمه کره داده می‌شوند.
اگر کره به عنوان جایگذاری شده در $\R^3$ در نظر گرفته شود، نگاشت نمایی به صراحت توسط معادله~\eqref{eq:sphere_expmap_explicit} داده می‌شود.
ژئودزیک‌ها روی مش‌های سطوح عمومی با راه‌حل‌های فرم بسته توصیف نمی‌شوند بلکه به صورت عددی محاسبه می‌شوند؛ به بخش~\ref{sec:surfaces_geom_mesh} مراجعه کنید.
در مقایسه با محیط هموار، باید بین ژئودزیک‌های «کوتاه‌ترین» و «مستقیم‌ترین» روی مش‌ها تمایز قائل شد~\cite{polthier1998straightest}.

پول‌بک میدان‌های ویژگی به مختصات نرمال ژئودزیک تنها راه نمایش میدان‌های ویژگی روی فضاهای مماس نیست.
در مقالات مربوط به \CNN های کروی، استفاده از تصویرهای نومونیک، که در شکل~\ref{fig:gnomonic_proj} به تصویر کشیده شده‌اند، نسبتاً رایج است.
قضیه~\ref{thm:gnomonic} نشان می‌دهد که این تصویر را می‌توان به عنوان یک مورد خاص از پارامترسازی ژئودزیک عمومی‌تر ما پس از اعمال یک تاب شعاعی به کرنل‌ها در نظر گرفت.
بنابراین مدل‌های مربوطه دقیقاً به عنوان کانولوشن‌های $\GM$ شناسایی می‌شوند.
سطوحی که در یک فضای محیطی مانند $\R^3$ جایگذاری شده‌اند، ممکن است علاوه بر این بر تصویرهای مختلفی در فضای جایگذاری تکیه کنند؛ به عنوان مثال سه مدل آخر که در بخش~\ref{sec:e_surface_conv} مورد بحث قرار گرفته‌اند را ببینید.
توجه داشته باشید که این رویکردها واقعاً با رویکرد ما متفاوت هستند، یعنی این سه مدل دقیقاً کانولوشن‌های $\GM$ نیستند.

\paragraph{نمایش‌های \textit{\lr{G}} و غیرخطی‌ها:}
تقریباً تمام مدل‌ها یا نمایش بدیهی، نمایش‌های تحویل‌ناپذیر یا نمایش‌های منظم را به عنوان انواع میدان در نظر می‌گیرند.
استثنائات شامل نمایش‌های خارج‌قسمتی، نمایش‌های القایی عمومی‌تر، نمایش‌های حاصل‌ضرب تانسوری و، به طور خاص برای $G=\SO3$، نمایش کواترنیونی هستند.
نمایش‌های با ابعاد بی‌نهایت، به ویژه نمایش‌های منظم و خارج‌قسمتی گروه‌های لی، در پیاده‌سازی‌ها گسسته‌سازی می‌شوند.
این کار می‌تواند یا از طریق نمونه‌برداری مونت کارلو یا با بازگشت به نمایش‌های متناظر زیرگروه‌های متناهی همانطور که در بالا بحث شد، انجام شود.

غیرخطی‌ها باید نسبت به عمل نمایش‌های $G$ انتخاب شده هموردا باشند.
از آنجا که میدان‌های اسکالر $G$-ناوردا هستند، غیرخطی‌های معمول مانند \lr{ReLU} روی آنها عمل می‌کنند.
میدان‌های ویژگی که مطابق با نمایش‌های جایگشتی، به ویژه نمایش‌های منظم، تبدیل می‌شوند، به صورت کانال به کانال عمل می‌کنند.
سایر انواع میدان به غیرخطی‌های سفارشی نیاز دارند -- ما برای بحث در مورد انتخاب‌های خاص به~\cite{Weiler2019_E2CNN} ارجاع می‌دهیم.

\paragraph{فضاهای کرنل \textit{G}-راهبری‌پذیر:}
کانولوشن‌های $\GM$ میدان‌های ورودی از نوع $\rhoin$ را به میدان‌های خروجی از نوع~$\rhoout$ با کانوالو کردن آنها با کرنل‌های $G$-راهبری‌پذیر~$K\in\KG$ نگاشت می‌دهند.
از آنجا که فضای $\KG$ از کرنل‌های $G$-راهبری‌پذیر، تعریف~\ref{dfn:G-steerable_kernel_def_43}، یک فضای برداری است، معمولاً بر حسب یک پایه $\{K_1,\,\dots,\,K_N\}$ از~$\KG$ پارامتری می‌شود.
قبل از محاسبه کانولوشن، کرنل یادگرفته شده $K = \sum_{i=1}^N w_i K_i$ در این پایه بسط داده می‌شود، که در آن $\{w_1,\dots,w_N\}$ وزن‌های با مقادیر حقیقی هستند که باید بهینه‌سازی شوند.
فضاهای کرنل اثبات‌شده کامل برای گروه‌های $G\leq\OO2$ در~\cite{Weiler2019_E2CNN} پیاده‌سازی شده‌اند.%
\footnote{\url{https://quva-lab.github.io/e2cnn/api/e2cnn.kernels.html}}
یک تعمیم از قضیه ویگنر-اکارت، پایه‌های فضای کرنل را برای گروه‌های ساختاری فشرده عمومی~$G$ مشخص می‌کند~\cite{lang2020WignerEckart}.

در عمل، اکثر نویسندگان از فرمول‌بندی مبتنی بر نظریه نمایش برای میدان‌های ویژگی و کرنل‌های راهبری‌پذیر استفاده نمی‌کنند، بلکه آنها را بر اساس شهود فرمول‌بندی می‌کنند.
به طور خاص، اکثر نویسندگان یک نوع میدان ورودی داده شده را فرض می‌کنند و عملیات کانولوشن مختلفی را پیشنهاد می‌دهند که به گونه‌ای مهندسی شده‌اند که میدان خروجی حاصل به شیوه‌ای هموردا (یا مستقل از مختصات) تبدیل شود.%
\footnote{
	این برخلاف رویکرد ماست که میدان‌های ورودی و خروجی را ثابت می‌کند و سپس به دنبال محدودیت حاصل بر کرنل‌های کانولوشن می‌گردد.
}
در حالی که این رویکردها کرنل‌های $G$-راهبری‌پذیر خاصی را پیشنهاد می‌کنند که بین میدان‌های $\rhoin$ و میدان‌های $\rhoout$ نگاشت انجام می‌دهند، این کرنل‌ها گاهی اوقات فضای کامل کرنل‌های ممکن را پوشش نمی‌دهند.
این موضوع به عنوان مثال در مورد \emph{\lr{MDGCNN}ها} و \emph{\lr{PFCNN}ها}، که در بخش~\ref{sec:so2_surface_conv} بحث می‌شوند، صدق می‌کند.