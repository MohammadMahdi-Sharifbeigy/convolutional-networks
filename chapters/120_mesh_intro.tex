%!TEX root=../GaugeCNNTheory.tex


\section{\lr{CNN}های مستقل از مختصات روی سطوح عمومی}
\label{sec:instantiations_mesh}


به جای عمل روی یک هندسه ثابت، کانولوشن‌های $\GM$ در این بخش روی منیفلدهای عمومی تعریف می‌شوند.
ما مرور خود را به سطوح ($d=2$) محدود می‌کنیم زیرا از پیاده‌سازی‌ها روی منیفلدهای (عمومی) با ابعاد بالاتر آگاه نیستیم.
سیگنال‌هایی که باید پردازش شوند، می‌توانند یا مستقیماً توسط مجموعه داده ارائه شوند یا از هندسه سطوح محاسبه شوند.
مثال‌هایی برای دسته اول شامل بافت‌های رنگی یا کمیت‌های فیزیکی مانند میدان‌های دما یا تنش دیواره یک ظرف تحت فشار است.
دسته دوم می‌تواند به عنوان مثال شامل انحناهای گاوسی و اصلی، توصیفگرهای \lr{SHOT} یا امضاهای کرنل موج باشد.
اکثر کاربردها تاکنون بر روی طبقه‌بندی سطوح~\cite{huang2019texturenet,jin2018learning,Wiersma2020}، قطعه‌بندی بخش‌هایی از آنها~\cite{poulenard2018multi,huang2019texturenet,Wiersma2020,Yang2020parallelFrameCNN} یا یافتن تناظر بین سطوح مختلف~\cite{masci2015geodesic,boscaini2016learning,schonsheck2018parallel,Wiersma2020,deHaan2020meshCNNs} متمرکز شده‌اند.
کاربردهای دیگر شامل پیش‌بینی کمیت‌های فیزیکی مانند تنش مکانیکی~\cite{sun2018zernet} یا سنتز بافت‌های رنگی~\cite{turk2001texture,ying2001texture} یا تغییر شکل‌های هندسی~\cite{hertz2020GeomTextureSynthesis} است.


طراحی \lr{CNN}های اقلیدسی و کروی به شدت تحت تأثیر نیاز به هموردایی تقارن سراسری است.
از آنجا که سطوح عمومی معمولاً دارای گروه‌های ایزومتری بدیهی هستند، این اصل راهنما از بین می‌رود، که آزادی زیادی را در انتخاب $G$-ساختارها به ما می‌دهد.
مدل‌هایی که در این بخش مرور می‌کنیم را می‌توان به کانولوشن‌های سطوح \emph{راهبری‌پذیر-دورانی} و \emph{$\{e\}$-راهبری‌پذیر} طبقه‌بندی کرد.
هر دو رویکرد به مسئله عدم وجود یک جهت کانونی روی سطوح می‌پردازند، با این حال، آنها این کار را به روشی اساساً متفاوت انجام می‌دهند.
مدل‌های راهبری‌پذیر-دورانی، عدم وجود جهت مرجع را با طراحی هموردای خود در نظر می‌گیرند و با همه جهات به طور یکسان رفتار می‌کنند.
$\SO2$-ساختار زیربنایی آنها -- به جز یک انتخاب عملاً بی‌اهمیت از جهت‌گیری%
\footnote{
	جهت‌گیری انتخاب شده روی یک منیفلد (همبند، جهت‌پذیر) دلخواه است زیرا کرنل‌ها یاد گرفته می‌شوند.
	اگر جهت‌گیری مخالف انتخاب می‌شد، آموزش فقط منجر به کرنل‌های با جهت‌گیری مخالف می‌شد.
}
-- توسط متریک ریمانی ثابت می‌شود.
بنابراین مدل‌های راهبری‌پذیر-دورانی عمدتاً در انتخاب انواع میدان متفاوت هستند.
مدل‌های $\{e\}$-راهبری‌پذیر، غیرهموردا هستند و بنابراین با یک نوع میدان (غیربدیهی) مرتبط نیستند.
با این حال، آنها از یکدیگر با انتخاب خاص $\{e\}$-ساختار که برای تعیین ترازهای کرنل استفاده می‌شود، متفاوت هستند.


\etocsettocdepth{3}
\etocsettocstyle{}{} % from now on only local tocs
\localtableofcontents


این بخش به صورت زیر سازماندهی شده است:
ما در بخش~\ref{sec:surfaces_geom_classical_smooth} با یک مقدمه (بسیار) کوتاه بر هندسه دیفرانسیل کلاسیک سطوح شروع می‌کنیم و به ویژه تفاوت بین هندسه ذاتی و خارجی آنها را مورد بحث قرار می‌دهیم.
در عمل، اکثر پیاده‌سازی‌ها روی سطوح گسسته‌سازی شده عمل می‌کنند.
بخش~\ref{sec:surfaces_geom_mesh} یک نمای کلی از هندسه مش‌های سطحی مثلثی ارائه می‌دهد، که مسلماً رایج‌ترین گسسته‌سازی‌های سطح در مقالات یادگیری عمیق هستند.
در بخش~\ref{sec:so2_surface_conv} ما کانولوشن‌های سطحی راهبری‌پذیر-دورانی را مورد بحث قرار می‌دهیم.
روش‌های ابتکاری برای ثابت کردن میدان‌های چارچوب که کانولوشن‌های سطحی $\{e\}$-راهبری‌پذیر را تعریف می‌کنند، در بخش~\ref{sec:e_surface_conv} مرور می‌شوند.


برای کامل بودن، در پاراگراف بعدی به چند رویکرد جایگزین برای تعریف کانولوشن‌های سطح اشاره می‌کنیم
قبل از اینکه به محتوای اصلی این بخش بپردازیم.

\paragraph{\cnnهای سطوح فراتر از کانولوشن‌های \textit{\lr{GM}}:}

در حالی که تعداد قابل توجهی از \lr{CNN}های سطح را می‌توان به عنوان کانولوشن‌های $\GM$ تفسیر کرد، بسیاری از طراحی‌های شبکه جایگزین پیشنهاد شده‌اند.
این روش‌ها به عنوان مثال بر موارد زیر تکیه دارند:
کانولوشن‌های گراف روی مش‌های سطح،
رویکردهای طیفی،
رندرهای چند-نما از جایگذاری‌های سطح،
روش‌های حجمی در فضای جایگذاری،
عملگرهای دیفرانسیل،
یا عملگرهای دیگری که بلافاصله روی ساختارهای داده مش عمل می‌کنند.
مرور مختصر زیر به منظور ارائه یک نمای کلی از جهات مختلفی است که مورد بررسی قرار گرفته‌اند.


یک روش برای طبقه‌بندی یا قطعه‌بندی سطوح جایگذاری شده، \emph{رندر کردن آنها از چندین دیدگاه} و پردازش رندرها با \lr{CNN}های اقلیدسی متعارف است.
ویژگی‌های حاصل سپس با
تجمیع روی دیدگاه‌ها~\cite{su2015multi,qi2016volumetric}
یا از طریق یک روش اجماع~\cite{paulsen2018multi} \lr{agregaste} می‌شوند.
استیوس و همکاران\cite{esteves2019multiView} انتخاب می‌کنند که دیدگاه‌های دوربین را روی یک کره مطابق با یک زیرگروه گسسته از $\SO3$ قرار دهند، به عنوان مثال گروه بیست‌وجهی.
سپس ویژگی‌های حاصل به طور مشترک از طریق یک کانولوشن گروهی گسسته (نه یک کانولوشن سطح) پردازش می‌شوند.


به جای تصویر کردن سطح با رندر کردن آن، می‌توان آن را با تعریف یک \emph{چارت} به $\R^2$ تصویر کرد.
سینها و همکاران\cite{sinha2016deep} چارت‌های سراسری تقریباً مساحت-نگهدار (\lr{authalic}) را روی توپولوژی‌های کروی تعریف می‌کنند.
این چارت‌ها ناپیوسته و به طور کلی زاویه-نگهدار (\lr{conformal}) نیستند.
یک \lr{CNN} اقلیدسی متعارف برای پردازش تصاویر حاصل استفاده می‌شود.
ناپیوستگی‌ها را می‌توان با پول‌بک کردن ویژگی‌های سطح در امتداد نگاشت‌های پوششی چنبره‌ای (\lr{toric})~\cite{maron2017convolutional} یا عمومی‌تر~\cite{haim2018surface,benhamu2018multichart} دور زد.
کانولوشن اقلیدسی بعدی روی پول‌بک را نمی‌توان به عنوان یک کانولوشن $\GM$ تفسیر کرد زیرا لایه‌های نگاشت پوششی، $\{e\}$-ساختارهای متفاوت و ناسازگاری را روی سطح القا می‌کنند.
لیو و همکاران\cite{li2019crossAtlas} از یک اطلس از چارت‌های (تقریباً) ایزومتریک استفاده می‌کنند ـ همانطور که در انتهای بخش~\ref{sec:e_surface_conv} بحث شد، این در واقع متناظر با یک کانولوشن $\GM$ است.


\emph{روش‌های حجمی} سطوح جایگذاری شده را با \lr{CNN}ها در فضای جایگذاری~$\R^3$ پردازش می‌کنند، به عنوان مثال با تفسیر رئوس یک مش سطح به عنوان یک \emph{ابر نقطه}~\cite{qi2017pointnet,qi2017pointnet++,thomas2019kpconv} یا با \emph{واکسل‌بندی} ورودی.
روش‌های مبتنی بر ابر نقطه در~\cite{guo2020deep} مرور شده‌اند.
مشدر و همکاران\cite{mescheder2019occupancyNets} و پنگ و همکاران\cite{peng2020occupancyCNNs} استدلال می‌کنند که یک پارامترسازی ضمنی سطح اقتصادی‌تر است و شبکه‌هایی را پیشنهاد می‌کنند که سطوح را به عنوان مرزهای تصمیم‌گیری مدل‌سازی می‌کنند.


\emph{رویکردهای طیفی} از قضیه کانولوشن الهام گرفته‌اند.
پایه فوریه روی یک منیفلد در اینجا با ویژه‌توابع عملگر لاپلاس-بلترامی داده می‌شود.
شبکه‌های عصبی طیفی، نقشه‌های ویژگی را با دستکاری طیف فوریه آنها با عملگرهای خطی یادگرفته شده پردازش می‌کنند.
از آنجا که پایه فوریه غیرمحلی است، بوسکاینی و همکاران\cite{boscaini2015learning} به جای آن از یک تبدیل فوریه پنجره‌ای استفاده می‌کنند؛ یک جایگزین، هارمونیک‌های منیفلد محلی‌شدهملزی و همکاران~\cite{melzi2018localized} است.
برونا و همکاران\cite{bruna2013spectral} مش‌های سطح را به عنوان گراف تفسیر می‌کنند.
بنابراین آنها از تبدیلات فوریه گراف استفاده می‌کنند، که بر اساس ویژه‌توابع لاپلاسین گراف هستند.


شارپ وهماران\cite{sharp2020diffusion} مدلی را پیشنهاد می‌کنند که بر اساس \emph{عملگرهای دیفرانسیل} است.
ویژگی‌های اسکالر از طریق نفوذ گرما با یک زمان نفوذ یادگرفتنی منتشر می‌شوند.
از آنجا که لاپلاسین (که در معادله گرما ظاهر می‌شود) همسانگرد است، نمی‌تواند به طور انتخابی به الگوها در دوران‌های خاص پاسخ دهد.
بنابراین نویسندگان علاوه بر این یک عملگر گرادیان اعمال می‌کنند و سپس حاصلضرب‌های داخلی از ویژگی‌های حاصل با مقادیر بردار مماس را می‌گیرند.
توجه داشته باشید که هر دو عملیات ناوردای پیمانه هستند.
این شبکه‌ها را می‌توان بر روی تمام ساختارهای داده‌ای که عملگرهای دیفرانسیل جزئی را می‌پذیرند، مانند ابرهای نقطه یا مش‌ها، پیاده‌سازی کرد.


تعداد قابل توجهی از شبکه‌ها بر روی \emph{ساختار منیفلد ریمانی} عمل نمی‌کنند، بلکه بر روی \emph{ساختار داده} که سطوح را به صورت عددی نمایش می‌دهد، عمل می‌کنند.
یک مثال، شبکه‌هایی هستند که گره‌ها و یال‌های یک مش سطح را به عنوان تشکیل دهنده یک گراف تفسیر می‌کنند و در نتیجه از \emph{شبکه‌های گراف} استفاده می‌کنند.
هموردایی ایزومتری شبکه‌های گراف در~\cite{khasanova2018isometric,horie2020isometric} بررسی شده است.
ورما و همکاران\cite{verma2018feastnet} یک شبکه گراف با فیلترهای پویا، یعنی فیلترهایی که در طول پاس مستقیم از ویژگی‌ها پیش‌بینی می‌شوند، پیشنهاد کردند.
مدل میلانو و همکاران\cite{milano2020primaldual} بر روی گراف‌های اولیه و دوگان مش‌ها عمل می‌کند و از مکانیزم‌های توجه استفاده می‌کند.

شبکه‌های مارپیچی (S\lr{piral nets}) ویژگی‌ها را روی مش‌ها از طریق عملگرهای مارپیچی محلی پردازش می‌کنند~\cite{lim2018simple,gong2019spiralnet++}.
این عملگرها ویژگی‌ها را با دنبال کردن یک مسیر مارپیچی به سمت بیرون از گره مرکزی شمارش می‌کنند.
یک پاسخ با اعمال یک \lr{LSTM} به دنباله حاصل از ویژگی‌ها یا یک \lr{MLP} به الحاق آنها محاسبه می‌شود.
انتخاب اولین همسایه و جهت مارپیچ متناظر با یک انتخاب از $\{e\}$-ساختار است.
هانوکا و همکاران\cite{hanocka2019meshcnn} وهرتز و همکاران \cite{hertz2020GeomTextureSynthesis} کانولوشن‌ها را به ترتیب روی وجوه و یال‌های مش تعریف می‌کنند.
هر دو مدل نسبت به دلخواه بودن ترتیب عناصر مش ناوردا ساخته شده‌اند، که می‌تواند به یک طراحی هموردای جایگشتی تعمیم یابد.

برای مرورهای عمیق‌تر از چنین روش‌هایی، خواننده را به برونشتاین و همکاران\cite{bronstein2017geometric} و گوو و همکاران\cite{guo2020deep} ارجاع می‌دهیم.