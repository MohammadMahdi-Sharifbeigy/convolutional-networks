%!TEX root=../GaugeCNNTheory.tex


\section{وجود و همواری تبدیلات میدان کرنل}
\label{apx:smoothness_kernel_field_trafo}


در تعریف~\ref{dfn:kernel_field_trafo}، ما \emph{تبدیلات میدان کرنل} $\TK$ را به عنوان تبدیلات انتگرالی هموار
\begin{align}
    \TK: \Gamma(\Ain)\to \Gamma(\Aout)
\end{align}
پیشنهاد کردیم که توسط یک میدان کرنل $\K$ (تعریف~\ref{dfn:kernel_field_general}) پارامتری شده و به صورت نقطه‌ای با
\begin{align}\label{eq:APX_kernel_field_trafo_def_ptwise}
    \big[ \TK (f)\big] (p)
    \ \ :=\ 
    \int\limits_{\TpM}\!\!
    \K(v) \,
    \Expspf(v)
    \ dv
    \ \ =\ 
    \int\limits_{\TpM}\!\!
    \K(v) \ 
    \PAinexppv \,
    f(\exp_p\!v)
    \ dv \,.
\end{align}
داده می‌شوند. تبدیلات میدان کرنل شامل کانولوشن‌های $\GM$ از تعریف~\ref{dfn:coord_free_conv} به عنوان موارد خاص برای میدان‌های کرنل کانولوشنی $\GM$ هستند.

در اینجا ما به طور خلاصه خوش‌تعریف بودن تبدیلات میدان کرنل را مورد بحث قرار می‌دهیم.
روشن است که انتگرال‌دهنده معادله~\eqref{eq:APX_kernel_field_trafo_def_ptwise} برای هر $p\in M$ و $v \in \TpM$ در $\Aoutp$ قرار دارد.
آنچه باقی می‌ماند نشان دادن \emph{وجود} انتگرال و \emph{همواری} میدان ویژگی حاصل است.
در ادامه ابتدا چند ملاحظه کلی در مورد چگونگی پرداختن به این پرسش‌ها ارائه می‌دهیم.
سپس قضیه~\ref{thm:existence_kernel_field_trafo_compact_kernels} را اثبات خواهیم کرد، یعنی خوش‌تعریف بودن تبدیلات میدان کرنل برای مورد خاص میدان‌هایی از کرنل‌ها که دارای تکیه‌گاه فشرده روی یک گوی با شعاع ثابت حول مبدأ هستند.







\paragraph{وجود:}
وجود و همواری تبدیلات میدان کرنل نیازمند یک انتخاب مناسب از میدان کرنل~$\K$ است.
مشابه مورد کانولوشن‌های متعارف روی $M=\R$، الزامات بر روی $\K$ برای اینکه تبدیل میدان کرنل وجود داشته باشد، به ویژگی‌های خاص میدان ویژگی ورودی $f\in\Gamma(\Ain)$ بستگی دارد.%
\footnote{
    به بحث در \url{https://en.wikipedia.org/wiki/Convolution\#Domain_of_definition} مراجعه کنید.
}
به طور کلی، $\K$ باید به اندازه کافی سریع افت کند تا انتگرال‌دهنده در معادله~\eqref{eq:APX_kernel_field_trafo_def_ptwise} انتگرال‌پذیر شود.


یک مورد خاص با اهمیت عملی بالا، کرنل‌های $\Kp: \TpM \to \Hom(\Ainp,\Aoutp)$ هستند که در هر $p\in M$ دارای \emph{تکیه‌گاه فشرده} هستند.
در این حالت، وجود انتگرال همیشه تضمین شده است.
برای دیدن این، توجه داشته باشید که میدان‌های ویژگی (ورودی) و میدان‌های کرنل به عنوان هموار تعریف شده‌اند.
همواری متریک علاوه بر این دلالت بر این دارد که
چگالی حجم ریمانی، نگاشت نمایی و انتقال موازی هموار هستند~\cite{gallier2019diffgeom1}.
در ترکیب، کل انتگرال‌دهنده در معادله~\eqref{eq:APX_kernel_field_trafo_def_ptwise} به عنوان یک تابع هموار و در نتیجه پیوسته از~$TM$ به~$\Aout$ دیده می‌شود.
اگر~$\Kp$ علاوه بر این دارای تکیه‌گاه فشرده باشد، انتگرال‌دهنده پیوسته و دارای تکیه‌گاه فشرده روی $\TpM$ می‌شود، که با تعمیمی از قضیه مقدار اکسترمم، دلالت بر این دارد که تصویر آن فشرده است (و در یک تریویالیزاسیون محلی $\R^{c}$ از $\Aoutp$ کران‌دار است)~\cite{rudin1976analysis}.
این وجود انتگرال را تضمین می‌کند~\cite{forster2012analysis3,spivak2019calculus}.


بسته به کاربرد، الزام بر روی تکیه‌گاه $\K$ ممکن است تسهیل شود.
به عنوان مثال، تصاویر روی $M=\R^d$ معمولاً خود دارای تکیه‌گاه فشرده هستند، به طوری که هیچ ویژگی اضافی از $\K$ به جز همواری آن لازم نیست.







\paragraph{همواری:}

ما به بحث در مورد همواری تبدیلات میدان کرنل می‌پردازیم، یعنی ویژگی آنها برای نگاشت میدان‌های ورودی هموار $\fin\in\Gamma(\Ain)$ به میدان‌های خروجی هموار $\fout := \TK(\fin) \in \Gamma(\Aout)$.
بنا به تعریف، یک نگاشت $\fout: M\to \Aout$ \emph{بین منیفلدها} $M$ و $\Aout$ هموار نامیده می‌شود اگر نمایش‌های مختصاتی آن هموار باشند.
در معادلات، $\fout$ هموار است اگر برای هر $p\in M$ چارت‌های هموار $(U,\phi)$ حول $p$ در $M$ و $(\widetilde{U}, \widetilde{\phi})$ حول $\fout(p)$ در $\Aout$ با $\fout(U) \subseteq \widetilde{U}$ وجود داشته باشند به طوری که $\widetilde{\phi} \circ \fout \circ \phi^{-1}: \phi(U) \to \widetilde{\phi}(\widetilde{U})$ به عنوان یک نگاشت بین (زیرمجموعه‌هایی از) فضاهای اقلیدسی هموار باشد.
با توجه به $(U,\phi)$، یک انتخاب مناسب برای $(\widetilde{U}, \widetilde{\phi})$
$ \widetilde{\phi} := \big(\phi \times \id\big) \circ \PsiAout\!:\ 
  \piAout^{-1}(U) \mapsto \phi(U) \times \R^c\, \subseteq\, \R^d \times \R^c \,, $
خواهد بود، با این حال، بحث زیر مستقل از این انتخاب است.%
\footnote{
    توجه داشته باشید که تضمین می‌شود $\piAout^{-1}(U)$ تریویالیزه‌پذیر باشد با توجه به اینکه $(\phi,U)$ یک چارت از $M$ است.
    این روشن است زیرا پایه‌های مختصاتی $\big[ \frac{\partial}{\partial \phi_\mu} \big]_{\mu=1}^d$ از $(\phi,U)$ یک تریویالیزاسیون از $\piTM^{-1}(U)$ را به دست می‌دهند (به پیوست~\ref{apx:coordinate_bases} مراجعه کنید) و از آنجا که تریویالیزاسیون‌های محلی $FM$ و $\A$ در بخش~\ref{sec:bundle_trivializations} از آنهای $TM$ القا شده‌اند.
}
یک نگاشت بین (زیرمجموعه‌هایی از) فضاهای اقلیدسی هموار است اگر در هر مؤلفه از تصویر خود، در اینجا در هر یک از $d+c$ بعد از $\phi(U)\times\R^c$ هموار باشد.
بنابراین ما به همواری نگاشت‌های
\begin{align}
    F_i: \phi(U) \to \R,\ \ x \mapsto \Big[\, \widetilde{\phi} \circ \fout \circ \phi^{-1} \Big]_i
\end{align}
برای هر $i = 1,\dots,d+c$ علاقه‌مندیم.
با بازنویسی $\fout$ و بیان انتگرال روی $\TpM$ با یک انتگرال روی $\R^d$ همانطور که در پیوست~\ref{apx:correspondences_bundle_trivializations} بحث شد، دیده می‌شود که $F_i$ به شکل
\begin{align}\label{eq:integral_smoothness_component}
    F_i(x) = \int_{\R^d} I_i(\mathscr{v},x)\,\ d\mathscr{v} \,.
\end{align}
هستند. عبارات مختصاتی انتگرال‌دهنده‌های $I_i$ در اینجا برای هر $i = 1,\dots,d+c$ با
\begin{align}\label{eq:integrand_smoothness_full}
    I_i\!: \R^d \mkern-2mu\times\mkern-2mu \phi(U) &\to \R, \\
    (\mathscr{v},x) &\mapsto
    \bigg[ \widetilde{\phi} \circ
    \K\pig(\mkern-1mu \psi_{\protect\scalebox{.65}{$T\!M\mkern-1mu,\mkern2mu$}\protect\scalebox{.75}{$\phi^{\mkern-1mu\shortminus1}\mkern-1mu(x)$}}^{-1} \mkern-2mu(\mathscr{v}) \mkern-1mu\pig) \circ
    \mathcal{P}_{\mkern-4mu\protect\scalebox{.8}{$\!\Ain$},\protect\scalebox{.83}{$\,\phi^{\mkern-1mu\shortminus1}(x)\!\leftarrow\!\exp \circ\mkern1mu \psi_{\protect\scalebox{.7}{$T\!M\mkern-1mu,\mkern1mu$}\protect\scalebox{.9}{$\phi^{\mkern-1mu\shortminus1}(x)$}}\!(\mathscr{v}) $}} \mkern-1mu\circ
    \fin \circ \exp \circ \,
    \psi_{\protect\scalebox{.65}{$T\!M\mkern-1mu,$}\protect\scalebox{.7}{$\phi^{\mkern-1mu\shortminus1}(x)$}}^{-1} \mkern-2mu(\mathscr{v})
    \bigg]_i
    \nonumber \,,
\end{align}
داده می‌شوند، که در آن ما برای راحتی و بدون از دست دادن کلیت، فرض کردیم که
$\psi_{\protect\scalebox{.65}{$T\!M\mkern-1mu,\mkern2mu$}\protect\scalebox{.75}{$\phi^{\mkern-1mu\shortminus1}\mkern-1mu(x)$}}$
یک پیمانه \emph{ایزومتریک} از $T_{\phi^{\mkern-1mu\shortminus1}\mkern-1mu(x)}M$ است، به طوری که ضریب مقیاس‌بندی حجم $\sqrt{|\! \det\eta_p|} = 1$ حذف می‌شود.
توجه داشته باشید که انتگرال‌دهنده‌های $I_i$ از نگاشت‌های هموار تشکیل شده‌اند و بنابراین خود نیز هموار هستند.


از بحث قبلی روشن است که همواری $\fout$ برقرار است اگر همه $F_i$ هموار باشند، یعنی بی‌نهایت بار مشتق‌پذیر جزئی باشند.
برای اثبات همواری $F_i$ کافی است نشان دهیم که مشتق‌گیری‌های جزئی و انتگرال‌گیری در معادله~\eqref{eq:integral_smoothness_component} جابجا می‌شوند -- که این همیشه صادق نیست.
اگر آنها جابجا شوند، مشتقات جزئی از مراتب دلخواه $(n_1,\dots,n_d) \in \N^d$ با
\begin{align}\label{eq:APX_smoothness_partial_diff_swapping}
    \Big[ \partial_{x_1}^{n_1} \dots \partial_{x_d}^{n_d}\: F_i \Big](x)
    \ =\ \int_{\R^d} \Big[ \partial_{x_1}^{n_1} \dots \partial_{x_d}^{n_d}\: I_i\Big] (\mathscr{v},x)\,\ d\mathscr{v}
\end{align}
داده می‌شوند، که در آن
۱) مشتقات جزئی $\pig[ \partial_{x_1}^{n_1} \dots \partial_{x_d}^{n_d}\: I_i\pig]$ از انتگرال‌دهنده وجود دارند (به دلیل همواری $I_i$ وجود آنها تضمین شده است) و
۲) انتگرال آنها وجود دارد.
اینکه آیا مشتق‌گیری‌ها با انتگرال جابجا می‌شوند یا نه را می‌توان با استفاده از لم زیر
از~\cite{forster2012analysis3}،%
\footnote{
    نسخه‌های مشابهی از این لم به زبان انگلیسی را می‌توان در \cite{klenke2006probability} یا در
    \url{https://en.wikipedia.org/wiki/Leibniz_integral_rule\#Measure_theory_statement} یافت.
    برخلاف آن نسخه‌ها، نسخه~\cite{forster2012analysis3} اجازه می‌دهد که $T$ هر بازه غیرتبهگن، از جمله بازه‌های بسته، باشد، که ما را از چند مرحله اضافی در ادامه بی‌نیاز می‌کند.
}
که نتیجه‌ای از قضیه همگرایی تحت سلطه است، بررسی کرد.
\begin{thm}[لم مشتق‌گیری~\cite{forster2012analysis3}]
\label{thm:differentiation_lemma}
    فرض کنید $\mathscr{V}$ یک فضای اندازه، $T\subset\R$ یک بازه غیرتبهگن و $I: \mathscr{V} \times T \to \R$ یک نگاشت با ویژگی‌های زیر باشد:
    \begin{itemize}
        \item[$(i)$] برای هر $t\in T$ ثابت، نگاشت $\mathscr{v} \mapsto I(\mathscr{v},t)$ روی $\mathscr{V}$ لبگ-انتگرال‌پذیر است.
        \item[$(ii)$] برای هر $\mathscr{v} \in \mathscr{V}$ ثابت، نگاشت $t\mapsto I(\mathscr{v},t)$ در $T$ مشتق‌پذیر است.
        \item[$(iii)$] یک تابع لبگ-انتگرال‌پذیر $\mathscr{B}: \mathscr{V} \to \R$ وجود دارد به طوری که
                $\big| \frac{\partial}{\partial t} I(\mathscr{v},t) \big| \leq \mathscr{B}(\mathscr{v})$
                برای هر $(\mathscr{v},t) \in \mathscr{V} \times T$ برقرار است.
    \end{itemize}
    آنگاه تابع $F: T \to \R,\ t \mapsto \int_\mathscr{V} I(\mathscr{v},t)\, d\mathscr{v}$ مشتق‌پذیر است
    با مشتقی برابر با
    \begin{align*}
        \frac{\partial}{\partial t} F(t)\ =\ \int_\mathscr{V} \frac{\partial}{\partial t} I(\mathscr{v},t)\; d\mathscr{v} \,.
    \end{align*}
\end{thm}
کاربردی بودن این لم (به طور مکرر برای هر مشتق‌گیری جزئی منفرد) به ویژگی‌های انتگرال‌دهنده بستگی دارد، که به نوبه خود به ویژگی‌های خاص میدان کرنل $\K$ و میدان ویژگی ورودی $\fin$ بستگی دارد.
برای مورد یک میدان کرنل که دارای تکیه‌گاه فشرده روی گوی‌هایی با شعاع ثابت حول مبدأ هر فضای مماس است، این لم اعمال می‌شود.
بر این اساس، ما در باقیمانده این پیوست اثباتی برای قضیه~\ref{thm:existence_kernel_field_trafo_compact_kernels} ارائه می‌دهیم.










\toclesslab\subsection{اثبات قضیه~\ref{thm:existence_kernel_field_trafo_compact_kernels} (کفایت تکیه‌گاه فشرده کرنل روی گوی‌های مماس)}{apx:proof_sufficiency_ball_kernel_support}

فرض کنید
$B_{\TpM}^{\mkern1.5mu\textup{\lr{closed}}}(0,R) := \big\{ v\in\TpM \,\big|\, \lVert v\rVert \leq R \big\}$
گوی بسته با شعاع $R>0$ حول مبدأ $\TpM$ و
$B_{\R^d}^{\mkern1.5mu\textup{\lr{closed}}}(0,R) := \big\{ v\in\R^d \,\big|\, \lVert v\rVert \leq R \big\}$
گوی متناظر حول مبدأ $\R^d$ باشد.
توجه داشته باشید که هر پیمانه ایزومتریک $\psiTMp\big(B_{\TpM}^{\mkern1.5mu\textup{\lr{closed}}}(0,R)\big) = B_{\R^d}^{\mkern1.5mu\textup{\lr{closed}}}(0,R)$ را برآورده می‌کند.
فرض کنید $\widetilde{\K}$ یک میدان کرنل باشد که تکیه‌گاه آن در داخل گوی‌هایی با شعاع یکسان $R$ در هر فضای مماس قرار دارد، یعنی
\begin{align}
    \supp\!\big(\widetilde{\K}_p\big) \subseteq B_{\TpM}^{\mkern1.5mu\textup{\lr{closed}}}(0,R) \quad \forall p\in M
\end{align}
و در نتیجه، برای هر پیمانه ایزومتریک $\psiTMp$:
\begin{align}
    \supp\! \pig(\widetilde{\K}_p \circ \big(\psiTMp\big)^{-1} \pig) \subseteq B_{\R^d}^{\mkern1.5mu\textup{\lr{closed}}}(0,R) \quad \forall p\in M
\end{align}
را برآورده می‌کند. طبق قضیه~\ref{thm:existence_kernel_field_trafo_compact_kernels} این ویژگی برای تضمین اینکه تبدیل میدان کرنل متناظر $\mathscr{T}_{\mathcal{K}_{\mkern-1muR}}$ خوش‌تعریف است، کافی است.
اثبات این گزاره در ادامه ارائه می‌شود.

\begin{proof}
    همانطور که قبلاً در ابتدای این پیوست بیان شد، وجود انتگرال با توجه به اینکه تکیه‌گاه‌های کرنل فشرده هستند، تضمین شده است:
    فشردگی کرنل‌ها به انتگرال‌دهنده‌های تبدیل میدان کرنل منتقل می‌شود.
    همواری آنها علاوه بر این، پیوستگی آنها را نتیجه می‌دهد و انتگرال‌های توابع پیوسته با تکیه‌گاه فشرده همیشه وجود دارند.

    برای اثبات همواری میدان ویژگی خروجی حاصل $\fout$ ما با بحث قبلی در این بخش ادامه می‌دهیم.
    هدف ما اعمال لم مشتق‌گیری~\ref{thm:differentiation_lemma} برای جابجایی مشتقات جزئی $\frac{\partial}{\partial x_\mu}$ برای هر $\mu=1,\dots,d$ در معادله~\eqref{eq:APX_smoothness_partial_diff_swapping} در هر $x_0 \in \phi(U)$ با انتگرال‌گیری روی $\R^d$ است.
    برای این منظور، ما توابع کمکی
    \begin{align}
        I_{i,x_0,\mu}: \R^d \times [-\varepsilon,\varepsilon] \to \R,\ \ (\mathscr{v},t) \mapsto I(\mathscr{v}, x_0 + t\epsilon_\mu)
    \end{align}
    و
    \begin{align}
        F_{i,x_0,\mu}: [-\varepsilon,\varepsilon] \to \R,\ \ t \mapsto F(x_0 + t\epsilon_\mu) = \int_{\R^d} I_{i,x_0,\mu}(\mathscr{v},t)\ d\mathscr{v} \,,
    \end{align}
    را معرفی می‌کنیم، که در آن $\epsilon_\mu \in \R^d$ بردار واحد در جهت $\mu$ است و
    $\varepsilon > 0$ به گونه‌ای انتخاب شده است که $\big\{ x_0 + t\epsilon_\mu \,\big|\, t\in [-\varepsilon,\varepsilon] \big\} \subset \phi(U)$ باشد، که این کار همیشه ممکن است زیرا $\phi(U)$ باز است.
    سپس $I_{i,x_0,\mu}$ با شناسایی‌های $\mathscr{V}=\R^d$ و $T=[-\varepsilon,\varepsilon]$ به شکلی است که توسط لم~\ref{thm:differentiation_lemma} لازم است.
    این لم ویژگی $(i)$ را با فرض وجود تبدیل میدان کرنل همانطور که قبلاً بحث شد، برآورده می‌کند.
    ویژگی $(ii)$ به دلیل همواری کل انتگرال‌دهنده در معادله~\eqref{eq:integrand_smoothness_full} برقرار است.
    برای ویژگی $(iii)$ مشاهده کنید که هم $I_{i,x_0,\mu}$ و هم مشتق آن هموار هستند به طوری که قدر مطلق $\big| \frac{\partial}{\partial t} I_{i,x_0,\mu} \big|$ پیوسته است.
    از آنجا که علاوه بر این دارای تکیه‌گاه فشرده روی $B_{\R^d}^{\mkern1.5mu\textup{\lr{closed}}}(0,R) \times [-\varepsilon,\varepsilon]$ است، با (تعمیمی از) قضیه مقدار اکسترمم با یک عدد $b\geq0$ کران‌دار است.
    بنابراین ما $\mathscr{B}(\mathscr{v}) = b\cdot \mathbb{I}_{B_{\R^d}^{\mkern1.5mu\textup{\lr{closed}}}(0,R) \times [-\varepsilon,\varepsilon]}$ را قرار می‌دهیم که در آن $\mathbb{I}$ تابع مشخصه است.
    این انتخاب
    $\big| \frac{\partial}{\partial t} I(\mathscr{v},t) \big| \leq \mathscr{B}(\mathscr{v})$
    را برای هر $(\mathscr{v},t) \in \mathscr{V} \times T$
    برآورده می‌کند و انتگرال‌پذیر است به طوری که ویژگی $(iii)$ نیز برآورده می‌شود.
    بنابراین ما می‌توانیم ترتیب مشتق‌گیری و انتگرال‌گیری را برای انتخاب‌های دلخواه از $x_0$ و $\mu$ جابجا کنیم، که از آن برای کشیدن مشتقات جزئی دلخواه به داخل انتگرال استفاده می‌کنیم:
    \begin{align}
        \bigg[ \frac{\partial}{\partial x_\mu} F_i \bigg](x_0)
        \,=\, \bigg[ \frac{\partial}{\partial t} F_{i,x_0,\mu} \bigg](0)
        \,=\, \int_{\R^d} \frac{\partial}{\partial t} I_{i,x_0,\mu}(\mathscr{v},t) \Big|_{t=0}\ d\mathscr{v}
        \,=\, \int_{\R^d} \frac{\partial}{\partial x_\mu} I_i(\mathscr{v},x) \Big|_{x=x_0}\ d\mathscr{v}
    \end{align}

    به دلیل همواری و تکیه‌گاه فشرده انتگرال‌دهنده $I_i$ مشتقات جزئی آن $\frac{\partial}{\partial x_\mu} I$ نیز هموار و دارای تکیه‌گاه فشرده هستند.
    بنابراین آنها نیز ویژگی‌های $(i)$، $(ii)$ و $(iii)$ را (با یک کران $b$ بالقوه تطبیق یافته) برآورده می‌کنند.
    بنابراین امکان تکرار مشتق‌گیری جزئی از $F_i$ به طور نامحدود وجود دارد، که همواری آن را اثبات می‌کند.
    از آنجا که استخراج‌ها مستقل از انتخاب‌های خاص برای نقطه $p\in M$ چارت‌های $(U,\phi)$ و $(\widetilde{U},\widetilde{\phi})$ نقاط $x_0\in\phi(U)$ و اندیس‌های $i$ و $\mu$ بودند، این نتیجه همواری کل میدان ویژگی خروجی $\fout = \TK(\fin)$ را اثبات می‌کند.

\NoEndMark
$~\hfill\Box$
\end{proof}