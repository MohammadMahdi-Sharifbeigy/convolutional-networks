%!TEX root=../GaugeCNNTheory.tex


\section{هم‌متغیری ایزومتری}
\label{sec:isometry_intro}


یک ویژگی اصلی عمل کانولوشن و تعمیم‌های مختلف آن، هم‌متغیری آنها نسبت به تقارن‌های منیفلد زیربنایی است.
به عنوان مثال، کانولوشن مرسوم در فضاهای اقلیدسی نسبت به انتقال هم‌متغیر است در حالی که کانولوشن‌های کروی نسبت به دوران هم‌متغیر هستند.
به طور کلی‌تر، هر گروه فشرده موضعی و فضاهای همگن آنها، کانولوشن‌های گروهی را می‌پذیرند
~\cite{gurarie1992symmetries,kowalski2010introduction,chirikjian2001engineering,gallier2019harmonicRepr}،
که اخیراً توسط جامعه یادگیری عمیق برای تعمیم شبکه‌های کانولوشنی به چنین فضاهایی مورد توجه قرار گرفته‌اند~\cite{Cohen2016-GCNN,Kondor2018-GENERAL,Cohen2019-generaltheory,bekkers2020bspline}.
با این حال، از آنجایی که این رویکردها اساساً به تقارن‌های سراسری و \emph{تعدی‌پذیر} فضای همگن متکی هستند، بلافاصله برای منیفلدهای ریمانی عمومی قابل اعمال نیستند.

از سوی دیگر، کانولوشن‌های $\GM$ تمرکز را از \emph{تقارن‌های سراسری خود فضا} به \emph{تقارن‌های محلی در مختصاتی‌سازی فضا} منتقل می‌کنند.
همانطور که مشخص می‌شود، هم‌متغیری پیمانه محلی کانولوشن‌های $\GM$، همراه با اشتراک‌گذاری وزن کانولوشنی، هم‌متغیری آنها را تحت عمل تقارن‌های سراسری القا می‌کند.
به بیان دقیق‌تر، کانولوشن‌های $\GM$ تحت عمل \emph{ایزومتری‌های حافظ \lr{G}-ساختار} (تعریف~\ref{dfn:IsomGM}) هم‌متغیر هستند، که زیرگروهی $\IsomGM \leq \IsomM$ از گروه کامل ایزومتری را تشکیل می‌دهند.
الزام به اینکه تقارن یک ایزومتری باشد (یعنی متریک را حفظ کند) در اینجا از استفاده از نگاشت‌های نمایی ناشی می‌شود، که به اتصال لوی-چیویتا و در نتیجه به متریک ریمانی متکی هستند.
الزام اضافی بر این ایزومتری‌ها برای حفظ \lr{G}-ساختار، نتیجه تعریف کلاف‌های بردار ویژگی به عنوان کلاف‌های \lr{G}-الحاقی است، که عناصر آنها تنها نسبت به آن قاب‌های مرجعی که در $\GM$ قرار دارند، معنای خوش‌تعریفی دارند.
توجه داشته باشید که مورد دوم واقعاً یک محدودیت نیست، زیرا همیشه می‌توان گروه‌های ساختار $G\geq\OO{d}$ را انتخاب کرد، که برای آنها \emph{هر} ایزومتری به \lr{G}-ساختار متناظر احترام می‌گذارد.
برعکس، این طراحی امکان کنترل دقیق سطح هم‌متغیری ایزومتری را فراهم می‌کند.
به عنوان مثال، کانولوشن مرسوم در فضاهای برداری اقلیدسی به $\{e\}$-ساختار کانونی $\R^d$ متکی است که در شکل~\ref{fig:frame_field_automorphism_1} به تصویر کشیده شده است، و بنابراین تنها نسبت به انتقال هم‌متغیر است.
یک $\SO{d}$-ساختار روی $\R^d$، که در شکل~\ref{fig:SO2_structure_SE2} به تصویر کشیده شده است، علاوه بر این توسط دوران‌ها نیز حفظ می‌شود، و بنابراین متناظر با کانولوشن‌های $\SE{d}$-هم‌متغیر است.
هم‌متغیری تحت گروه کامل ایزومتری $\E{d}$ از $\R^d$ هنگام انتخاب یک $\OO{d}$-ساختار روی $\R^d$ ایجاب می‌شود.


\etocsettocdepth{3}
\etocsettocstyle{}{} % from now on only local tocs
\localtableofcontents


هدف این بخش استخراج قضایایی است که به طور رسمی هم‌متغیری ایزومتری کانولوشن‌های $\GM$ و تبدیلات میدان کرنل را مشخص می‌کنند.
بخش~\ref{sec:isom_background} با معرفی گروه‌های ایزومتری منیفلدهای ریمانی و بحث در مورد طیفی از روابط و ساختارهای شناخته‌شده‌ای که آنها القا می‌کنند، پایه‌های این تحقیق را بنا می‌نهد.
به طور خاص، بخش~\ref{sec:isometry_groups} ایزومتری‌ها و گروه‌های ایزومتری را معرفی می‌کند در حالی که بخش~\ref{sec:isom_action_bundles} عمل القایی آنها («پیش‌ران‌ها») را بر روی کلاف‌های الحاقی در یک محیط مستقل از مختصات تعریف می‌کند.
در بخش~\ref{sec:isom_coordinatization} ما این اعمال روی کلاف‌ها را نسبت به بدیهی‌سازی‌های محلی بیان می‌کنیم و تفسیر غیرفعال آنها را به عنوان تبدیلات پیمانه القاشده از ایزومتری، که در شکل‌های~\ref{fig:intro_gauge_isom_induction} (راست) و~\ref{fig:pushforward_vector_components} به تصویر کشیده شده‌اند، مورد بحث قرار می‌دهیم.
بخش~\ref{sec:isom_expmap_transport} به طور خلاصه بیان می‌کند که مقادیر درگیر در تبدیلات میدان کرنل تحت عمل ایزومتری‌ها چگونه رفتار می‌کنند.


بر اساس این ویژگی‌ها، ما هم‌متغیری ایزومتری تبدیلات میدان کرنل و کانولوشن‌های $\GM$ را در بخش~\ref{sec:isometry_equivariance} مطالعه می‌کنیم.
پس از تعریف رسمی عبارت «هم‌متغیری ایزومتری»، بخش~\ref{sec:isometry_constraint} یک نتیجه مرکزی را اثبات می‌کند که تأکید می‌کند تقاضا برای \emph{هم‌متغیری ایزومتری مستلزم نامتغیر بودن میدان کرنل تحت ایزومتری‌ها است}؛ به شکل~\ref{fig:isom_invariant_kernel_field_multiple_orbits} مراجعه کنید.
بخش~\ref{sec:isom_equiv_GM_conv} کانولوشن‌های خاص‌تر $\GM$ را در نظر می‌گیرد و اثبات می‌کند که آنها بنا به طراحی، تحت هر ایزومتری که \lr{G}-ساختار را حفظ می‌کند، هم‌متغیر هستند.
این نتیجه به طور خاص ایجاب می‌کند که کانولوشن‌های $\OM$ نسبت به هر ایزومتری هم‌متغیر هستند.


قید نامتغیر بودن بر روی میدان‌های کرنل ایجاب می‌کند که آنها وزن‌ها را بر روی مدارهای گروه ایزومتری به اشتراک بگذارند.
این نشان می‌دهد که میدان‌های کرنل نامتغیر را می‌توان به طور معادل با کرنل‌های نماینده روی نمایندگان مدار توصیف کرد، که ما آن را در بخش~\ref{sec:quotient_kernel_fields} رسمی می‌کنیم.
بخش~\ref{sec:isom_quotients} فضاهای خارج‌قسمتی القاشده از ایزومتری و نمایندگان آنها را مورد بحث قرار می‌دهد.
در بخش~\ref{sec:quotient_kernels_stabilizers} ما از این تعاریف ریاضی برای اثبات این موضوع استفاده می‌کنیم که فضای میدان‌های کرنل نامتغیر نسبت به ایزومتری در واقع با میدان‌های کرنل روی نمایندگان خارج‌قسمتی ایزومورف است.
این به طور خاص ایجاب می‌کند که تبدیلات میدان کرنل هم‌متغیر نسبت به ایزومتری در فضاهای همگن لزوماً کانولوشن هستند، که حلقه را به کارهای پیشین متصل می‌کند.