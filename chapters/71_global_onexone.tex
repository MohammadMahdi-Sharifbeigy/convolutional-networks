%!TEX root=../GaugeCNNTheory.tex


\subsection[\texorpdfstring{کانولوشن‌های \lr{GM} با اندازه ${1\kern-2.7pt\times\kern-2.9pt1}$}{1x1} ]%
         {کانولوشن‌های \lr{GM} با اندازه \texorpdfstring{${1\kern-2.7pt\times\kern-2.9pt1}$}{1x1}}
\label{sec:onexone}


\onexoneGMs میدان‌های ویژگی ورودی $\fin\in\Gamma(\Ain)$ را به میدان‌های ویژگی خروجی $\fout\in\Gamma(\Aout)$ با نگاشت خطی هر بردار ویژگی ورودی منفرد $\fin(p)\in\Ainp\cong\R^{\cin}$ به یک بردار ویژگی خروجی $\fout(p)\in\Aoutp\cong\R^{\cout}$ در همان مکان $p\in M$ نگاشت می‌کنند.
ویژگی کانولوشنی با \emph{اشتراک‌گذاری} نگاشت خطی از $\Ainp$ به $\Aoutp$ بین مکان‌های فضایی مختلف پیاده‌سازی می‌شود.
با این حال، در حالی که فضاهای ویژگی $\Ainp$ و $\Ainq$ و همچنین $\Aoutp$ و $\Aoutq$ برای $p,q\in M$ مختلف با یکدیگر ایزومورف هستند، اگر گروه ساختار $G$ در نظر گرفته شده غیربدیهی باشد، هیچ ایزومورفیسم کانونی بین آنها وجود ندارد.
بنابراین، مشخص نیست که چگونه نگاشت خطی می‌تواند بین مکان‌های مختلف به اشتراک گذاشته شود.
همانطور که قبلاً در مقدمه این بخش اشاره شد، این مسئله با در نظر گرفتن کرنل‌های \lr{G}-هم‌متغیر که نسبت به انتخاب خاص ایزومورفیسم یا پیمانه بی‌تفاوت هستند، حل می‌شود.
اختیاری بودن بدیهی‌سازی که از \lr{G}-اطلس انتخاب می‌شود، استقلال از مختصات $\GM$ را در \onexoneGMs منعکس می‌کند.


از نظر ریاضی، \onexoneGMs را می‌توان یا به عنوان \lr{M}-مورفیسم‌های کلاف برداری خاص یا از طریق مقاطع متناظر کلاف‌های هومومورفیسم (الحاقی) $\Hom(\Ain,\Aout)$ فرمول‌بندی کرد.
از آنجایی که بعداً به هر دو مفهوم نیاز داریم، هر دو دیدگاه را در بخش‌های بعدی~\ref{sec:onexone_M_morphism} و~\ref{sec:onexone_hom_section} معرفی خواهیم کرد.


\subsubsection[\texorpdfstring{کانولوشن‌های \lr{GM} با اندازه ${1\kern-2.7pt\times\kern-2.9pt1}$}{1x1} به عنوان \lr{M}-مورفیسم‌های کلاف برداری]%
          {کانولوشن‌های \lr{GM} با اندازه \texorpdfstring{${1\kern-2.7pt\times\kern-2.9pt1}$}{1x1} به عنوان \lr{M}-مورفیسم‌های کلاف برداری}
\label{sec:onexone_M_morphism}


\onexoneGMs را می‌توان بر حسب \emph{\lr{M}-مورفیسم‌های کلاف برداری} هموار خاصی که وزن‌ها را در موقعیت‌های مکانی به اشتراک می‌گذارند، رسمی‌سازی کرد.
با نادیده گرفتن الزام به اشتراک‌گذاری وزن‌ها در حال حاضر، چنین \lr{M}-مورفیسم کلاف برداری $\mathcal{C}$ یک نگاشت کلاف هموار است که نمودار جابجایی زیر را برآورده می‌کند:
\begin{equation}\label{eq:bundle_morphism_onexone}
    \begin{tikzcd}[row sep=3.5em, column sep=2.5em]
        % ROW 1
        \Ain
            \arrow[rd, "\piAin"']
            \arrow[rr, "\mathcal{C}"]
        & &
        \Aout
            \arrow[ld, "\piAout"]
        \\
        % ROW 2
        & M
    \end{tikzcd}
\end{equation}
جابجایی‌پذیری $\piAin = \piAout \!\circ\, \mathcal{C}$ تضمین می‌کند که هر تار $\Ainp$ به تار $\Aoutp$ روی همان نقطه $p\in M$ نگاشت می‌شود (که باعث پیدایش «\lr{M}» در عبارت \lr{M}-مورفیسم می‌شود).
به عنوان یک مورفیسم کلاف برداری، محدودیت $\mathcal{C}\!\!\;|_p: \Ainp\to\Aoutp$ به یک تار منفرد، علاوه بر این، به صورت خطی تعریف می‌شود.
نسبت به یک بدیهی‌سازی محلی $\PsiAin^A$ از $\Ain$ و $\PsiAout^A$ از $\Aout$، نگاشت کلاف در هر نقطه $p\in U^A$ توسط یک ماتریس نمایش داده می‌شود:
\begin{align}\label{eq:bundle_morphism_onexone_triv_local}
    \mathcal{C}^A\!\!\;|_p\ :=\  \psiAoutp^A \circ \mathcal{C}\!\;|_p \circ \big(\psiAinp^A\big)^{-1}\ \in\, \R^{\cout\times\cin} \,.
\end{align}
ارتباط آن با یک مختصاتی‌سازی دوم $\mathcal{C}^B$ در $p\in U^A\cap U^B$ به صورت زیر داده می‌شود:
\begin{align}\label{eq:onexone_gaugetrafo}
    \mathcal{C}^B\!\!\;|_p \ =\ \rhoout\big(g_p^{BA}\big)\; \mathcal{C}^A\!\!\;|_p\; \rhoin\big(g_p^{BA}\big)^{-1} \,,
\end{align}
که از نمودار جابجایی زیر مشهود است:
\begin{equation}
\begin{tikzcd}[column sep=65pt, row sep=30, font=\normalsize]
    \R^{\cin}
        \arrow[dd, "\rhoin\big(g_p^{BA}\big)\ "']
        \arrow[rrr, "\mathcal{C}^A\!\!\;|_p"]
    & &[-3ex] &
    \R^{\cout}
        \arrow[dd, "\ \rhoout\big(g_p^{BA}\big)"]
    \\
    &
    \Ainp
        \arrow[r, "\mathcal{C}\!\!\;|_p"]
        \arrow[ul, "\psiAinp^A"]
        \arrow[dl, "\psiAinp^B"']
    &
    \Aoutp
        \arrow[ur, "\psiAoutp^A"']
        \arrow[dr, "\psiAoutp^B"]
    \\
    \R^{\cin}
        \arrow[rrr, "\mathcal{C}^B\!\!\;|_p"']
    & & &
    \R^{\cout}
\end{tikzcd}
\end{equation}


نگاشت کلاف $\mathcal{C}$ بر روی میدان‌های ویژگی ورودی $\fin \in \Gamma(\Ain)$ عمل می‌کند تا میدان‌های ویژگی خروجی تولید کند:
\begin{align}
    \fout = \mathcal{C} \circ \fin
    \quad \in\ \ \Gamma(\Aout) \,.
\end{align}
بر حسب یک نمودار جابجایی، این نگاشت به صورت زیر به تصویر کشیده می‌شود:
\begin{equation}\label{eq:bundle_morphism_onexone_section}
    \begin{tikzcd}[row sep=3.5em, column sep=2.5em]
        % ROW 1
        \Ain
            \arrow[rr, "\mathcal{C}"]
        & &
        \Aout
        \\
        % ROW 2
        & M \arrow[ul, "\fin"]
            \arrow[ur, "\fout"']
    \end{tikzcd} ,
\end{equation}


برای اینکه یک \lr{M}-مورفیسم کلاف برداری $\mathcal{C}_{K_{\!1\!\times\!1}}$ نماینده یک \onexoneGM باشد، باید بر حسب یک الگوی کرنل \onexoneGM\ یعنی $K_{\!1\!\times\!1} \in \R^{\cout\times\cin}$ پارامتری شود که با مختصاتی‌سازی‌ها در تمام موقعیت‌های مکانی به اشتراک گذاشته شده است.
همانطور که قبلاً بحث شد، برای تضمین \emph{استقلال از مختصات $\GM$}، هیچ پیمانه خاصی نباید ترجیح داده شود.
بنابراین لازم است که \emph{وزن‌ها با تمام بدیهی‌سازی‌های $X \in \mathfrak{X}$ از \lr{G}-اطلس $\mathscr{A}^G$ به طور همزمان به اشتراک گذاشته شوند}، یعنی لازم است:
\begin{align}\label{eq:weight_sharing_onexone}
    \mathcal{C}_{K_{\!1\!\times\!1}}^X\!\big|_p\ =\ K_{\!1\!\times\!1}
    \qquad \textup{برای \emph{هر} پیمانه}\,\ X \in \mathfrak{X}\,\ \textup{با}\,\ p\in U^X \,.
\end{align}
از رفتار تبدیل بین مختصاتی‌سازی‌های مختلف در معادله~\eqref{eq:onexone_gaugetrafo} نتیجه می‌شود که الگوی کرنل باید قید خطی زیر را برآورده کند:
\begin{align}\label{eq:onexone_kernel_constraint}
    \rhoout(g)\, K_{\!1\!\times\!1}\, \rhoin(g)^{-1}  =\, K_{\!1\!\times\!1} \qquad\forall g\in G,
\end{align}
یعنی، باید یک درهم‌تننده (یک نگاشت خطی هم‌متغیر) باشد.
فضای برداری
\begin{align}
    \Hom_G(\rhoin,\rhoout)\ :=\ 
    \pig\{ K_{\!1\!\times\!1} \in \R^{\cout\times\cin}\ \pig|\ 
    K_{\!1\!\times\!1} \rhoin(g) = \rhoout(g) K_{\!1\!\times\!1}\ \ \forall g\in G \pig\}
\end{align}
نگاشت‌های درهم‌تننده، فضای کرنل‌های \onexone\ مستقل از مختصات $\GM$ را به طور کامل مشخص می‌کند.
همانطور که قبلاً در بخش~\ref{sec:gauge_1x1} ذکر شد، \emph{لم شور}~\cite{gallier2019harmonicRepr} ایجاب می‌کند که الزام به اینکه $K_{\!1\!\times\!1}$ یک درهم‌تننده باشد، از نگاشت بین میدان‌هایی که تحت نمایش‌های کاهش‌ناپذیر غیرایزومورف تبدیل می‌شوند، از طریق \onexoneGMs جلوگیری می‌کند.
کانولوشن‌های عمومی‌تر $\GM$ با کرنل‌های با گستره فضایی، که در بخش~\ref{sec:global_conv} تعریف شده‌اند، این مسئله را حل خواهند کرد.


با این مقدمات، آماده‌ایم تا تعریف دقیقی از \onexoneGMs ارائه دهیم:
\begin{dfn}[\onexoneGM]
\label{dfn:onexone}
    یک \onexoneGM یک نگاشت است
    \begin{align}
        K_{\!1\!\times\!1} \ostar :\ \Gamma(\Ain) \to \Gamma(\Aout),\ \ \ 
        \fin \,\mapsto\, K_{\!1\!\times\!1} \,\ostar\, \fin \,:=\, \mathcal{C}_{K_{\!1\!\times\!1}}\! \circ \fin
    \end{align}
    که توسط یک کرنل \emph{درهم‌تننده} \onexoneGM یعنی $K_{\!1\!\times\!1} \in \Hom_G(\rhoin,\rhoout)$ پارامتری می‌شود.
    در اینجا $\mathcal{C}_{K_{\!1\!\times\!1}}$ \lr{M}-مورفیسم کلاف برداری هموار منحصر به فرد بین $\Ain$ و $\Aout$ است که در پیمانه‌های \emph{دلخواه} $\psiAinp$ و $\psiAoutp$ از \lr{G}-اطلس مورد نظر به صورت نقطه‌ای تعریف می‌شود:
    \begin{align}
        \mathcal{C}_{K_{\!1\!\times\!1}}|_p\ :=\ \psiAoutp^{-1} \circ K_{\!1\!\times\!1} \circ \psiAinp \,.
    \end{align}
    استقلال از پیمانه‌های انتخاب شده (استقلال از مختصات $\GM$) با درهم‌تننده بودن $K_{\!1\!\times\!1}$ تضمین می‌شود.
\end{dfn}
برای نشان دادن صریح استقلال از پیمانه انتخاب‌شده، هر بدیهی‌سازی مرتبط با \lr{G} یعنی $\rhoin(g)\,\psiAinp$ و $\rhoout(g)\,\psiAoutp$ را برای یک عنصر گروه ساختار دلخواه $g\in G$ در نظر بگیرید، که ساختار
\begin{align}
    \mathcal{C}_{K_{\!1\!\times\!1}} \mkern-2mu\big|_p\ 
    =\ \ &\big(\rhoout(g)\, \psiAoutp \big)^{-1} \circ K_{\!1\!\times\!1} \circ \big(\rhoin(g)\, \psiAinp\big) \notag \\
    =\ \ &\psiAoutp^{-1} \circ \big( \rhoout(g)^{-1} K_{\!1\!\times\!1}\, \rhoin(g) \big) \circ \psiAinp \notag \\
    =\ \ &\psiAoutp^{-1} \circ K_{\!1\!\times\!1} \circ \psiAinp
\end{align}
را نامتغیر باقی می‌گذارد.
اینکه \onexoneGMs تعریف‌شده به این شکل واقعاً به مقاطعی در $\Gamma(\Aout)$ نگاشت می‌شوند، از این واقعیت ناشی می‌شود که $\mathcal{C}_{K_{\!1\!\times\!1}}$ یک نگاشت کلاف است.
یک نمای کلی از مختصاتی‌سازی‌های محلی \onexoneGMs در شکل~\ref{fig:triv_bundle_morphism_onexone} آورده شده است.

\begin{figure}
    \centering
    \begin{tikzcd}[row sep=4.5em, column sep=4.35em, crossing over clearance=.6ex,
                   execute at end picture={
                        \node [] at (-1.16, -1.05) {$\noncommutative$};
                        \node [] at ( 1.09, -1.05) {$\noncommutative$};
                        }]
        % ROW 1
          U\times \R^{\cin}
                        \arrow[rrrr, pos=.5, rounded corners, to path={ 
                                -- ([yshift=2.5ex]\tikztostart.north) 
                                --node[above]{\small$
                                    \mathcal{C}_{K_{\!1\!\times\!1}}^B
                                    := (\id\times K_{\!1\!\times\!1})
                                $} ([yshift=2.5ex]\tikztotarget.north) 
                                -- (\tikztotarget.north)
                                }]
        &[-3.0ex] & &
        &[-3.0ex] U\times \R^{\cout}
        \\
        % ROW 2
          U\times \R^{\cin}
                        \arrow[drr, pos=.5, "\proj_1"']
                        \arrow[u, "\big(\id\!\times\! \rhoin\big(g^{BA}\big)\!\cdot\big)"]
                        \arrow[rrrr, pos=.5, rounded corners, to path={ 
                                -- ([yshift=-17.5ex]\tikztostart.south) 
                                --node[below]{\small$
                                    \mathcal{C}_{K_{\!1\!\times\!1}}^A
                                    := (\id\times K_{\!1\!\times\!1})
                                $} ([yshift=-17.5ex]\tikztotarget.south) 
                                -- (\tikztotarget.south)
                                }]
        & \piAin^{-1}(U)   \arrow[dr, shorten <=-3pt, shift right=.25, pos=.2, "\piAin\mkern-12mu"']
                        \arrow[l,  "\PsiAin^A"']
                        \arrow[lu, "\PsiAin^B"']
                        \arrow[rr,  "\mathcal{C}_{K_{\!1\!\times\!1}}"]
        &
        & \piAout^{-1}(U)  \arrow[dl, shorten <=-2pt, shift left=.25, pos=.2, "\mkern-6mu\piAout"]
                        \arrow[r,  "\PsiAout^A"]
                        \arrow[ru, "\PsiAout^B"]
        & U\times \R^{\cout}
                        \arrow[u, swap, "\big(\id\!\times\! \rhoout\big(g^{BA}\big)\!\cdot\big)"]
                        \arrow[lld, pos=.5, "\proj_1"]
        \\[1.5ex]
        % ROW 3
        &&
          U \arrow[ul, shorten >=-5pt, bend right=22, looseness=.5, pos=.6, "\!\fin"']
            \arrow[ur, shorten >=-5pt, bend left =22, looseness=.5, pos=.6, "K_{\!1\!\times\!1} \mkern-1mu\ostar\mkern-1mu \fin\!\!"]
    \end{tikzcd}
    \caption{\small
        مختصاتی‌سازی یک \onexoneGM\ $K_{\!1\!\times\!1} \protect\ostar: \Gamma(\Ain) \to \Gamma(\Aout)$ و \lr{M}-مورفیسم کلاف برداری متناظر آن $\mathcal{C}_{K_{\!1\!\times\!1}}$.
        ویژگی کانولوشنی با اشتراک‌گذاری یک ماتریس کرنل $K_{\!1\!\times\!1}\in\R^{\cout\times\cin}$ در موقعیت‌های مکانی مختلف $p\in M$ در مورفیسم کدگذاری می‌شود.
        از آنجایی که هیچ پیمانه‌ای نباید ترجیح داده شود، کرنل علاوه بر این در بدیهی‌سازی‌های مختلف
        $\mathcal{C}_{K_{\!1\!\times\!1}}^A$ و $\mathcal{C}_{K_{\!1\!\times\!1}}^B$ به اشتراک گذاشته می‌شود.
        بنابراین، جابجایی‌پذیری نمودار برای هر انتخاب
        $\Psi_{\!\!\A_\text{in} }^A$،
        $\Psi_{\!\!\A_\text{out}}^A$ و
        $\Psi_{\!\!\A_\text{in} }^B$،
        $\Psi_{\!\!\A_\text{out}}^B$
        قید
        $\rho_\text{out}(g) K_{\!1\!\times\!1} \rho_\text{in}(g)^{-1} = K_{\!1\!\times\!1}\,\ \forall g\!\in G$
        را اعمال می‌کند که ماتریس کرنل را به یک درهم‌تننده (یک نگاشت خطی هم‌متغیر) محدود می‌کند، یعنی
        $K_{\!1\!\times\!1} \in \Hom_G(\rhoin,\rhoout) \subseteq \R^{\cout\times\cin}$.
        به جز $\fin \circ \piAin \neq \id_{\Ain}$ و $\big[K_{\!1\!\times\!1} \protect\ostar \fin\big] \circ \piAout \neq \id_{\Aout}$، نمودار جابجایی‌پذیر است.
    }
    \label{fig:triv_bundle_morphism_onexone}
\end{figure}










\subsubsection[\texorpdfstring{کانولوشن‌های \lr{GM} با اندازه ${1\kern-2.7pt\times\kern-2.9pt1}$}{1x1} به عنوان مقاطع کلاف هومومورفیسم]%
          {کانولوشن‌های \lr{GM} با اندازه \texorpdfstring{${1\kern-2.7pt\times\kern-2.9pt1}$}{1x1} به عنوان مقاطع کلاف هومومورفیسم}
\label{sec:onexone_hom_section}


درحالی‌که \lr{M}-مورفیسم کلاف برداری با مختصاتی‌سازی‌های مستقل از پیمانه از تعریف~\ref{dfn:onexone} و شکل~\ref{fig:triv_bundle_morphism_onexone} یک \onexoneGM را به طور کامل مشخص می‌کند، اکنون دیدگاه جایگزینی را اتخاذ می‌کنیم که \onexoneGMs را بر حسب \emph{کلاف هومومورفیسم} $\Hom(\Ain,\Aout) \xrightarrow{\,\piHom\,} M$ توصیف می‌کند.
برای این منظور، به یاد بیاورید که مورفیسم کلاف برداری $\mathcal{C}$ در معادله~\eqref{eq:bundle_morphism_onexone} به نگاشت‌های خطی ${\mathcal{C}\!\:|_p}: \Ainp\to\Aoutp$ روی هر $p\in M$ محدود می‌شود.
مجموعه چنین نگاشت‌های خطی (یا هومومورفیسم‌های فضای برداری) بین $\Ainp$ و $\Aoutp$ به صورت $\Hom(\Ainp,\Aoutp)$ نشان داده می‌شود.
از آنجایی که این مجموعه تحت ترکیب‌های خطی بسته است، خود یک فضای برداری تشکیل می‌دهد.
می‌توان نشان داد که اجتماع مجزای
\begin{align}
    \Hom(\Ain,\Aout)\ :=\ \coprod_{p\in M} \Hom(\Ainp,\Aoutp)
\end{align}
این فضاهای هومومورفیسم، هنگامی که با نگاشت تصویر $\piHom: \Hom(\Ain,\Aout) \to M$ که عناصر در $\Hom(\Ainp,\Aoutp)$ را به $p$ می‌فرستد و یک ساختار هموار القا شده از ساختار $\Ain$ و $\Aout$ مجهز شود، یک کلاف برداری، یعنی کلاف هومومورفیسم بین $\Ain$ و $\Aout$ را تشکیل می‌دهد~\cite{dundas2018differentialTopology}.
تارهای روی $p$ در رابطه $\Hom(\Ainp,\Aoutp) \cong \Hom(\R^\cin,\R^\cout) \cong \R^{\cout\times\cin}$ صدق می‌کنند، به طوری که می‌توانیم تار نمونه‌ای را فضای برداری ماتریس‌های ${\cout\!\times\!\cin}$ با مقادیر حقیقی در نظر بگیریم.
بدیهی‌سازی‌های
\begin{align}
    \PsiHom:\ \piHom^{-1}(U) \to U\!\times\R^{\cout\times\cin},\ \ H\mapsto \big(p,\ \psiHomp(H)\big) ,
\end{align}
که در آن $p=\piHom(H)$ را به اختصار آورده‌ایم، از بدیهی‌سازی‌های $\Ain$ و $\Aout$ با تعریف
\begin{align}\label{eq:Hom_bdl_triv_ptwise}
    \psiHomp\!:\ \Hom(\Ainp,\Aoutp) \to \R^{\cout\times\cin} ,\ \ \ H\mapsto \psiAoutp \circ H \circ \big(\psiAinp\big)^{-1}
\end{align}
به قیاس با معادلات~\eqref{eq:bundle_morphism_onexone_triv_local} و \eqref{eq:matrix_trivialization} \emph{القا} می‌شوند.
این به معنای نگاشت‌های گذار
\begin{alignat}{3}
    H^B
    \ &=&\ \psiAoutp^B \circ\,&H \circ \big(\psiAinp^B\big)^{-1} \notag \\
    \ &=&\ \psiAoutp^B \circ \big(\psiAoutp^A\big)^{-1}\,&H^A\ \psiAinp^A \circ \big(\psiAinp^B\big)^{-1} \notag \\
    \ &=&\ \rhoout\big(g^{BA}\big)\, &H^A\, \rhoin\big(g^{BA}\big)^{-1} \notag \\
    \ &=:&\ \rhoHom\big(g^{BA}\big)\, &H^A
\end{alignat}
بین پیمانه‌های $\PsiHom^A$ و $\PsiHom^B$ روی $U^A\cap U^B$ است، که در آن برای راحتی نوشتاری، نمایش گروهی هومومورفیسم $\rhoHom:G\to\GL{\R^{\cout\times\cin}}$ را به عنوان ضرب چپ و راست با $\rhoout$ و $\rhoin$ معرفی کردیم.%
\footnote{
    به طور کلی، یک کلاف هومومورفیسم بین دو کلاف برداری \emph{غیرالحاقی} با گروه‌های ساختار $G_1$ و $G_2$ یک گروه ساختار $G_1\times G_2$ خواهد داشت.
    از آنجایی که $\Ain$ و $\Aout$ الحاقی هستند، آنها به طور همزمان تحت همان گروه ساختار $G_1=G_2=G$ تبدیل می‌شوند به طوری که نگاشت‌های گذار آنها در زیرگروه قطری $G$ از $G\times G$ مقدار می‌گیرند.
}
کلاف هومومورفیسم $\Hom(\Ain,\Aout)$ بنا به ساختار، به $\TM$، $\GM$، $\Ain$ و $\Aout$ الحاقی است، یعنی بدیهی‌سازی‌های آن به طور همزمان با بدیهی‌سازی‌های کلاف‌های دیگر تبدیل می‌شوند.
به عنوان یک کلاف برداری \lr{G}-الحاقی، می‌توان آن را با $(\GM\times\R^{\cout\times\cin})/\!\sim_{\!\rhoHom}$ شناسایی کرد.
شکل~\ref{fig:trivialization_hom} یک نمای کلی از بدیهی‌سازی‌های محلی $\Hom(\Ain,\Aout)$ ارائه می‌دهد.
به شباهت با بدیهی‌سازی‌های دیگر کلاف‌های \lr{G}-الحاقی در شکل~\ref{fig:trivializations_TM_FM_A} توجه کنید.

\begin{figure}
    \centering
    \begin{subfigure}[b]{0.48\textwidth}
        \begin{tikzcd}[row sep=4.em, column sep=5.5em]
            & U\times \R^{\cout\times\cin} \\
              \piHom^{-1}(U)
                    \arrow[d, swap, "\piHom"]
                    \arrow[r, "\PsiHom^A"]
                    \arrow[ru, "\PsiHom^B"]
            & U\times \R^{\cout\times\cin}
                    \arrow[u, swap, "\big( \id\times \rhoHom\big(g^{BA}\big)\!\cdot \big)"]
                    \arrow[ld, "\proj_1"] \\
              U
        \end{tikzcd}
        \centering
        \caption{\small
            بدیهی‌سازی $\Hom(\Ain,\Aout)$.
            از آنجایی که به $\TM$، $\GM$، $\Ain$ و $\Aout$ الحاقی است، نگاشت‌های گذار کلاف هومومورفیسم توسط همان عنصر گروهی $g^{BA}$ از گروه ساختار مشترک $G$ تعیین می‌شوند (این را با شکل~\ref{fig:trivializations_TM_FM_A} مقایسه کنید).
            \lr{M}-مورفیسم‌های کلاف برداری بدون قید همانطور که در معادله~\eqref{eq:bundle_morphism_onexone} نشان داده شده است، متناظر با مقاطع هموار بدون قید از $\Hom(\Ain,\Aout)$ هستند.
        }
        \label{fig:trivialization_hom}
    \end{subfigure}
    \hfill
    \begin{subfigure}[b]{0.49\textwidth}
        \centering
        \begin{tikzcd}[row sep=5.5em, column sep=7.em, crossing over clearance=.6ex,
                       execute at end picture={
                            \node [] at (-2.98, -.11) {$\noncommutative$};
                            }]
            % ROW 1
              \piHom^{-1}(U)   \arrow[d, "\piHom", shift left=.2]
                                \arrow[r, "\PsiHom^A"', bend right=3, shift right=.5, looseness=.5, pos=.45]
                                \arrow[r, "\PsiHom^B",  bend left=3,  shift left=.5,  looseness=.5, pos=.45]
            & U\!\times\! \underbrace{\Hom_G(\rhoin,\rhoout)}_{\subseteq\ \R^{\cout\times\cin}}
                                \arrow[loop, distance=3.5em, in=125, out=55, "\big(\id\times \rhoHom\big(g^{BA}\big)\!\cdot\!\big)"']
                                \arrow[ld, "\proj_1", shorten <= -15pt]
            \\
            % ROW 2
              U               \arrow[u, bend left=20, shift left=.5, "\sigma_{K_{1\!\times\!1}}"]
        \end{tikzcd}
        \caption{\small
            مقاطع $\sigma_{K_{1\!\times\!1}}: M\to \Hom(\Ain,\Aout)$ از کلاف هومومورفیسم که متناظر با \onexoneGMs هستند، دقیقاً آنهایی هستند که به \emph{همان} ماتریس (درهم‌تننده)
            $K_{\!1\!\times\!1} \in \Hom_G(\rhoin,\rhoout) \subseteq \R^{\cout\times\cin}$
            در همه پیمانه‌ها بدیهی‌سازی می‌شوند.
            چنین مقاطعی متناظر با نگاشت‌های کلاف هستند که همانطور که در شکل~\ref{fig:triv_bundle_morphism_onexone} مشخص شده است، بدیهی‌سازی می‌شوند.
            \\~
        }
        \label{fig:trivialization_hom_onexone_section}
    \end{subfigure}
    \caption{\small
        بدیهی‌سازی‌های محلی کلاف هومومورفیسم $\Hom(\Ain,\Aout)$، که کلاف برداری نگاشت‌های خطی بین فضاهای $\Ainp$ و $\Aoutp$ برای هر $p\in M$ است.
        طبق معمول $U=U^A\cap U^B$ را به اختصار می‌نویسیم.
        به جز $\sigma_{K_{1\!\times\!1}} \circ \piHom \neq \id_{\Hom(\Ain,\Aout)}$، نمودارها جابجایی‌پذیر هستند.
    }
    \label{fig:trivializations_hom_bundle}
\end{figure}


از دیدگاه کلاف‌های هومومورفیسم، نگاشت‌های کلاف بدون قید مانند معادله~\eqref{eq:bundle_morphism_onexone} متناظر با عمل مقاطع کلاف هومومورفیسم هموار بدون قید هستند
\begin{align}\label{eq:hom_bdl_section_unconstrained}
    \sigma_{\Hom}:M \mapsto \Hom(\Ain,\Aout)
    \quad \textup{به طوری که} \quad
    \piHom \circ \sigma_{\Hom} = \id_M
\end{align}
که می‌توان آنها را به عنوان \emph{میدان‌های کرنل} $1\!\times\!1$ تفسیر کرد که وزن‌ها را به اشتراک نمی‌گذارند.
وجود سراسری آنها با این واقعیت که $\Hom(\Ain,\Aout)$ یک کلاف برداری است تضمین می‌شود.
مقاطع متناظر با \onexoneGMsit علاوه بر این نیاز دارند که تبدیلات خطی $\sigma_{\Hom}(p)\in\Hom(\Ainp,\Aoutp)$ توسط یک کرنل الگو $K_{\!1\!\times\!1} \in\R^{\cout\times\cin}$ تعیین شوند که در موقعیت‌های مختلف $p\in M$ و هر انتخاب پیمانه به اشتراک گذاشته شده است.
بنابراین آنها را می‌توان برای هر $p \in \!M$ به صورت زیر تعریف کرد
\begin{align}
    \sigma_{K_{1\!\times\!1}}(p)\ :=\ \psiHomp^{-1}\big(K_{\!1\!\times\!1}\big), \qquad K_{\!1\!\times\!1} \in \Hom_G(\rhoin,\rhoout) \,,
\end{align}
که در آن بدیهی‌سازی انتخاب‌شده $\PsiHom$ دلخواه است اگر (و تنها اگر) $K_{\!1\!\times\!1}$ قید درهم‌تننده را برآورده کند
\begin{align}\label{eq:onexone_intertwiner_constraint_rhoHom}
    \rhoHom(g) K_{\!1\!\times\!1} = K_{\!1\!\times\!1} \qquad\forall g\in G \,,
\end{align}
که معادل با معادله~\eqref{eq:onexone_kernel_constraint} است.%
\footnote{
    همواری مورد نیاز مقطع از همواری بدیهی‌سازی‌های محلی ناشی می‌شود.
}
بی‌اهمیت بودن پیمانه در چنین مقاطعی در نمودار جابجایی در شکل~\ref{fig:trivialization_hom_onexone_section} به تصویر کشیده شده است (این را با بدیهی‌سازی نگاشت کلاف معادل در شکل~\ref{fig:triv_bundle_morphism_onexone} مقایسه کنید).


\paragraph{ملاحظات پایانی:}
یک لایه هموار \onexoneGM\ $K_{\!1\!\times\!1} \ostar: \Gamma(\Ain)\to \Gamma(\Aout),\ \fin\mapsto \fout$ را می‌توان به طور معادل از طریق یک نگاشت کلاف هموار به صورت $\fout(p) := \mathcal{C}_{K_{\!1\!\times\!1}} \!\circ \fin(p)$ یا از طریق یک مقطع کلاف هومومورفیسم هموار به صورت $\fout(p) := \sigma_{K_{\!1\!\times\!1}}(p) \circ \fin(p)$ تعریف کرد.
بنا به تعریف، هر دو در یک پیمانه دلخواه $\PsiHom^A$ به $\fout^A(p) = K_{\!1\!\times\!1} \fin^A(p)$ بدیهی‌سازی می‌شوند.
استقلال از مختصات $\GM$ این تعریف با خاصیت درهم‌تنندگی کرنل در معادله~\eqref{eq:onexone_kernel_constraint} یا، به طور معادل، معادله~\eqref{eq:onexone_intertwiner_constraint_rhoHom} تضمین می‌شود.
این را می‌توان با در نظر گرفتن یک بدیهی‌سازی متفاوت از طریق $\PsiHom^B$ مشاهده کرد:
\begin{align}
    K_{\!1\!\times\!1} \fin^B(p)
    \ &=\ K_{\!1\!\times\!1} \left( \rhoin\big(g^{BA}_p\big) \fin^A(p) \right) \notag \\
    \ &=\ \rhoout\big(g^{BA}_p\big) K_{\!1\!\times\!1} \fin^A(p) \notag \\
    \ &=\ \rhoout\big(g^{BA}_p\big) \fout^A(p) \notag \\
    \ &=\ \fout^B(p)
\end{align}