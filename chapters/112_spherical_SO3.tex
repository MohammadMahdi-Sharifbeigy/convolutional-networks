%!TEX root=../GaugeCNNTheory.tex


\subsection{\lr{CNN}های کروی کاملاً هموردای دورانی}
\label{sec:spherical_CNNs_fully_equivariant}


این بخش به بحث در مورد کانولوشن‌های کروی کاملاً $\SO3$- یا $\OO3$-هموردا می‌پردازد که در ردیف‌های (۳۱-۳۳) جدول~\ref{tab:network_instantiations} فهرست شده‌اند.
همه آنها را می‌توان به عنوان نمونه‌های خاصی از کانولوشن‌های $\GM$ روی $\SO2$-ساختار در شکل~\ref{fig:G_structure_S2_1} یا $\OO2$-ساختار متناظر، که علاوه بر آن تحت بازتاب‌های چارچوب نیز بسته است، درک کرد.

به جای سازماندهی این بحث بر حسب گروه‌های ساختاری و نمایش‌های گروهی در نظر گرفته شده، ما مدل‌ها را بر اساس چارچوب‌های نظری که در آنها توسعه یافته‌اند، دسته‌بندی می‌کنیم:
\citet{kicanaoglu2019gaugeSphere} یک شبکه پیکسلی را روی کره تعریف کرده و کانولوشن را مستقیماً به عنوان کانولوشن $\GM$ فرمول‌بندی می‌کنند، یعنی بر حسب پیمانه‌ها، کرنل‌های راهبری‌پذیر و انتقال‌دهنده‌های بردار ویژگی.
یک چارچوب جایگزین، کانولوشن‌های گراف روی مش‌های پیکسلی کروی است~\cite{perraudin2018DeepSphere,yang2020rotation}.
چنین کانولوشن‌های گرافی متناظر با کانولوشن‌های $\GM$ با کرنل‌های همسانگرد هستند.
بنابراین آنها بین میدان‌های اسکالر (غیرحساس به جهت) نگاشت انجام می‌دهند.
در آخر، به پیاده‌سازی‌هایی می‌رسیم که کرنل‌های کانولوشن (راهبری‌پذیر) را روی $S^2$ به جای کرنل‌های ما روی فضاهای مماس در نظر می‌گیرند~\cite{esteves2018zonalSpherical,Cohen2018-S2CNN,kondor2018ClebschGordan,esteves2020spinweighted}.
قضیه~\ref{thm:spherical_kernel_space_iso} اثبات می‌کند که چنین کرنل‌های راهبری‌پذیر کروی را می‌توان با کرنل‌های $G$-راهبری‌پذیر روی فضاهای مماس، هنگامی که در مختصات نرمال ژئودزیک بیان می‌شوند، یکی گرفت.
بر اساس این نتیجه، ما در قضیه~\ref{thm:spherical_conv_GM_conv} اثبات می‌کنیم که کانولوشن‌ها با کرنل‌های کروی معادل کانولوشن‌های $\GM$ ما هستند.
برای کامل بودن، باید اشاره کنیم که چنین مدل‌هایی معمولاً در حوزه طیفی پیاده‌سازی می‌شوند.
ما بر این دیدگاه تمرکز نمی‌کنیم اما خواننده علاقه‌مند را به مرور~\citet{esteves2020theoretical} ارجاع می‌دهیم.


\paragraph{کانولوشن‌های \textit{GM} کروی:}

ما با \lr{CNN} کروی توسط \citet{kicanaoglu2019gaugeSphere} شروع می‌کنیم زیرا فرمول‌بندی آن دقیقاً با نظریه عمومی‌تر ما هنگام اعمال بر هندسه کروی مطابقت دارد.
نویسندگان $\SO2$-ساختار را از شکل~\ref{fig:G_structure_S2_1} فرض می‌کنند و بنابراین میدان‌های ویژگی و کرنل‌های کانولوشن $\SO2$-راهبری‌پذیر را در نظر می‌گیرند.
میدان‌های ویژگی بر حسب بردارهای ویژگی که به یک شبکه نمونه‌برداری روی کره اختصاص داده شده‌اند، گسسته‌سازی می‌شوند.
در حالی که این روش در اصل مستقل از طرح نمونه‌برداری خاص است، نویسندگان پیشنهاد می‌کنند که هندسه کروی را با یک مش ایکوسفر گسسته‌سازی کنند.
این مش با گرفتن یک بیست‌وجهی جایگذاری شده، تقسیم مکرر وجوه آن همانطور که در شکل~\ref{fig:ico_neighborhoods} نشان داده شده، و در نهایت تصویر کردن رئوس شبکه به صورت شعاعی روی کره، یعنی به نرم واحد، ساخته می‌شود.
میدان‌های ویژگی نمونه‌برداری شده به صورت عددی با مجموعه‌ای از بردارهای ضریب $f^A(p) \in \R^c$ در رئوس شبکه $p$ نمایش داده می‌شوند، که نسبت به برخی چارچوب‌های راست‌هنجار راست‌گرد دلخواه $\big[e_1^A, e_2^A \big]$ در رئوس بیان می‌شوند.%
\footnote{
	این متناظر با یک انتخاب مستقل از پیمانه $\psiTMp^{A_p}$ روی هر همسایگی باز $U^{A_p}$ از هر رأس~$p$ است.
}
در عمل، چارچوب‌ها با یک بردار مماس منفرد با نرم واحد نمایش داده می‌شوند، که بردار چارچوب دوم از آن به طور یکتا نتیجه می‌شود زیرا چارچوب‌ها راست‌گرد هستند.


برای محاسبه کانولوشن مستقل از مختصات $[K \star f](p)$ از معادله~\eqref{eq:gauge_conv_coord_expression}، \citet{kicanaoglu2019gaugeSphere} باید کرنل $\SO2$-راهبری‌پذیر $K$ را با پول‌بک انتقال‌دهنده $[\Expspf]^A$ از میدان ویژگی $f$ (معادله~\eqref{eq:transporter_pullback_in_coords}) منقبض کنند.
همانطور که در یادگیری عمیق معمول است، در اینجا فرض می‌شود که $K$ دارای تکیه‌گاه فشرده است، به طوری که فقط چند رأس را در یک همسایگی یک-حلقه یا دو-حلقه $\mathcal{N}_p$ حول یک رأس مرکزی $p$ پوشش می‌دهد.
در نظریه پیوسته، پول‌بک انتقال‌دهنده ویژگی‌ها را از تمام نقاط $\exp_p (\psiTMp^A)^{-1}(\mathscr{v})$ برای $\mathscr{v} \in \R^2$ می‌گیرد و آنها را به~$p$ منتقل می‌کند.
در عمل، میدان‌های ویژگی فقط در رئوس شبکه $q$ نمونه‌برداری می‌شوند، که متناظر با ضرایب بردار مماس $v^A_{pq} = \psiTMp^A \log_p(q) \in \R^2$ نسبت به پیمانه~$A$ در رأس~$p$ هستند.%
\footnote{
	اگر نگاشت نمایی به شعاع انژکتیویته محدود نشود، هر رأس $q$ با چندین بردار مماس نمایش داده می‌شود.
	این در عمل مشکلی ایجاد نمی‌کند زیرا فرض می‌شود که کرنل به صورت محلی در داخل شعاع انژکتیویته دارای تکیه‌گاه است.
}
نگاشت‌های لگاریتمی $\log_p(q)$ در اینجا همانطور که در معادله~\eqref{eq:sphere_logmap_explicit} تعریف شده، محاسبه می‌شوند.
انتقال‌دهنده‌های لوی-چیویتا $\rho\big( g_{p\leftarrow q}^{A\widetilde{A}}\big)$ در امتداد ژئودزیک‌ها از $q$ به $p$ در اصل توسط معادله~\eqref{eq:sphere_transporter_explicit_in_coords} داده می‌شوند.
از آنجا که چارچوب‌ها همگی راست‌گرد و راست‌هنجار هستند، و از آنجا که انتقال متناظر با اتصال لوی-چیویتا روی $S^2$ است، اعضای گروه $g_{p\leftarrow q}^{A\widetilde{A}}$ مقادیری در $\SO2$ دارند.
بنابراین آنها به طور کامل توسط زاویه بین محور اول چارچوب منتقل شده $\PTMgamma\big(e_1^{\widetilde{A}}\big)$ از $q$ و محور اول چارچوب $e_1^A$ در~$p$ تعیین می‌شوند.
با این اجزا در دست، نویسندگان پیشنهاد می‌کنند که انتگرال کانولوشن پیوسته را با مجموع گسسته
\begin{align}
	\big[K\star f\big]^A(p)
	\ =\ \int_{\R^2} K(\mathscr{v})\, \big[\!\Expspf]^A(\mathscr{v}) \,\ d\mathscr{v}
	\ \approx\ \sum_{q\in\mathcal{N}_p} K\mkern-1.5mu\big(v^A_{pq}\big)\, \rho\big( g^{A\widetilde{A}}_{p\leftarrow q} \big)\, f^{\widetilde{A}}(q)
\end{align}
روی گره‌های مش همسایه تقریب بزنند.
ضریب نرمال‌سازی گمشده را می‌توان به عنوان جذب شده در پارامترهای یادگرفتنی $w_i \in\R$ از کرنل کانولوشن $\SO2$-راهبری‌پذیر $K = \sum_i w_i K_i$ در نظر گرفت.
به عنوان جایگزینی برای این تقریب ساده، نویسندگان یک طرح انتگرال‌گیری تربیعی بهینه را پیشنهاد می‌کنند، که به طور تجربی نشان داده شده است که هموردایی ایزومتری $\SO3$ مدل را بهبود می‌بخشد.


این مدل در جدول~\ref{tab:network_instantiations} به عنوان پردازشگر میدان‌های ویژگی که مطابق با نمایش منظم $\SO2$ تبدیل می‌شوند، فهرست شده است.
در پیاده‌سازی خود، \citet{kicanaoglu2019gaugeSphere} میدان‌های نمایش تحویل‌ناپذیر از $\SO2$ را در کانولوشن‌ها در نظر می‌گیرند.
یک تغییر پایه قبل و بعد از کانولوشن‌ها این میدان‌های ویژگی را به میدان‌های ویژگی منظم تبدیل می‌کند، که سپس غیرخطی‌های نقطه‌ای مانند \lr{ReLU} روی آنها عمل می‌کنند.
نمایش منظم بی‌نهایت-بعدی $\SO2$ در اینجا با نمایش‌های منظم زیرگروه‌های دوری گسسته $\Operatorname{C}_N$ تقریب زده می‌شود، که نمایش‌های تحویل‌ناپذیر آنها فقط نمایش‌های تحویل‌ناپذیر $\SO2$ تا یک فرکانس محدودکننده باند $\lfloor N/2 \rfloor$ هستند؛ به عنوان مثال به پیوست~F.2 از~\cite{Weiler2019_E2CNN} مراجعه کنید.
تغییر پایه بین نمایش‌ها در این مورد خاص فقط تبدیل فوریه گسسته معمول است.


\paragraph{کانولوشن‌های گراف کروی:}

\lr{CNN}های کروی توسط \citet{perraudin2018DeepSphere} و \citet{yang2020rotation}، که در ردیف (۳۳) جدول~\ref{tab:network_instantiations} فهرست شده‌اند، بر اساس کانولوشن‌های گراف متعارف هستند~\cite{kipf2016semi}.
مش‌های پیکسلی روی کره در اینجا به عنوان گراف تفسیر می‌شوند.
شبکه‌های کانولوشنی گراف سیگنال‌ها را روی کره با ضرب آنها در چندجمله‌ای‌های درجه $\kappa$ به صورت $\sum_{k=0}^\kappa w_k L^k$ از ماتریس لاپلاسین گراف $L$ پردازش می‌کنند، که در آن $w_k \in \R$ پارامترهای قابل آموزش هستند.
از آنجا که ماتریس لاپلاسین فقط برای گره‌های مجاور ورودی‌های غیرصفر دارد، جمله مرتبه $k$-ام فقط بر همسایگی $k$-هاپ حول هر گره تأثیر می‌گذارد.
روی یک مش منظم با یال‌های گراف بدون وزن، سهم یک گره همسایه $q$ در ویژگی انباشته شده در $p$ فقط به فاصله گراف آنها («شعاع») بستگی دارد، اما نه به همسایه خاص («جهت»).
بنابراین کانولوشن گراف در چنین مواردی کرنل‌های \emph{همسانگرد} را روی گراف اعمال می‌کند.
گراف پیکسلی در نظر گرفته شده روی کره این ویژگی‌ها را تقریباً برآورده می‌کند.
از آنجا که \emph{جایگذاری} آنها روی کره به گونه‌ای است که گره‌ها تقریباً به صورت ژئودزیکی هم‌فاصله هستند، همسانگردی توپولوژیکی کرنل‌های کانولوشن گراف متناظر با همسانگردی متریک آنها روی کره است.


گروه ایزومتری $\OO3$ کره، تبدیلات پیمانه‌ای با مقادیر $\OO2$ را القا می‌کند، یعنی با جابجایی الگوها به یک مکان جدید و در یک جهت جدید عمل می‌کند.
به دلیل اشتراک وزن کانولوشنی و همسانگردی کرنل‌ها، کانولوشن‌های گراف به طور بدیهی هموردای ایزومتری هستند.
همانطور که قبلاً در معادله~\eqref{eq:Euc3_punctured_O2_constraint} استدلال شد، کرنل‌های همسانگرد در چارچوب ما به عنوان کرنل‌های $\OO2$-راهبری‌پذیر که بین \emph{میدان‌های اسکالر} نگاشت انجام می‌دهند، بازیابی می‌شوند.
هموردایی $\OO3$ کانولوشن در نظریه ما با ناوردایی $\OO3$ از $\OO2$-ساختار کره توضیح داده می‌شود.


\paragraph{کانولوشن‌های کروی با کرنل‌های روی $S^2$:}

به عنوان یک فضای همگن، کره کانولوشن‌های گروهی (یا فضای خارج‌قسمتی)~\cite{Kondor2018-GENERAL} و کانولوشن‌های راهبری‌پذیر عمومی‌تر روی فضاهای همگن~\cite{Cohen2019-generaltheory} را می‌پذیرد.%
\footnote{
	یک مرور عمومی‌تر از کانولوشن‌ها روی فضاهای همگن در پیوست~\ref{apx:homogeneous_conv} یافت می‌شود.
}
به جای تعریف کرنل‌های کانولوشن روی فضاهای مماس یا روی همسایگی‌های گراف، این رویکردها کرنل‌ها را بلافاصله به عنوان توابع ماتریسی روی کره تعریف می‌کنند، یعنی به صورت
\begin{align}\label{eq:spherical_kernel}
	\kappa:\, S^2 \to \R^{\cout\times\cin} \,.
\end{align}
\citet{Cohen2019-generaltheory} نشان دادند که این کرنل‌ها برای تضمین هموردایی کانولوشن باید یک محدودیت تقارنی را برآورده کنند.
ما در ادامه استدلال می‌کنیم که چنین کرنل‌هایی روی $S^2$ معادل کرنل‌های $G$-راهبری‌پذیر روی فضاهای مماس هستند (قضیه~\ref{thm:spherical_kernel_space_iso})، که این دلالت بر این دارد که \lr{CNN}های کروی پوشش داده شده در~\cite{Cohen2019-generaltheory} و~\cite{Kondor2018-GENERAL} را می‌توان به عنوان کانولوشن‌های $\GM$ در نظر گرفت (قضیه~\ref{thm:spherical_conv_GM_conv}).
یکی گرفتن بین این دو نوع کرنل در اینجا با پول‌بک کردن کرنل‌های کروی از طریق نگاشت نمایی به فضاهای مماس انجام می‌شود.
قبل از توضیح این عملیات، ما به طور خلاصه مدل‌های پیشنهادی در \cite{Cohen2018-S2CNN,esteves2018zonalSpherical,esteves2020spinweighted,kondor2018ClebschGordan} را به عنوان نمونه‌های خاصی از کانولوشن‌های کروی با کرنل‌های کروی مورد بحث قرار می‌دهیم.
برای جزئیات بیشتر در مورد این مدل‌ها، به ویژه در مورد فرمول‌بندی آنها در فضای فوریه، خواننده را به مرور جامع~\citet{esteves2020theoretical} ارجاع می‌دهیم.



بحث خود را با \lr{CNN} کروی کانولوشنی گروهی توسط \citet{Cohen2018-S2CNN} که در ردیف (۳۲) جدول~\ref{tab:network_instantiations} فهرست شده است، آغاز می‌کنیم.
این مدل پشته‌هایی از $\cin$ \emph{میدان اسکالر}
\begin{align}
	f:\, S^2 \to \R^{\cin}
\end{align}
را روی کره با تطبیق آنها با کرنل‌های کروی (معادله~\eqref{eq:spherical_kernel}) در هر ژست تبدیل‌شده با $\SO3$ پردازش می‌کند.
در معادلات، این عملیات به صورت زیر تعریف می‌شود:
\begin{alignat}{3}
	\label{eq:spherical_lifting_conv}
	\big[\kappa \star_{\mkern-2mu S^2}\! f\big](\phi)\ &:=\, \int_{S^2} \kappa(\phi^{-1}(p))\, f(p)\ dp \qquad\quad &&\phi\in\SO3 \,.
	\intertext{
		توجه داشته باشید که نقشه ویژگی حاصل به عنوان پشته‌ای از $\cout$ تابع اسکالر روی گروه تقارنی $\SO3$ در نظر گرفته می‌شود.
		چنین نقشه‌های ویژگی به شکل $f:\SO3 \to \R^{\cin}$ (با تعداد جدید کانال‌های ورودی که متناظر با کانال‌های خروجی لایه قبل است) با کانولوشن‌های گروهی به شکل زیر بیشتر پردازش می‌شوند:
	}
	\label{eq:spherical_group_conv}
	\big[\kappa \star_{\SO3}\! f\big](\phi)\ &:=\, \int_{\SO3}\! \kappa(\phi^{-1}\Omega)\, f(\Omega)\ d\Omega \qquad\quad &&\phi\in\SO3 \,,
\end{alignat}
که در آن $\kappa: \SO3 \to \R^{\cout\times\cin}$ اکنون یک تابع ماتریسی روی $\SO3$ است و $d\Omega$ اندازه هار روی~$\SO3$ است.
از دیدگاه \lr{CNN}های راهبری‌پذیر روی فضاهای همگن~\cite{Cohen2019-generaltheory} و کانولوشن‌های $\GM$، توابع اسکالر روی $\SO3$ به عنوان میدان‌های ویژگی روی $S^2 \cong \SO3/\SO2$ در نظر گرفته می‌شوند، که مطابق با \emph{نمایش منظم} تارهای (زیرگروه‌های پایدارساز)~$\SO2$ تبدیل می‌شوند.
کانولوشن اولیه در معادله~\eqref{eq:spherical_lifting_conv} در این تفسیر، کرنل‌های $\SO2$-راهبری‌پذیر را بین میدان‌های اسکالر و منظم اعمال می‌کند، در حالی که کانولوشن گروهی در معادله~\eqref{eq:spherical_group_conv} کرنل‌های $\SO2$-راهبری‌پذیر را بین میدان‌های منظم روی~$S^2$ اعمال می‌کند.


\citet{esteves2018zonalSpherical} کانولوشن‌های کروی را مانند معادله~\eqref{eq:spherical_lifting_conv} با این فرض اضافی اعمال می‌کنند که کرنل‌ها \emph{ناحیه‌ای} هستند، یعنی تحت دوران‌های $\SO2$ حول محور قطبی ناوردا هستند؛ شکل~\ref{fig:zonal_kernel} را مقایسه کنید.
در حالی که انتگرال از نظر فنی هنوز پاسخ‌هایی در $\SO3$ می‌دهد، تقارن کرنل دلالت بر این دارد که این پاسخ‌ها روی تارهای $\SO2$ از $\SO3$، هنگامی که به عنوان کلاف روی $S^2$ تفسیر می‌شوند، ثابت هستند.
بنابراین میدان‌های ویژگی حاصل به عنوان میدان‌های اسکالر روی $S^2$ شناسایی می‌شوند، که اجازه کاربرد مکرر این نوع کانولوشن را می‌دهد.
توجه داشته باشید که تقارن ناحیه‌ای کرنل با محدودیت کرنل راهبری‌پذیر بین میدان‌های اسکالر (نمایش‌های بدیهی) که قبلاً در معادله~\eqref{eq:Euc3_punctured_SO2_constraint} با آن مواجه شدیم، سازگار است.
همانطور که قبلاً در بخش~\ref{sec:punctured_euclidean_3dim} بحث شد، این محدودیت معادل محدودیت راهبری‌پذیری $\OO2$ بین میدان‌های اسکالر در معادله~\eqref{eq:Euc3_punctured_O2_constraint} است، که این دلالت بر این دارد که مدل \citet{esteves2018zonalSpherical} در واقع $\OO3$-هموردا است.
این مدل از نظر روحی شبیه به کانولوشن‌های گراف کروی است که در بالا مورد بحث قرار گرفت، اما از دیدگاه متفاوتی استخراج شده و در پیاده‌سازی به طور متفاوتی گسسته‌سازی می‌شود.


\citet{esteves2020spinweighted} این مدل را از میدان‌های اسکالر به \emph{توابع کروی با وزن اسپین} تعمیم می‌دهند.
این توابع نه تنها به موقعیت $p\in S^2$ روی کره بستگی دارند، بلکه علاوه بر آن به انتخاب خاص چارچوب مرجع راست‌هنجار و راست‌گرد در آن نقطه نیز بستگی دارند.
آنها با \emph{نمایش‌های تحویل‌ناپذیر} $\rho_s$ از $\SO2$ مرتبط هستند، که در آن عدد صحیح $s\in\Z$ به عنوان وزن اسپین توابع شناخته می‌شود.%
\footnote{
	می‌توان این مفهوم را به نمایش‌های اسپین، که با وزن‌های اسپین نیمه‌صحیح برچسب‌گذاری شده‌اند، تعمیم داد.
}
مقادیر آنها برای چارچوب‌های مختلف $\SOpM$ از $\SO2$-ساختار $\SOM$ به گونه‌ای محدود شده‌اند که تبدیلات پیمانه‌ای چارچوب با $g\in\SO2$ منجر به تبدیل مقدار تابع با $\rho_s(g)$ می‌شود.
بنابراین، در معادلات، آنها با%
\footnote{
	یک پیاده‌سازی با مقادیر حقیقی به جای آن توابع با وزن اسپین به شکل $\prescript{}{s}f: \SOM \to \R^{\dim(\rho_s)}$ را در نظر می‌گیرد، که در آن~$\rho_s$ نمایش‌های تحویل‌ناپذیر $\SO2$ روی اعداد حقیقی هستند.
}
\begin{align}
	\prescript{}{s}f: \SOM \to \Cm
	\ \ \ \textup{such that} \ \ \
	\prescript{}{s}f\big( [e_1,e_2] \lhd g \big) = \rho_s(g) \prescript{}{s}f \big( [e_1,e_2] \big)
	\ \ \ \forall\ [e_1,e_2] \in \SOpM,\ g\in\SO2 \,;\!
\end{align}
تعریف می‌شوند؛ برای جزئیات بیشتر و تعاریف جایگزین به~\cite{boyle2016should} مراجعه کنید.
به شباهت این محدودیت تقارنی با رابطه هم‌ارزی
\begin{align}
	\big[ [e_i]_{i=1}^2 \lhd g,\, \mathscr{f} \big]\ \sim_{\rho_s}\ \big[ [e_i]_{i=1}^2,\, \rho_s(g) \mathscr{f} \big]
\end{align}
از معادله~\eqref{eq:equiv_relation_A} که زیربنای تعریف کلاف‌های همبسته است، توجه کنید.
توابع کروی با وزن اسپین در واقع معادل مقاطعی از کلاف‌های همبسته
$(\SOM \times \Cm)/\!\sim_{\rho_s}$ هستند؛
به عنوان مثال به گزاره~۱.۶.۳ در~\cite{wendlLectureNotesBundles2008} مراجعه کنید.
آنها در نظریه ما به سادگی به عنوان میدان‌های نمایش تحویل‌ناپذیر $\SO2$ ظاهر می‌شوند، از جمله میدان‌های اسکالر برای $s=0$ و میدان‌های برداری برای~$s=1$.
شبکه‌های عصبی پیشنهادی توسط~\citet{esteves2020spinweighted} ویژگی‌های با وزن اسپین را با کرنل‌های با وزن اسپین روی کره کانوالو می‌کنند.
این عملیات متناظر با یک کانولوشن با کرنل‌های $\SO2$-راهبری‌پذیر است که در آن $\rhoin$ و $\rhoout$ نمایش‌های تحویل‌ناپذیر هستند.


مدل‌های \cite{Cohen2018-S2CNN,esteves2018zonalSpherical,esteves2020spinweighted} در ابتدا در حوزه فضایی فرمول‌بندی شده‌اند، یعنی به عنوان پردازش توابع روی $S^2$ همانطور که در بالا بحث شد.
با این حال، آنها در حوزه طیفی پیاده‌سازی می‌شوند، که به لطف قضایای کانولوشن تعمیم‌یافته روی $S^2$ و روی $\SO3$ ممکن است~\cite{makadia2006rotation,Kondor2018-GENERAL,vilenkin2013representation}.
\citet{kondor2018ClebschGordan} این رویکردها را تعمیم می‌دهند و مدلی را پیشنهاد می‌کنند که بر اساس ترکیبات خطی یادگرفته شده از تمام مُدهای فوریه میدان‌های ویژگی با فرکانس یکسان است.
نویسندگان استدلال می‌کنند که این رویکرد فضای کامل نگاشت‌های خطی $\SO3$-هموردا را بین میدان‌های ویژگی روی کره پوشش می‌دهد.
از سوی دیگر، \citet{Cohen2019-generaltheory} نشان می‌دهند که هر چنین نگاشتی را می‌توان در حوزه فضایی به عنوان یک کانولوشن با کرنل‌های کروی $\SO2$-راهبری‌پذیر نوشت.
یک ویژگی قابل توجه مدل پیشنهادی توسط \citet{kondor2018ClebschGordan} این است که به طور کامل در فضای فوریه عمل می‌کند:
به جای تبدیل بازگشت به حوزه فضایی و اعمال غیرخطی‌های نقطه‌ای مانند \lr{ReLU} در آنجا، همانطور که در رویکردهای قبلی انجام می‌شد، نویسندگان حاصلضرب تانسوری را بین تمام میدان‌های ویژگی محاسبه کرده و متعاقباً آنها را از طریق تجزیه کلبش-گوردون به ویژگی‌های تحویل‌ناپذیر (مُدهای فوریه) باز می‌گردانند.
این از نظر محاسباتی سودمند است، با این حال، به قیمت از دست دادن محلی بودن غیرخطی‌ها تمام می‌شود.
وظایف یادگیری خاص، به ویژه در علوم طبیعی، ممکن است از چنین غیرخطی‌هایی بهره‌مند شوند زیرا تعاملات فیزیکی اغلب با حاصلضرب‌های تانسوری توصیف می‌شوند.


همانطور که در~\cite{Cohen2019-generaltheory,Cohen2018-intertwiners} استدلال شده است، تمام این مدل‌ها را می‌توان به عنوان اعمال کرنل‌های راهبری‌پذیر روی $S^2$ در نظر گرفت که بین میدان‌های اسکالر~\cite{esteves2018zonalSpherical}، میدان‌های ویژگی منظم~\cite{Cohen2018-S2CNN} یا میدان‌های نمایش تحویل‌ناپذیر~\cite{esteves2020spinweighted,kondor2018ClebschGordan} نگاشت انجام می‌دهند.
در باقیمانده این بخش و پیوست~\ref{apx:spherical_conv_main} ما نشان می‌دهیم که آنها را نیز می‌توان به عنوان کانولوشن‌های $\GM$ در نظر گرفت.
ادعای اینکه کانولوشن‌های کروی با کرنل‌های راهبری‌پذیر روی $S^2$ معادل کانولوشن‌های $\GM$ هستند، در قضیه~\ref{thm:spherical_conv_GM_conv} دقیقاً بیان می‌شود.
این قضیه به طور حیاتی به قضیه~\ref{thm:spherical_kernel_space_iso} متکی است، که یک ایزومورفیسم را بین کرنل‌های راهبری‌پذیر کروی و کرنل‌های $G$-راهبری‌پذیر روی فضاهای مماس برقرار می‌کند.

فرض کنید $\I$ هر گروه ایزومتری متعدی از کره باشد، یعنی $\I=\OO3$ یا $\I=\SO3$.
\citet{Cohen2019-generaltheory} کانولوشن‌های کروی $\I$-هموردا را بر حسب کرنل‌های کروی $\Stab{n}$-راهبری‌پذیر ${\kappa: S^2 \to \R^{\cout\times\cin}}$ توصیف می‌کنند، که در آن $\Stab{n} < \I$ زیرگروه پایدارساز هر نقطه $n\in S^2$ است، به عنوان مثال قطب شمال.
از آنجا که این کرنل‌ها روی کره تعریف شده‌اند، که از نظر توپولوژیکی با $\R^2$ متمایز است، تعریف مستقیم یک ایزومورفیسم بین آنها و کرنل‌های $G$-راهبری‌پذیر ممکن نیست.
با این حال، از آنجا که قطب جنوب $-n$ یک مجموعه با اندازه صفر است، می‌توانیم دامنه انتگرال‌گیری $S^2$ کانولوشن‌های کروی را با $S^2\backslash \mkern-1mu\minus\mkern1mu n$ جایگزین کنیم بدون اینکه نتیجه تغییر کند.
با این تطبیق، کرنل‌های راهبری‌پذیر کروی \citet{Cohen2019-generaltheory} به صورت زیر تعریف می‌شوند:
\begin{align}\label{eq:spherical_steerable_kernel_space}
	\mathscr{K}^{\Stab{n}}_{\rhoin\mkern-1mu,\rhoout}
	:= \pig\{ \kappa: S^2\backslash \mkern-1mu\minus\mkern1mu n \to \R^{\cout\times\cin}
	\,\pig|\ \kappa\big(\xi (p)\big) = \rhoout\big( g_\xi^{NN}(n) \big) \mkern-1.5mu\cdot\mkern-1.5mu \kappa(p) \mkern-1.5mu\cdot\mkern-1.5mu \rhoin\big( g_\xi^{XP}(p) \big)^{-1}
	\qquad \\ \notag
	\forall\,\ p\in S^2\backslash \mkern-1mu\minus\mkern1mu n,\ \ \xi\in\Stab{n} \pig\} \,,
\end{align}
هنگامی که به نمادگذاری ما ترجمه شود.
از آنجا که کرنل‌ها به صورت سراسری روی کره تعریف شده‌اند، مقادیر آنها در $\R^{\cout\times\cin}$ نسبت به پیمانه‌های بالقوه متفاوت $N$ در $n$ که کرنل در آن متمرکز است، $P$ در $p\in S^2$ که کرنل یک ویژگی $f^P(p) \in \R^\cin$ را منقبض می‌کند و $X$ در $\xi(p)$ که این ویژگی تحت عمل $\xi \in \Stab{n}$ جابجا می‌شود، بیان می‌شوند.
این محدودیت کرنل تمام مقادیر کرنل را که روی مدارهای ${\Stab{n}\mkern-4mu.\mkern1mup} = \{ \xi(p) \,|\, \xi\in\Stab{n} \}$ قرار دارند، از طریق تبدیلات پیمانه القا شده توسط ایزومتری آنها $g_\xi^{XP}(p)$ و $g_\xi^{NN}(n)$ به هم مرتبط می‌کند؛ به معادلات~\eqref{cd:pushforward_GM_coord_extended} و~\eqref{cd:pushforward_A_coord} مراجعه کنید.%
\footnote{
	\citet{Cohen2019-generaltheory} تبدیلات پیمانه القا شده توسط ایزومتری را با $\Operatorname{h}(p,\xi)$ به جای $g_\xi^{XP}(p)$ نشان می‌دهند، با این فرض که پیمانه‌های $X$ در $\xi(p)$ و $P$ در $p$ یکسان هستند.
	تعریف آنها از $\Operatorname{h}(p,\xi)$ مشابه معادله ما~\eqref{eq:pushfwd_section_right_action} است.
}
کرنل‌های $G$-راهبری‌پذیر معادل ما، که در آن $G \cong \Stab{n}$، با
\begin{align}\label{eq:G_steer_kernel_space_open_ball_pi}
	\mathscr{K}^{G,B_{\R^2}(0,\pi)}_{\rhoin\mkern-1mu,\rhoout}
	:= \Big\{ K\!: B_{\R^2}(0,\pi) \to \R^{\cout\times\cin} \mkern1.5mu\Big|\,
	K(g\mkern1mu \mathscr{v}) =
	\rhoout(g) \mkern-2mu\cdot\mkern-2mu K(\mathscr{v}) \mkern-2mu\cdot\mkern-2mu \rhoin(g)^{-1} \ \ \ \forall\ \mathscr{v}\in B_{\R^2}(0,\pi),\,\ g\in G \Big\} .
\end{align}
داده می‌شود. دامنه کرنل در اینجا از $\R^2$ به گوی باز $B_{\R^2}(0,\pi) := \{ \mathscr{v}\in\R^2 \,|\, \lVert \mathscr{v}\rVert < \pi \}$ با شعاع $\pi$ حول مبدأ $\R^2$ محدود شده است -- که می‌توان آن را از طریق نگاشت نمایی با $S^2\backslash \mkern-1mu\minus\mkern1mu n$ یکی گرفت.
توجه داشته باشید که $\mathscr{K}^{G,B_{\R^2}(0,\pi)}_{\rhoin\mkern-1mu,\rhoout}$ خوش‌تعریف است زیرا $\Stab{n} \cong G$ شامل ایزومتری‌ها است، که این دلالت بر $G=\OO2$ یا $G=\SO2$ دارد، که تحت عمل آنها $B_{\R^2}(0,\pi)$ بسته است.
ما علاوه بر این ضریب دترمینان را از محدودیت $G$-راهبری‌پذیری عمومی‌تر در معادله~\eqref{eq:G-steerable_kernel_space} حذف کردیم زیرا ${|\!\det g| = 1}$ برای $G\leq\OO2$.
محدودیت کرنل ما به طور قابل توجهی ساده‌تر از محدودیت \citet{Cohen2019-generaltheory} است زیرا کرنل را به صورت محلی نسبت به یک پیمانه منفرد توصیف می‌کند، به جای اینکه به صورت سراسری نسبت به یک اطلس از پیمانه‌ها توصیف کند.
توجه داشته باشید که ما فرض همواری را روی کرنل‌ها حذف کردیم، زیرا همواری یا پیوستگی میدان‌های ویژگی توسط~\citet{Cohen2019-generaltheory} مورد بحث قرار نگرفته است.
این ویژگی را می‌توان به راحتی با خواستن اینکه کرنل‌های $G$-راهبری‌پذیر برای $\lVert\mathscr{v}\rVert$ که به سمت $\pi$ می‌رود، به مقدار یکسانی همگرا شوند، که از طریق نگاشت نمایی متناظر با قطب جنوب است، اضافه کرد.

فضاهای کرنل‌های $\Stab{n}$-راهبری‌پذیر روی $S^2\backslash \mkern-1mu\minus\mkern1mu n$ و کرنل‌های $G$-راهبری‌پذیر روی $B_{\R^2}(0,\pi)$ ایزومورف هستند، یعنی کرنل‌های آنها با یک نگاشت وارون‌پذیر $\Omega$ که محدودیت‌های کرنل را رعایت می‌کند، یکی گرفته می‌شوند:
\begin{equation}
	\begin{tikzcd}[row sep=3.5em, column sep=12.em]
		\mathscr{K}^{G,B_{\R^2}\mkern-1mu(0,\pi)}_{\rhoin\mkern-1mu,\rhoout}
		\arrow[r, bend left=8, shift left=2pt, "\Omega"]
		&
		\mathscr{K}^{\Stab{n}}_{\rhoin\mkern-1mu,\rhoout}
		\arrow[l, bend left=8, shift left=2pt, "\Omega^{-1}"]
	\end{tikzcd}
\end{equation}
این ایزومورفیسم (یا بهتر بگوییم وارون آن $\Omega^{-1}$) را می‌توان به عنوان مشابه \emph{پول‌بک انتقال‌دهنده} از میدان‌های ویژگی در نظر گرفت:
این ایزومورفیسم مقادیر کرنل را از نقاط $\exp_n\! \big(\psiTMn^N\big)^{\!-1} \mathscr{v}$ در $S^2\backslash \mkern-1mu\minus\mkern1mu n$ به \emph{مختصات نرمال ژئودزیک} $\mathscr{v}\in B_{\R^2}(0,\pi)$ پول‌بک می‌کند.
برای بیان مقادیر کرنل از تمام نقاط $p\in S^2\backslash \mkern-1mu\minus\mkern1mu n$ نسبت به همان پیمانه، این ایزومورفیسم انتقال‌دهنده‌های لوی-چیویتا $\rho\big( g_{n\leftarrow p}^{NP} \big)$ را از $p$ در امتداد ژئودزیک‌ها به قطب شمال~$n$ اعمال می‌کند.
علاوه بر این، مقادیر کرنل را با عنصر حجم ریمانی
$\sqrt{\big|\eta_p^{\partial\mkern-2mu/\mkern-2mu\partial\mathscr{v}}\big|}
:= {\sqrt{\big| \!\det\!\big( \eta_p\big( \frac{\partial}{\partial \mathscr{v}_i} \mkern-2mu\big|_p, \frac{\partial}{\partial \mathscr{v}_j} \mkern-2mu\big|_p \big)_{ij} \big)\big|} }$
نسبت به سیستم مختصات نرمال ژئودزیک (چارت مختصاتی)
$\mathscr{v}: S^2\backslash \mkern-1mu\minus\mkern1mu n \to B_{\R^2}(0,\pi),\ \ p \mapsto \mathscr{v}(p) := \psiTMn^N \log_n p$ تغییر مقیاس می‌دهد.%
\footnote{
	توجه داشته باشید که پایه‌های مختصاتی
	$\big[ \frac{\partial}{\partial\mathscr{v}_1} \mkern-2mu|_p,\, \frac{\partial}{\partial\mathscr{v}_2} \mkern-2mu|_p \big]$
	که توسط مختصات نرمال ژئودزیک
	$\mathscr{v}: S^2\backslash \mkern-1mu\minus\mkern1mu n \to B_{\R^2}(0,\pi)$
	القا می‌شوند، برای $G\leq\OO2$ در~$\GM$ موجود \emph{نیستند}.
	این پایه‌ها نقشی در کانولوشن $\GM$ ندارند اما فقط برای تصحیح حجم ریمانی هنگام انتگرال‌گیری در مختصات نرمال ژئودزیک روی کره ظاهر می‌شوند.
}
قضیه زیر ایزومورفیسم فضای کرنل را به طور رسمی تعریف و اثبات می‌کند.
\begin{thm}[کرنل‌های راهبری‌پذیر کروی در مختصات ژئودزیک]
	\label{thm:spherical_kernel_space_iso}
	فرض کنید $\I$ هر گروه ایزومتری متعدی از $S^2$ باشد و $\Stab{n}$ زیرگروه پایدارساز آن در قطب شمال $n\in S^2$ باشد.
	با توجه به هر انتخاب از پیمانه $\psiTMn^N$ در این قطب، فرض کنید $G\leq \GL{2}$ گروه ساختاری ایزومورف باشد که $\Stab{n}$ را در مختصات مطابق با
	$\Stab{n} \xrightarrow{\sim} G,\ \xi \mapsto \psiTMn^N \circ \dxiTM \circ \big(\psiTMn^N \big)^{-1}$ نمایش می‌دهد.
	فضای $\mathscr{K}^{\Stab{n}}_{\rhoin\mkern-1mu,\rhoout}$ از کرنل‌های $\Stab{n}$-راهبری‌پذیر روی $S^2\backslash \mkern-1mu\minus\mkern1mu n$ توسط \citet{Cohen2019-generaltheory} (معادله~\eqref{eq:spherical_steerable_kernel_space}) سپس با فضای $\mathscr{K}^{G,B_{\R^2}\mkern-1mu(0,\pi)}_{\rhoin\mkern-1mu,\rhoout}$ از کرنل‌های $G$-راهبری‌پذیر روی گوی باز $B_{\R^2}(0,\pi)$ (معادله~\eqref{eq:G_steer_kernel_space_open_ball_pi}) ایزومورف است.
	ایزومورفیسم فضای کرنل
	\begin{align}
		\Omega:\ 
		\mathscr{K}^{G,B_{\R^2}\mkern-1mu(0,\pi)}_{\rhoin\mkern-1mu,\rhoout}
		\xrightarrow{\,\sim\,}\,
		\mathscr{K}^{\Stab{n}}_{\rhoin\mkern-1mu,\rhoout}
	\end{align}
	با
	\begin{alignat}{4}
		\label{eq:spherical_kernel_space_iso_Omega}
		\Omega(K)\! &:&\,\ S^2 \backslash \mkern-1mu\minus\mkern1mu n \,&\to&\, \R^{\cout\times\cin},
		\quad p \,&\mapsto\,
		\big[\Omega(K)\big](p)\ &:=&\ K\big( \psiTMn^N \log_n p \big)\, \rhoin\big( g_{n\leftarrow p}^{NP} \big)\, \sqrt{\big|\eta_p^{\partial\mkern-2mu/\mkern-2mu\partial\mathscr{v}}\big|}^{\,-1}
		\intertext{
			داده می‌شود اگر کرنل نسبت به پیمانه‌های (بالقوه مستقل) $N$ در $n$ و $P$ در~$p$ بیان شود.
			وارون آن با
		}
		\Omega^{-1}(\kappa)\! &:&\,\ B_{\R^2}\mkern-1mu(0,\pi) &\to& \R^{\cout\times\cin},
		\quad \mathscr{v} \,&\mapsto\,
		\big[\Omega^{-1}(\kappa)\big](\mathscr{v})\ &:=&\ \kappa\big(\! \exp_n\! \big(\psiTMn^N\big)^{\!-1} \mathscr{v} \big)\, \rhoin\big( g_{n\leftarrow p}^{NP} \big)^{\!-1} \sqrt{\big|\eta_p^{\partial\mkern-2mu/\mkern-2mu\partial\mathscr{v}}\big|} ,
	\end{alignat}
	داده می‌شود، که در آن ما $p := \exp_n\! \big(\psiTMn^N\big)^{\!-1} \mathscr{v}$ را به صورت مخفف نوشته‌ایم.
\end{thm}
\begin{proof}
	با جایگذاری این دو عبارت، به راحتی می‌توان دید که $\Omega^{-1}$ یک وارون خوش‌تعریف برای $\Omega$ است زیرا
	$\Omega \circ \Omega^{-1} = \id_{\mathscr{K}^{\Stab{n}}_{\rhoin\mkern-1mu,\rhoout}}$
	و
	$\Omega^{-1} \circ \Omega = \id_{\mathscr{K}^{G,B_{\R^2}\mkern-1mu(0,\pi)}_{\rhoin\mkern-1mu,\rhoout}}$.
	بخش فنی اثبات این است که نشان دهیم این دو محدودیت کرنل یکدیگر را نتیجه می‌دهند، که این کار در پیوست~\ref{apx:spherical_conv_kernel_space_iso} انجام شده است.
\end{proof}
توجه داشته باشید که ضریب مقیاس‌بندی حجم برای برقراری ایزومورفیسم بین فضاهای کرنل ضروری نیست اما برای معادل ساختن انتگرال کانولوشن کروی روی $S^2\backslash \mkern-1mu\minus\mkern1mu n$ با انتگرال کانولوشن $\GM$ روی $B_{\R^2}(0,\pi)$ لازم است.


\citet{Cohen2018-intertwiners} کانولوشن ${[\kappa \star_{\mkern-2mu S^2}\! f]}$ از یک میدان ویژگی $f\in\Gamma(\Ain)$ را با کرنل‌های راهبری‌پذیر کروی $\kappa \in \mathscr{K}^{\Stab{n}}_{\rhoin\mkern-1mu,\rhoout}$ در مختصات تعریف می‌کنند.
با توجه به پیمانه‌های $P$ در $p$ و $Q$ در $q$ (اشتباه تایپی در متن اصلی، باید $P$ در $p$ و $Q$ در $q$ باشد)، فرض کنید $\phi_p \in \I$ ایزومتری یکتایی باشد که قطب شمال را به $p$ منتقل می‌کند، یعنی $\phi_p(n) = p$، و چارچوب در $n$ را به چارچوب در $p$ نگاشت می‌دهد، یعنی $\dphipGM \sigma^N(n) = \sigma^P(p)$ یا به طور معادل $g_{\phi_p}^{PN}(n) = e$.
علاوه بر این، فرض کنید $X$ پیمانه در $\phi_p^{-1}(q)$ باشد.
کانولوشن کروی سپس در~\cite{Cohen2018-intertwiners} نسبت به این پیمانه‌ها به صورت نقطه‌ای با
\begin{align}\label{eq:spherical_steerable_conv}
	\big[\kappa \star_{\mkern-2mu S^2}\! f\big]^P(p)
	\ :=\ \int\limits_{S^2} \kappa\big(\phi_p^{-1}q)\, \rhoin\big( g_{\phi_p^{-1}}^{XQ}(q) \big)\, f^Q(q)\ dq
	\ = \int\limits_{S^2 \backslash \mkern-2mu -p} \mkern-8mu \kappa\big(\phi_p^{-1}q)\, \rhoin\big( g_{\phi_p^{-1}}^{XQ}(q) \big)\, f^Q(q)\ dq \,,
\end{align}
تعریف می‌شود، که در آن ما نقطه متقابل $-p$ را در مرحله دوم بدون تغییر نتیجه حذف کردیم.%
\footnote{
	این فرمول‌بندی عمومی‌تر از فرمول‌بندی در معادله~\eqref{eq:spherical_lifting_conv} است.
	دومی برای کرنل‌هایی که میدان‌های اسکالر را به میدان‌های ویژگی منظم نگاشت می‌دهند، بازیابی می‌شود.
}
به طور شهودی، این عملیات یک ویژگی خروجی را در $p$ با
۱) گرفتن هم کرنل و هم میدان ورودی،
۲) دوران دادن آنها از طریق $\phi_p^{-1}$ به طوری که $p$ به قطب شمال منتقل شود (از طریق تبدیل پیمانه القا شده برای بردار ویژگی) و
۳) انتگرال‌گیری از حاصلضرب آنها روی کره، محاسبه می‌کند.
بنابراین، این عملیات به جای اشتراک مستقیم وزن‌ها روی فضاهای مماس، همانطور که ما انجام می‌دهیم، وزن‌ها را از طریق عمل ایزومتری به اشتراک می‌گذارد.
بنا به تعریف $\phi_p$, هر دو تعریف از اشتراک وزن، کرنل را در مکان هدف $p$ به گونه‌ای جهت‌دهی می‌کنند که با چارچوب انتخاب شده $\sigma^P(p)$ در این مکان تراز شود.
قضیه زیر اثبات می‌کند که کانولوشن $\GM$ با یک کرنل $K \in \mathscr{K}^{G,B_{\R^2}(0,\pi)}_{\rhoin\mkern-1mu,\rhoout}$ معادل کانولوشن کروی با کرنل کروی متناظر $\Omega(K) \in \mathscr{K}^{\Stab{n}}_{\rhoin\mkern-1mu,\rhoout}$ است.
\begin{thm}[کانولوشن‌های راهبری‌پذیر کروی به عنوان کانولوشن‌های \textit{GM}]
	\label{thm:spherical_conv_GM_conv}
	فرض کنید $\Stab{n}$ زیرگروه پایدارساز هر گروه ایزومتری متعدی $\I$ از $S^2$ باشد و $G \leq \GL{2}$ هر گروه ساختاری ایزومورف باشد.
	علاوه بر این، فرض کنید $K \in \mathscr{K}^{G,B_{\R^2}(0,\pi)}_{\rhoin\mkern-1mu,\rhoout}$ هر کرنل $G$-راهبری‌پذیر روی گوی باز $B_{\R^2}(0,\pi)$ با شعاع $\pi$ باشد (معادله~\eqref{eq:G_steer_kernel_space_open_ball_pi})
	و فرض کنید $\Omega(K) \in \mathscr{K}^{\Stab{n}}_{\rhoin\mkern-1mu,\rhoout}$ کرنل $\Stab{n}$-راهبری‌پذیر متناظر آن روی $S^2\backslash \mkern-1mu\minus\mkern1mu n$ باشد (معادلات~\eqref{eq:spherical_steerable_kernel_space} و~\eqref{eq:spherical_kernel_space_iso_Omega}).
	کانولوشن $\GM$ (در اینجا برای وضوح با $\star_{\mkern-1mu\scalebox{.64}{$\GM$}}$ نشان داده شده) با $K$ سپس معادل کانولوشن کروی ($\star_{\mkern-2mu S^2}$، معادله~\eqref{eq:spherical_steerable_conv}) توسط \citet{Cohen2018-intertwiners} با کرنل کروی $\Omega(K)$ است، یعنی،
	\begin{align}
		\Omega(K) \star_{\mkern-2mu S^2}\! f\ =\ K \star_{\mkern-1mu\scalebox{.64}{$\GM$}} f
	\end{align}
	برای هر میدان ویژگی کروی $f \in \Gamma(\Ain)$ برقرار است.
\end{thm}
\begin{proof}
	اثبات در پیوست~\ref{apx:spherical_conv_equivalence} ارائه شده است.
\end{proof}
این اثبات ادعای ما را توجیه می‌کند که مدل‌های \cite{Cohen2018-S2CNN,esteves2018zonalSpherical,esteves2020spinweighted,kondor2018ClebschGordan} که در این بخش مورد بحث قرار گرفتند، همگی موارد خاصی از کانولوشن‌های $\GM$ هستند.