%!TEX root=../GaugeCNNTheory.tex


\section{تبدیلات میدان کرنل مستقل از مختصات و کانولوشن‌های \lr{GM}}
\label{sec:gauge_CNNs_global}


کلاف‌های \lr{G}-الحاقی معرفی‌شده در بخش~\ref{sec:bundles_fields} امکان توصیف میدان‌های ویژگی - و در نتیجه شبکه‌های کانولوشنی - را در سطح سراسری فراهم می‌کنند.
با داشتن دنباله‌ای از کلاف‌های بردار ویژگی \lr{G}-الحاقی روی \lr{M}
${\A_0\! \xrightarrow{\scalebox{.85}{$\pi_{{\scalebox{.65}{$\!\!\A_0$}}}$}}\! M ,}\ \dots,\ 
 {\A_N\! \xrightarrow{\scalebox{.85}{$\pi_{{\scalebox{.65}{$\!\!\A_N$}}}$}}\! M}$،
ما شبکه‌های کانولوشنی مستقل از مختصات را به عنوان دنباله‌هایی
\begin{align}
    \Gamma(\A_0)\, \xrightarrow{\ \ L_1\ \ }\, \Gamma(\A_1)\, \xrightarrow{\ \ L_2\ \ }\ \ \dots\ \ \xrightarrow{\ \ L_N\ \ }\, \Gamma(\A_N)
\end{align}
از لایه‌های پارامتری‌شده $L_1,\dots, L_N$ توصیف می‌کنیم که بین فضاهای ویژگی $\Gamma(\A_0), \dots, \Gamma(\A_N)$، یعنی بین فضاهای میدان‌های ویژگی مدل‌شده توسط کلاف‌های متناظر، نگاشت می‌کنند.
درحالی‌که انواع میدان (یا قوانین تبدیل) ${\rho_i:G\to\GL{{c_i}}}$ برای کلاف‌های میانی
$\A_i := (\GM\times\R^{c_i})/\!\sim_{\!\rho_i}$ برای $i=1,\dots,N-1$
باید توسط کاربر به عنوان یک اَبَرپارامتر مشخص شوند، انواع میدان $\rho_0:G\to\GL{{c_0}}$ و $\rho_N:G\to\GL{{c_N}}$ برای ورودی و خروجی شبکه معمولاً توسط وظیفه یادگیری تعیین می‌شوند.
ساختار ماژولار شبکه‌های عصبی اجازه می‌دهد تا توجه را به لایه‌های منفرد محدود کنیم که بین فضاهای ویژگی $\Gamma(\Ain)$ و $\Gamma(\Aout)$ با ابعاد $\cin$ و $\cout$ و نوع $\rhoin$ و $\rhoout$ نگاشت انجام می‌دهند.


\etocsettocdepth{3}
\etocsettocstyle{}{} % from now on only local tocs
\localtableofcontents


هدف اصلی این بخش، معرفی کانولوشن‌های مستقل از مختصات $\GM$ است که بلوک‌های سازنده اصلی شبکه‌های مستقل از مختصات $\GM$ روی منیفلدهای ریمانی هستند.
برای شروع، و برای معرفی مفاهیمی که بعداً مورد نیاز خواهند بود، ما در بخش~\ref{sec:onexone} ابتدا بر روی مورد ساده‌تر\onexoneGMsitfarsi تمرکز خواهیم کرد که کرنل‌های نقطه‌مانند را به کار می‌برند.
بخش~\ref{sec:global_conv} تمرکز را به کانولوشن‌های $\GM$ و تبدیلات میدان کرنل با کرنل‌های با گستره فضایی منتقل می‌کند.
آن‌ها بر حسب \emph{میدان‌های کرنل} سراسری و هموار پارامتری می‌شوند که در بخش~\ref{sec:kernel_fields} معرفی شده‌اند.
\emph{میدان‌های کرنل کانولوشنی} $\GM$ ملزم به اشتراک‌گذاری وزن‌ها بین موقعیت‌های مکانی مختلف هستند.
برای اینکه این اشتراک‌گذاری وزن مستقل از مختصات $\GM$ باشد، کرنل‌های الگو که میدان‌های کرنل کانولوشنی $\GM$ را پارامتری می‌کنند، باید \lr{G}-هدایت‌پذیر باشند (معادله~\eqref{eq:kernel_constraint_rhohom}).
تبدیلات واقعی میدان کرنل و کانولوشن‌های $\GM$ در بخش~\ref{sec:KFTs_GM-conv_global} معرفی می‌شوند.
تعریف سراسری آن‌ها با جایگزینی عبارات مختصاتی محلی از بخش~\ref{sec:gauge_conv_main} با همتاهای سراسری و مستقل از مختصاتشان هدایت می‌شود.
همانطور که در بخش~\ref{sec:KFTs_GM-conv_local} نشان داده شده است، این تعاریف مستقل از مختصات در بدیهی‌سازی‌های محلی به عبارات مختصاتی از بخش~\ref{sec:gauge_conv_main} کاهش می‌یابند.