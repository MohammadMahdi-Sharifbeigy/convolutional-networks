%!TEX root=../GaugeCNNTheory.tex


\subsection{\lr{CNN}های اقلیدسی در مقالات}
\label{sec:euclidean_literature}

تمام مدل‌های موجود در ردیف‌های (۱-۲۶) جدول~\ref{tab:network_instantiations}، کانولوشن‌های $\GM$ هموردا نسبت به $\Aff(G)$ در فضاهای اقلیدسی~$\Euc_d$ هستند که در بخش‌های قبل مورد بحث قرار گرفتند.
آنها در ابعاد~$d$ فضای اقلیدسی، گروه ساختار~$G$ و در نتیجه گروه تقارن سراسری $\Aff(G)$، نمایش‌های گروهی یا انواع میدان~$\rho$ و انتخاب‌های گسسته‌سازی متفاوت هستند.
این بخش به طور خلاصه مدل‌ها را با گروه‌بندی آنها بر اساس انواع میدانشان به مدل‌های نمایش تحویل‌ناپذیر (irrep)، مدل‌های نمایش منظم (متناظر با کانولوشن‌های گروهی) و انواع آنها، مدل‌های نمایش خارج‌قسمتی و غیره مورد بحث قرار می‌دهد.
\lr{CNN}های متعارف، که ابتدا آنها را مرور می‌کنیم، کمی از این طبقه‌بندی خارج می‌شوند زیرا گروه ساختاری بدیهی آنها منجر به میدان‌های ویژگی و کرنل‌هایی بدون هیچ‌گونه محدودیت تقارنی می‌شود.


ردیف (۱) کانولوشن‌های $\GM$ اقلیدسی را روی $\{e\}$-ساختارهای ناوردای انتقالی فهرست می‌کند که در شکل~\ref{fig:G_structure_R2_1} به تصویر کشیده شده است.
به دلیل بدیهی بودن گروه ساختار $G=\{e\}$، هیچ تبدیل پیمانه‌ای (غیربدیهی) وجود ندارد و تنها انتخاب ممکن برای نمایش گروه، نمایش بدیهی است.
بنابراین، محدودیت راهبری‌پذیری-$G$ بدیهی می‌شود، به طوری که فضای کرنل‌های کانولوشن قابل قبول، نامحدود باقی می‌ماند.
هنگامی که از طریق یک چارت به $\R^d$ پول‌بک شود، کانولوشن $\GM$ طبق قضیه~\ref{thm:Euclidean_GM_conv_is_conventional_conv} به یک کانولوشن (همبستگی) متعارف تبدیل می‌شود.
قضیه~\ref{thm:affine_equivariance_Euclidean_GM_conv} هموردایی انتقالی آن را تأیید می‌کند.
بنابراین، دیده می‌شود که این مدل‌ها با شبکه‌های کانولوشنی متعارف توسط~\citet{LeCun1990CNNs} مطابقت دارند.


تمام مدل‌های اقلیدسی دیگر در ردیف‌های (۲-۲۶) گروه‌های ساختاری غیربدیهی را در نظر می‌گیرند.
می‌توان آنها را به عنوان کانولوشن‌های متعارف روی~$\R^d$ با محدودیت اضافی بر روی کرنل‌ها برای $G$-راهبری‌پذیر بودن در نظر گرفت، که هموردایی $\Aff(G)$ آنها را تضمین می‌کند.


\paragraph{ویژگی‌های نمایش تحویل‌ناپذیر \lr{(Irrep)}:}
شبکه‌های موجود در ردیف‌های (۴، ۹، ۱۰، ۱۷، ۲۳) و (۲۶) بر روی میدان‌های ویژگی عمل می‌کنند که مطابق با \emph{نمایش‌های تحویل‌ناپذیر} \lr{(irreps)} از~$G$ تبدیل می‌شوند.
برای $G=\SO2$ که در ردیف (۴) فهرست شده است، این امر به شبکه‌های به اصطلاح هارمونیک منجر می‌شود~\cite{Worrall2017-HNET,Weiler2019_E2CNN}.
این نام از این واقعیت نشأت می‌گیرد که محدودیت کرنل در این حالت فقط به هارمونیک‌های دایره‌ای با مکان‌یابی طیفی فرکانس $m-n$ اجازه می‌دهد، هنگام نگاشت بین میدان‌هایی که مطابق با نمایش‌های تحویل‌ناپذیر مرتبه~$n$ در ورودی و مرتبه~$m$ در خروجی تبدیل می‌شوند.%
\footnote{
	هنگام در نظر گرفتن نمایش‌های تحویل‌ناپذیر \emph{مختلط} از~$\SO2$، فقط فرکانس $m-n$ مجاز است.
	برای نمایش‌های تحویل‌ناپذیر \emph{حقیقی}، محدودیت به فرکانس‌های $|m-n|$ و $m+n$ اجازه می‌دهد.
	جزئیات بیشتر را می‌توان در پیوست‌های F.5 از~\cite{Weiler2019_E2CNN} و E.1 و E.2 از~\cite{lang2020WignerEckart} یافت.
}
محدودیت بازتابی اضافی برای $G=\O2$ که در ردیف (۱۰) فهرست شده است، قوانین انتخاب پاریته را اضافه می‌کند که فاز هارمونیک‌های دایره‌ای را ثابت می‌کند و نیمی از درجات آزادی را در مقایسه با حالت $G=\SO2$ سرکوب می‌کند~\cite{Weiler2019_E2CNN}.
مدل‌های \cite{3d_steerableCNNs,Thomas2018-TFN,miller2020relevance,Kondor2018-NBN,anderson2019cormorant} در ردیف (۱۷) نمایش‌های تحویل‌ناپذیر $G=\SO3$ را در نظر می‌گیرند و بنابراین می‌توان آنها را به عنوان مشابه مدل‌های ردیف (۴) در سه بعد در نظر گرفت.
فضای کرنل‌های معتبر برای نگاشت بین میدان‌هایی که مطابق با نمایش‌های تحویل‌ناپذیر (ماتریس‌های D ویگنر) از مرتبه‌های $n$ و $m$ تبدیل می‌شوند، در اینجا توسط تمام هارمونیک‌های کروی از $2(\min(m,n)+1)$ مرتبه $j$ با $|m-n| \leq j \leq m+n$ تولید می‌شود.
همانطور که در~\cite{lang2020WignerEckart} اثبات شده است، این امر به هر گروه ساختاری فشرده تعمیم می‌یابد، و فرکانس‌های مجاز هارمونیک‌ها توسط ضرایب متناظر کلبش-گوردون که با $m، n$ و~$j$ برچسب‌گذاری شده‌اند، تعیین می‌شوند.
نوعی از این رویکرد در ردیف (۲۳) فهرست شده است \cite{poulenard2019effective}.
یک کانولوشن از یک میدان اسکالر ورودی با هارمونیک‌های کروی، میدان‌های نمایش تحویل‌ناپذیر از مرتبه متناظر را تولید می‌کند.
با این حال، به جای پردازش بیشتر این ویژگی‌های نمایش تحویل‌ناپذیر از طریق کانولوشن‌ها، نویسندگان نرم آنها را محاسبه می‌کنند.
این کار منجر به میدان‌های اسکالر می‌شود که در لایه بعدی به همین روش پردازش می‌شوند.
مدل~\cite{shutty2020learning} در ردیف (۲۶) متریک استاندارد اقلیدسی را فرض نمی‌کند بلکه متریک مینکوفسکی را در نظر می‌گیرد.
گروه ساختاری آن گروه لورنتس $G=\SO{d-1,1}$ و گروه تقارن سراسری، گروه پوانکاره است.
علاوه بر ساختن شبکه هموردا، نویسندگان الگوریتمی را برای محاسبه نمایش‌های تحویل‌ناپذیر گروه‌های لی از ثوابت ساختاری جبر لی آنها پیشنهاد می‌کنند.


یک مورد خاص از نمایش‌های تحویل‌ناپذیر، \emph{نمایش‌های بدیهی} هستند که بردارهای ویژگی $G$-ناوردا (اسکالرها) را توصیف می‌کنند.
به دلیل ناوردایی، چنین ویژگی‌هایی نمی‌توانند تفاوت بین هیچ الگویی را در ژست‌های مرتبط با $G$ کدگذاری کنند.
محدودیت روی کرنل‌هایی که بین میدان‌های اسکالر نگاشت انجام می‌دهند، به $K(g\mkern1mu \mathscr{v}) = K(\mathscr{v})$ برای هر $\mathscr{v} \in \R^d$ و هر $g\in G$ تبدیل می‌شود، که کرنل‌هایی را که (در هر کانال به طور جداگانه) تحت عمل $G$ ناوردا هستند، تحمیل می‌کند.
این برای بازتاب‌های $G=\Flip$ در ورودی بالا سمت چپ جدول~\ref{tab:reflection_steerable_kernels} به تصویر کشیده شده است.
تفسیر شبکه پیکسلی یک تصویر به عنوان یک گراف و اعمال یک کانولوشن گراف استاندارد بر روی آن، متناظر با یک کانولوشن با راهبری‌پذیری بدیهی با کرنل‌های $\O2$-ناوردا است زیرا کانولوشن‌های گراف استاندارد کرنل‌های همسانگرد را اعمال می‌کنند~\cite{khasanova2018isometric}.

یک مزیت ویژگی‌های نمایش تحویل‌ناپذیر از دیدگاه عملی، ابعاد پایین و در نتیجه مصرف حافظه کمتر برای هر میدان ویژگی است.
با این حال، نتایج تجربی نشان می‌دهد که کانولوشن‌های راهبری‌پذیر مبتنی بر میدان نمایش تحویل‌ناپذیر معمولاً عملکرد پایین‌تری نسبت به انواع دیگر میدان‌ها، به عنوان مثال آنهایی که بر اساس نمایش‌های منظم هستند، کسب می‌کنند.
این گزاره در ارزیابی ما از کانولوشن‌های موبیوس در بخش~\ref{sec:mobius_evaluation} و بنچمارک کانولوشن‌های اقلیدسی هموردای ایزومتری در~\cite{Weiler2019_E2CNN} منعکس شده است.



\paragraph{ویژگی‌های منظم و کانولوشن‌های گروهی:}
احتمالاً برجسته‌ترین دسته از نمایش‌های گروهی در یادگیری عمیق هموردا، \emph{نمایش‌های منظم} از گروه‌های ساختاری هستند.
نمایش‌های منظم بر روی یک فضای مناسب%
\footnote{
	به عنوان مثال، برای گروه‌های توپولوژیکی، توابع معمولاً باید پیوسته باشند.
	برای گروه‌های فشرده محلی معمولاً فضای $L^2(G)$ از توابع انتگرال‌پذیر مربع روی $G$ در نظر گرفته می‌شود.
}
از توابع $\digamma: G \to \R$ با انتقال آنها عمل می‌کنند، یعنی
$\big[ \rho_\textup{reg} (\tilde{g}) \digamma \big](g) = \digamma\big( \tilde{g}^{-1}g \big)$.%
\footnote{
	نمایش‌های منظم روی یک میدان $\mathbb{K}$ متفاوت از اعداد حقیقی، مقادیری در این میدان می‌گیرند، یعنی $\digamma: G \to \mathbb{K}$.
}
برای گروه‌های متناهی، این به معنای میدان‌های ویژگی با تعداد کانال‌های $c = |G|$ است که توسط مرتبه گروه داده می‌شود.
از آنجا که گروه‌های غیرمتناهی به معنای نمایش‌های منظم غیرمتناهی هستند، ویژگی‌های مربوطه در عمل گسسته‌سازی می‌شوند، که عمدتاً با در نظر گرفتن یک زیرگروه متناهی از گروه ساختاری انجام می‌شود.
از آنجا که میدان‌های ویژگی منظم $f \in \Gamma(\A)$ یک تابع $f^A(p): G \to \R$ را به هر نقطه $p\in M$ اختصاص می‌دهند (هنگامی که نسبت به هر پیمانه $A$ در $p$ بیان می‌شوند)، آنها با توابع با مقادیر حقیقی $\tilde{f}: \GM \to \R$ روی $G$-ساختار~$\GM$ معادل هستند.%
\footnote{
	قضیه~\ref{thm:regular_field_scalar_GM} در پیوست~\ref{apx:regular_field_scalar_GM} این ایزومورفیسم
	$C^\infty(\GM)\ \cong\ \Gamma(\A_{\rho_\textup{reg}})$
	را برای حالت عملی مرتبط که $G$ یک گروه متناهی است، اثبات می‌کند.
}
برای حالتی که $\GM$ توسط یک اطلس $\Aff(G)$ القا می‌شود، این علاوه بر این با توابع با مقادیر حقیقی $\accentset{\approx}{f}: \Aff(G) \to \R$ روی $\Aff(G) \cong \GM$ (در امتداد ایزومورفیسم در معادله~\eqref{eq:principal_bundle_iso_AffG_GM}) معادل است.
نگاشت‌های خطی هموردا بین توابع روی گروه $\Aff(G)$ \emph{کانولوشن‌های گروهی} هستند (به معادله~\eqref{eq:group_conv_def} در بخش~\ref{apx:homogeneous_conv} و بخش~۷.۱۱ در~\cite{gallier2019harmonicRepr} مراجعه کنید)، که به این معنی است که \lr{CNN}های مبتنی بر کانولوشن گروه افاین توسط چارچوب ما پوشش داده می‌شوند~\cite{Cohen2016-GCNN,Kondor2018-GENERAL,bekkers2020bspline}.


کانولوشن‌های گروهی $\Aff(G)$ در جدول~\ref{tab:network_instantiations} در ردیف‌های (۲،۳،۵،۱۱،۱۵،۱۹،۲۱،۲۴) و (۲۵) فهرست شده‌اند.
از آنجا که این مدل‌ها معمولاً تصاویر خاکستری یا اسکالر را پردازش می‌کنند، آنها یک کانولوشن اولیه از میدان‌های اسکالر به میدان‌های منظم اعمال می‌کنند، و به دنبال آن کانولوشن‌های گروهی، یعنی کانولوشن‌ها از میدان‌های منظم به منظم را انجام می‌دهند.
از آنجا که نمایش‌های منظم، نمایش‌های جایگشتی هستند، آنها معمولاً غیرخطی‌های نقطه‌ای مانند \lr{ReLU} را به هر یک از کانال‌های میدان به طور جداگانه اعمال می‌کنند.%
\footnote{
	از آنجا که عمل نگاشت‌های غیرخطی به پایه انتخاب شده بستگی دارد، این همان چیزی است که واقعاً میدان‌های ویژگی منظم (یا هر میدان غیر تحویل‌ناپذیر دیگر) را از تجزیه آنها به میدان‌های نمایش تحویل‌ناپذیر متمایز می‌کند؛ به پاورقی~\ref{footnote:feature_field_irrep_decomposition} و بحث در بخش~\ref{sec:mobius_representations} مراجعه کنید.
}
\lr{CNN} هموردای بازتابی روی $\Euc_2$ از~\cite{Weiler2019_E2CNN} در 
ردیف (۳)
کرنل‌های $\Flip$-راهبری‌پذیر را اعمال می‌کند، همانطور که در بخش~\ref{sec:mobius_conv} استخراج شده و در ورودی پایین سمت راست جدول~\ref{tab:reflection_steerable_kernels} به تصویر کشیده شده است.
از آنجا که گروه بازتاب متناهی با مرتبه $|\Flip| = 2$ است، میدان‌های ویژگی منظم دارای دو کانال هستند که هر کدام به یکی از دو جهت‌گیری چارچوب از $\Flip$-ساختار در شکل~\ref{fig:G_structure_R2_3} مرتبط است.
مدل حاصل به صورت سراسری $\Trans_2 \rtimes \Flip = \Aff(\Flip)$-هموردا است.
%
برای ساخت کانولوشن‌های گروهی $\SE2 = \Aff(\SO2)$-هموردا، در تئوری باید $\SO2$-ساختار را در شکل~\ref{fig:G_structure_R2_2} با میدان‌های ویژگی که مطابق با نمایش منظم $\SO2$ تبدیل می‌شوند، در نظر گرفت.
در عمل، اکثر مدل‌های 
ردیف (۵)
از جدول~\ref{tab:network_instantiations}
این را از طریق نمایش‌های منظم گروه‌های دوری $\CN \leq \SO2$ که زیرگروه‌های متناهی از دوران‌های گسسته با مضرب‌هایی از $2\pi/N$ هستند، تقریب می‌زنند.
از آنجا که مرتبه این گروه‌ها $|\!\CN\!|=N$ است، میدان‌های ویژگی متناظر $N$-بعدی هستند.
در حالی که عملکرد مدل در ابتدا به طور قابل توجهی با $N$ افزایش می‌یابد، به طور تجربی مشخص شده است که در حدود ۸ تا ۱۲ جهت نمونه‌برداری شده به اشباع می‌رسد~\cite{Weiler2018SFCNN,Weiler2019_E2CNN,bekkers2020bspline}.
برای درک شهودی از فضاهای کرنل‌های $\CN$-راهبری‌پذیر به تصاویر در~\cite{Weiler2018SFCNN,bekkers2018roto,bekkers2020bspline} مراجعه می‌کنیم.
%
کانولوشن‌های گروهی $\E2 = \Aff(\O2)$-هموردا در 
ردیف (۱۱)
به طور مشابه از طریق زیرگروه‌های دووجهی $\DN \leq \O2$ که از $N$ دوران، هر کدام در دو بازتاب، تشکیل شده‌اند، تقریب زده می‌شوند.
میدان‌های ویژگی در این حالت $|\!\DN\!| = 2N$-بعدی هستند.
%
هموردایی همزمان تحت انتقال و مقیاس‌بندی توسط کانولوشن‌های گروهی $\Trans_d \rtimes \Scale = \Aff(\Scale)$ در
ردیف‌های (۲) و (۱۵)
به دست می‌آید.
گروه مقیاس‌بندی در اینجا معمولاً گسسته‌سازی می‌شود.
از آنجا که این کار هنوز به یک مرتبه گروه (شمارا) نامتناهی منجر می‌شود، پیاده‌سازی‌ها برش‌هایی را معرفی می‌کنند، یعنی مقیاس‌های حداقل و حداکثر همانطور که توسط چارچوب‌ها در شکل~\ref{fig:G_structure_R2_4} نشان داده شده است.
توجه داشته باشید که این امر به اثرات مرزی مشابه کانولوشن‌های متعارف در لبه یک تصویر منجر می‌شود.
%
مدل‌های 
ردیف‌های (۱۹) و (۲۱)
نسبت به انتقال‌ها، دوران‌ها و برای مورد دوم، بازتاب‌ها در فضاهای اقلیدسی سه‌بعدی~$\Euc_3$ هموردا هستند.
در حالی که \citet{finzi2020generalizing} یک گسسته‌سازی مونت کارلو از نمایش منظم را انتخاب می‌کنند،
مدل‌های \cite{Worrall2018-CUBENET,winkels3DGCNNsPulmonary2018} بر اساس زیرگروه‌های گسسته مختلف از $\SO3$ یا $\O3$ هستند.
یک محدودیت فعلی مدل‌های هموردای دورانی و بازتابی مبتنی بر کانولوشن گروهی در سه بعد، نیاز بالای آنها به حافظه و محاسبات است.
به عنوان مثال، گروه تقارن مکعب، که هنوز وضوح نسبتاً درشتی از دوران‌ها با $\pi/2$ دارد، در حال حاضر از ۴۸ عضو گروه تشکیل شده است که به معنای میدان‌های ویژگی ۴۸ بعدی در فضای سه‌بعدی است.
از سوی دیگر، تعداد زیاد تقارن‌ها نشان‌دهنده کارایی داده بسیار بهبود یافته چنین مدل‌هایی است:
نویسندگان \cite{winkels3DGCNNsPulmonary2018} عملکرد یکسانی را از یک مدل هموردا در مقایسه با یک شبکه غیرهموردا (قابل راهبری با $\{e\}$) گزارش می‌دهند علی‌رغم اینکه روی یک مجموعه داده ۱۰ برابر کوچکتر آموزش دیده‌اند.
%
مدل‌های 
ردیف‌های (۲۴) و (۲۵)
روی $\Euc_3$ کانوالو می‌کنند، با این حال، آنها گروه‌های ساختاری دوری و دووجهی $\operatorname{C}_4$ و $\operatorname{D}_4$ را در نظر می‌گیرند، یعنی دوران‌ها و بازتاب‌های صفحه‌ای حول محور $z$ (که به این ترتیب تعریف شده است).
بنابراین کرنل‌های راهبری‌پذیر آنها مشابه کرنل‌های مدل‌های 
ردیف‌های (۵) و (۱۱)
هستند اما علاوه بر این در یک جهت جدید $z$ گسترش می‌یابند.


\paragraph{تجمیع از منظم به اسکالر و بردار:}
یک تنوع از شبکه‌های کانولوشنی گروهی، مدل‌های 
ردیف‌های (۷،۸،۱۳،۱۶) و (۲۰)
هستند که با \lr{regular}$\xrightarrow{\textup{pool}}$\lr{trivial} و \lr{regular}$\xrightarrow{\textup{pool}}$\lr{vector} برچسب‌گذاری شده‌اند.
پس از اعمال یک کانولوشن به میدان‌های ویژگی منظم، آنها یک عملیات \emph{تجمیع} $\operatorname{max}$ را روی کانال‌ها انجام می‌دهند که منجر به میدان‌های اسکالر (بدیهی) می‌شود \cite{Cohen2016-GCNN,marcos2016learning,Weiler2019_E2CNN,ghosh2019scale,andrearczyk2019exploring}، یا یک تجمیع $\operatorname{max}$ به همراه یک $\operatorname{argmax}$، که از آن می‌توان میدان‌های برداری را محاسبه کرد~\cite{Marcos2017-VFN,Weiler2019_E2CNN}.
کانولوشن‌های بعدی از میدان‌های اسکالر یا برداری حاصل به میدان‌های ویژگی منظم نگاشت می‌یابند.
از آنجا که عملیات تجمیع تعداد کانال‌ها را به طور قابل توجهی از $|G|$ به ترتیب به ۱ یا $d$ کاهش می‌دهد، مدل‌ها از نظر حافظه و محاسبات کارآمدتر از کانولوشن‌های گروهی متعارف می‌شوند.
نقطه ضعف این است که تجمیع با از دست دادن اطلاعات همراه است، که به طور تجربی مشخص شده است که عملکرد مدل را کاهش می‌دهد~\cite{Weiler2019_E2CNN}.



\paragraph{ویژگی‌های خارج‌قسمتی:}
ردیف‌های (۶،۱۲) و (۲۲) مدل‌هایی را فهرست می‌کنند که میدان‌های ویژگی آنها مطابق با \emph{نمایش‌های خارج‌قسمتی} از گروه ساختاری تبدیل می‌شوند، که نمایش‌های جایگشتی مشابه نمایش‌های منظم هستند.
با توجه به یک زیرگروه~$\widehat{G}$ از~$G$، نمایش خارج‌قسمتی متناظر بر روی توابع اسکالر $\digamma: G/\widehat{G} \to \R$ در فضای خارج‌قسمتی $G/\widehat{G}$ از طریق انتقال عمل می‌کند، یعنی
$\big[\rho_\textup{quot}^{G/\widehat{G}} (\tilde{g})\mkern2mu \digamma\big] (g\mkern1mu \widehat{G}) = \digamma(\tilde{g}^{-1}g \mkern2mu \widehat{G})$.
بنابراین ابعاد میدان‌های ویژگی با اندیس ${|G:\mkern-2mu\widehat{G}|}$ از $\widehat{G}$ در $G$ داده می‌شود که برای گروه‌های متناهی برابر با~$|G|/|\widehat{G}|$ است.
میدان‌های ویژگی که تحت نمایش‌های خارج‌قسمتی تبدیل می‌شوند را می‌توان به عنوان میدان‌های ویژگی منظم با تقارن محدود در نظر گرفت که مجبورند مقدار یکسانی را روی تمام اعضای گروه در یک هم‌دسته $g\widehat{G}$ از $\widehat{G}$ در~$G$ بگیرند.
یک مثال خاص، نمایش‌های ردیف (۲۲) هستند که با خارج‌ قسمت $\O3/\O2 \cong S^2$ مرتبط هستند.
به جای اجازه دادن به کرنل‌های کانولوشن دلخواه، محدودیت کرنل در اینجا به کرنل‌هایی منجر می‌شود که تحت دوران حول محور $z$ ناوردا هستند؛ به تصاویر در~\cite{janssen2018design} مراجعه کنید.
جزئیات بیشتر و یک شهود گرافیکی در مورد میدان‌های ویژگی مبتنی بر نمایش خارج‌قسمتی را می‌توان در پیوست~C از~\cite{Weiler2019_E2CNN} یافت.
نظریه پیشنهادی در \cite{Kondor2018-GENERAL} میدان‌های خارج‌قسمتی را از دیدگاه جایگزین کانولوشن‌های گروهی روی فضاهای خارج‌قسمتی راست پوشش می‌دهد.


\paragraph{نمایش‌های القایی:}
یک تعمیم از نمایش‌های منظم و خارج‌قسمتی، نمایش‌های القایی مانند \emph{نمایش‌های تحویل‌ناپذیر القایی $\SO2$} در ردیف (۱۴)
از جدول~\ref{tab:network_instantiations} هستند.
با توجه به هر نمایش تحویل‌ناپذیر $\SO2$ به صورت $\rho: \SO2 \to \GL{n}$، نمایش القایی $\Ind_{\SO2}^{\O2}\rho: \O2 \to \GL{c}$ از~$\O2$ با $c = n \mkern-2mu\cdot\! {|\O2:\SO2|} = 2n$
به روش زیر عمل می‌کند:
بازتاب‌ها دو زیرفضای $n$-بعدی و متعامد از $\R^{2n}$ را که متناظر با دو هم‌دسته در $\O2/\SO2$ هستند، جایگشت می‌کنند در حالی که دوران‌ها بر روی زیرفضاهای جداگانه از طریق~$\rho$ عمل می‌کنند.
برای $\rho$ که نمایش بدیهی از $\SO2$ است، این کار نمایش‌های خارج‌قسمتی را همانطور که در بالا بحث شد، بازیابی می‌کند.
در مقایسه با میدان‌های ویژگی نمایش تحویل‌ناپذیر $\O2$ ، میدان‌های نمایش تحویل‌ناپذیر القایی $\SO2$ عملکرد به طور قابل توجهی بهبود یافته‌ای را نشان می‌دهند.
توصیف دقیق‌تر و ارزیابی تجربی این انواع میدان را می‌توان در~\cite{Weiler2019_E2CNN} یافت.


آخرین نوع نمایش فهرست شده در جدول~\ref{tab:network_instantiations}، نمایش کواترنیونی از دوران‌های سه‌بعدی در ردیف (۱۸) است~\cite{zhang2019quaternion}.
این از نمایش معمول دوران‌ها از طریق کواترنیون‌ها استفاده می‌کند، که بر شناسایی کواترنیون‌های واحد با $\operatorname{SU}(2)$ و وجود یک همومورفیسم گروهی پوشا از $\operatorname{SU}(2)$ به~$\SO3$ تکیه دارد.
توجه داشته باشید که نمایش کواترنیونی در واقع یک نمایش تصویری از $\SO3$ است.


در حالی که نظریه ما بر روی فضاهای اقلیدسی پیوسته فرمول‌بندی شده است، پیاده‌سازی‌ها میدان‌های ویژگی را بر روی زیرمجموعه‌های گسسته نمونه‌برداری می‌کنند.
رایج‌ترین گسسته‌سازی $\Euc_d$ بر حسب شبکه پیکسلی $\Z^d$ است.
یک جایگزین، شبکه‌های صفحه‌ای شش‌ضلعی روی $\Euc_2$ است که توسط~\citet{Hoogeboom2018-HEX} بررسی شده است.
اگر چنین شبکه‌های پیکسلی منظمی انتخاب شوند، یک پایه از کرنل‌های $G$-راهبری‌پذیر را می‌توان از پیش محاسبه کرد و بر روی این شبکه نمونه‌برداری کرد.
داده‌هایی مانند رویدادها در فضازمان~\cite{shutty2020learning} یا مولکول‌ها در $\R^3$ \cite{Thomas2018-TFN,Kondor2018-NBN,anderson2019cormorant,miller2020relevance} در عوض معمولاً با ابرهای نقطه نامنظم نمایش داده می‌شوند.
در این حالت، کرنل‌ها باید به صورت تحلیلی داده شوند، که امکان نمونه‌برداری آنلاین آنها را در طول پاس مستقیم فراهم می‌کند.


در نهایت، می‌خواهیم اشاره کنیم که مدل‌های \emph{سراسری} $\Aff(G)$-هموردا وجود دارند که \emph{به صورت محلی} $G$-هموردا \emph{نیستند}.
یک مثال، تجمیع ناوردای-تبدیل (TI-pooling)~\cite{Laptev_2016_CVPR} است، که مجموعه‌ای از میدان‌های ویژگی تبدیل‌شده سراسری را به یک \lr{CNN} متعارف می‌دهد و در نهایت ویژگی‌های حاصل را روی این تبدیلات تجمیع می‌کند، که منجر به یک توصیفگر ناوردا می‌شود.