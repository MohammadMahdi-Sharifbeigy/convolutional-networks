%!TEX root=../GaugeCNNTheory.tex


\subsection%
[ساختارهای \lr{\textit{G}} \lr{\textit{GM}} و کلاف‌های برداری ویژگی وابسته \texorpdfstring{$\mathcal{A}$}{A}]%
{ساختارهای \lr{\textit{G}} \lr{\textit{GM}} و کلاف‌های برداری ویژگی وابسته $\mathcal{A}$}
\label{sec:G_associated_bundles}


اکنون ما $G$-ساختارها $\GM$ را به عنوان زیرمجموعه‌های متمایزی از چارچوب‌ها در $\FM$ معرفی می‌کنیم که ساختار هندسی اضافی روی $M$ را کدگذاری می‌کنند که باید توسط \CNN های مستقل از مختصات رعایت شود.
کلاف مماس از طریق یک ساختار کلاف وابسته مشابه با بخش قبل، به عنوان یک $G$-کلاف وابسته بازمعرفی می‌شود.
این رویکرد را می‌توان برای ساخت هر کلاف $G$-وابسته دیگر تعمیم داد، که ما از آن برای تعریف کلاف‌های برداری ویژگی $\A$ استفاده می‌کنیم.
تمام چنین کلاف‌های ساخته‌شده‌ای به یکدیگر وابسته‌اند، یعنی فقط در تار خود $F$ تفاوت دارند اما فضای پایه $M$، گروه ساختار $G$ و توابع گذار $g^{BA}$ بین همسایگی‌های بدیهی‌ساز را مشترکاً دارند.
بدیهی‌سازی‌های محلی کلاف‌ها و تبدیلات پیمانه متقابل آنها در بخش بعدی~\ref{sec:bundle_trivializations} به تفصیل مورد بحث قرار می‌گیرد.


\paragraph{\lr{\textit{G}}-ساختارها \lr{\textit{GM}}:}
همانطور که در بخش~\ref{sec:21_main} و جدول~\ref{tab:G_structures} بحث شد، اغلب ممکن است با یک \emph{زیرمجموعه متمایز از چارچوب‌های مرجع} کار کرد که توسط عمل یک \emph{گروه ساختار کاهش‌یافته} $G \leq \GL{d}$ به هم مرتبط هستند.
این موضوع با بحث در مورد چند مثال قبل از رسیدن به یک تعریف فنی در ادامه، به بهترین وجه قابل درک است.
به عنوان مثال، یک محدودیت به چارچوب‌های متعامد
\begin{align}
	\OpM\ :=\ \pig\{ \big[e_{1},\dots,e_{d}\big]\, \pig|\ 
	\{e_{1},\dots,e_{d}\}\ \text{یک پایه \emph{متعامد} از } \TpM\ \text{نسبت به}\ \eta \text{ است} \pig\}\ \cong\ \OO{d}
\end{align}
منجر به یک زیرکلاف اصلی $\OM$ از $\FM$ با گروه ساختار $\OO{d}$ می‌شود.
توجه داشته باشید که تعامد چارچوب‌های مرجع توسط متریک $\eta$ روی $M$ قضاوت می‌شود -- بنابراین انتخاب‌های مختلف متریک روی یک خمینه متناظر با زیرمجموعه‌های مختلفی از چارچوب‌های مرجع مرجح برای همان گروه ساختار $\OO{d}$ است.
به عنوان مثال دوم، یک انتخاب جهت‌گیری روی یک خمینه جهت‌پذیر را در نظر بگیرید، که امکان مشخص کردن یک مفهوم مرجح از چارچوب‌ها را فراهم می‌کند%
\footnote{
	برعکس، خمینه‌های غیرجهت‌پذیر اجازه کاهش گروه ساختار به $\operatorname{GL}^{\!+}\!(d)$ را نمی‌دهند.
}
\begin{align}
	\operatorname{GL}^{\!+}_p\!M\ :=\ \pig\{ \big[e_{1},\dots,e_{d}\big]\, \pig|\ \{e_{1},\dots,e_{d}\}\ \text{یک پایه با \emph{جهت‌گیری مثبت} از } \TpM \text{ است} \pig\}\ \cong\ \operatorname{GL}^{\!+}\!(d)
\end{align}
و یک زیرکلاف اصلی متناظر $\operatorname{GL}^{\!+}\!(d)M$ از $\FM$ با گروه ساختار $\operatorname{GL}^{\!+}\!(d)$.
باز هم، دو انتخاب مختلف از جهت‌گیری‌ها متناظر با دو انتخاب مختلف از زیرکلاف‌های چارچوب‌های جهت‌دار متناسب است.
ترکیب هر دو شرط برای تعامد و راست‌گردی چارچوب‌ها منجر به یک $\SO{d}$-ساختار با تارهای زیر می‌شود:
\begin{align}
	\SOpM\ :=\ \pig\{ \big[e_{1},\dots,e_{d}\big]\, \pig|\ 
	\{e_{1},\dots,e_{d}\}\ \text{یک پایه با \emph{جهت‌گیری مثبت} و \emph{متعامد} از } \TpM \text{ است} \pig\}\ \cong\ \SO{d} \,,
\end{align}
می‌توان شکل~\ref{fig:frame_bundle} را به عنوان نمایش‌دهنده یک $\SO{2}$-ساختار در نظر گرفت زیرا تنها چارچوب‌های راست‌گرد و متعامد نشان داده شده‌اند (در این صورت تار نوعی $\GL{d}$ باید با $\SO2$ برچسب‌گذاری شود).
انتخاب‌های مختلف از $\SO{d}$-ساختارها یا متناظر با یک دست‌سانی مخالف از چارچوب‌ها، با پایبندی به همان مفهوم تعامد، یا متناظر با یک انتخاب متفاوت از متریک (یا هر دو) است.
دقیقا همین الگو برای فرم‌های حجم $\omega$ (روی خمینه‌های جهت‌پذیر $M$) تکرار می‌شود:
آنها امکان مشخص کردن یک مفهوم مرجح از چارچوب‌ها را فراهم می‌کنند
\begin{align}
	\operatorname{SL}_p\!M\ :=\ \pig\{ \big[e_{1},\dots,e_{d}\big]\, \pig|\ \{e_{1},\dots,e_{d}\}\ \text{یک پایه از }\, \TpM\ \text{با \emph{حجم واحد} نسبت به}\ \omega \text{ است} \pig\}\ \cong\ \operatorname{SL}(d)
\end{align}
و در نتیجه زیرکلاف‌های اصلی $\operatorname{SL}\!M$ از $\FM$ با گروه ساختار $\operatorname{SL}(d)$.
مجموعه خاص چارچوب‌هایی که مرجح هستند در اینجا به انتخاب خاص فرم حجم بستگی دارد.
به عنوان آخرین مثال، $\{e\}$-ساختارها را در نظر بگیرید که متناظر با یک گروه ساختار بدیهی $G=\{e\}$ هستند و بنابراین در هر نقطه~$p$ از یک چارچوب واحد تشکیل شده‌اند.
طبق تعریف، $\{e\}$-ساختارها با میدان‌های چارچوب (هموار) سراسری $\sigma \in \Gamma(\FM)$ معادل هستند:
\begin{align}
	\epM\ :=\ \pig\{ \big[e_{1},\dots,e_{d}\big] = \sigma(p) \pig\}\ \cong\ \{e\}
\end{align}
بنابراین آنها فقط روی خمینه‌های بدیهی وجود دارند.
شکل‌های~\ref{fig:frame_field_automorphism_1} و~\ref{fig:frame_field_automorphism_2} دو انتخاب مختلف از $\{e\}$-ساختارها~$\eM$ روی~$M=\R^2$ را به تصویر می‌کشند.


تمام این مثال‌ها انتخاب‌های خاصی از $G$\emph{-ساختارها} $\GM$ روی $M$ را نشان می‌دهند.
به طور کلی، یک $G$-ساختار روی~$M$ یک \emph{زیرکلاف} اصلی $G$ از $\FM$ است، یعنی یک انتخاب "هموار متغیر" از \emph{زیرمجموعه‌های} $\GpM \subseteq \FpM$ که نسبت به $\lhd$ برای هر $p\in M$ ، $G$-تورسورهای راست هستند \cite{sternberg1999lectures,piccione2006theory,crainic2013GStructuresExamples}.%
\footnote{\label{footnote:GpM_G_orbit_in_FpM}
	از آنجا که $\FpM$ یک $\GL{d}$-تورسور راست است، هر $G$-مدار $\GpM$ در $\FpM$ به طور خودکار تضمین می‌شود که یک $G$-تورسور راست باشد.
}
همواری در اینجا می‌تواند با این شرط فرمول‌بندی شود که در اطراف هر چارچوب $[e_i]_{i=1}^d \in \GpM$ یک همسایگی $U$ از $p$ وجود داشته باشد که روی آن یک برش هموار $\sigma: U \mapsto \piGM^{-1}(U) \subseteq GM$ با $\sigma(p) = [e_i]_{i=1}^d$ وجود داشته باشد.
تصویر
\begin{align}
	\piGM :=\, \piFM\mkern-2mu \big|_{\scalebox{.6}{$\GM$}}:\ \ \GM \to M
\end{align}
از $\GM$ در اینجا به سادگی با محدود کردن نگاشت تصویر $\FM$ به $\GM$ داده می‌شود.
به همراه محدودیت
\begin{align}\label{eq:rightaction_GM}
	\lhd:\ \GM\times G \to \GM, \quad
	\big( [e_i]_{i=1}^d,\ g \big)
	\ \mapsto\ 
	[e_i]_{i=1}^d\! \lhd g \ :=\ 
	\left[ \sum\nolimits_j e_j\, g_{ji} \right]_{i=1}^d
\end{align}
از عمل راست $\GL{d}$ روی $\FM$ در معادله~\eqref{eq:rightaction_FM} به یک عمل $G \leq \GL{d}$ روی $\GM \subseteq \FM$ ، این $G$-ساختار را به یک \emph{کلاف اصلی $G$} $\GM\!\xrightarrow{\piGM}\!M$ تبدیل می‌کند.
با این حال، مهم است توجه داشته باشید که \emph{چندین انتخاب} از چنین زیرکلاف‌هایی وجود دارد، که متناظر با $G$-ساختارهای مختلف برای همان گروه ساختار~$G$ است؛ این ادعا را با مثال‌های بالا مقایسه کنید.
همانطور که قبلاً بحث شد، توپولوژی یک کلاف ممکن است مانع از کاهش به یک گروه ساختار~$G$ و در نتیجه وجود یک $G$-ساختار متناظر $\GM$ شود.


در حالی که تعریف بالا از $G$-ساختارها کافی خواهد بود، مفید است که به طور خلاصه برخی از تعاریف معادل جایگزین را مرور کنیم.
ادعای اینکه $\GM$ یک \emph{زیرکلاف} اصلی $G$ از $\FM$ است، با تعریف آن به عنوان یک تاپل $(P, \mathscr{E})$ متشکل از یک انتخاب از یک کلاف اصلی $G$ (همچنین غیرمنحصربه‌فرد) $P$ روی $M$ به همراه یک نشاندن هموردای راست $G$ و هموار $\mathscr{E}: P \to \FM$ (روی $M$) دقیق می‌شود.%
\footnote{
	این نشاندن یک ریخت $M$-کلاف اصلی $G$ است همانطور که در بخش~\ref{sec:fiber_bundles_general} با هم‌ریختی گروهی $\theta:G\to\GL{d}$ که الحاق کانونی زیرگروه $G\leq\GL{d}$ به $\GL{d}$ است، معرفی شد.
}
این موضوع توسط نمودار زیر به تصویر کشیده شده است، که لازم است برای هر $g\in G$ جابجا باشد:
\begin{equation}\label{cd:GM_def_embedding}
	\begin{tikzcd}[row sep=3.em, column sep=2.5em]
		P
		\arrow[rr, "\mathscr{E}", hook]
		&& \FM
		\\
		P
		\arrow[rr, "\mathscr{E}", hook]
		\arrow[u, "\lhd_{\overset{}{\mkern-2muP}}\mkern2mu g\:"]
		\arrow[dr, "\pi_P"']
		&& \FM
		\arrow[u, "\:\lhd\mkern2mu g"']
		\arrow[dl, "\piFM"]
		\\
		& M
	\end{tikzcd}
\end{equation}
زیرمجموعه‌های مختلف از چارچوب‌های مرجح در این دیدگاه متناظر با انتخاب‌های مختلف از نشاندن‌های ${\GM = \mathscr{E}(P)}$ از $P$ در $\FM$ است.
$G$-ساختارها علاوه بر این با برش‌هایی به شکل $s: M \mapsto \FM/G$ با $\GM = s(M)$ معادل هستند، که تأکید می‌کند $\GpM = s(p) \in \FpM/G$ در واقع یک انتخاب از $G$-مدار در $\FpM$ است همانطور که در پانوشت~\ref{footnote:GpM_G_orbit_in_FpM} بیان شد.
تعریف دیگری از $G$-ساختارها بر حسب (کلاس‌های هم‌ارزی) اطلس‌های $G$ است~\cite{wendlLectureNotesBundles2008}.
از آنجا که این دیدگاهی است که ممکن است در یک پیاده‌سازی از کانولوشن‌های $\GM$ اتخاذ شود، ما آن را با جزئیات بیشتری در بخش بعدی~\ref{sec:bundle_trivializations} مورد بحث قرار می‌دهیم.
برای خواننده علاقه‌مند می‌خواهیم اشاره کنیم که $G$-ساختارها یک مورد خاص از مفهوم کلی‌تر \emph{کاهش (یا لیفت) گروه‌های ساختار} هستند~\cite{sternberg1999lectures,piccione2006theory,crainic2013GStructuresExamples}.


$G$-ساختارها برای نظریه کانولوشن‌های $\GM$ از اهمیت محوری برخوردارند.
انتخاب خاص $G$-ساختار، مجموعه خاصی از چارچوب‌های مرجع را تعیین می‌کند که کرنل الگوی $G$-راهبر بر روی آن به اشتراک گذاشته می‌شود.
با توجه به هموردایی پیمانه‌ای کرنل‌ها، تضمین می‌شود که کانولوشن‌های $\GM$ ، $G$-ساختار را رعایت کنند، یعنی مستقل از مختصات $\GM$ باشند.
همانطور که در بخش~\ref{sec:isometry_intro} استنتاج شده است، ایزومتری‌هایی که یک کانولوشن $\GM$ نسبت به آنها هموردا است، دقیقاً آنهایی هستند که $G$-ساختار را حفظ می‌کنند (یعنی آنهایی که خودریختی‌های~$\GM$ را القا می‌کنند).


\paragraph{کلاف \lr{\textit{TM}} به عنوان کلاف برداری \lr{\textit{G}}-وابسته ($(\GM\times \mathds{R}^d)/$\lr{\textit{G}}):}

با داشتن یک $G$-ساختار $\GM$ ، می‌توان ساختار کلاف $\GL{d}$-وابسته $\TM$ از $\FM$ در بخش~\ref{sec:GL_associated_bundles} را به یک ساختار کلاف $G$-وابسته مشابه $\TM$ بر اساس $\GM$ تطبیق داد.
به جای بیان بردارهای مماس نسبت به چارچوب‌های عمومی در $\FM$ ، آنها در اینجا نسبت به چارچوب‌های متمایز در~$\GM$ بیان می‌شوند و خارج‌قسمت نسبت به گروه ساختار کاهش‌یافته~$G$ به جای~$\GL{d}$ گرفته می‌شود.
کلاف حاصل طبق طراحی به $\GM$ وابسته است (یا به $\FM$ با یک اطلس $G$ ، که معادل است همانطور که در بخش بعدی توضیح داده می‌شود) و بنابراین دارای توابع گذاری است که مقادیری در~$G$ می‌گیرند.
محدودیت $\chi$ در معادله~\eqref{eq:A_TM_isomorphism} به $(\GM\times\R^d)/G$ یک یکریختی کلاف برداری به دست می‌دهد:
\begin{align}
	\TM\, \cong\, (\GM\times\R^d)/G \,.
\end{align}
در حالی که هر سه کلاف $\TM$ ، $(\FM\times\R^d)/\GL{d}$ و $(\GM\times\R^d)/G$ بنابراین \emph{به عنوان کلاف‌های برداری} یکریخت هستند، آنها تنها در صورتی به عنوان کلاف‌های $G$-وابسته یکریخت هستند که $\TM$ و $(\FM\times\R^d)/\GL{d}$ به یک $G$-ساختار (یا اطلس $G$) مجهز شوند، که به طور پیشینی اینطور نیست.
در مقابل، کلاف $(\GM\times\R^d)/G$ طبق طراحی با یک $G$-ساختار همراه است.
برای تعریف دقیق یکریختی‌های کلاف $G$-وابسته، ما به \cite{schullerGeometricalAnatomy2016} ارجاع می‌دهیم.


\paragraph{کلاف‌های برداری ویژگی وابسته $\mathcal{A}$:}
ساختار کلاف $G$-وابسته $(\GM\times\R^d)/G$ را می‌توان برای چسباندن تارهای دیگر با عمل‌های گروهی دیگر به $G$-ساختار $\GM$ تعمیم داد.
در واقع، \emph{هر} کلاف وابسته به $\GM$ را می‌توان به این روش ساخت.
مثال‌های مهم در هندسه دیفرانسیل عبارتند از کلاف هم‌مماس $\TsM$ با تار نوعی آن که دوگان $\R^{d*}$ از $\R^d$ است و عمل دوگان بر آن اثر می‌گذارد، یا کلاف‌های تانسوری $(r,s)$ $T^r_s\!M$ با تارهای $\big(\R^d\big)^{\otimes r}\! \otimes\! \big(\R^{d*}\big)^{\otimes s}$ که نمایش حاصلضرب تانسوری متناظر~$G$ بر آن عمل می‌کند.

در ادامه ما \emph{کلاف‌های برداری ویژگی وابسته} با ضرایب بردار ویژگی $\R^c$ به عنوان تارهای نوعی را در نظر می‌گیریم.
تحت تبدیلات پیمانه، این تارها از چپ توسط یک ضرب با یک نمایش گروهی $\rho:G\to\GL{c}$ عمل می‌شوند، یعنی معادله~\eqref{eq:gauge_trafo_leftaction} با $\blacktriangleright_\rho:\ {G\times\R^c \to \R^c,}\ \ {(g,\mathscr{f}) \mapsto \rho(g)\mathscr{f}}$ نمونه‌سازی می‌شود.
مشابه قبل، کلاف‌های برداری ویژگی در این صورت به عنوان یک خارج‌قسمت
\begin{align}\label{eq:associated_bundle_def}
	\A\ :=\ (\GM\times\R^c)/\!\sim_{\!\rho}
\end{align}
ساخته می‌شوند که در آن رابطه هم‌ارزی $\sim_{\!\rho}$ توسط
\begin{align}\label{eq:equiv_relation_A}
	\big([e_i]_{i=1}^d,\, \mathscr{f} \mkern1mu\big)\ \sim_{\!\rho}\ 
	\big([e_i]_{i=1}^d\!\lhd g^{-1},\ \rho(g) \mathscr{f} \mkern1mu\big) \qquad\forall g\in G.
\end{align}
داده می‌شود.
عناصر $\A$ کلاس‌های هم‌ارزی $\big[[e_i]_{i=1}^d,\ \mathscr{f}\big]$ از ضرایب بردار ویژگی نسبت به چارچوب‌های مرجع هستند و بنابراین \emph{مستقل از مختصات} هستند.
یک نگاشت تصویر (خوش‌تعریف) دوباره از تصویر $G$-ساختار القا می‌شود:
\begin{align}\label{eq:associated_A_proj}
	\piA: \A \to M,\ \ 
	\big[[e_i]_{i=1}^d,\,\mathscr{f}\big] \mapsto \piGM \big( [e_i]_{i=1}^d \big)
\end{align}
ترکیبات خطی روی تارها به قیاس با معادله~\eqref{eq:associated_bdl_linear_combination} تعریف می‌شوند.
از آنجا که چنین کلاف‌های برداری ویژگی تعریف شده‌ای به $\GM$ وابسته‌اند، گروه ساختار آنها $G\leq\GL{d}$ است، همانطور که به طور صریح در بخش بعدی~\ref{sec:bundle_trivializations} استنتاج خواهیم کرد.%
\footnote{
	به طور دقیق، توابع گذار مقادیری در $\rho(G) \leq \GL{c}$ به جای $G \leq \GL{d}$ خواهند گرفت، با این حال، از آنجا که گذارهای حاصل هنوز با $G$ سازگار هستند، اصطلاح "$G$-مقدار" معمولاً برای شامل شدن چنین مواردی تطبیق داده می‌شود~\cite{wendlLectureNotesBundles2008}.
}
توجه داشته باشید که این تعریف شامل میدان‌های برداری مماس و میدان‌های اسکالر، که البته می‌توانند به عنوان میدان‌های ویژگی پردازش شوند، به ترتیب برای $\rho(g)=g$ و $\rho(g)=1$ می‌شود.


ساختار $\A$ به عنوان یک کلاف $G$-وابسته، بردارهای ویژگی مستقل از مختصات $\GM$ روی~$M$ را مدل می‌کند:
ویژگی‌های $f_p \in \A$ به طور معادل نسبت به چارچوب‌های دلخواه در $\GM$ بیان می‌شوند، با ضرایب ویژگی در مختصات‌دهی‌های مختلف که از طریق معادله~\eqref{eq:equiv_relation_A} به هم مرتبط هستند، اما بیان مختصاتی خوش‌تعریفی نسبت به چارچوب‌های دیگر ندارند.
از دیدگاه مهندسی، این موضوع در $G$-راهبری کرنل‌های کانولوشن منعکس می‌شود که تضمین می‌کند اندازه‌گیری‌های ویژگی‌ها به صورت \emph{نسبی} به چارچوب‌ها در $\GM$ انجام شود اما امکان تمایز بین الگوهایی را فراهم می‌کند که ژست‌های آنها با یک تبدیل پیمانه $G$-مقدار به هم مرتبط نیستند، به معنای \emph{مطلق}.


\paragraph{میدان برداری ویژگی وابسته و فضاهای ویژگی:}
میدان‌های ویژگی هموار و مستقل از مختصات به عنوان برش‌های سراسری هموار $f\in\Gamma(\A)$ از کلاف‌های برداری ویژگی تعریف می‌شوند، یعنی به عنوان نگاشت‌های هموار $f:M\to\A$ که شرط $\piA\circ f=\id_M$ را برآورده می‌کنند.
همانطور که قبلاً بحث شد، وجود چنین میدان‌های ویژگی تضمین شده است زیرا کلاف‌های برداری همیشه برش‌های سراسری هموار را می‌پذیرند.
در بخش بعدی~\ref{sec:bundle_trivializations} نشان می‌دهیم که چگونه یک بدیهی‌سازی کلاف محلی روی $U^A$ امکان نمایش $f$ توسط یک میدان $f^A:U^A\to\R^c$ از ضرایب بردار ویژگی را فراهم می‌کند.
یک بدیهی‌سازی متفاوت روی $U^B$ منجر به یک میدان ضریب متفاوت $f^B:U^B\to\R^c$ می‌شود که $f$ را به صورت محلی نمایش می‌دهد.
از نگاشت‌های گذار بین بدیهی‌سازی‌های کلاف نتیجه خواهد شد که هر دو میدان ضریب روی همپوشانی $U^{AB}=U^A\cap U^B$ از دامنه‌های خود با $f^B(p)=\rho\big(g_p^{BA}\big) f^A(p)$ به هم مرتبط هستند.
نمودار جابجایی در شکل~\ref{fig:trivialization_A_sections} روابط بین میدان‌های برداری ویژگی و بدیهی‌سازی‌های محلی آنها را به تصویر می‌کشد.


فضاهای ویژگی \CNNهای مستقل از مختصات معمولاً از چندین میدان ویژگی مستقل روی همان فضای پایه تشکیل شده‌اند.
کلاف توصیف‌کننده یک فضای ویژگی به عنوان یک کل، \emph{مجموع ویتنی} $\bigoplus_i\A_i$ از کلاف‌های برداری ویژگی $\A_i\xrightarrow{\scalebox{.85}{$\pi_{\scalebox{.6}{$\!\A_i$}}$}}M$ است که زیربنای میدان‌های منفرد آن هستند.
به این ترتیب، این کلاف همان فضای پایه $M$ را دارد، یک تار نوعی $\bigoplus_i\R^{c_i} \cong \R^{\sum_i\!c_i}$ که به عنوان مجموع مستقیم تارهای میدان‌های منفرد تعریف شده و به نگاشت تصویر بدیهی مجهز است.
این کلاف به $\TM$ ، $\FM$ ، $\GM$ و $\A_i$ به عنوان $G$-کلاف‌ها وابسته است و بنابراین می‌تواند به طور معادل به صورت
\begin{align}
	\scalebox{1.1}{$\bigoplus$}_i\,\A_i\ \ \cong\,\ \left(\GM\times\R^{\sum_i\!c_i}\right)\!\!\Big/\!\!\sim_{\oplus_i\rho_i}
\end{align}
تعریف شود.
توجه داشته باشید که مجموع مستقیم $\bigoplus_i\rho_i$ از نمایش‌های $\rho_i$ که $\A_i$ را تعریف می‌کنند، تضمین می‌کند که نگاشت‌های گذار $\bigoplus_i\A_i$ هر میدان منفرد را به طور مستقل تبدیل کنند.
فضاهای ویژگی در این صورت به عنوان فضاهای $\Gamma\!\left(\bigoplus_i\A_i\right)$ از برش‌های سراسری کلاف مجموع ویتنی تعریف می‌شوند.