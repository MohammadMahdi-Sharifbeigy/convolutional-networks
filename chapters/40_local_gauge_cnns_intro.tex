%!TEX root=../GaugeCNNTheory.tex


\section{شبکه‌های مستقل از مختصات و \textit{\lr{GM}}-کانولوشن‌ها}
\label{sec:gauge_CNNs_local}


شبکه‌های عصبی داده‌ها را با اعمال مجموعه‌ای از نگاشت‌های پارامتری (لایه‌ها) به یک سیگنال ورودی پردازش می‌کنند ـ در مورد ما به مجموعه‌ای از میدان‌های ویژگی بر روی یک منیفولد ریمانی.
\emph{اصل کوواریانس} بدین ترتیب ایجاب می‌کند که لایه‌های منفرد شبکه باید عملیات مستقل از مختصات $\GM$ باشند.
نمایش‌های مختصاتی چنین لایه‌هایی بنابراین باید طوری تبدیل شوند که قوانین تبدیل میدان‌های ویژگی ورودی و خروجی خود را رعایت کنند.
به جز این الزام سازگاری، لایه‌های مستقل از مختصات عمومی \emph{بدون محدودیت} باقی می‌مانند.


یک اصل طراحی رایج شبکه‌های عصبی که بر روی سیگنال‌های فضایی (میدان‌های ویژگی) عمل می‌کنند این است که آن‌ها به معنای تعمیم یافته‌ای کانولوشنی هستند.
ویژگی اصلی که اکثر تعمیم‌های عملیات کانولوشن در آن مشترک هستند این است که استنتاج آن‌ها \emph{مستقل از موقعیت} است.
این امر با \emph{اشتراک توابع قالب}، به عنوان مثال کرنل‌های کانولوشن یا بایاس‌ها، بین مکان‌های مختلف حاصل می‌شود.
هر زمان که گروه ساختار $G$ غیربدیهی باشد، فرآیند اشتراک وزن مبهم است زیرا توابع قالب می‌توانند نسبت به چارچوب‌های مرجع مختلف به اشتراک گذاشته شوند.
همانطور که در ادامه استدلال خواهیم کرد، این ابهام با طراحی توابع قالب مشترک برای \emph{تناوب‌پذیر بودن تحت تبدیل‌های گیج با مقدار $G$} حل می‌شود.
توابع قالب تناوب‌پذیر گیج نسبت به چارچوب مرجع خاصی که در آن اعمال می‌شوند بی‌تفاوت خواهند بود و بنابراین امکان اشتراک وزن مستقل از مختصات را فراهم می‌کنند.



\etocsettocdepth{3}
\etocsettocstyle{}{} % from now on only local tocs
\localtableofcontents



در این بخش ما لایه‌های شبکه‌ای را در نظر خواهیم گرفت که میدان‌های $\fin$ از نوع $\rhoin$ را به عنوان ورودی می‌گیرند و میدان $\fout$ از نوع $\rhoout$ را به عنوان خروجی تولید می‌کنند.
بخش~\ref{sec:pointwise_operations} حالت خاص لایه‌هایی را که به صورت نقطه‌ای عمل می‌کنند مورد بحث قرار می‌دهد، یعنی آن‌هایی که خروجی آن‌ها $\fout(p)$ در هر $p\in M$ تنها به بردار ویژگی ورودی منفرد $\fin(p)$ در همان مکان بستگی دارد.
مثال‌های عملی مرتبط که در اینجا در نظر گرفته شده‌اند عبارتند از \onexonesitfarsi\ تناوب‌پذیر گیج در بخش~\ref{sec:gauge_1x1}، جمع بایاس در بخش~\ref{sec:gauge_bias_summation} و غیرخطی‌ها در بخش~\ref{sec:gauge_nonlinearities}.
حالت پیچیده‌تر کانولوشن‌ها با کرنل‌های گسترده فضایی در بخش~\ref{sec:gauge_conv_main} بررسی شده است.
به عنوان آماده‌سازی، بخش~\ref{sec:observers_view} میدان‌های ویژگی را همانطور که از دیدگاه مشاهده‌گران محلی (چارچوب‌های مرجع) دیده می‌شوند مورد بحث قرار می‌دهد، که کرنل‌های (کانولوشن) نسبت به آن‌ها اعمال خواهند شد.
چنین مشاهداتی به عنوان پس‌کشی میدان ویژگی به فضای مماس یک مشاهده‌گر رسمی‌سازی می‌شوند؛ به شکل~\ref{fig:pullback_field_exp_TpM} مراجعه کنید.
بخش~\ref{sec:kernel_field_trafos} به اصطلاح تبدیل‌های میدان کرنل را معرفی می‌کند، که شبیه کانولوشن‌ها هستند اما اشتراک وزن فضایی را فرض نمی‌کنند و بنابراین توسط یک میدان کرنل (به صورت هموار متغیر) بر روی~$M$ پارامتری می‌شوند.
$\GM$-کانولوشن‌های واقعی در بخش~\ref{sec:gauge_conv} به عنوان آن تبدیل‌های میدان کرنل تعریف می‌شوند که توسط یک کرنل قالب منفرد مشترک پارامتری می‌شوند.
به منظور تضمین استقلال از مختصات فرآیند اشتراک وزن، کرنل‌های کانولوشن مطالبه می‌شوند که $G$-\emph{استیریبل} باشند، یعنی قید تناوب‌پذیری گیج را برآورده کنند.
بخش~\ref{sec:gauge_conv_isom_equiv} نشان می‌دهد که $\GM$-کانولوشن‌ها به طور خودکار تحت آن ایزومتری‌هایی که تقارن‌های $G$-ساختار هستند ($\IsomGM$-تناوب‌پذیر) تناوب‌پذیر هستند.
این بدان معنی است که $\GM$-کانولوشن‌ها با عمل ایزومتری‌ها بر میدان‌های ویژگی همانطور که در شکل~\ref{fig:lizard_conv_egg} بصری‌سازی شده است، جابجا می‌شوند.