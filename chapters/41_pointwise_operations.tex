%!TEX root=../GaugeCNNTheory.tex


\subsection{عملیات نقطه‌ای تناوب‌پذیر گیج}
\label{sec:pointwise_operations}


برای شروع، ما برخی عملیات شبکه عصبی را در نظر می‌گیریم که محدودیت‌های ناشی از استقلال از مختصات مورد نیاز و اشتراک وزن به ویژه ساده قابل استخراج هستند.
همه این عملیات این ویژگی مشترک را دارند که به صورت نقطه‌ای بر بردارهای ویژگی عمل می‌کنند، یعنی بردارهای ویژگی خروجی $\fout(p)$ در $p\in M$ را صرفاً بر اساس بردارهای ویژگی ورودی $\fin(p)$ در همان مکان محاسبه می‌کنند.
به منظور ارضای اصل کوواریانس، مختصات‌بندی‌های این عملیات همگی مطالبه می‌شوند که مطابق پیش‌ترکیب با $\rhoin$ و پس‌ترکیب با $\rhoout$ تبدیل شوند.
هنگام تقاضای اینکه عملیات بر حسب وزن‌های مشترک تعیین شوند، این قوانین تبدیل الزامی برای تناوب‌پذیری گیج (یا ناورداییی) عملیات را دلالت می‌کنند.


استخراج‌ها برای عملیات نقطه‌ای مختلف در بخش‌های بعدی~\ref{sec:gauge_1x1}، \ref{sec:gauge_bias_summation} و~\ref{sec:gauge_nonlinearities} در گام‌های اول عمدتاً مشابه هستند و به محدودیت‌های کوواریانس و تناوب‌پذیری اساساً یکسانی بر توابع قالب منجر می‌شوند.
بنابراین آن‌ها می‌توانند با هم بررسی شوند، عملیات خاص (یا تابع قالب) را انتزاعی نگه داشته.
با این حال، از آنجا که مفاهیم محدودیت‌های حاصل برای نمونه‌سازی‌های خاص متفاوت است، و از آنجا که می‌خواهیم بحث را نزدیک به کاربرد نگه داریم، چنین فرمول‌بندی انتزاعی را حذف خواهیم کرد و مستقیماً نمونه‌سازی‌های خاص را در نظر خواهیم گرفت.


\subsubsection{کانولوشن‌های $1\times1$ تناوب‌پذیر گیج}
\label{sec:gauge_1x1}


به عنوان اولین مثال از عملیات نقطه‌ای، ما عمل خانواده‌ای از \emph{نگاشت‌های خطی} $\mathcal{C}_p$ را در نظر می‌گیریم، که بردار ویژگی ورودی $\fin(p)$ را در هر $p\in M$ به بردار ویژگی خروجی
\begin{align}
	\fout(p) := \mathcal{C}_p\, \fin(p) \,.
\end{align}
ارسال می‌کنند.
اگر فرض اشتراک وزن فضایی را اضافه کنیم، نگاشت‌های خطی $\mathcal{C}_p$ و $\mathcal{C}_q$ در مکان‌های مختلف $p$ و~$q$ جفت خواهند شد، و عملیات را می‌توان به عنوان کانولوشنی با کرنل دلتای دیراک با مقدار عملگر خطی دید.
این عملیات در بینایی کامپیوتر بسیار رایج است، جایی که معمولاً به عنوان کانولوشن $1\times1$ نشان داده می‌شود، زیرا گسسته‌سازی فضایی یک کرنل دیراک خطی که بر روی تصاویر دو بعدی عمل می‌کند توسط کرنل (با مقدار ماتریس) با گستره فضایی $1\times1$ پیکسل داده می‌شود.
ما در ادامه استخراج خواهیم کرد که تقاضای اشتراک وزن فضایی منجر به محدودیتی خواهد شد که کرنل‌های قالب با مقدار ماتریس را مجبور می‌کند که \emph{درهم‌تنیده} باشند، یعنی ماتریس‌های تناوب‌پذیر گیج.


پیش از فرض اشتراک وزن، بیان‌های مختصاتی نگاشت‌های خطی $\mathcal{C}_p$ و تبدیل‌های گیج بین آن‌ها بسیار شبیه آن‌هایی از نگاشت‌های خطی بر روی $\TpM$ رفتار می‌کنند، که در بخش~\ref{sec:gauges_TpM_functions} مورد بحث قرار دادیم.
از آنجا که بردارهای ویژگی ورودی و خروجی در مختصات توسط بردارهای ضریب $\fin^A(p) \in \R^{\cin}$ و $\fout^A(p) \in \R^{\cout}$ نمایش داده می‌شوند، نگاشت خطی به طور طبیعی توسط آن ماتریس $\mathcal{C}_p^A \in \R^{\cout\times\cin}$ نمایش داده می‌شود که
\begin{align}\label{eq:linear_op_coord_A}
	\fout^A(p) = \mathcal{C}_p^A \cdot \fin^A(p) \,.
\end{align}
را برآورده می‌کند.
این رابطه البته برای مختصات‌بندی‌های دلخواه برقرار است، به طوری که برای هر گیج دیگری که با $B$ برچسب گذاری شده است، $\fout^B(p) = \mathcal{C}_p^B \cdot \fin^B(p)$ داریم.
قانون تبدیلی که $\mathcal{C}_p^B$ را به $\mathcal{C}_p^A$ مرتبط می‌کند از اصل کوواریانس از قوانین تبدیل ویژگی‌های ورودی و خروجی پیروی می‌کند.
از آنجا که این‌ها توسط $\fin^B(p) = \rhoin\big( g_p^{BA}\big) \fin^A(p)$ و $\fout^B(p) = \rhoout\big( g_p^{BA}\big) \fout^A(p)$ داده می‌شوند، داریم:
\begin{alignat}{3}
	&& \fout^B(p)\ &=\ \mathcal{C}_p^B \cdot \fin^B(p)
	\notag \\ \Leftrightarrow \qquad
	&& \rhoout\big( g_p^{BA}\big)\, \fout^A(p)\ &=\ \mathcal{C}_p^B\, \rhoin\big( g_p^{BA}\big)\, \fin^A(p)
	\notag \\ \Leftrightarrow \qquad
	&& \fout^A(p)\ &=\ \rhoout\big( g_p^{BA}\big)^{-1}\, \mathcal{C}_p^B\, \rhoin\big( g_p^{BA}\big)\, \fin^A(p) \,.
\end{alignat}
مقایسه با معادله~\eqref{eq:linear_op_coord_A} دلالت دارد بر اینکه دو بیان مختصاتی $\mathcal{C}_p$ لزوماً با
\begin{align}\label{eq:linear_op_trafo_law}
	\mathcal{C}_p^B\ =\ \rhoout\big( g_p^{BA}\big)\, \mathcal{C}_p^A\, \rhoin\big( g_p^{BA}\big)^{-1}
\end{align}
مرتبط هستند اگر قرار باشد قوانین تبدیل بردارهای ویژگی را رعایت کنند.
همانطور که معمول است، این ملاحظات به طور مختصر توسط نمودار جابجایی ضبط می‌شوند:
\begin{equation}\label{cd:linear_op_trafo_law}
	\begin{tikzcd}[column sep=60pt, row sep=30pt, font=\normalsize]
		\R^{\cin}
		\arrow[d, "\rhoin\big(g_p^{BA}\big)\cdot\,"']
		\arrow[r, "\mathcal{C}_p^A \cdot"]
		&
		\R^{\cout}
		\arrow[d, "\ \rhoout\big(g_p^{BA}\big) \cdot"]
		\\
		\R^{\cin}
		\arrow[r, "\mathcal{C}_p^B \cdot"']
		&
		\R^{\cout}
	\end{tikzcd}
\end{equation}
مفهوم عملی مهم این نتیجه تا کنون این است که نگاشت خطی $\mathcal{C}_p$ به هیچ وجه محدود نیست.
به بیان دیگر: تا زمانی که بیان‌های مختصاتی در گیج‌های مختلف توسط معادله~\eqref{eq:linear_op_trafo_law} مرتبط باشند، آزاد هستیم که $\mathcal{C}_p$ را در یک گیج دلخواه ثابت $A$ توسط ماتریس \emph{بدون محدودیت} $\mathcal{C}_p^A$ پارامتری کنیم.
همانطور که خواهیم دید، وضعیت زمانی تغییر می‌کند که نگاشت‌های خطی اشتراک وزن داشته باشند.


حال حالتی را در نظر بگیرید که نگاشت‌های خطی $\mathcal{C}_p$ و $\mathcal{C}_q$ وزن‌ها را به اشتراک می‌گذارند.
این بدان معنی است که فرض می‌کنیم آن‌ها توسط مجموعه مشترکی از پارامترها پارامتری شوند، که توسط کرنل قالب $1\times1$ $K_{\!1\!\times\!1} \in \R^{\cout\times\cin}$ داده می‌شود.
سؤال باز این است که دقیقاً چگونه نگاشت‌های آزاد از مختصات باید بر حسب این کرنل قالب پارامتری شوند.
الزام ما برای استقلال از مختصات $\GM$ تقاضا می‌کند که هیچ چارچوب مرجع خاصی را در فرآیند اشتراک وزن ترجیح ندهیم، یعنی همه مختصات‌بندی‌ها را به همان شیوه درمان کنیم.
بنابراین لازم است که \emph{کرنل قالب را با همه مختصات‌بندی‌ها به طور همزمان به اشتراک بگذاریم}، یعنی
\begin{align}\label{eq:weight_sharing_1x1}
	\mathcal{C}_p^X = K_{\!1\!\times\!1}
	\quad \textup{برای \emph{هر} گیج}\ \ \big(U^X,\psi^X) \in \mathscr{A}^G\ \ \textup{با}\ \ p\in U^X \,,
\end{align}
تنظیم کنیم، که در آن $\mathscr{A}^G$ (حداکثر) $G$-اطلس متناظر با $G$-ساختار در نظر گرفته شده است؛ به معادله~\eqref{eq:G_atlas_dfn} مراجعه کنید.
از آنجا که محدودیت کوواریانس در معادله~\eqref{eq:linear_op_trafo_law} نیاز دارد که برای گیج‌های دلخواه مرتبط با $G$ برقرار باشد، و مختصات‌بندی‌های $\mathcal{C}_p^A = \mathcal{C}_p^B = K_{\!1\!\times\!1}$ نگاشت‌های خطی همگی مطابقت دارند، تقاضای مشترک برای اشتراک وزن و استقلال از مختصات $\GM$ منجر به محدودیت
\begin{align}\label{eq:Konexone_constraint_intertwiner}
	K_{\!1\!\times\!1}\ =\ \rhoout(g)\, K_{\!1\!\times\!1}\, \rhoin(g)^{-1} \qquad \forall\ g\in G
\end{align}
بر کرنل قالب می‌شود.
تطبیق متناظر نمودار جابجایی در معادله~\eqref{cd:linear_op_trafo_law} با اشتراک وزن توسط:
\begin{equation}
	\begin{tikzcd}[column sep=60pt, row sep=30pt, font=\normalsize]
		\R^{\cin}
		\arrow[d, "\rhoin\big(g_p^{BA}\big)\cdot\,"']
		\arrow[r, "K_{\!1\!\times\!1} \cdot"]
		&
		\R^{\cout}
		\arrow[d, "\ \rhoout\big(g_p^{BA}\big) \cdot"]
		\\
		\R^{\cin}
		\arrow[r, "K_{\!1\!\times\!1} \cdot"']
		&
		\R^{\cout}
	\end{tikzcd}
\end{equation}
داده می‌شود.


نتیجه‌گیری این تحلیل این است که کرنل‌های قالبی که می‌توانند \emph{بدون ابهام به اشتراک گذاشته شوند} دقیقاً آن‌هایی هستند که تحت \emph{عمل گیج ناوردا} هستند.
فضای برداری چنین کرنل‌های $1\times1$ ناوردا گیج ساده فضای \emph{نگاشت‌های درهم‌تنیده} بین نمایش‌های $\rhoin$ و $\rhoout$ است، یعنی
\begin{align}\label{eq:gauge_onexone_solution_space}
	\Hom_G(\rhoin,\rhoout)\ :=\ 
	\pig\{ K_{\!1\!\times\!1} \in \R^{\cout\times\cin}\ \pig|\ 
	K_{\!1\!\times\!1} = \rhoout(g)\, K_{\!1\!\times\!1}\, \rhoin(g)^{-1}\ \ \ \forall g\in G \pig\}
	\,\ \subseteq\ \R^{\cout\times\cin} \,.
\end{align}
توجه داشته باشید که، طبق \emph{لم شور}~\cite{gallier2019harmonicRepr}، الزام بر $K_{\!1\!\times\!1}$ برای درهم‌تنیده بودن مانع از نگاشت بین میدان‌هایی که تحت نمایش‌های کاهش‌ناپذیر غیرهم‌ریخت تبدیل می‌شوند از طریق کانولوشن‌های $1\times1$ می‌شود.
این محدودیت شدید با کرنل‌های $1\times1$ اجتناب‌ناپذیر است اما بعداً هنگام اجازه دادن به کرنل‌های گسترده فضایی حل خواهد شد.

در این نقطه می‌خواهیم اشاره کنیم که اصطلاحات «تابع قالب تناوب‌پذیر گیج» و «تابع قالب ناوردا گیج» را به طور متقابل استفاده می‌کنیم.
این با مشاهده توجیه می‌شود که محدودیت ناورداییی در معادله~\eqref{eq:Konexone_constraint_intertwiner} می‌تواند به عنوان محدودیت تناوب‌پذیری
$K_{\!1\!\times\!1}\, \rhoin(g) = \rhoout(g)\, K_{\!1\!\times\!1}\ \ \forall g\in G$
نوشته شود.
به طور کلی امکان مشاهده توابعی که نسبت به عمل گروهی در دامنه و دامنه مشترک آن‌ها تناوب‌پذیر هستند به عنوان ناوردای عمل متناظر بر خود تابع وجود دارد.
در کاربرد ما، دیدگاه تناوب‌پذیری برجسته می‌کند که تبدیل میدان ورودی منجر به تبدیل متناظر میدان خروجی خواهد شد، که تضمین می‌کند همه کمیت‌های درگیر به طور کوواریانت با یکدیگر تبدیل شوند.
از طرف دیگر، دیدگاه ناورداییی تأکید می‌کند که تابع قالب می‌تواند در گیج دلخواه به اشتراک گذاشته شود.


\subsubsection{جمع بایاس تناوب‌پذیر گیج}
\label{sec:gauge_bias_summation}


پس از اعمال عملیات کانولوشن، رایج است که بردار بایاس (مشترک) به بردارهای ویژگی منفرد جمع شود.
همراه با الزام استقلال از مختصات، اشتراک وزن دوباره منجر به محدودیت خطی خواهد شد.
این محدودیت تنها اجازه جمع شدن بایاس‌ها به زیرفضاهای ناوردا عمل گیج بر میدان ویژگی ورودی را خواهد داد.


همانند قبل، ابتدا جمع بایاس را بدون اشتراک وزن در نظر می‌گیریم.
بنابراین بایاس‌های~$\mathscr{b}_p$ داریم، که به موقعیت $p$ بر روی منیفولد بستگی دارند، که به بردار ویژگی ورودی جمع می‌شوند تا بردار ویژگی خروجی
\begin{align}
	\fout(p) = \fin(p) + \mathscr{b}_p \,.
\end{align}
تولید کنند.
نسبت به گیج‌های $\psi_p^A$ و $\psi_p^B$، بایاس توسط آن بردارهای ضریب $\mathscr{b}_p^A$ و $\mathscr{b}_p^B$ در $\R^c$ نمایش داده می‌شود که $\fout^A(p) = \fin^A(p) + \mathscr{b}_p^A$ و $\fout^B(p) = \fin^B(p) + \mathscr{b}_p^B$ را برآورده می‌کنند.
از آنجا که جمع بردارها اجازه تغییر قوانین تبدیل آن‌ها را نمی‌دهد، نمایش‌های گروهی مرتبط با ویژگی‌های ورودی و خروجی لزوماً مطابقت دارند، یعنی
\begin{align}
	\rhoin = \rhoout =: \rho \,.
\end{align}
همراه با الزام برای استقلال از مختصات، این دلالت دارد بر اینکه نمودار
\begin{equation}\label{eq:bias_trafo_law}
	\begin{tikzcd}[column sep=75pt, row sep=30pt, font=\normalsize]
		\R^c
		\arrow[d, "\rho \big(g_p^{BA}\big)\cdot\,"']
		\arrow[r, "+\mathscr{b}_p^A"]
		&
		\R^c
		\arrow[d, "\ \rho\big( g_p^{BA}\big) \cdot"]
		\\
		\R^c
		\arrow[r, "+\mathscr{b}_p^B"']
		&
		\R^c
	\end{tikzcd}
	\ \ ,
\end{equation}
که معادل آن در معادله~\eqref{cd:linear_op_trafo_law} است، نیاز دارد که جابجا شود.
به صورت معادله نوشته شده، این رابطه
$\rho\big(g_p^{BA}\big) f^A_p + \mathscr{b}_p^B \ =\ \rho\big(g_p^{BA}\big) \big(f^A_p + \mathscr{b}_p^A\big)$
را تقاضا می‌کند که برقرار باشد.
از آنجا که خطی بودن $\rho(g)$ امکان بازنویسی سمت راست را به عنوان
$\rho\big(g_p^{BA}\big) f^A_p + \rho\big(g_p^{BA}\big) \mathscr{b}_p^A$
می‌دهد، تفریق بردار ویژگی ورودی منجر به
\begin{align}\label{eq:bias_trafo_non_shared}
	\mathscr{b}_p^B\ =\ \rho\big(g_p^{BA}\big) \, \mathscr{b}_p^A \,.
\end{align}
می‌شود.
بردارهای ضریبی که یک بایاس مستقل از مختصات را نسبت به گیج‌های مختلف نمایش می‌دهند بنابراین نیاز دارند که دقیقاً مانند بردارهای ویژگی که به آن‌ها جمع می‌شوند تبدیل شوند.
همانند مورد کانولوشن‌های $1\times1$, استقلال از مختصات بایاس $\mathscr{b}_p$ را به هیچ وجه محدود \emph{نمی‌کند}، بلکه فقط مختصات‌بندی‌های مختلف همان بایاس را مطالبه می‌کند که با یکدیگر سازگار باشند.
یک پیاده‌سازی بنابراین می‌تواند گیج دلخواه انتخاب کند و بایاس را در آن گیج آزادانه توسط پارامترهایی در $\R^{\cin}$ پارامتری کند.


وضعیت دوباره زمانی تغییر می‌کند که اشتراک وزن فضایی تقاضا شود.
فرض کنید $b \in \R^{\cin}$ بردار بایاس قالبی باشد که بر روی منیفولد به اشتراک گذاشته شود.
از آنجا که تنها راه انجام این کار بدون ترجیح دلخواه هر مختصات‌بندی، اشتراک بردار بایاس در همه گیج‌ها به طور همزمان است، باید
\begin{align}
	\mathscr{b}_p^X = b
	\quad \textup{برای \emph{هر} گیج}\ \ \big(U^X,\psi^X) \in \mathscr{A}^G\ \ \textup{با}\ \ p\in U^X \,.
\end{align}
را در قیاس با معادله~\eqref{eq:weight_sharing_1x1} مطالبه کنیم.
ترکیب محدودیت کوواریانس در معادله~\eqref{eq:bias_trafo_non_shared} با این اشتراک وزن مستقل از گیج سپس منجر به محدودیت ناورداییی
\begin{align}\label{eq:bias_invariance_constraint}
	b\ =\ \rho(g)\, b \qquad \forall\ g\in G
\end{align}
بر قالب بردار بایاس می‌شود.
برای تکمیل قیاس با مورد کانولوشن‌های $1\times1$, نسخه تطبیق یافته نمودار جابجایی در معادله~\eqref{eq:bias_trafo_law} با وزن‌های مشترک را نشان می‌دهیم:
\begin{equation}
	\begin{tikzcd}[column sep=75pt, row sep=30pt, font=\normalsize]
		\R^c
		\arrow[d, "\rho\big( g_p^{BA}\big)\cdot\,"']
		\arrow[r, "+b"]
		&
		\R^c
		\arrow[d, "\ \rho\big( g_p^{BA}\big)\cdot"]
		\\
		\R^c
		\arrow[r, "+b"']
		&
		\R^c
	\end{tikzcd}
\end{equation}


برای بصیرت در مفاهیم محدودیت ناورداییی در معادله~\eqref{eq:bias_invariance_constraint}، فرض کنید برای بردار قالب داده شده~$b$ برآورده شود.
به دلیل خطی بودن محدودیت، هر بردار مقیاس شده $\alpha \!\cdot\! b$ برای $\alpha\in\R$ نیز آن را برآورده خواهد کرد، یعنی هر حل \emph{زیرفضای یک بعدی از~$\R^c$} را پوشش می‌دهد که تحت \emph{عمل~$\rho$ ناوردا} است.
چنین زیرفضای ناوردا به عنوان زیرنمایش~$\rho$ نشان داده می‌شود.
از آنجا که زیرفضاهای در نظر گرفته شده یک بعدی هستند، خودشان هیچ زیرفضای مناسب ندارند و بنابراین زیرنمایش‌های کاهش‌ناپذیر بدیهی هستند.
نتیجه می‌شود که فضای برداری
\begin{align}\label{eq:gauge_bias_solution_space}
	\mathscr{B}^G_\rho\ :=\ \big\{ b \in\R^c \;\big|\; b = \rho(g)\mkern2mu b\ \ \ \forall g\in G \big\}
\end{align}
بایاس‌های تناوب‌پذیر گیج با (زیرفضاهای) زیرنمایش‌های بدیهی $\rho$ مطابقت دارد.
ابعاد $\mathscr{B}^G_\rho$ -- و بنابراین تعداد پارامترهای قابل یادگیری -- با کثرت زیرنمایش‌های بدیهی موجود در~$\rho$ مطابقت دارد.
برای گروه‌های فشرده~$G$، روابط متعامد شور دلالت دارند بر اینکه این ابعاد توسط $\dim\!\big(\mathscr{B}^G_\rho\big) = \int_G \tr\big(\rho(g)\big) dg$ داده می‌شود.
این گزاره موارد مهم عملی گروه‌های متعامد $G=\OO{d}$ و همه زیرگروه‌های آن را پوشش می‌دهد.


دو مثال ساده از میدان‌های ویژگی که ممکن است بخواهیم بایاس‌های مشترک به آن‌ها جمع کنیم، میدان‌های اسکالر و میدان‌های بردار مماس هستند.
بر اساس تعریف، میدان ضریب یک میدان اسکالر تحت تبدیل‌های گیج ناوردا است، یعنی طبق نمایش بدیهی ${\rho(g)=1\ \ \forall g\in G}$ تبدیل می‌شود.
بنابراین می‌توان بایاس (اسکالر) $b \in \R$ به آن‌ها جمع کرد.
در مقابل، میدان ضریب یک میدان بردار مماس طبق نمایش گروهی غیربدیهی، کاهش‌ناپذیر $\rho(g)=g$ تبدیل می‌شود.
از آنجا که این نمایش هیچ زیرنمایش بدیهی ندارد، جمع بردار بایاس مشترک به میدان‌های بردار مماس با حفظ استقلال از مختصات غیرممکن است.
به عنوان مثال سوم، نمایش‌های منظم گروه‌های فشرده را در نظر بگیرید، که به عنوان مثال میدان‌های ویژگی شبکه‌های کانولوشنی گروهی را توصیف می‌کنند.
توسط قضیه پیتر-ویل، شناخته شده است که نمایش‌های منظم دقیقاً یک زیرنمایش بدیهی دارند~\cite{gurarie1992symmetries,gallier2019harmonicRepr}.
بنابراین بایاس که به میدان‌های ویژگی منظم جمع می‌شود توسط یک پارامتر منفرد توصیف می‌شود.


\subsubsection{غیرخطی‌های تناوب‌پذیر گیج}
\label{sec:gauge_nonlinearities}


به جز عملیات خطی (کانولوشن) و جمع بایاس، اساسی‌ترین عملیات استفاده شده در هر شبکه عصبی غیرخطی‌ها هستند.
ما در اینجا حالت معمول غیرخطی‌های $\sigma_p$ را در نظر خواهیم گرفت که به روشی محلی فضایی عمل می‌کنند، یعنی بردارهای ویژگی خروجی را به عنوان $\fout(p) = \sigma_p\big( \fin(p) \big)$ محاسبه می‌کنند.
یک غیرخطی مشترک دوباره مطالبه می‌شود که تناوب‌پذیر گیج باشد.
از آنجا که استدلالی که به این نتیجه‌گیری منجر می‌شود شبیه آن در موارد قبلی است، تنها به طور خلاصه آن را خلاصه خواهیم کرد.
به دلیل عمومیت نگاشت‌های غیرخطی استخراج فضاهای حل خطی همانند معادلات~\eqref{eq:gauge_onexone_solution_space} و~\eqref{eq:gauge_bias_solution_space} غیرممکن است، با این حال، برخی مثال‌های خاص را مورد بحث قرار خواهیم داد.


مشابه قبل، هر غیرخطی آزاد از مختصات $\sigma_p$ نسبت به گیج‌های $A$ و $B$ توسط بیان‌های مختصاتی $\sigma_p^A: \R^{\cin} \to \R^{\cout}$ و $\sigma_p^B: \R^{\cin} \to \R^{\cout}$ داده می‌شود، که توسط تقاضای استقلال از مختصات مطالبه می‌شوند که با $\sigma_p^B = \rhoout\big( g_p^{BA}\big) \circ \sigma_p^A \circ \rhoin\big( g_p^{BA}\big)^{-1}$ مرتبط باشند.
یک تابع قالب غیرخطی $\mathscr{s}: \R^{\cin} \to \R^{\cout}$ تنها زمانی می‌تواند به روشی مستقل از مختصات به اشتراک گذاشته شود که به طور همزمان با همه گیج‌ها به اشتراک گذاشته شود.
این محدودیت کوواریانس را به محدودیت ناورداییی $\mathscr{s} = \rhoout(g) \circ \mathscr{s} \circ \rhoin(g)^{-1}\ \ \forall g\in G$ بر تابع قالب تبدیل می‌کند، یا، به طور معادل، به محدودیت تناوب‌پذیری متناظر
\begin{align}\label{eq:gauge_constraint_nonlinearities}
	\rhoout(g) \circ \mathscr{s} = \mathscr{s} \circ \rhoin(g)^{-1} \qquad \forall\ g\in G \,.
\end{align}


به دلیل غیرخطی بودن این محدودیت مجبور هستیم آن را مورد به مورد بررسی کنیم -- بنابراین بحث خود را به چند مثال خاص محدود خواهیم کرد.
بحث‌برانگیزترین ساده‌ترین حالت، غیرخطی‌هایی است که بین میدان‌های اسکالر نگاشت می‌کنند، یعنی برای آن‌ها $\rhoin(g) = \rhoout(g) = 1$ برای هر $g \in G$ ناوردا هستند.
در این حالت محدودیت تناوب‌پذیری در معادله~\eqref{eq:gauge_constraint_nonlinearities} به $\mathscr{s} = \mathscr{s}$ تبدیل می‌شود، که به طور بدیهی برای \emph{هر} غیرخطی $\mathscr{s}: \R \to \R$ برآورده می‌شود.
مثال جالب‌تری نمایش یکانی $\rhoin$ است.
یک غیرخطی ممکن برای این حالت توسط نرم بردارهای ویژگی داده می‌شود.
از آنجا که $\lVert \rhoin(g) \fin^A(p) \rVert = \lVert \fin^A(p) \rVert$ به دلیل یکانی بودن $\rhoin$ ناوردا است، $\rhoout$ نمایش بدیهی خواهد بود.
گرفتن نرم بنابراین به عنوان عملیات غیرخطی، تناوب‌پذیر گیج دیده می‌شود که هر میدان یکانی را به میدان اسکالر نگاشت می‌کند.
یک نگاشت غیرخطی که نوع میدان را حفظ می‌کند، یعنی $\rhoout = \rhoin$ را برآورده می‌کند، می‌تواند به طور مشابه به عنوان $\fin^A(p) \mapsto \lVert\fin^A(p)\rVert \cdot \fin^A(p)$ تعریف شود.
گزینه دیگر، که ممکن است در یادگیری تعاملات فیزیکی نقش داشته باشد و در~\cite{kondor2018ClebschGordan,Kondor2018-NBN,anderson2019cormorant,alex2020lorentz} مورد بررسی قرار گرفته است، غیرخطی‌های حاصل‌ضرب تانسوری هستند.
با داشتن دو میدان $f_\textup{\;\!\!in,1}^A(p)$ و $f_\textup{\;\!\!in,2}^A(p)$، که به ترتیب طبق
$\rho_{\overset{}{\protect\scalebox{.6}{\;\!\!\textup{in},1}}}$ و $\rho_{\overset{}{\protect\scalebox{.6}{\;\!\!\textup{in},2}}}$
تبدیل می‌شوند، چنین غیرخطی‌هایی ویژگی حاصل‌ضرب تانسوری $\fout^A = f_\textup{\;\!\!in,1}^A(p) \otimes f_\textup{\;\!\!in,2}^A(p)$ را محاسبه می‌کنند، که به طور تناوب‌پذیر طبق نمایش حاصل‌ضرب تانسوری
$\rhoout = \rho_{\overset{}{\protect\scalebox{.6}{\;\!\!\textup{in},1}}} \otimes \rho_{\overset{}{\protect\scalebox{.6}{\;\!\!\textup{in},2}}}$
تبدیل می‌شود.


همه این مثال‌ها محدودیت تناوب‌پذیری گیج در معادله~\eqref{eq:gauge_constraint_nonlinearities} را برآورده می‌کنند.
اینکه کدام غیرخطی خاص در عمل خوب کار می‌کند، با این حال، هنوز سؤال تحقیقاتی بازی است که نیازمند بررسی تجربی بسیار بیشتری قبل از پاسخ دادن است.
اولین تلاش در این جهت در~\cite{Weiler2019_E2CNN} انجام شده است.