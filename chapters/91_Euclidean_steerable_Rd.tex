%!TEX root=../GaugeCNNTheory.tex


\subsection%
[فرمول‌بندی کلاسیک \lr{CNN}های \texorpdfstring{$G$}{G}-هدایت‌پذیر بر روی \texorpdfstring{$\R^d$}{R^d}]%
{فرمول‌بندی کلاسیک \lr{CNN}های \textit{G}-هدایت‌پذیر بر روی $\fakebold{\R^d}$}
\label{sec:steerable_cnns_in_coords}


در این بخش، ما مفهوم معمول کانولوشن‌ها (یا همبستگی‌های متقابل%
\footnote{
	در یادگیری عمیق، استفاده از اصطلاحات «کانولوشن» و «همبستگی متقابل» به صورت مترادف رایج شده است.
})
را بر روی $\R^d$ مرور می‌کنیم.
هنگام کانوالو کردن با یک کرنل \lr{G}-هدایت‌پذیر، کانولوشن‌ها تحت عمل گروه آفین $\Aff(G)$ هم‌متغیر می‌شوند.
بخش‌های بعدی~\ref{sec:euclidean_geometry} و~\ref{sec:euclidean_affine_equiv} این عملیات را بر روی $\R^d$ به عنوان عبارات مختصاتی کانولوشن‌های $\GM$ مستقل از مختصات بر روی فضاهای اقلیدسی $\Euc_d$ شناسایی خواهند کرد.


\paragraph{\lr{CNN}های هدایت‌پذیر اقلیدسی:}
\lr{CNN}های مرسوم \emph{نقشه‌های ویژگی} را بر روی $\R^d$ در نظر می‌گیرند، که توابعی به شکل زیر هستند:
\begin{align}
	F: \R^d \to \R^c \,.
\end{align}
یک کانولوشن (در واقع یک همبستگی) با یک کرنل با مقدار ماتریسی $K: \R^d \to \R^{\cout\times\cin}$ سپس به عنوان تبدیل انتگرالی زیر تعریف می‌شود:
\begin{align}
	\Fout(\mathscr{x}) \,:=\, [K * \Fin] (\mathscr{x}) \,:= \int_{\R^d} K(\mathscr{v})\, \Fin(\mathscr{x} + \mathscr{v})\ d\mathscr{v} \,,
\end{align}
که یک نقشه ویژگی ورودی با $\cin$ کانال را به یک نقشه ویژگی خروجی با $\cout$ کانال نگاشت می‌کند.
می‌توان نشان داد که این عملیات عمومی‌ترین نگاشت خطی و هم‌متغیر نسبت به انتقال بین نقشه‌های ویژگی است~\cite{Cohen2019-generaltheory}.%
\footnote{
	برای کلیت کامل، در واقع باید کرنل‌ها را به معنای توزیعی (توابع تعمیم‌یافته) مجاز دانست.
}


\lr{CNN}های هدایت‌پذیر اقلیدسی~\cite{Cohen2017-STEER,3d_steerableCNNs,Weiler2019_E2CNN} \lr{CNN}های مرسوم را به شبکه‌های کانولوشنی تعمیم می‌دهند که تحت عمل گروه‌های آفین بر روی میدان‌های ویژگی هم‌متغیر هستند.
گروه‌های آفین $\R^d$ در اینجا به عنوان حاصلضرب‌های نیمه‌مستقیم به شکل زیر تعریف می‌شوند:
\begin{align}\label{eq:AffG_def}
	\Aff(G)\ :=\ \Trans_d \rtimes G \,,
\end{align}
که در آن $G\leq\GL{d}$ است.
% NOTE: Changed \O{d} to O(d) to fix math mode error.
گروه‌های آفین شامل ایزومتری‌های $\R^d$ که در شکل~\ref{fig:isometries_plane} به تصویر کشیده شده‌اند، به عنوان یک مورد خاص برای $G\leq O(d)$ هستند اما گروه‌های نقطه‌ای (گروه‌های ساختار) $G$ عمومی‌تری، به عنوان مثال مقیاس‌بندی یکنواخت $\Scale$، را نیز مجاز می‌دانند.
معادلات زیر یک نمای کلی از رایج‌ترین گروه‌های آفین در ادبیات (تا گسسته‌سازی‌ها) و راه‌های جایگزین نوشتن آنها را ارائه می‌دهند (با فرض اینکه $\IsomGM$ توسط \lr{G}-ساختارهای نامتغیر نسبت به $\Aff(G)$ تعیین می‌شود؛ به زیر مراجعه کنید):
\begin{alignat}{5}
	&\Aff(\{e\}) \ &&=\ \Trans_d               \ &&=\ (\R^d,+)\ &&=\ \IsomeM \notag \\
	&\Aff(\Flip) \ &&=\ \Trans_d \rtimes \Flip   &&                \ &&=\ \IsomRM \notag \\
	&\Aff(\SO{d})\ &&=\ \Trans_d \rtimes \SO{d} \ &&=\ \SE{d}  \ &&=\ \IsomSOM\ &&=\ \Isom_+\!(\R^d) \\
	&\Aff(O(d)) \ &&=\ \Trans_d \rtimes O(d)  \ &&=\  \E{d}  \ &&=\ \IsomOM \ &&=\ \Isom(\R^d) \notag \\
	&\Aff(\Scale)\ &&=\ \Trans_d \rtimes \Scale \notag
\end{alignat}
گروه $\Aff(\GL{d})$ شامل \emph{تمام} تبدیلات آفین $\R^d$ است.
از آنجایی که گروه‌های آفین به عنوان حاصلضرب‌های نیمه‌مستقیم تعریف می‌شوند، هر یک از عناصر آنها $tg \in \Aff(G)$ را می‌توان به طور منحصر به فرد به یک انتقال $t\in \Trans_d$ و یک عنصر گروه نقطه‌ای $g\in G$ تجزیه کرد.
عمل (کانونی) آنها بر روی $\R^d$ به صورت زیر داده می‌شود:
\begin{align}
	\Aff(G)\times\R^d \to \R^d, \quad (tg,\, \mathscr{x})\ \mapsto\ g\mkern1mu \mathscr{x} + t \,.
\end{align}
عمل یک عنصر گروه معکوس $(tg)^{-1}$ به صورت زیر به دست می‌آید:
\begin{align}
	\big( (tg)^{-1},\, \mathscr{x}\big)\ \mapsto\ g^{-1} (\mathscr{x} - t) \,.
\end{align}


یک میدان ویژگی از نوع $\rho$ بر روی $\R^d$ طبق \emph{نمایش القایی} $\Ind_G^{\Aff(G)} \rho$ از $\rho$ همانطور که توسط
\begin{align}\label{eq:induced_rep_affine}
	(tg) \mkern2mu\rhd_\rho F \ :=\ \big[\Ind_G^{\Aff(G)} \rho\big](tg)\, F \ :=\ \rho(g)\, F\, (tg)^{-1} \,,
\end{align}
مشخص شده است، تبدیل می‌شود، که می‌توان آن را به عنوان مشابه عمل مستقل از مختصات بر روی مقاطع در معادله~\eqref{eq:pushforward_section_A} در نظر گرفت.%
\footnote{
	نمایش‌های القایی به روشی مشابه با آنچه در شکل~\ref{fig:active_TpM_equivariance} نشان داده شده است، بر روی میدان‌ها عمل می‌کنند.
	برخلاف تبدیل در این شکل، نمایش‌های القایی علاوه بر این به انتقال‌ها نیز اجازه می‌دهند (قانون تبدیل در بخش~\ref{sec:gauge_conv} محدودیت $\Res_G^{\Aff(G)}\Ind_G^{\Aff(G)}\!\rho$ از نمایش القایی به $G$ است، یعنی نمایش القایی بدون انتقال‌ها).
}
یک کانولوشن با یک کرنل \lr{G}-هدایت‌پذیر $K \in \KG$ نسبت به این اعمال بر روی میدان ورودی و خروجی هم‌متغیر است، یعنی،
\begin{align}\label{eq:Euclidean_conv_equiv_in_coords_Rd}
	K *\, \big(tg \,\rhd_{\rhoin}\! \Fin \big)\ =\ tg \,\rhd_{\rhoout} \big( K*\Fin \big) \qquad \forall\ \ tg \,\in\, \Aff(G) \,.
\end{align}
این به راحتی با یک محاسبه صریح بررسی می‌شود:
\begin{align}
	\pig[ K * (tg \rhd_{\rhoin}\! \Fin) \pig] (\mathscr{x})
	\ =&\ \pig[ K * \big(\rhoin(g)\, \Fin\, (tg)^{-1}\big) \pig] (\mathscr{x}) \notag \\
	\ =&\ \int_{\R^d} K(\mathscr{v})\; \rhoin(g)\, \Fin \big((tg)^{-1} (\mathscr{x} + \mathscr{v}) \big)\ d\mathscr{v} \notag \\
	\ =&\ \int_{\R^d} K(\mathscr{v})\; \rhoin(g)\, \Fin \big(g^{-1}( \mathscr{x} + \mathscr{v} - t)\big)\ d\mathscr{v} \notag \\
	\ =&\ \int_{\R^d} K(g \tilde{\mathscr{v}})\; \rhoin(g)\, \Fin \big(g^{-1}(\mathscr{x} - t) + \tilde{\mathscr{v}}\big)\ \detg\ d\tilde{\mathscr{v}} \notag \\
	\ =&\ \int_{\R^d} \rhoout(g)\, K(\tilde{\mathscr{v}})\; \Fin \big(g^{-1}(\mathscr{x} - t) + \tilde{\mathscr{v}}\big)\ d\tilde{\mathscr{v}} \notag \\
	\ =&\ \rhoout(g)\, \big[ K * \Fin \big] \big(g^{-1}(\mathscr{x}-t)\big) \notag \\
	\ =&\ tg \rhd_{\rhoout} \big[K * \Fin \big] (\mathscr{x}) \,,
\end{align}
که از \lr{G}-هدایت‌پذیری $K$ در مرحله پنجم استفاده کرد و برای هر $\mathscr{x} \in \R^d$ و هر $tg\in \Aff(G)$ برقرار است.
همانطور که در~\cite{3d_steerableCNNs} اثبات شده است، چنین
\emph{کانولوشن‌های \lr{G}-هدایت‌پذیر، عمومی‌ترین نگاشت‌های خطی هم‌متغیر نسبت به $\Aff(G)$ بین میدان‌های ویژگی اقلیدسی هستند}.%
\footnote{
	با فرض اینکه میدان‌های ویژگی طبق نمایش القایی، معادله~\eqref{eq:induced_rep_affine}، تبدیل می‌شوند، که برای رسیدن به یک کانولوشن لازم است.
}%
\footnote{
	این گزاره، گزاره شناخته‌شده‌ای را تعمیم می‌دهد که کانولوشن‌های اقلیدسی مرسوم، عمومی‌ترین نگاشت‌های خطی هم‌متغیر نسبت به انتقال بین توابع (یا نقشه‌های ویژگی) در فضاهای اقلیدسی هستند، که برای $G=\{e\}$ بازیابی می‌شود.
}




\paragraph{رابطه با کانولوشن‌های \lr{GM} اقلیدسی:}
این کانولوشن‌های هدایت‌پذیر بر روی $\R^d$ چگونه به کانولوشن‌های $\GM$ بر روی فضاهای اقلیدسی ${M = \Euc_d}$ مرتبط می‌شوند؟
این واقعیت که کانولوشن‌های هدایت‌پذیر به کرنل‌های \lr{G}-هدایت‌پذیر متکی هستند، نشان می‌دهد که آنها نه تنها به صورت سراسری نسبت به $\Aff(G)$ هم‌متغیر هستند، بلکه (به طور کلی‌تر) به صورت محلی نسبت به $G$ نیز هم‌متغیر هستند.
برای برقراری ارتباط بین \lr{CNN}های کلاسیک \lr{G}-هدایت‌پذیر بر روی $\R^d$ و کانولوشن‌های $\GM$ مستقل از مختصات ما، باید ساختار هندسی را که به طور ضمنی توسط اولی در نظر گرفته شده است، شناسایی کنیم.


به طور کلی، $\R^d$ به طور کانونی با یک $\{e\}$-ساختار، که در شکل~\ref{fig:G_structure_R2_1} به تصویر کشیده شده است، مجهز است.%
\footnote{
	به طور رسمی، $\{e\}$-ساختار کانونی $\R^d$ به صورت زیر به وجود می‌آید:
	فضای برداری $M=\R^d$ خود با یک پایه کانونی، که توسط بردارهای پایه $e_i \in \R^d$ با عناصر $(e_i)_j = \delta_{ij}$ داده می‌شود، همراه است.
	قاب‌های مرجع کانونی فضاهای مماس $T_p\R^d$ از این پایه از طریق ایزومورفیسم‌های کانونی
	$\iota_{\R^d,p}: T_p{\R^d} \xrightarrow{\sim} \R^d$ از معادله~\eqref{eq:canonical_iso_TRk_Rk} به دست می‌آیند.
	به طور شهودی، قاب‌های محلی فضاهای مماس $T_p\R^d$ با قاب سراسری $\R^d$ «هم‌تراز» هستند.
	این معادل با معرفی نگاشت همانی به عنوان چارت مختصاتی سراسری $x = \id_{\R^d} : M=\R^d \to \R^d$ و سپس تعریف $\{e\}$-ساختار کانونی به عنوان میدان پایه‌های مختصاتی القایی $\big[\frac{\partial}{\partial x_i} \big]_{i=1}^d$ است.
}
علاوه بر این، با یک متریک ریمانی متناظر با ضرب داخلی استاندارد $\R^d$ همراه است.%
\footnote{
	این متریک استاندارد $\eta$ به عنوان پول‌بک ضرب داخلی استاندارد
	$\langle\cdot,\cdot\rangle_{\R^d}: \R^d\times\R^d \to \R$
	بر روی $\R^d$ از طریق ایزومورفیسم‌های کانونی
	$\iota_{\R^d,p}: T_p{\R^d} \xrightarrow{\sim} \R^d$ از معادله~\eqref{eq:canonical_iso_TRk_Rk}
	به فضاهای مماس تعریف می‌شود.
	بنابراین برای هر $v,w\in T_p\R^d$ به صورت
	${\eta_p(v,w) := \langle \iota_{\R^d,p}(v) ,\, \iota_{\R^d,p}(w) \rangle_{\R^d}}$ داده می‌شود.
}
اتصال لوی-چیویتای متناظر، منتقل‌کننده‌های موازی را در شکل~\ref{fig:transport_flat} ایجاد می‌کند، که بردارها را به معنای معمول در فضاهای اقلیدسی موازی نگه می‌دارد.
هنگامی که نسبت به قاب‌های $\{e\}$-ساختار کانونی بیان می‌شود، منتقل‌کننده‌های موازی بدیهی می‌شوند و بنابراین حذف می‌شوند.
نگاشت‌های نمایی به یک جمع ساده کاهش می‌یابند (پس از اعمال برخی ایزومورفیسم‌ها، به زیر مراجعه کنید).


درحالی‌که ما یک $\{e\}$-ساختار بر روی $\R^d$ داریم، کانولوشن‌های $\GM$ به \lr{G}-ساختارهای کمتر خاصی متکی هستند.
این \lr{G}-ساختارها را می‌توان به عنوان \emph{بالابرهای} $G$ (کانونی)
\begin{align}\label{eq:G_lifted_G_structure_Rd}
	\GM\ =\ \eM\lhd G\ :=\ \pig\{\, [e_i]_{i=1}^d \lhd g \;\pig|\; [e_i]_{i=1}^d \in \eM,\ g\in G \,\pig\}
\end{align}
از $\{e\}$-ساختار کانونی $\eM$ از $\R^d$ در نظر گرفت.
به طور شهودی، این \lr{G}-ساختارهای بالابر شده با افزودن هر قاب \lr{G}-مرتبط دیگر (مدار راست-\lr{G} آن در $\FM$) به هر قاب مرجع کانونی در $\eM$ تعریف می‌شوند.
شکل~\ref{fig:G_structures_R2_main} چنین \lr{G}-ساختارهای بالابر شده‌ای را برای گروه‌های ساختار مختلف نشان می‌دهد.
همانطور که در قضیه~\ref{thm:Aff_GM_in_charts} در زیر اثبات شده است، آنها تحت عمل $\Aff(G)$ نامتغیر هستند -- که در قضیه~\ref{thm:affine_equivariance_Euclidean_GM_conv} نشان داده شده است که هم‌متغیری $\Aff(G)$ کانولوشن‌ها را توضیح می‌دهد.
آنها علاوه بر این با اتصال لوی-چیویتا \lr{G}-سازگار هستند.


ادعاهای مطرح شده در اینجا در دو بخش بعدی با دقت بیشتری مورد بحث قرار می‌گیرند.
این فرمول‌بندی رسمی، با این حال، برای درک طبقه‌بندی ما از \lr{CNN}های اقلیدسی در ادبیات به شدت ضروری نیست، به طوری که خواننده می‌تواند از آنها عبور کرده و مستقیماً به بخش~\ref{sec:euclidean_literature} برود.