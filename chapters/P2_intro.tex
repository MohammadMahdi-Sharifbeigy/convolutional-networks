%!TEX root=../GaugeCNNTheory.tex

\mypart{نظریه شبکه‌های کانولوشنی مستقل از مختصات}
\label{part:bundle_theory}

بخش~\ref{part:local_theory} میدان‌های ویژگی و لایه‌های شبکه را بر حسب عبارات مختصاتی آن‌ها نسبت به انتخابی از گیج روی محله‌های \emph{محلی} $U\subseteq M$ معرفی کرد.
از آنجا که وجود گیج‌های \emph{سراسری} به طور کلی از نظر توپولوژیکی مانع دارد، نمایش‌های مختصاتی سراسری میدان‌های ویژگی به طور کلی وجود ندارند.
بخش~\ref{part:local_theory} این مسئله را با تجمیع محتوای سراسری میدان‌های ویژگی از عبارات مختصاتی محلی آن‌ها نسبت به اطلسی از گیج‌هایی که~$M$ را پوشش می‌دهند، برطرف کرد.
یک جایگزین ظریف‌تر، تعریف میدان‌های ویژگی سراسری در یک فرمالیسم انتزاعی و \emph{مستقل از مختصات} بر حسب بندل‌های فیبر است.
تسهیم‌های بندل امکان بازیابی عبارات مختصاتی محلی میدان‌های ویژگی و لایه‌های شبکه را فراهم می‌کنند.

\etocsettocdepth{2}
\etocsettocstyle{}{} % from now on only local tocs
\localtableofcontents

~

بخش‌های زیر توصیف سراسری و مستقل از مختصات شبکه‌های عصبی و فضاهای ویژگی از بخش~\ref{part:local_theory} را توسعه می‌دهند.
بخش~\ref{sec:bundles_fields} بندل‌های فیبر، به ویژه بندل مماس، $G$-ساختارها و بندل‌های بردار ویژگی $G$-مرتبط را معرفی می‌کند.
عملیات شبکه عصبی مانند تبدیل‌های میدان کرنل و کانولوشن‌های $\GM$ در بخش~\ref{sec:gauge_CNNs_global} تعریف می‌شوند.
بخش~\ref{sec:isometry_intro} هموردایی ایزومتری این عملیات را بررسی می‌کند.