%!TEX root=../GaugeCNNTheory.tex


\section{انتگرال‌گیری روی فضاهای مماس}
\label{apx:tangent_integral}


روی یک منیفلد ریمانی $(M,\eta)$، چگالی حجم%
\footnote{
    برخلاف یک \emph{فرم} حجم $\omega$, \emph{چگالی‌های} حجم $|\omega|$ یک حجم مثبت را به هر چارچوب اختصاص می‌دهند.
    آنها هم روی منیفلدهای جهت‌پذیر و هم غیرجهت‌پذیر وجود دارند.
}
$dp$ روی $M$ به طور یکتا با این الزام مشخص می‌شود که به \emph{چارچوب‌های راست‌هنجار} $\big[ e_1^O, \,\dots,\, e_d^O \big]$ نسبت به متریک $\eta$ \emph{حجم واحد} اختصاص داده شود:
\begin{align}
    dp\big(e_1^O, \,\dots,\, e_d^O \big) \ =\ 1
    \mkern36mu &\textup{برای هر چارچوب \emph{راست‌هنجار}
    $\ \big[ e_1^O, \,\dots,\, e_d^O \big]\ $ از $\ \TpM$}
\intertext{
به طور مشابه، یک چگالی حجم $dv$ روی \emph{فضاهای مماس} $\TpM$ از یک منیفلد ریمانی به طور یکتا با اختصاص حجم واحد به چارچوب‌های راست‌هنجار آن نسبت به $\eta_p$ تعریف می‌شود:
}
    dv\big(\mathfrak{e}_1^O, \,\dots,\, \mathfrak{e}_d^O \big) \ =\ 1
    \mkern36mu &\textup{برای هر چارچوب \emph{راست‌هنجار}
    $\ \big[ \mathfrak{e}_1^O, \,\dots,\, \mathfrak{e}_d^O \big]\ $ از $\ \TvTpM$}
\end{align}


برای جلوگیری از یک بحث بی‌جهت پیچیده در مورد کلاف مماس مضاعف $\TTM$, ما انتگرال‌گیری روی $\TpM$ را به طور معادل با پول‌بک کردن آن از طریق یک پیمانه \emph{ایزومتریک} (و در نتیجه حجم-نگهدار) به~$\R^d$ تعریف می‌کنیم.
فرض کنید $\psiTMp^O$ چنین پیمانه ایزومتریکی از یک اطلس $\OO{d}$ باشد، که چارچوب‌های راست‌هنجار را در $\TpM$ با چارچوب‌های راست‌هنجار در $\R^d$ یکی می‌گیرد.
سپس انتگرال یک تابع $f: \TpM \to \R$ از طریق پول‌بک آن تعریف می‌شود
\begin{align}
    \int_{\TpM} f(v)\, dv
    \ :=&\ \int_{\R^d} f \mkern-2mu\circ\mkern-2mu \big(\psiTMp^O \big)^{\!-1} (v^O)\,\ dv^O \notag \\
    \ =&\ \int_{\R^d} f^O(v^O)\, dv^O \,,
\end{align}
که در آن عبارت مختصاتی $f^O := f \circ \big(\psiTMp^O \big)^{-1} : \R^d \to \R$ از $f$ را طبق معمول تعریف کردیم.
این واقعیت که $\psiTMp^O$ ایزومتریک است در اینجا تضمین می‌کند که $dv$ واقعاً حجم واحد را به چارچوب‌های راست‌هنجار اختصاص می‌دهد اگر $dv^O$ این کار را بکند.
از آنجا که دومی فقط اندازه لبگ استاندارد روی $\R^d$ است، این مورد برقرار است.



حال فرض کنید $\psiTMp^A$ \emph{هر} پیمانه‌ای در $p$ باشد، که ممکن است بخواهیم انتگرال را نسبت به آن بیان کنیم.
نگاشت گذار بین هر دو مختصاتی‌سازی به سادگی با تبدیل پیمانه
$v^O = \psi^O \circ (\psi^A )^{-1} (v^A) = g^{OA}_p (v^A)$ داده می‌شود.
طبق قوانین استاندارد برای تغییر متغیرها در انتگرال‌های چندبعدی، دیفرانسیل‌ها باید مطابق با دترمینان ژاکوبین این تبدیل تغییر کنند تا حجم حفظ شود.
از آنجا که تبدیل خطی است، ژاکوبین با خود $g_p^{OA}$ داده می‌شود، به طوری که به دست می‌آوریم
\begin{align}\label{eq:integral_gOA}
    \int_{\TpM} f(v)\, dv
    \ &=\ \int_{\R^d} f^A(v^A)\; \pig|\mkern-2mu \det \!\big(g_p^{OA} \big)\mkern-1mu\pig|\; dv^A \,.
\end{align}


از طریق تبدیل پیمانه، این عبارت هنوز به انتخاب دلخواه از پیمانه ایزومتریک~$\psiTMp^O$ بستگی دارد.
این وابستگی را می‌توان با بیان مستقیم اندازه انتگرال بر حسب متریک~به صورت
\begin{align}\label{eq:integral_etaA}
    \int_{\TpM} f(v)\, dv
    \ &=\ \int_{\R^d} f^A(v^A)\; \sqrt{|\eta_p^A|}\ dv^A \,,
\end{align}
از بین برد، که در آن ضریب
\begin{align}\label{eq:volume_element_def}
    \sqrt{|\eta_p^A|}\ :=\ \sqrt{\mkern2mu \pig|\det\!\pig( \big[ \eta_p(e_i^A, e_j^A) \big]_{ij} \pig)\pig| \,}
\end{align}
حجم (مطلق) چارچوب مرجع $[e_i^A]_{i=1}^d$ را نسبت به متریک $\eta$ اندازه‌گیری می‌کند.
برای تأیید برابری سمت راست معادلات~\eqref{eq:integral_gOA} و~\eqref{eq:integral_etaA}، ما متریک $\eta_p$ از $\TpM$ را بر حسب حاصلضرب داخلی استاندارد $\langle\cdot,\, \cdot\rangle$ از~$\R^d$ بیان می‌کنیم، که دوباره با استفاده از پیمانه ایزومتریک $\psiTMp^O$ از اطلس $\OO{d}$ انجام می‌شود:
\begin{align}
    \eta_p\big( e_i^A,\, e_j^A \big)
    \ &=\ \pig\langle \psiTMp^O\big( e_i^A\big),\; \psiTMp^O\big( e_j^A\big) \pig\rangle \notag \\
    \ &=\ \pig\langle \psiTMp^O \circ \big(\psiTMp^A \big)^{-1} (\epsilon_i),\; \psiTMp^O \circ \big(\psiTMp^A \big)^{-1} (\epsilon_j) \pig\rangle \notag \\
    \ &=\ \pig\langle g_p^{OA} \epsilon_i,\; g_p^{OA} \epsilon_j \pig\rangle \notag \\
    \ &=\ \epsilon_i^\top \big(g_p^{OA} \big)^\top \, g_p^{OA} \epsilon_j \notag \\
    \ &=\ \Big( \big(g_p^{OA} \big)^\top \, g_p^{OA} \Big)_{ij}
\end{align}
بنابراین قدر مطلق دترمینان در معادله~\eqref{eq:volume_element_def} با
\begin{align}
    \pig|\det\!\pig( \big[ \eta_p(e_i^A, e_j^A) \big]_{ij} \pig)\pig|
    \ &=\ \pig| \det\pig( \big(g_p^{OA} \big)^\top \, g_p^{OA} \pig) \pig| \notag \\
    \ &=\ \pig| \det\pig( \big(g_p^{OA} \big)^\top \pig) \, \det\pig( g_p^{OA} \pig) \pig| \notag \\
    \ &=\ \pig| \det\big( g_p^{OA} \big) \big|^2 \,,
\end{align}
داده می‌شود، که از آن برابری سمت راست معادلات~\eqref{eq:integral_gOA} و~\eqref{eq:integral_etaA} با گرفتن ریشه دوم نتیجه می‌شود.


از آنجا که ضرایب $\sqrt{|\eta_p^A|}$ و $\sqrt{|\eta_p^B|}$ حجم‌های چارچوب‌های مربوطه خود را اندازه‌گیری می‌کنند، به راحتی می‌توان نشان داد که آنها با تغییر حجم \emph{وارون} $\big|\mkern-2mu \det g_p^{BA} \big|$ به هم مرتبط هستند:
\begin{alignat}{3}
    \sqrt{|\eta_p^B|}\ &=\ \frac{1}{\big|\mkern-2mu \det g_p^{BA} \big|}\, \sqrt{|\eta_p^A|}
    && \qquad\quad \big(\Rightarrow \quad & -1 & \textup{-چگالی} \,\big)
\intertext{
این به همراه فرمول معمول تغییر متغیرها
}
    dv^B\ &=\ \big|\mkern-2mu \det g_p^{BA} \big|\ dv^A
    && \qquad\quad \big(\Rightarrow \quad  & +1 & \textup{-چگالی} \,\big) \,,
\intertext{
دلالت بر این دارد که مختصاتی‌سازی‌های عنصر حجم ریمانی $dv$ بنا به طراحی تحت تبدیلات پیمانه ناوردا هستند، یعنی،
}
    \qquad\qquad
    \sqrt{|\eta_p^B|}\ dv^B\ &=\ \sqrt{|\eta_p^A|}\ dv^A
    && \qquad\quad \big(\Rightarrow \quad  & 0 & \textup{-چگالی} \,\big) \,.
\end{alignat}
این رابطه تضمین می‌کند که انتگرال‌گیری در معادله~\eqref{eq:integral_etaA} خوش‌تعریف است، یعنی مستقل از مختصات است.