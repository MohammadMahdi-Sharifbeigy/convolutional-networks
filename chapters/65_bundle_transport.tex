%!TEX root=../GaugeCNNTheory.tex


\subsection{منتقل‌کننده‌های موازی روی کلاف‌های الحاقی}
\label{sec:bundle_transport}

بخش~\ref{sec:transport_local} مقدمه‌ای شهودی بر انتقال موازی بردارهای مماس و بردارهای ویژگی در امتداد یک مسیر~$\gamma$ از $q\in M$ به $p\in M$ ارائه داد.
در اینجا به طور خلاصه بحث می‌کنیم که چگونه منتقل‌کننده‌های موازی مستقل از مختصات روی کلاف‌های تاری یکدیگر را القا می‌کنند و عبارات مختصاتی آن‌ها را نسبت به بدیهی‌سازی‌های داده‌شده استخراج می‌کنیم.
ما با فرض \emph{داده‌شده} بودن منتقل‌کننده‌های مستقل از مختصات
\begin{alignat}{2}
	\PTMgamma:&\ &\TqM &\to \TpM \\
	\intertext{
		روی کلاف مماس $\TM$ شروع می‌کنیم و توضیح می‌دهیم که چگونه آن‌ها منتقل‌کننده‌های
	}
	\PFMgamma:&\ &\FqM &\to \FpM \\
	\intertext{
		روی کلاف قاب $\FM$ را \emph{القا} می‌کنند.
		اگر این منتقل‌کننده‌ها، همانطور که در ادامه بحث می‌شود، با \lr{G}-ساختار انتخاب‌شده \lr{G}-\emph{سازگار} باشند، در ادامه منتقل‌کننده‌های
	}
	\PGMgamma:&\ &\GqM &\to \GpM \\
	\PAgamma:&\  &\A_q &\to \A_p
\end{alignat}
را روی کلاف‌های \lr{G}-الحاقی $\GM$ و $\A$ القا می‌کنند.
در عمل، یعنی در مرور ادبیات ما در بخش~\ref{part:literature_review}، اکثر شبکه‌های کانولوشنی یا منتقل‌کننده‌هایی را فرض می‌کنند که بر اساس اتصال لوی-چیویتا هستند یا یک اتصال بدیهی.

یک تعریف رسمی‌تر از منتقل‌کننده‌های کلاف ممکن است مسیر متفاوتی را در پیش بگیرد، که با معرفی یک اتصال به اصطلاح اصلی ارسمان روی کلاف اصلی \lr{G} یعنی $\GM$ شروع می‌شود (که بنا به تعریف، \lr{G}-سازگار خواهد بود).
چنین اتصال ارسمانی می‌تواند یا با انتخاب یک زیرکلاف افقی $HGM$ از کلاف مماس $TGM$ از $\GM$ تعریف شود یا، به طور معادل، با یک ۱-فرم اتصال با مقدار در جبر لی $\omega:TGM\to\mathfrak{g}$ روی $\GM$.
انتقال روی $\GM$ متعاقباً از طریق بالابر افقی $\gamma^\uparrow:[0,1]\to \GM$ از منحنی‌های $\gamma:[0,1]\to M$ روی فضای پایه تعریف می‌شود، به طوری که بردارهای مماس بالابر در $\GM$ افقی هستند، یعنی~$\dot{\gamma}^\uparrow \in HGM$.
سپس تمام منتقل‌کننده‌ها روی $\TM$، $\FM$ و $\A$ به عنوان کلاف‌های \lr{G}-الحاقی از منتقل‌کننده‌های روی \lr{G}-ساختار القا می‌شوند.
به جای دنبال کردن این رویکرد رسمی، که نسبتاً فنی خواهد بود و می‌توان آن را در ادبیات یافت~\cite{schullerGeometricalAnatomy2016,wendlLectureNotesBundles2008,husemollerFibreBundles1994a,nakahara2003geometry,marshGaugeTheoriesFiber2016,shoshichikobayashiFoundationsDifferentialGeometry1963}،
ما بر چگونگی ارتباط متقابل منتقل‌کننده‌های مختلف از طریق القای یکدیگر تمرکز می‌کنیم.









\paragraph{انتقال روی \lr{\textit{TM}}:}
برای این منظور، با فرض داده‌شده بودن منتقل‌کننده‌های مستقل از مختصات $\PTMgamma$ روی $\TM$، یک راه میان‌بر را انتخاب می‌کنیم.
به یاد بیاورید که، با داشتن پیمانه‌های $\PsiTM^{\widetilde{A}}$ روی یک همسایگی $U^{\widetilde{A}}$ از $q$ و $\PsiTM^A$ روی یک همسایگی $U^A$ از $p$، منتقل‌کننده بردار مماس طبق معادله~\eqref{eq:transporter_gauge} مختصاتی می‌شود، یعنی،
\begin{align}\label{eq:transporter_gauge_copy}
	g_\gamma^{A\widetilde{A}} \ :=\ \psiTMp^A \circ \PTMgamma \circ \Big( \psiTMq^{\widetilde{A}} \Big)^{-1} \ \in\, \GL{d} \,,
\end{align}
و اینکه مختصاتی‌سازی‌های آن تحت تبدیلات پیمانه در $q$ و $p$ طبق معادله~\eqref{eq:transporter_gauge_trafo} تبدیل می‌شوند:
\begin{align}\label{eq:transporter_gauge_trafo_copy}
	g_\gamma^{B\widetilde{B}} \ =\ g_p^{BA}\, g_\gamma^{A\widetilde{A}} \Big(g_q^{\widetilde{B}\widetilde{A}}\Big)^{-1}
\end{align}
برای نمایش تصویری این تعاریف در قالب یک نمودار جابجایی، به معادله~\eqref{cd:transporter_trivialization} بازمی‌گردیم.








\paragraph{انتقال روی \lr{\textit{FM}}:}
با داشتن منتقل‌کننده روی کلاف مماس، منتقل‌کننده روی کلاف قاب بلافاصله از انتقال محورهای قاب منفرد به دست می‌آید.
در معادلات، فرض کنید $[e_i]_{i=1}^d \in \FqM$ یک قاب در $q$ باشد، سپس محورهای منفرد $e_i$ برای $i=1,\dots,d$ بردارهای مماس در $\TqM$ هستند که می‌توانند از طریق $\PTMgamma$ منتقل شوند.
بنابراین منتقل‌کننده روی کلاف قاب را به صورت زیر تعریف می‌کنیم:%
\footnote{
	انتقال یک قاب در امتداد $\gamma$ یک منحنی $\gamma^\uparrow$ (بالابر افقی) را در $\FM$ توصیف می‌کند.
	فضای تولید شده توسط تمام بردارهای مماس $\dot{\gamma}^\uparrow$ در $TFM$ در امتداد چنین منحنی‌هایی، زیرکلاف افقی $HFM$ از $TFM$ است که در بالا ذکر شد.
}
\begin{align}\label{eq:transporter_FM_def}
	\PFMgamma\!:\ \FqM \to \FpM, \quad
	[e_i]_{i=1}^d \mapsto \PFMgamma\big([e_i]_{i=1}^d\big) := \big[\PTMgamma(e_i)\big]_{i=1}^d
\end{align}
برای استخراج شکل صریح مختصاتی‌سازی آن
$\psiFMp^A \circ \PFMgamma \circ \big(\psiFMq^{\widetilde{A}}\big)^{-1}\! \in \GL{d}$،
عمل آن را بر روی یک عنصر گروهی $h\in \GL{d}$ در نظر بگیرید که نماینده یک قاب بدیهی‌شده از $\R^d$ است که توسط ستون‌های ماتریس $h_{:,i}\in\R^d,\ i=1,\dots,d$ تولید می‌شود:
\begin{alignat}{3}\label{eq:transporter_gauge_FM}
	\Big[ \psiFMp^A \circ \PFMgamma \circ \big(\psiFMq^{\widetilde{A}}\big)^{-1} \Big](h)
	\ &=\ \Big[ \psiFMp^A \circ \PFMgamma \Big] \Big(\! \big[\big(\psiTMq^{\widetilde{A}}\big)^{-1}(h_{:,i})\big]_{i=1}^d \Big)
	\qquad && \big( \text{\small تعریف $\psiFMp^{\widetilde{A}}$، معادلۀ~\eqref{eq:trivialization_FM_p}} \big) \notag \\
	\ &=\ \psiFMp^A \Big( \big[\PTMgamma \circ \big(\psiTMq^{\widetilde{A}}\big)^{-1}(h_{:,i})\big]_{i=1}^d \Big)
	\qquad && \big( \text{\small تعریف $\PFMgamma$، معادلۀ~\eqref{eq:transporter_FM_def}} \big) \notag \\
	\ &=\ \Big( \psiTMp^A \circ \PTMgamma \circ \big(\psiTMq^{\widetilde{A}}\big)^{-1}(h_{:,i}) \Big)_{i=1}^d
	\qquad && \big( \text{\small تعریف $\psiFMp^{\widetilde{A}}$، معادلۀ~\eqref{eq:trivialization_FM_p}} \big) \notag \\
	\ &=\ \Big( g_\gamma^{A\widetilde{A}} (h_{:,i}) \Big)_{i=1}^d
	\qquad && \big( \text{\small بدیهی‌سازی $\PTMgamma$، معادلۀ~\eqref{eq:transporter_gauge_copy}} \big) \notag \\
	\ &=\ g_\gamma^{A\widetilde{A}} \, h
\end{alignat}
بنابراین، مختصاتی‌سازی‌های منتقل‌کننده‌های قاب معادل با مختصاتی‌سازی‌های منتقل‌کننده‌های بردار مماس در معادله~\eqref{eq:transporter_gauge_copy} هستند اما به جای عمل بر روی بردارهای ضریب در $\R^{d}$، بر روی قاب‌های بدیهی‌شده در $\GL{d}$ عمل می‌کنند.
تبدیلات پیمانۀ آن‌ها از نمودار جابجایی
\begin{equation}\label{cd:FM_transport_trivialization}
	\begin{tikzcd}[column sep=53pt, row sep=30, font=\normalsize]
		\GL{d}
		\arrow[dd, "g_q^{\widetilde{B}\widetilde{A}}\cdot\ "']
		\arrow[rrr, "g_\gamma^{A\widetilde{A}}\cdot"]
		& &[-3ex] &
		\GL{d}
		\arrow[dd, "\ g_p^{BA}\cdot"]
		\\
		&
		\FqM
		\arrow[ul, "\psiFMq^{\widetilde{A}}", pos=.45]
		\arrow[dl, "\psiFMq^{\widetilde{B}}"', pos=.45]
		\arrow[r, "\PFMgamma"]
		&
		\FpM
		\arrow[ur, "\psiFMp^A"', pos=.45]
		\arrow[dr, "\psiFMp^B", pos=.45]
		\\
		\GL{d}
		\arrow[rrr, "g_\gamma^{B\widetilde{B}}\cdot"']
		& & &
		\GL{d}
	\end{tikzcd}
	\quad
\end{equation}
دیده می‌شود که با تبدیلات پیمانۀ منتقل‌کننده‌های مختصاتی‌شده روی $\TM$ در معادله~\eqref{eq:transporter_gauge_trafo_copy} منطبق هستند.





\paragraph{سازگاری اتصالات و \lr{\textit{G}}-ساختارها:}

هر انتخابی از اتصال یا تعریف منتقل‌کننده‌ها روی کلاف‌های $\GL{d}$ یعنی $\TM$ و $\FM$ با هر \lr{G}-ساختاری سازگار نیست.
به طور خاص، یک \lr{G}-ساختار ممکن است تحت انتقال قاب‌ها بسته نباشد، یعنی،
درحالی‌که یک قاب در $\GqM \subseteq \FqM$ توسط $\PFMgamma$ به یک قاب در $\FpM$ منتقل می‌شود، این قاب \emph{لزوماً} در $\GpM$ قرار ندارد.%
\footnote{
	بر حسب یک اتصال اصلی ارسمان روی $\FM$، این حالت زمانی رخ می‌دهد که زیرکلاف افقی $HFM \subseteq TFM$ در $TGM \subseteq TFM$ قرار نداشته باشد.
	یک تعریف فوری از انتقال موازی بر حسب انتخاب یک زیرکلاف افقی $HGM$ روی \lr{G}-ساختار، همیشه (بنا به تعریف) منجر به یک انتقال خوش‌تعریف روی $\GM$ خواهد شد.
}
نسبت به بدیهی‌سازی‌های $\GM$، چنین ناسازگاری در منتقل‌کننده‌های مختصاتی‌شده ${g_\gamma^{A\widetilde{A}} \notin G}$ منعکس می‌شود، که ضرب چپ آن‌ها روی تارهای $\R^d$ و $\GL{d}$ از کلاف‌های $\GL{d}$ یعنی $\TM$ و $\FM$ خوش‌تعریف است، اما روی تار $G$ از $\GM$ خوش‌تعریف نیست.
اگر زیرکلاف $\GM$ تحت انتقال موازی روی $\FM$ بسته نباشد، این به معنای آن است که هیچ انتقال متناظر خوش‌تعریفی روی $\GM$ -- و در نتیجه روی هیچ کلاف \lr{G}-الحاقی $\A$ -- وجود ندارد.

به عنوان مثال، اتصال لوی-چیویتا را در فضاهای اقلیدسی در نظر بگیرید، که منتقل‌کننده‌های آن بردارهای مماس و قاب‌ها را به معنای معمول در $\Euc_d$ موازی نگه می‌دارند.
$\{e\}$-ساختار (میدان قاب) در شکل~\ref{fig:frame_field_automorphism_1} تحت این انتقال بسته است و بنابراین سازگار است.
از سوی دیگر، $\{e\}$-ساختار در شکل~\ref{fig:frame_field_automorphism_2} تحت این انتقال بسته نیست و بنابراین با اتصال لوی-چیویتا ناسازگار است.
به طور مشابه، $\SO2$-ساختار روی $S^2$ در شکل~\ref{fig:G_structure_S2_1} با اتصال لوی-چیویتا روی کره سازگار است، در حالی که $\{e\}$-ساختار در شکل~\ref{fig:G_structure_S2_2} سازگار نیست.

خواننده ممکن است بپرسد چه گزاره‌های کلی در مورد سازگاری اتصالات (یا منتقل‌کننده‌ها) و \lr{G}-ساختارها می‌توان بیان کرد.
به طور کلی، اتصال لوی-چیویتا، یا هر اتصال متریک دیگری، با $\O{d}$-ساختار $\OM$ که متناظر با متریک است، سازگار هستند.%
\footnote{
	این گزاره بنا به تعریف برقرار است زیرا اتصالات متریک زوایا و طول‌ها بین بردارها و در نتیجه اورتونرمال بودن قاب‌ها را حفظ می‌کنند.
	علاوه بر این، می‌توان اتصالات متریک را به عنوان اتصالات اصلی ارسمان روی $\OM$ \emph{تعریف} کرد.
}
اگر منیفلد جهت‌پذیر باشد، اتصال لوی-چیویتا علاوه بر این با هر $\SO{d}$-ساختاری که متناظر با متریک باشد، سازگار است.
یک مثال، $\SO2$-ساختار روی $S^2$ در شکل~\ref{fig:G_structure_S2_1} است.
یک شرط لازم (اما نه کافی) برای اینکه یک \lr{G}-ساختار با یک اتصال داده‌شده سازگار باشد این است که گروه هولونومی اتصال، زیرگروهی از گروه ساختار $G$ باشد.

یک مورد خاص مهم، $\{e\}$-ساختارها هستند، زیرا آن‌ها یک \emph{اتصال بدیهی منحصر به فرد} را القا می‌کنند.%
\footnote{
	یک اتصال \emph{بدیهی} است اگر گروه هولونومی آن، یعنی انتقال موازی آن حول هر حلقه بسته، بدیهی باشد~\cite{craneTrivialConnectionsDiscrete2010}.
}%
\footnote{
	فقط یک اتصال اصلی ارسمان $H\eM = T\eM$ را می‌توان روی $\eM$ انتخاب کرد زیرا زیرکلاف عمودی $V\eM$ مقطع صفر از $T\eM$ است.
}
منتقل‌کننده‌های متناظر، قاب‌ها را به گونه‌ای حرکت می‌دهند که با قاب‌های $\{e\}$-ساختار موازی باقی بمانند.
اتصالات بدیهی ممکن است برای نظریه کانولوشن‌های $\GM$ اهمیت خاصی نداشته باشند، با این حال، در واقع توسط بسیاری از شبکه‌های کانولوشنی استفاده می‌شوند.
به طور خاص، هر شبکه‌ای که به یک $\{e\}$-ساختار متکی است، به طور ضمنی یک اتصال بدیهی را فرض می‌کند.
این شامل تمام مدل‌های جدول~\ref{tab:network_instantiations} با $G=\{e\}$ می‌شود، به ویژه آن‌هایی که در بخش‌های~\ref{sec:spherical_CNNs_azimuthal_equivariant} و~\ref{sec:e_surface_conv} بررسی شده‌اند.%
\footnote{
	این مدل‌ها با مدل‌سازی نکردن منتقل‌کننده‌های غیربدیهی بردارهای ویژگی، به طور \emph{ضمنی} یک اتصال بدیهی را فرض می‌کنند:
	آن‌ها ضرایب بردار ویژگی را بدون تبدیل آن‌ها انباشت می‌کنند.
}
توجه داشته باشید که این مدل‌ها اتصال بدیهی را فقط برای انتقال بردار ویژگی خود فرض می‌کنند اما ژئودزیک‌ها را برای پول‌بک منتقل‌کننده، معادله~\eqref{eq:transporter_pullback_in_coords}، بر اساس اتصال اصلی لوی-چیویتا محاسبه می‌کنند.



\paragraph{انتقال روی \lr{\textit{GM}}:}
با فرض اینکه $\GM$ با انتقال روی $\FM$ سازگار است (یعنی تحت آن بسته است)، یک منتقل‌کننده خوش‌تعریف با محدود کردن منتقل‌کننده کلاف قاب به \lr{G}-ساختار داده می‌شود:
\begin{align}\label{eq:transporter_GM_def}
	\PGMgamma := \PFMgamma \big|_{\scalebox{.62}{$\GM$}}:\ \ \GqM \to \GpM
\end{align}
توابع گذار بین مختصاتی‌سازی‌های مختلف $\PGMgamma$ سپس با توابع گذار $\PFMgamma$ و در نتیجه با $\PTMgamma$ نیز منطبق خواهند بود.
ما نمودار جابجایی زیر را به دست می‌آوریم که محدودسازی نمودار در معادله~\eqref{cd:FM_transport_trivialization} از $\FqM$، $\FpM$ و $\GL{d}$ به $\GqM$، $\GpM$ و $G$ را به تصویر می‌کشد:
\begin{equation}\label{cd:transport_GM_triv}
	\begin{tikzcd}[column sep=60pt, row sep=30, font=\normalsize]
		G
		\arrow[dd, "g_q^{\widetilde{B}\widetilde{A}}\cdot\ "']
		\arrow[rrr, "g_\gamma^{A\widetilde{A}}\cdot"]
		& &[-3ex] &
		G
		\arrow[dd, "\ g_p^{BA}\cdot"]
		\\
		&
		\GqM
		\arrow[ul, "\psiGMq^{\widetilde{A}}", pos=.45]
		\arrow[dl, "\psiGMq^{\widetilde{B}}"', pos=.45]
		\arrow[r, "\PGMgamma"]
		&
		\GpM
		\arrow[ur, "\psiGMp^A"', pos=.45]
		\arrow[dr, "\psiGMp^B", pos=.45]
		\\
		G
		\arrow[rrr, "g_\gamma^{B\widetilde{B}}\cdot"']
		& & &
		G
	\end{tikzcd}
	\quad
\end{equation}
ما در ادامه این کار فرض می‌کنیم که انتقال روی $\GM$ خوش‌تعریف است.





\paragraph{انتقال روی $\A$:}
اگر منتقل‌کننده‌های یک اتصال روی $\GM$ خوش‌تعریف باشند، آن‌ها منتقل‌کننده‌هایی را روی هر کلاف \lr{G}-الحاقی، از جمله کلاف‌های بردار ویژگی $\A=(\GM\times\R^c)/\!\sim_\rho$ القا می‌کنند.
فرض کنید $f_q := \big[[e_i]_{i=1}^d,\,\mathscr{f}\:\!\big]$ یک بردار ویژگی مستقل از مختصات در $\A_q$ باشد.
انتقال موازی آن توسط آن کلاس هم‌ارزی داده می‌شود که با ثابت نگه داشتن برخی ضرایب نماینده $\mathscr{f}\in \R^c$ و انتقال قاب متناظر $[e_i]_{i=1}^d$ تعریف می‌شود:
\begin{align}\label{eq:transporter_A_def}
	\PAgamma: \A_q &\to \A_p, \quad
	f_q \,\mapsto\, \PAgamma(f_q) \,:=\, \pig[\PGMgamma\big([e_i]_{i=1}^d),\: \mathscr{f} \,\pig]
\end{align}
در بخش~\ref{sec:transport_local} ما ادعا کردیم که منتقل‌کننده ضرایب عددی بردار ویژگی توسط $\rho\big(g_\gamma^{A\widetilde{A}}\big)$ داده می‌شود به شرطی که $g_\gamma^{A\widetilde{A}}\in G$ باشد، که این حالت زمانی است که انتقال روی $\GM$ خوش‌تعریف باشد.
این عبارت مختصاتی از $\PAgamma$ را می‌توان با ارزیابی گام به گام عمل
$\psiAp^A \circ \PAgamma \circ \big(\psiAq^{\widetilde{A}}\big)^{-1} \in\, \rho(G)\, \leq\, \GL{c}$
روی یک بردار ضریب ویژگی $\mathscr{f}\in\R^c$ استخراج کرد:

\begin{alignat}{3}\label{eq:transporter_gauge_A}
	\Big[\psiAp^A \circ \PAgamma \circ \big(\psiAq^{\widetilde{A}}\big)^{-1}\Big] (\mathscr{f})
	\ &=\ \Big[\psiAp^A \circ \PAgamma\Big] \big(\big[\sigma^{\widetilde{A}}(q),\, \mathscr{f}\,\big]\big)
	\qquad && \big( \text{\small تعریف $\big(\psiAp^{\widetilde{A}}\big)^{-1}$، معادلۀ~\eqref{eq:trivialization_A_p_inv}} \big) \\
	\ &=\ \psiAp^A  \Big(\big[\PGMgamma\big(\sigma^{\widetilde{A}}(q)\big),\, \mathscr{f}\,\big]\Big)
	\qquad && \big( \text{\small تعریف $\PAgamma$، معادلۀ~\eqref{eq:transporter_A_def}} \big) \notag \\
	\ &=\ \rho\Big(\psiGMp^A \circ \PGMgamma \circ \sigma^{\widetilde{A}}(q)\Big) \cdot \mathscr{f}
	\qquad && \big( \text{\small تعریف $\psiAp^A$، معادلۀ~\eqref{eq:trivialization_A_p}} \big) \notag \\
	\ &=\ \rho\Big(\psiGMp^A \circ \PGMgamma \circ \big(\psiGMq^{\widetilde{A}}\big)^{-1}(e)\Big) \cdot \mathscr{f}
	\qquad && \big( \text{\small تعریف مقطع همانی $\sigma^{\widetilde{A}}$، معادلۀ~\eqref{eq:identity_section_def}} \big) \notag \\
	\ &=\ \rho\big(g_\gamma^{A\widetilde{A}}\big) \!\cdot\! \mathscr{f}
	\qquad && \big( \text{\small $\PGMgamma$ در مختصات، معادلۀ~\eqref{cd:transport_GM_triv}} \big) \notag
\end{alignat}
نمودار جابجایی
\begin{equation}
	\begin{tikzcd}[column sep=65pt, row sep=30, font=\normalsize]
		\R^c
		\arrow[dd, "\rho\big(g_q^{\widetilde{B}\widetilde{A}}\big)\ "']
		\arrow[rrr, "\rho\big(g_\gamma^{A\widetilde{A}}\big)"]
		& &[-3ex] &
		\R^c
		\arrow[dd, "\ \rho\big(g_p^{BA}\big)"]
		\\
		&
		\A_q
		\arrow[ul, "\psiAq^{\widetilde{A}}"]
		\arrow[dl, "\psiAq^{\widetilde{B}}"']
		\arrow[r, "\PAgamma"]
		&
		\A_p
		\arrow[ur, "\psiAp^A"']
		\arrow[dr, "\psiAp^B"]
		\\
		\R^c
		\arrow[rrr, "\rho\big(g_\gamma^{B\widetilde{B}}\big)"']
		& & &
		\R^c
	\end{tikzcd}
\end{equation}
نشان می‌دهد که تبدیلات پیمانه منتقل‌کننده‌های بردار ویژگی مختصاتی‌شده به صورت زیر است:
\begin{align}
	\rho\big(g_\gamma^{B\widetilde{B}}\big)
	\ =\
	\rho\big(g_p^{BA}\big)
	\rho\big(g_\gamma^{A\widetilde{A}}\big)
	\rho\big(g_q^{\widetilde{B}\widetilde{A}}\big)^{-1}
\end{align}
توجه داشته باشید که این قانون تبدیل با قانون تبدیل در معادله~\eqref{eq:transporter_gauge_trafo_copy} سازگار است.