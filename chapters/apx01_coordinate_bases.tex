%!TEX root=../GaugeCNNTheory.tex


\section{ارتباط با فرمالیسم چارت مختصاتی در هندسه دیفرانسیل}
\label{apx:coordinate_bases}


این پیوست به منظور برقراری ارتباط بین \emph{فرمالیسم کلاف}، که زیربنای نظریه \lr{CNN}های مستقل از مختصات است، و \emph{فرمالیسم چارت مختصاتی}، که احتمالاً در اولین مطالعه هندسه دیفرانسیل با آن مواجه می‌شویم، عمل می‌کند.
تفاوت اصلی بین این دو این است که فرمالیسم کلاف به نقاط~$p$ از فضای پایه $M$ به روشی \emph{مستقل از مختصات} اشاره می‌کند.
در صورت نیاز، مختصات مستقیماً از طریق تریویالیزاسیون‌های محلی کلاف به تارها (به عنوان مثال فضاهای مماس) اختصاص داده می‌شوند.
در مقابل، فرمالیسم چارت بر \emph{چارت‌های مختصاتی} (دیفئومورفیسم‌ها)
\begin{align}
	x: M \supseteq U \to V \subseteq \R^d \,,
\end{align}
تکیه دارد، که مختصات را به تکه‌های محلی $U$ از منیفلد اختصاص می‌دهند.
تریویالیزاسیون‌های محلی کلاف و تبدیلات پیمانه بین آنها به عنوان \emph{دیفرانسیل‌های چارت‌ها و توابع گذار چارت القا می‌شوند}.
در این بخش ما ارتباط بین هر دو فرمالیسم را بررسی می‌کنیم.
یک نمای کلی از نتایج در جدول~\ref{tab:coord_charts_gauge_trafos} ارائه شده است.


\etocsettocdepth{3}
\etocsettocstyle{}{} % from now on only local tocs
\localtableofcontents


ما در بخش~\ref{apx:tangent_cotangent_dual_bases} با معرفی مختصر فضاهای مماس $\TpM$ به عنوان فضاهایی از عملگرهای مشتق جهتی شروع می‌کنیم، که از آنها فضاهای هم‌مماس $\TspM$ به عنوان فضاهای دوگان نتیجه می‌شوند.
بخش~\ref{apx:differentials_gradients_jacobians} دیفرانسیل‌های عمومی و گرادیان‌ها و ژاکوبین‌های خاص‌تر را تعریف می‌کند.
بر اساس این مقدمات، ما در بخش~\ref{apx:coord_basis_def} \emph{پایه‌های مختصاتی} (پایه‌های هولونومیک)
${\pig[ \frac{\partial}{\partial x_1}\big|_p, \ \dots,\ \frac{\partial}{\partial x_d}\big|_p \pig]\ \in \FpM}$
از فضاهای مماس $\TpM$ را تعریف خواهیم کرد، که توسط عملگرهای مشتق جهتی در امتداد شبکه مختصاتی که توسط چارت از $V$ به~$U$ کشیده شده است، تولید می‌شوند.
پایه‌های دوگان
${\pig[ \hat{d}x_\mu|_p, \ \dots,\ \hat{d}x_\mu|_p \pig]}$
از فضاهای هم‌مماس $\TspM$ با گرادیان‌های مؤلفه‌های چارت $x_\mu$ داده می‌شوند.
نگاشت‌های گذار بین چارت‌ها، تبدیلات پیمانه کوواریانت و کانتراواریانت را بین پایه‌های متناظر القا می‌کنند، که در بخش~\ref{apx:chart_transition_induced_gauge_trafos} استخراج می‌شوند.
بخش~\ref{apx:correspondences_bundle_trivializations} پایه‌های مختصاتی را به عنوان تریویالیزاسیون‌های محلی کلاف تفسیر می‌کند و ارتباط بین فرمالیسم کلاف و فرمالیسم چارت را دقیق می‌کند.
پایه‌ها و تریویالیزاسیون‌های القا شده از چارت‌های مختصاتی تمام تریویالیزاسیون‌های ممکن را پوشش نمی‌دهند، به طوری که بین پایه‌های مختصاتی و پایه‌های غیرمختصاتی تمایز قائل می‌شویم (فرمالیسم کلاف به پایه‌های غیرمختصاتی عمومی اجازه می‌دهد).
در ادبیات فیزیک، پایه‌های غیرمختصاتی معمولاً از طریق \emph{میدان‌های ویل‌باین} معرفی می‌شوند.
بخش~\ref{apx:vielbein_fields} استدلال می‌کند که این میدان‌های ویل‌باین فقط تبدیلات پیمانه با مقادیر $\GL{d}$ از چارچوب‌های عمومی در $\FM$ به یک $G$-ساختار داده شده~$\GM$ هستند، که در داخل آن می‌توان متعاقباً تبدیلات پیمانه با مقادیر $G$ را که $G$-ساختار را حفظ می‌کنند، اعمال کرد.

مقدمه‌های جامعی بر فرمالیسم چارت در~\cite{nakahara2003geometry,schullerGeometricalAnatomy2016,carroll2004spacetime} ارائه شده است.
یک شرح دقیق‌تر در~\cite{schullerGeometricalAnatomy2016} یافت می‌شود.

ما می‌خواهیم به خواننده یادآوری کنیم که ما از اندیس‌های کوواریانت و کانتراواریانت استفاده \emph{نمی‌کنیم}.
اندیس‌ها همیشه به صورت زیرنویس ظاهر خواهند شد، با حروف یونانی $\mu,\nu,\dots$ که اندیس‌های مرتبط با چارت مختصاتی را نشان می‌دهند و حروف لاتین $i,j,\dots$ که اندیس‌های پیمانه‌های عمومی را نشان می‌دهند.
بالانویس‌های $A,B,\dots$ برای برچسب‌گذاری چارت‌ها یا پیمانه‌های مختلف حفظ می‌شوند.











\subsection{فضاهای مماس، فضاهای هم‌مماس و پایه‌های دوگان}
\label{apx:tangent_cotangent_dual_bases}

\subsubsection{فضاهای مماس بر حسب مشتقات جهتی}
یک تعریف رایج از فضاهای مماس $\TpM$ یک منیفلد $M$ به عنوان فضاهای برداری از عملگرهای مشتق جهتی در $p \in M$ است، که ما در اینجا به طور خلاصه آن را تشویق می‌کنیم.
فرض کنید $f\in C^\infty(M)$، یعنی $f:M\to\R$ یک نگاشت هموار باشد، و برای یک بازه $I\subseteq\R$ حاوی $0$ فرض کنید $\gamma: I \to M$ یک منحنی هموار باشد که در زمان $t = 0$ از~$p$ عبور می‌کند، یعنی $\gamma(0) = p$ را برآورده می‌کند.
سپس \emph{عملگر مشتق جهتی} در~$p$ در امتداد~$\gamma$ را به عنوان عملگر خطی
\begin{align}\label{eq:tangent_vector_directional_derivative}
	v_\gamma: C^\infty(M) \to \R,\ \ \ f \mapsto \big(f \circ \gamma \big)'(0) \,.
\end{align}
تعریف می‌کنیم. از آنجا که مشتق در امتداد جهت $\gamma$ گرفته می‌شود، یعنی مماس بر آن، $v_\gamma$ \emph{بردار مماس} نامیده می‌شود.
می‌توان آن را به عنوان سرعت یک ذره با مسیر $\gamma$ در زمان $t=0$ در نظر گرفت.
برای ارجاع بعدی، ما نمودار جابجایی ساده زیر را ارائه می‌دهیم، که پول‌بک $f\circ\gamma$ از $f$ از $M$ به $\R$ را از طریق $\gamma$ نشان می‌دهد، که مشتق جهتی بر حسب آن تعریف می‌شود:
\begin{equation}\label{cd:directional_derivative}
	\begin{tikzcd}[row sep=2.5em, column sep=4em]
		\R \supset I
		\arrow[r, "\gamma"]
		\arrow[rr, rounded corners, to path={ 
			|- node[below, pos=.75]{\small$f \circ \gamma$} ([yshift=-3.5ex]\tikztotarget.south)
			-- (\tikztotarget.south)
		}]
		& M   \arrow[r, "f"]
		& \R
	\end{tikzcd}
\end{equation}

می‌توان نشان داد که فضای تمام بردارهای مماس بر منحنی‌ها در $p$ یک فضای برداری $d$-بعدی را تشکیل می‌دهد
\begin{align}
	\TpM\ :=\ \big\{ v_\gamma \,\big|\, \gamma\ \text{یک منحنی هموار گذرنده از}\ p \text{ است} \big\} \,,
\end{align}
که به عنوان فضای مماس در~$p$ شناخته می‌شود.
برای جزئیات بیشتر در مورد تعریف بردارهای مماس و ساختار فضای برداری فضاهای مماس، به~\cite{schullerGeometricalAnatomy2016} مراجعه کنید.

با تعریف فضاهای مماس به عنوان فضاهای برداری، ممکن است انتخاب کنیم که با بردارهای مماس به عنوان بردارهای هندسی انتزاعی رفتار کنیم و در نتیجه تعریف آنها را از طریق مشتقات جهتی (یا هر تعریف جایگزین دیگری) «فراموش» کنیم.
ما این کار را در اکثر جاها انجام می‌دهیم، اما در بخش‌های بعدی برای استخراج دیفرانسیل‌های نگاشت‌های هموار و پایه‌های مختصاتی به تعریف از طریق مشتقات جهتی باز می‌گردیم.



\subsubsection{فضاهای هم‌مماس}
\label{apx:cotangent_spaces}
به عنوان فضاهای برداری حقیقی، فضاهای مماس $\TpM$ دارای \emph{فضاهای دوگان} متناظر $\TspM := (\TpM)^*$ هستند، یعنی \emph{فضاهای هم‌مماس}.
بنا به تعریف فضاهای دوگان، آنها از تابعی‌های خطی
\begin{align}
	\omega: \TpM \to \R \,,
\end{align}
تشکیل شده‌اند، که در هندسه دیفرانسیل معمولاً \emph{هم‌بردارها} یا \emph{۱-فرم‌ها} نامیده می‌شوند.
به همراه جمع (هم)برداری
$(\omega + \widetilde{\omega})(v) = \omega(v) + \widetilde{\omega}(v)$
و ضرب اسکالر
$(\lambda \cdot \omega)(v) = \lambda\cdot (\omega(v))$،
فضاهای هم‌مماس خود فضاهای برداری هستند.

به عنوان دوگان‌های متناهی-بعدی از یکدیگر، $\TpM$ و $\TspM$ ایزومورف هستند و بنابراین به طور خاص دارای بعد یکسان $d = \dim(M) = \dim(\TpM) = \dim(\TspM)$ هستند.
با این حال، ایزومورفیسم بین این دو کانونی نیست.
یک ایزومورفیسم فضای برداری را می‌توان از طریق یک فرم دوخطی (غیرتبهگن) $\eta_p: \TpM \times \TpM \to \R$ روی $\TpM$ مثلاً یک متریک ریمانی، از طریق
\begin{align}
	\widehat{\eta}_p:\ \TpM \to \TspM,\ \ v \mapsto \eta_p(v, \cdot) \,,
\end{align}
مشخص کرد، که تابعی خطی
$\widehat{\eta}_p(v):\ \TpM \to \R,\ \ w \mapsto \eta_p(v, w)$ را تعیین می‌کند.



\subsubsection{پایه‌های دوگان}
هر پایه $\big[e_i\big]_{i=1}^d$ از $\TpM$ به طور کانونی یک \emph{پایه دوگان} $\big[e_i^*\big]_{i=1}^d$ از $\TspM$ را القا می‌کند، که برای برآوردن روابط
\begin{align}
	e^*_i e_j = \delta_{ij} \quad \textup{برای هر }\ i,j \in 1,\dots,d \,.
\end{align}
تعریف شده است. فرض کنید $\big[e^A_i\big]_{i=1}^d$ و $\big[e^B_i\big]_{i=1}^d = \big[e^A_i\big]_{i=1}^d \lhd \big(g^{BA}\big)^{-1}$ دو پایه از $\TpM$ باشند، که با عمل راست~$\lhd$ از عضو گروه ساختاری (وارون) $\big(g^{BA}\big)^{-1} \in \GL{d}$ در معادله~\eqref{eq:frame_rightaction} به هم مرتبط هستند، یعنی برای $j=1,\dots,d$ :
\begin{align}\label{eq:general_tangent_basis_gauge_trafo}
	e^B_j\ =\ \sum_l e^A_l \big(g^{BA}\big)^{-1}_{lj}
\end{align}
پایه دوگان $\big[e^{A,*}_i\big]_{i=1}^d$ بر این اساس تحت آن عمل چپی که $e_i^{A,*}$ را به
\begin{align}\label{eq:general_cotangent_basis_gauge_trafo}
	e_i^{B,*}\ =\ \sum_k g^{BA}_{ik} e_k^{A,*} \,.
\end{align}
می‌فرستد، تبدیل می‌شود. این با جفت‌کردن تأیید می‌شود:
\begin{align}
	e_i^{B,*} e_j^B
	\ &=\ \sum_{k,l} g^{BA}_{ik} e_k^{A,*} e_l^A \big(g^{BA}\big)^{-1}_{lj}  \notag \\
	\ &=\ \sum_{k,l} g^{BA}_{ik} \delta_{kl} \big(g^{BA}\big)^{-1}_{lj} \notag \\
	\ &=\ \sum_{k} g^{BA}_{ik} \big(g^{BA}\big)^{-1}_{kj} \notag \\
	\ &=\ \delta_{ij} \,.
\end{align}
رفتار تبدیل وارون پایه‌ها و پایه‌های دوگان معمولاً به عنوان تبدیل \emph{کوواریانت} و \emph{کانتراواریانت} نامیده می‌شود.
به شباهت تبدیل پایه دوگان با تبدیلات کانتراواریانت $\psi^B = g^{BA} \psi^A$ پیمانه‌ها در معادله~\eqref{eq:gauge_trafo_local_def_21} و $v^B = g^{BA} v^A$ مؤلفه‌های برداری در \eqref{eq:components_leftaction} توجه کنید.
در واقع، پیمانه‌ها فقط انتخاب‌هایی از یک پایه هم‌مماس هستند که در ادامه بیشتر مورد بحث قرار می‌گیرد.








\subsection{دیفرانسیل‌ها، گرادیان‌ها و ژاکوبین‌ها}
\label{apx:differentials_gradients_jacobians}

در حساب برداری، توابع $\phi: \R^m \to \R^n$ را در نظر می‌گیریم، که می‌توان آنها را در هر نقطه $p\in \R^m$ به صورت خطی با ماتریس ژاکوبین آنها (یا مشتق کل یا دیفرانسیل) $d\phi_p = \big(\frac{\partial\phi_i}{\partial x_j} \big|_p \big)_{ij}$ تقریب زد.
در اینجا ما تعمیم این مفهوم را به دیفرانسیل‌های توابع هموار بین منیفلدهای هموار معرفی می‌کنیم.

\paragraph{دیفرانسیل‌ها به طور کلی:}
فرض کنید $\phi: M \to N$ یک نگاشت هموار بین منیفلدهای هموار $M$ و $N$ باشد.
در هر نقطه $p\in M$ چنین نگاشتی یک دیفرانسیل (یا پوش‌فوروارد)
\begin{align}
	d\phi_p : \TpM \to \TphipN,\ \ v \mapsto d\phi_p(v)
\end{align}
را القا می‌کند که بردارهای مماس را در~$p$ به صورت خطی به بردارهای مماس در $\phi(p)$ نگاشت می‌دهد.
برای تعریف فضاهای مماس بر حسب مشتقات جهتی در معادله~\eqref{eq:tangent_vector_directional_derivative}، پوش‌فوروارد $v \in \TpM$ در امتداد $\phi$ به صراحت با
\begin{align}
	d\phi_p(v): C^\infty(N) \to \R,\ \ \ f \mapsto \big( d\phi_p(v) \big)(f)\ :=\ v(f \circ \phi) \,,
\end{align}
داده می‌شود، یعنی با اعمال~$v$ بر روی پول‌بک $f \circ \phi: M \to \R$ از $f: N\to \R$ از طریق $\phi$.
این تعاریف با دو نمودار جابجایی زیر روشن می‌شوند:
\begin{equation}
	\begin{tikzcd}[column sep=70pt, row sep=30, font=\normalsize]
		M
		\arrow[r, "\phi"]
		\arrow[dr, "f \circ \phi"']
		&
		N
		\arrow[d, "\ f"]
		\\
		& \R
	\end{tikzcd}
	\qquad\qquad
	\begin{tikzcd}[column sep=60pt, row sep=30, font=\normalsize]
		C^\infty(M)
		\arrow[d, "v\ "']
		&
		C^\infty(N)
		\arrow[l, "{(\,\cdot\,) \circ \phi}"']
		\arrow[dl, "{d\phi(v)}"]
		\\
		\R
	\end{tikzcd}
\end{equation}

از این تعریف بلافاصله نتیجه می‌شود که دیفرانسیل ترکیب نگاشت‌های هموار برابر با ترکیب دیفرانسیل‌های منفرد آنها است، که همان قاعده زنجیره‌ای است:
\begin{align}
	d(\phi \circ \psi)_p\ =\ d\phi_{\psi(p)} \circ d\psi_p
\end{align}
اگر $\phi$ وارون‌پذیر باشد (یک دیفئومورفیسم)، علاوه بر این نتیجه می‌شود که دیفرانسیل آن یک ایزومورفیسم فضای برداری است که وارون آن برابر با دیفرانسیل $\phi^{-1}$ است، یعنی
\begin{align}\label{eq:differential_inverse}
	\big( d\phi_{p} \big)^{-1}\ =\ d\big( \phi^{-1} \big)_{\phi(p)} \,.
\end{align}

در مجموع، دیفرانسیل‌های $d\phi_p$ در نقاط منفرد $p\in M$ یک مورفیسم کلاف برداری (یک نگاشت کلاف خطی-تاری، به بخش‌های~\ref{sec:fiber_bundles_general} مراجعه کنید) را بین کلاف‌های مماس $M$ و $N$ القا می‌کنند:
\begin{equation}
	\begin{tikzcd}[column sep=60pt, row sep=35, font=\normalsize]
		TM
		\arrow[r, "d\phi"]
		\arrow[d, "\piTM"']
		&
		TN
		\arrow[d, "\piTM"]
		\\
		M
		\arrow[r, "\phi"']
		&
		N
	\end{tikzcd}
\end{equation}

توجه داشته باشید که ما در این پیوست از یک نمادگذاری متفاوت، یعنی $d\phi$ استفاده می‌کنیم، نسبت به مقاله اصلی، که در آن به جای آن $\dphiTM$ می‌نویسیم.
ما اولی را برای ارتباط با نمادگذاری معمول $dx_\mu$ برای پایه‌های القا شده توسط چارت از فضاهای هم‌مماس انتخاب کردیم.
دومی در متن اصلی برای تأکید بر شباهت با نگاشت‌های کلاف $\dphiFM$، $\dphiGM$ و $\dphiA$ که روی کلاف‌های همبسته $\FM$، $\GM$ و $\A$ القا می‌شوند، استفاده می‌شود.



\paragraph{گرادیان‌ها:}
در مورد توابع هموار با مقادیر حقیقی $\phi: M \to \R$، یعنی $\phi \in C^\infty(M)$، دیفرانسیل $d\phi_p: \TpM \to T_{\phi(p)}\R$ بردارهای $v$ را در $\TpM$ به بردارهای $d\phi(v): C^\infty(\R) \to \R,\ f \mapsto v(f \circ \phi)$ در $T_{\phi(p)}\R$ پوش‌فوروارد می‌کند.
با استفاده از ایزومورفیسم کانونی
\begin{align}\label{eq:canon_isom_TR_R}
	\iota_{\R}: T_{\phi(p)}\R \xrightarrow{\sim} \R,\ \ v \mapsto v(\id_{\R})
\end{align}
\emph{عملگر گرادیان}
\begin{align}
	\hat{d}_p: C^\infty(M) \to \TspM,\ \ \phi \mapsto  \hat{d}\phi_p := \iota_{\R} \circ d\phi_p = \big( d\phi_p(\,\cdot\,) \big)(\id_{\R}) \,,
\end{align}
را تعریف می‌کنیم، که توابع هموار $\phi$ را به هم‌بردارها%
\footnote{
	میدان گرادیان اغلب به عنوان یک میدان \emph{برداری} $\nabla f := (\hat{d}f)^{\sharp^\eta}$ تعریف می‌شود که از میدان \emph{هم‌برداری} $\hat{d}f$ از طریق ایزومورفیسم موسیقیایی $\sharp^\eta: \TsM \to \TM$ متناظر با متریک («بالا بردن اندیس‌ها») محاسبه می‌شود.
}
$\hat{d}\phi$ می‌فرستد، که به نوبه خود بر روی بردارها به صورت
\begin{align}\label{eq:gradient_vector_action}
	\hat{d}\phi_p: \TpM \to \R,\ \ v \mapsto \hat{d}\phi_p(v) = \big( d\phi_p(v) \big)(\id_{\R}) = v(\id_{\R} \circ \phi) = v(\phi) \,.
\end{align}
عمل می‌کنند. با یک \emph{سوءاستفاده از نمادگذاری} معمولاً «کلاه» روی $\hat{d}$ حذف می‌شود و بلافاصله $d\phi_p(v) := v(\phi)$ تعریف می‌شود.
در حالی که این نمادگذاری بسیار رایج است، ما در ادامه به «کلاه» پایبند می‌مانیم تا نیاز به ایزومورفیسم کانونی $\iota_{\R}$ را صریح کنیم.

در بخش~\ref{apx:coord_basis_def} در ادامه خواهیم دید که پایه‌های $\TspM$ که دوگان پایه‌های مختصاتی $\TpM$ هستند، با ۱-فرم‌های گرادیان $\hat{d}x_\mu|_p$ داده می‌شوند، که در آن $x_\mu$ مؤلفه‌های چارت مختصاتی هستند.



\paragraph{ژاکوبین‌ها:}
به طور خاص برای توابع $\phi: \R^n \to \R^m$ بین (زیرمجموعه‌هایی از) فضاهای اقلیدسی، دیفرانسیل
$d\phi_{x_0}: {T_{x_0}\R^n \to T_{\phi(x_0)}\R^m}$
به راحتی دیده می‌شود که با \emph{ژاکوبین} $\frac{\partial \phi}{\partial x} \big|_{x_0}: \R^n \to \R^m$
پس از یکی گرفتن کانونی $T_p\R^k \cong \R^k$ در هر دو دامنه و هم‌دامنه، منطبق است.
ایزومورفیسم کانونی در اینجا با
\begin{align}\label{eq:canonical_iso_TRk_Rk}
	\iota_{\R^k}: v \mapsto \big( v(\proj_1), \dots, v(\proj_k) \big) \,,
\end{align}
داده می‌شود، که $\iota_{\R}$ را از معادله~\eqref{eq:canon_isom_TR_R} به ابعاد چندگانه تعمیم می‌دهد.
از آنجا که محاسبه عمدتاً مشابه مورد گرادیان‌ها است، ما آن را در اینجا تکرار نخواهیم کرد بلکه ایده را از طریق یک نمودار جابجایی به تصویر می‌کشیم:
\begin{equation}\label{cd:jacobian_def}
	\begin{tikzcd}[row sep=2.5em, column sep=4em]
		\R^n
		\arrow[rrr, rounded corners, to path={ 
			|- node[below, pos=.75]{$\frac{\partial \phi}{\partial x} \Big|_{x_0}$} ([yshift=-3.5ex]\tikztotarget.south)
			-- (\tikztotarget.south)
		}]
		& T_{x_0}\R^n
		\arrow[l, "\iota_{\R^n}"']
		\arrow[r, "{d\phi |_{x_0}}"]
		& T_{\phi(x_0)}\R^m
		\arrow[r, "\iota_{\R^m}"]
		& \R^m
	\end{tikzcd}
\end{equation}

اگر $\phi$ وارون‌پذیر باشد، همانی در معادله~\eqref{eq:differential_inverse} به
\begin{align}\label{eq:inv_fct_thm_jacobian}
	\frac{\partial \phi}{\partial x} \bigg|_{x_0}^{-1} \ =\ 
	\frac{\partial \phi^{-1}}{\partial x} \bigg|_{\phi(x_0)} \,,
\end{align}
تبدیل می‌شود، که همان قضیه تابع وارون است.
ما بعداً از این همانی برای وارون کردن تبدیلات پیمانه بین پایه‌های مختصاتی مختلف که به عنوان ژاکوبین‌های نگاشت‌های گذار چارت القا می‌شوند، استفاده خواهیم کرد.











\subsection{پایه‌های مختصاتی القا شده توسط چارت}
\label{apx:chart_induced_bases_main}

در این بخش ما \emph{چارت‌های مختصاتی} به شکل
\begin{align}
	x: U \to V \,,
\end{align}
را در نظر می‌گیریم، که به صورت دیفئومورفیک مختصات $x(p) \in V \subseteq \R^d$ را به هر نقطه $p \in U \subseteq M$ اختصاص می‌دهند.
هر چنین چارتی یک انتخاب طبیعی از پایه‌ها را برای فضاهای مماس $\TpM$ روی~$U$ القا می‌کند، که به عنوان \emph{پایه‌های مختصاتی} شناخته می‌شوند.
فضاهای دوگان $\TspM$ از فضاهای مماس روی~$U$ بر این اساس با پایه‌های مختصاتی دوگان از هم‌بردارهای مماس مجهز می‌شوند.
نگاشت‌های گذار بین مختصات دو چارت، تبدیلات پیمانه‌ای را القا می‌کنند که بین پایه‌های مختصاتی متناظر ترجمه می‌کنند.
این تبدیلات پیمانه با ژاکوبین‌های نگاشت‌های گذار داده می‌شوند.




\subsubsection{چارت‌ها و پایه‌های مختصاتی القا شده}
\label{apx:coord_basis_def}

\paragraph{پایه‌های مختصاتی برای $\TpM$:}
برای تشویق تعریف پایه‌های مختصاتی، مشاهده کنید که~$x$ یک «شبکه مختصاتی» را روی~$U$ با پول‌بک کردن شبکه مختصاتی کانونی روی $V$ به منیفلد القا می‌کند.
سپس پایه مختصاتی در یک نقطه خاص $p \in U$ را می‌توان به عنوان متشکل از آن $d$ \emph{عملگر مشتق جهتی} که در \emph{امتداد خطوط شبکه مختصاتی~$x$ روی~$U$} می‌روند، در نظر گرفت.

برای دقیق‌تر کردن این موضوع، ابتدا منحنی‌های
\begin{align}
	\widetilde{\gamma}_\mu: I \to V,\ \ \ t \mapsto x(p) + t \epsilon_\mu \quad\qquad \mu = 1,\dots,d
\end{align}
را در نظر بگیرید که در زمان $t=0$ با سرعت واحد در جهت $\mu$ از~$x(p) \in V$ عبور می‌کنند.
نگاشت آن $\widetilde{\gamma}_\mu$ از طریق چارت به~$U$ منحنی‌های ذکر شده در بالا را تعریف می‌کند
\begin{align}
	\gamma_\mu: I \to U,\ \ \ t \mapsto\,
	x^{-1}\mkern-2mu \circ \widetilde{\gamma}_\mu (t) \ =\ 
	x^{-1} \big( x(p) + t \epsilon_\mu \big)
\end{align}
که در زمان $t=0$ در امتداد شبکه مختصاتی~$x$ روی~$U$ از~$p$ عبور می‌کنند.
پایه مختصاتی $d$-بعدی~$\TpM$ القا شده توسط~$x$ سپس با عملگرهای مشتق جهتی در معادله~\eqref{eq:tangent_vector_directional_derivative} در امتداد مسیرهای~$\gamma_\mu$ داده می‌شود.
با نشان دادن بردار پایه $\mu$-ام با سوءاستفاده معمول از نمادگذاری به صورت $\frac{\partial}{\partial x_\mu}\big|_p$ بنابراین تعریف می‌کنیم:
\begin{align}\label{eq:coord_basis_def}
	\frac{\partial}{\partial x_\mu} \bigg|_p \!:\ \ f\, \mapsto\,
	\frac{\partial}{\partial x_\mu} \bigg|_p f
	\ \ :=&\ \ \big(f \circ \gamma_\mu \big)'(0) \notag \\
	\ \  =&\ \ \big(f \circ x^{-1} \circ \widetilde{\gamma}_\mu \big)'(0) \notag \\
	\ \  =&\ \ \big(f \circ x^{-1}\big( x(p) + t\epsilon_\mu \big) \big)'(0) \notag \\
	\ \  =&\ \ \pig[\mkern1.5mu \partial_\mu \big(f \circ x^{-1} \big)\pig] \big(x(p)\big)
\end{align}
در مرحله آخر ما مشتق جزئی $\mu$-ام معمول از پول‌بک $f\circ x^{-1}: V \to \R$ را شناسایی کردیم، که نمادگذاری $\frac{\partial}{\partial x_\mu}\big|_p$ را توجیه می‌کند.
این تعاریف در نمودار جابجایی زیر که نمودار در معادله~\eqref{cd:directional_derivative} را گسترش می‌دهد، به تصویر کشیده شده‌اند:
\begin{equation}
	\begin{tikzcd}[row sep=4.em, column sep=6.em]
		& V     \arrow[rd, "{f\circ x^{-1}}"]
		\\
		\R \supset I
		\arrow[r, "\gamma_\mu"]
		\arrow[ru, "\widetilde{\gamma}_\mu"]
		\arrow[rr, rounded corners, to path={ 
			|- node[below, pos=.75]{\small$f \circ \gamma_\mu$} ([yshift=-3.5ex]\tikztotarget.south)
			-- (\tikztotarget.south)
		}]
		& U     \arrow[r, pos=.4, "f"]
		\arrow[u, pos=.4, "x"]
		&[1.4em] \R
	\end{tikzcd}
\end{equation}




\paragraph{پایه‌های مختصاتی دوگان برای $\TspM$:}

همانطور که در بخش~\ref{apx:tangent_cotangent_dual_bases} گفته شد، هر پایه از $\TpM$ یک \emph{پایه دوگان} از $\TspM$ را القا می‌کند.
به طور خاص برای پایه‌های مختصاتی، که توسط بردارهای $\frac{\partial}{\partial x_\mu} \big|_p$ تولید می‌شوند، عناصر پایه دوگان با \emph{گرادیان‌های $\hat{d}x_\mu|_p = \hat{d}(x_\mu)_p \in \TspM$ از مؤلفه‌های چارت} $x_\mu = \proj_\mu \circ x: U \to \R$ داده می‌شوند.
اینکه این گرادیان‌ها در واقع پایه دوگان را تشکیل می‌دهند، به راحتی با عمل بر روی بردارهای پایه همانطور که در معادله~\eqref{eq:gradient_vector_action} تعریف شده است، دیده می‌شود:
\begin{align}
	\hat{d}x_\mu \big|_p\ \frac{\partial}{\partial x_\nu} \bigg|_p
	\ &=\ \frac{\partial}{\partial x_\nu} \bigg|_p x_\mu \notag \\
	\ &=\ \pig[ \partial_\nu \big( x_\mu \circ x^{-1} \big) \pig] \big(x(p)\big) \notag \\
	\ &=\ \pig[ \partial_\nu \big( \proj_\mu \big) \pig] \big(x(p)\big) \notag \\
	\ &=\ \delta_{\mu\nu} \,.
\end{align}





\paragraph{دیفرانسیل‌های چارت به عنوان تریویالیزاسیون محلی کانونی:}

با توجه به اینکه چارت از $U \subseteq M$ به $V \subseteq \R^d$ نگاشت می‌دهد، دیفرانسیل‌های آن در $p\in U$ نگاشت‌هایی به شکل
\begin{align}
	dx_p: \TpM \to T_{x(p)}\R^d \,.
\end{align}
هستند. با استفاده مجدد از ایزومورفیسم کانونی $\iota_{\R^d}$ از $T_{x(p)}\R^d$ به $\R^d$ از معادله~\eqref{eq:canonical_iso_TRk_Rk}، ما یک نگاشت
\begin{align}\label{eq:chart_differential_via_gradients}
	\qquad
	\hat{d}x_p: \TpM \to \R^d,\ \ \ v\ \mapsto\ \hat{d}x_p (v)
	:=\ &\iota_{\R^d} \circ dx_p (v) \notag \\
	=\ & \Big( \big(dx_p(v) \big)(\proj_1) \,,\,\dots,\, \big(dx_p(v) \big)(\proj_d) \Big)^\top \notag \\
	=\ & \Big( v\big(\proj_1 \circ x \circ x^{-1}\big)(x(p)) \,,\,\dots,\, v\big(\proj_1 \circ x \circ x^{-1}\big)(x(p)) \Big)^\top \notag \\
	=\ & \Big( v(x_1(p)) \,,\,\dots,\, v(x_d(p)) \Big)^\top \notag \\
	=\ & \Big( \hat{d}x_1 |_p(v) \,,\,\dots,\, \hat{d}x_d |_p(v) \Big)^\top
\end{align}
به دست می‌آوریم، پس از شناسایی گرادیان‌های مؤلفه چارت منفرد در مرحله آخر.
توجه داشته باشید که عمل این دیفرانسیل چارت بر روی پایه مختصاتی $\mu$-ام نتیجه می‌دهد
\begin{align}
	\hat{d}x_p \: \frac{\partial}{\partial x_\mu} \bigg|_p
	\ &=\ \bigg( \hat{d}x_1|_p \: \frac{\partial}{\partial x_\mu} \bigg|_p \,,\,\dots,\, \hat{d}x_d|_p \: \frac{\partial}{\partial x_\mu} \bigg|_p \bigg)^\top \notag \\
	\ &=\ \big( \delta_{\mu1} \,,\,\dots,\, \delta_{\mu d} \big)^\top \notag \\
	\ &=\ \epsilon_\mu \,,
\end{align}
یعنی بردار واحد $\mu$-ام $\epsilon_\mu$ از $\R^d$.
این دلالت بر این دارد که $\hat{d}x_p: \TpM \to \R^d$ نقش یک \emph{پیمانه} $\psi_p$ را در~$p$ ایفا می‌کند.
بنابراین می‌توان به همان اندازه با تعریف یک پایه هم‌مماس شروع کرد و
\begin{align}\label{eq:coord_basis_vector_via_chart_differential}
	\frac{\partial}{\partial x_\mu} \bigg|_{x(p)}\ =\ \hat{d}x_p^{-1} (\epsilon_\mu) \,,
\end{align}
را قرار داد، که آنالوگ معادله~\eqref{eq:framefield_gauge_equivalence} در فرمالیسم چارت است.











\subsubsection{نگاشت‌های گذار چارت و تبدیلات پیمانه القا شده}
\label{apx:chart_transition_induced_gauge_trafos}

چارت‌های مختلف پایه‌های مختصاتی مختلفی را القا می‌کنند.
بنابراین گذار چارت‌ها تبدیلات پیمانه، یعنی تبدیلات پایه‌ها و ضرایب برداری را القا می‌کنند، که ما در این بخش آنها را استخراج می‌کنیم.

در ادامه ما دو چارت دلخواه و همپوشان $x^A: U^A \to V^A$ و $x^B: U^B\to V^B$ را در نظر می‌گیریم.
مختصات مختلفی که آنها به همپوشانی $U^A \cap U^B \neq \varnothing$ اختصاص می‌دهند، سپس از طریق \emph{نگاشت‌های گذار چارت} به هم مرتبط می‌شوند
\begin{align}\label{eq:chart_transition_fct}
	x^B\circ\left(x^A\right)^{-1} \!:\ x^A\big(U^A\cap U^B\big)\to x^B\big(U^A\cap U^B\big) \,.
\end{align}



\paragraph{تبدیل پایه‌های مختصاتی مماس:}
پایه‌های مختصاتی $\TpM$ که توسط دو چارت القا می‌شوند، طبق خط آخر معادله~\eqref{eq:coord_basis_def} با عمل آنها بر روی $f \in C^\infty(M)$ به صورت
\begin{align}
	\frac{\partial}{\partial x^A_\mu} \bigg|_p f
	\ =\ \Big[\mkern1.5mu \partial_\mu \pig(f \circ \big(x^A\big)^{-1} \pig)\Big] \big(x^A(p)\big)
	\qquad \text{and} \qquad
	\frac{\partial}{\partial x^B_\mu} \bigg|_p f
	\ =\ \Big[\mkern1.5mu \partial_\mu \pig(f \circ \big(x^B\big)^{-1} \pig)\Big] \big(x^B(p)\big) \ ,
	\quad
\end{align}
تعریف می‌شوند، که با نمودار جابجایی زیر به تصویر کشیده شده است:
\begin{equation}\label{cd:scalar_field_chart_expressions}
	\begin{tikzcd}[row sep=4.em, column sep=6.em] %,
		V^A \supset x^A \big( U^A \cap U^B \big)
		\arrow[rd, "{f\circ \big(x^A\big)^{-1}}"]
		\arrow[dd, rounded corners, to path={ 
			|- node[left, pos=.75]{\small$x^B \circ \big(x^A\big)^{-1}$} ([xshift=-3.5ex]\tikztotarget.west)
			-- (\tikztotarget.west)
		}]
		\\
		U^A \cap U^B
		\arrow[r, pos=.4, "f"]
		\arrow[u, "x^A"]
		\arrow[d, "x^B"']
		&
		\R
		\\
		V^B \supset x^B \big( U^A \cap U^B \big)
		\arrow[ru, "{f\circ \big(x^B\big)^{-1}}"']
	\end{tikzcd}
\end{equation}

از طریق نگاشت‌های گذار چارت، پایه‌های مختصاتی مختلف با
\begin{align}\label{eq:coord_basis_trafo_action_f}
	\frac{\partial}{\partial x^B_\mu} \bigg|_p f
	\ &=\ \Big[\mkern1.5mu \partial_\mu \pig(f \circ \big(x^B\big)^{-1} \pig) \Big] \big(x^B(p)\big) \notag \\
	\ &=\ \Big[\mkern1.5mu \partial_\mu \pig(f \circ \big(x^A\big)^{-1} \circ x^A \circ \big(x^B\big)^{-1} \pig) \Big] \big(x^B(p)\big) \,, \notag
	\intertext{
		به هم مرتبط می‌شوند، که با استفاده از قاعده زنجیره‌ای چندمتغیره، بیشتر منجر به این می‌شود:
	}
	\frac{\partial}{\partial x^B_\mu} \bigg|_p f
	\ &=\ \sum_{\nu=1}^d
	\Big[\mkern1.5mu \partial_\nu \pig(f \circ \big(x^A\big)^{-1} \pig)\Big] \big(x^A(p)\big) \cdot
	\Big[\mkern1.5mu \partial_\mu \pig(x^A_\nu \circ \big(x^B\big)^{-1} \pig)\Big] \big(x^B(p)\big) \notag \\
	\ &=\ \sum_{\nu=1}^d \,
	\frac{\partial f}{\partial x^A_\nu} \bigg|_p \ 
	\frac{\partial x^A_\nu}{\partial x^B_\mu} \bigg|_{x^B(p)}
\end{align}
در مرحله آخر ما از سوءاستفاده معمول از نمادگذاری%
\footnote{
	«سوءاستفاده» این است که $x^A$ به عنوان تابعی از $x^B(p)$ تفسیر می‌شود، و بنابراین باید به طور دقیق‌تر به صورت $x^A \circ \big(x^B\big)^{-1}$ نوشته شود.
}
\begin{align}\label{eq:abuse_of_notation_jacobian}
	\frac{\partial x^A_\nu}{\partial x^B_\mu} \bigg|_{x^B(p)}
	:=\ \partial_\mu \pig(x^A_\nu \circ \big(x^B\big)^{-1} \pig) \big( x^B(p) \big)
\end{align}
برای مؤلفه‌های \emph{ژاکوبین}
\begin{align}
	\frac{\partial x^A}{\partial x^B} \bigg|_{x^B(p)}
	=\ \hat{d}x^A_p \circ \hat{d}(x^B_p)^{-1}
\end{align}
\emph{از نگاشت‌های گذار} استفاده کردیم. با حذف~$f$ از معادله~\eqref{eq:coord_basis_trafo_action_f}، ما قانون تبدیل
\begin{align}\label{eq:coord_bases_trafo_law}
	\frac{\partial}{\partial x^B_\mu} \bigg|_p
	\ =\ \sum_{\nu=1}^d \,
	\frac{\partial}{\partial x^A_\nu} \bigg|_p \ 
	\frac{\partial x^A_\nu}{\partial x^B_\mu} \bigg|_{x^B(p)}
\end{align}
از پایه‌های مختصاتی مماس را شناسایی می‌کنیم.
ما در اینجا انتخاب کردیم که ژاکوبین را در سمت راست بردار پایه بنویسیم تا تأکید کنیم که تغییر پایه باید به عنوان یک \emph{عمل راست} درک شود.
با انجام این کار، باید به خواننده هشدار دهیم که $\frac{\partial}{\partial x_\nu}\big|_p$ فقط یک سوءاستفاده از نمادگذاری برای بردار پایه است اما به معنای عمل یک عملگر دیفرانسیل بر روی ژاکوبین در سمت راست نیست.




\paragraph{تبدیل پایه‌های مختصاتی هم‌مماس:}
قانون تبدیل کانتراواریانت پایه‌های مختصاتی فضای هم‌مماس از تبدیل وارون پایه‌های دوگان در معادله~\eqref{eq:general_cotangent_basis_gauge_trafo} نسبت به~\eqref{eq:general_tangent_basis_gauge_trafo} نتیجه می‌شود.
برای اعمال این رابطه، ما ابتدا معادله~\eqref{eq:coord_bases_trafo_law} را با قرارداد خود مبنی بر اینکه پایه‌ها مطابق با یک عمل راست با یک عضو گروه \emph{وارون} تبدیل می‌شوند، تطبیق می‌دهیم.
این کار با اعمال معادله~\eqref{eq:inv_fct_thm_jacobian} برای وارون کردن ژاکوبین (سوءاستفاده از نمادگذاری را به یاد بیاورید)
\begin{align}
	\frac{\partial x^A}{\partial x^B} \bigg|_{x^B(p)} \ =\ 
	\frac{\partial x^B}{\partial x^A} \bigg|_{x^A(p)}^{-1}
\end{align}
انجام می‌شود که نتیجه می‌دهد:
\begin{align}\label{eq:coord_bases_trafo_law_with_inv}
	\frac{\partial}{\partial x^B_\mu} \bigg|_p
	\ =\ \sum_{\nu=1}^d \,
	\frac{\partial}{\partial x^A_\nu} \bigg|_p \ 
	\frac{\partial x^A_\nu}{\partial x^B_\mu} \bigg|_{x^B(p)}
	\ =\ \sum_{\nu=1}^d \,
	\frac{\partial}{\partial x^A_\nu} \bigg|_p \ 
	\bigg( \frac{\partial x^A}{\partial x^B} \bigg|_{x^B(p)} \bigg)_{\nu\mu}
	\ =\ \sum_{\nu=1}^d \,
	\frac{\partial}{\partial x^A_\nu} \bigg|_p \ 
	\bigg( \frac{\partial x^B}{\partial x^A} \bigg|_{x^A(p)}^{-1} \bigg)_{\nu\mu}
\end{align}
بنابراین عناصر پایه هم‌مماس مطابق با معادلات~\eqref{eq:general_tangent_basis_gauge_trafo} و ~\eqref{eq:general_cotangent_basis_gauge_trafo} مانند
\begin{align}\label{eq:chart_component_gradient_trafo_law}
	\hat{d}x^B_\mu|_p \ =\ 
	\sum_{\nu=1}^d\ 
	\frac{\partial x^B_\mu}{\partial x^A_\nu} \bigg|_{x^A(p)}
	\hat{d}x^A_\nu|_p \,.
\end{align}
تبدیل می‌شوند.



\paragraph{تبدیل دیفرانسیل‌های چارت:}
بیان دیفرانسیل‌های چارت $\hat{d}x^A|_p$ بر حسب گرادیان‌های مؤلفه چارت $\hat{d}x^A_\mu|_p$ در معادله~\eqref{eq:chart_differential_via_gradients} اجازه می‌دهد تا قانون تبدیل آنها را از قانون در معادله~\eqref{eq:chart_component_gradient_trafo_law} استنتاج کنیم.
به طور جایگزین، قانون تبدیل با ضرب راست در همانی به شکل $\id_{\TpM} = \hat{d}x^A|_p \circ \big( \hat{d}x^A|_p \big)^{-1}$ و شناسایی یک ضرب چپ با ژاکوبین نگاشت‌های گذار چارت به دست می‌آید:
\begin{align}\label{eq:chart_differential_trafo_law}
	\hat{d}x^B|_p
	\ &=\ \hat{d}x^B|_p \circ \big( \hat{d}x^A|_p \big)^{-1} \circ \hat{d}x^A|_p \notag \\
	\ &=\ \frac{\partial x^B}{\partial x^A} \bigg|_{x^A(p)} \hat{d}x^A|_p
\end{align}
توجه داشته باشید که این نتیجه به سادگی بیان ماتریسی معادله~\eqref{eq:chart_component_gradient_trafo_law} است.


\paragraph{تبدیل ضرایب برداری:}
بردارهای $v \in \TpM$ نسبت به یک پایه مختصاتی
$\big[\frac{\partial}{\partial x^B_\mu} \big|_p \big]_{\mu=1}^d$
با ضرایب $v^A \in \R^d$ بیان می‌شوند:
\begin{align}
	v\ =\
	\sum_{\mu=1}^d \,
	v_\mu^A \frac{\partial}{\partial x^A_\mu} \bigg|_p
\end{align}
ضرایب منفرد با عمل پایه هم‌مماس بازیابی می‌شوند:
\begin{align}
	\hat{d}x^A_\mu \big|_p(v)
	\ =\ \hat{d}x^A_\mu \big|_p\ \sum_{\nu=1}^d \, v_\nu^A \frac{\partial}{\partial x^A_\nu} \bigg|_p
	\ =\ \sum_{\nu=1}^d \, v_\nu^A \delta_{\mu\nu}
	\ =\ v^A_\mu
\end{align}
این دلالت بر این دارد که ضرایب به صورت کانتراواریانت تبدیل می‌شوند، درست مانند پایه مختصاتی هم‌مماس:
\begin{align}
	v^B_\mu
	\ =\ \hat{d}x^B_\mu \big|_p (v)
	\ =\ \sum_{\nu=1}^d\ 
	\frac{\partial x^B_\mu}{\partial x^A_\nu} \bigg|_{x^A(p)}
	\hat{d}x^A_\nu|_p (v)
	\ =\ \sum_{\nu=1}^d\ 
	\frac{\partial x^B_\mu}{\partial x^A_\nu} \bigg|_{x^A(p)}
	v^A_\nu
\end{align}
به راحتی تأیید می‌شود که این قانون تبدیل واقعاً به یک نمایش مستقل از مختصات از بردارهای مستقل از مختصات $v\in\TpM$ منجر می‌شود:
\begin{align}
	\sum_\mu \frac{\partial}{\partial x^B_\mu} \bigg|_p v_\mu^B
	\ =\ \sum_{\mu,\nu,\rho} \frac{\partial}{\partial x^A_\nu} \bigg|_p \,
	\frac{\partial x^A_\nu}{\partial x^B_\mu} \bigg|_{x^B(p)} \,
	\frac{\partial x^B_\mu}{\partial x^A_\rho} \bigg|_{x^A(p)} \,
	v^A_\rho
	\ =\ \sum_{\nu,\rho} \frac{\partial}{\partial x^A_\nu} \bigg|_p \,
	\delta_{\nu\rho} \,
	v^A_\rho
	\ =\ \sum_\nu \frac{\partial}{\partial x^A_\nu} \bigg|_p v_\nu^A
\end{align}



















\subsection{پایه‌های مختصاتی به عنوان تریویالیزاسیون‌های محلی کلاف}
\label{apx:correspondences_bundle_trivializations}

قوانین تبدیل القا شده توسط نگاشت گذار چارت در بخش~\ref{apx:chart_transition_induced_gauge_trafos} با تبدیلات پیمانه همانطور که در بخش~\ref{sec:21_main} فرمول‌بندی شده است، هنگام یکی گرفتن ژاکوبین‌ها
$\frac{\partial x^B}{\partial x^A} \big|_{x^A(p)}$ با $g_p^{BA}$ منطبق هستند.
در بخش~\ref{apx:correspondences_chart_gauge_ptwise} ما این ارتباطات را با فهرست کردن تمام تناظرات دقیق می‌کنیم.
بخش~\ref{apx:correspondences_chart_gauge_local} این نتایج را با استخراج عبارات برای تریویالیزاسیون‌های کلاف القا شده توسط چارت روی دامنه‌های گسترده~$U \subseteq M$ همانطور که در بخش~\ref{sec:bundles_fields} معرفی شد، گسترش می‌دهد.
یک فرهنگ لغت که تناظرات را خلاصه می‌کند در جدول~\ref{tab:coord_charts_gauge_trafos} ارائه شده است.



\subsubsection[تناظرات با تریویالیزاسیون‌های نقطه‌ای \texorpdfstring{$ \TpM$}{TpM}]%
{تناظرات با تریویالیزاسیون‌های نقطه‌ای \texorpdfstring{$\TpM$}{TpM}}
\label{apx:correspondences_chart_gauge_ptwise}


\paragraph{پیمانه‌ها و دیفرانسیل‌های چارت:}
فرمالیسم کلاف بر تعریف پیمانه‌ها (معادله~\eqref{eq:gauge_definition})
\begin{align}
	\psiTMp^A: \TpM \to \R^d \,,
\end{align}
تکیه دارد، که ایزومورفیسم‌های کلاف برداری هستند و مختصات را به فضاهای مماس با $p\in U^A$ اختصاص می‌دهند.
در فرمالیسم چارت، پیمانه‌ها روی $U^A$ به عنوان دیفرانسیل‌های چارت \emph{القا} می‌شوند (معادله~\eqref{eq:chart_differential_via_gradients}):
\begin{align}
	\hat{d}x^A_p: \TpM \to \R^d
\end{align}
پیمانه‌های مختلف با تبدیلات پیمانه به هم مرتبط هستند (معادله~\eqref{eq:gauge_trafo_local_def_21})
\begin{align}
	\psiTMp^B\ =\ g^{BA}_p\, \psiTMp^A \,\ \qquad
	\qquad &\textup{with} \qquad
	g^{BA}_p\ :=\ \psiTMp^B \circ \big(\psiTMp^A\big)^{-1} \ \ \in\ G \,.
	\intertext{
		همین تعریف برای پیمانه‌های القا شده توسط چارت نیز برقرار است، که در آن تبدیلات پیمانه با ژاکوبین نگاشت‌های گذار چارت منطبق می‌شوند (معادله~\eqref{eq:chart_differential_trafo_law}):
	}
	\hat{d}x^B_p\ =\ \frac{\partial x^B}{\partial x^A} \bigg|_{x^A(p)} \hat{d}x^A_p
	\qquad &\textup{with} \qquad
	\frac{\partial x^B}{\partial x^A} \bigg|_{x^A(p)}
	\!=\ \hat{d}x^B_p \circ \big( \hat{d}x^A_p \big)^{-1}
	\ \   \in\ \GL{d}
\end{align}



\paragraph{مؤلفه‌های برداری:}
از آنجا که مؤلفه‌های برداری $v^A = \psiTMp^A(v)$ یا $v^A = \hat{d}x^A|_p(v)$ با عمل پیمانه‌ها داده می‌شوند، آنها همان رفتار تبدیل کوواریانت را نشان می‌دهند
\begin{align}
	v^B = g_p^{BA} v^A
	\qquad &\textup{and} \qquad
	v^B = \frac{\partial x^B}{\partial x^A} \bigg|_{x^A(p)} v^A \,.
	\intertext{بر حسب مؤلفه‌ها، این روابط به صورت زیر نوشته می‌شوند}
	v^B_i = \sum_{j=1}^d \big(g_p^{BA}\big)_{ij}\, v^A_j
	\qquad &\textup{and} \qquad
	v^B_\mu = \sum_{\nu=1}^d \frac{\partial x^B_\mu}{\partial x^A_\nu} \bigg|_{x^A(p)} v^A_\nu \,.
\end{align}




\paragraph{چارچوب‌های مرجع القا شده:}

چارچوب‌های مرجع در فرمالیسم کلاف با نگاشت بردارهای~$\epsilon_i$ از چارچوب استاندارد $e\in G$ از $\R^d$ از طریق نگاشت پیمانه به $\TpM$ القا می‌شوند (معادله~\ref{eq:framefield_gauge_equivalence}):
\begin{align}
	\big[ e_i^A \big]_{i=1}^d\ =\ \Big[ \big(\psiTMp^A \big)^{-1} (\epsilon_i) \Big]_{i=1}^d
\end{align}
رابطه متناظر در فرمالیسم چارت طبق معادله~\eqref{eq:coord_basis_vector_via_chart_differential} با 
\begin{align}
	\bigg[ \frac{\partial}{\partial x^A_\mu} \bigg|_p \bigg]_{\mu=1}^d\ =\ \Big[ \big(\hat{d}x_p^A \big)^{-1} (\epsilon_\mu) \Big]_{\mu=1}^d
\end{align}
داده می‌شود. معادله~\eqref{eq:frame_rightaction} نشان می‌دهد که قوانین تبدیل چارچوب‌های مرجع با عمل راست
\begin{align}\label{eq:trafo_law_comparison_basis_gauge}
	\left[e_{i}^B\right]_{i=1}^d
	\  =\ \left[ e_{i}^A \right]_{i=1}^d \!\lhd \left(g_p^{BA}\right)^{-1}
	\ :=\ \left[ \sum\nolimits_{j=1}^d e_{j}^A\, \big(g_p^{BA}\big)^{-1}_{ji} \right]_{i=1}^d
	\ =\ \left[ \sum\nolimits_{j=1}^d e_{j}^A\, \big(g_p^{AB}\big)_{ji} \right]_{i=1}^d \,.
\end{align}
داده می‌شود. به طور مشابه، قانون تبدیل پایه‌های مختصاتی از معادله~\eqref{eq:coord_bases_trafo_law_with_inv} به صورت
\begin{align}\label{eq:trafo_law_comparison_basis_chart}
	\bigg[\frac{\partial}{\partial x^B_\mu} \bigg|_p \bigg]_{\mu=1}^d
	=\ \bigg[\frac{\partial}{\partial x^A_\mu} \bigg|_p \bigg]_{\mu=1}^d \mkern-6mu\lhd \frac{\partial x^B}{\partial x^A} \bigg|_{x^{\mkern-1muA}\mkern-1mu(p)}^{\;-1}
	&=\ \Bigg[ \sum_{\nu=1}^d \,
	\frac{\partial        }{\partial x^A_\nu} \bigg|_p \ 
	\bigg( \frac{\partial x^B}{\partial x^A} \bigg|_{x^{\mkern-1muA}\mkern-1mu(p)} \bigg)^{-1}_{\nu\mu}
	\Bigg]_{\mu=1}^d 
	\notag \\
	&=\ \Bigg[ \sum_{\nu=1}^d \,
	\frac{\partial        }{\partial x^A_\nu} \bigg|_p \ 
	\frac{\partial x^A_\nu}{\partial x^B_\mu} \bigg|_{x^{\mkern-1muB}\mkern-1mu(p)}
	\Bigg]_{\mu=1}^d 
\end{align}
دیده می‌شود.











\subsubsection[تریویالیزاسیون‌های محلی القا شده توسط چارت از \texorpdfstring{$ \pi_{\TM}^{-1}(U) $}{TU}]%
{تریویالیزاسیون‌های محلی القا شده توسط چارت از \texorpdfstring{$\pi_{\TM}^{-1}(U)$}{TU}}
\label{apx:correspondences_chart_gauge_local}

تناظراتی که در بخش قبل بیان شدند، تریویالیزاسیون‌های \emph{نقطه‌ای} $\psiTMp$ از $\TpM$ را به دیفرانسیل‌های چارت $\hat{d}x_p$ مرتبط می‌کردند.
برای تکمیل این تصویر، این بخش عباراتی را برای تریویالیزاسیون‌های محلی
\begin{align}
	\PsiTM: \piTM^{-1}(U) \to U \times \R^d
\end{align}
که توسط چارت‌ها القا می‌شوند، اضافه می‌کند.

یک کاندیدای خوب برای ساخت $\PsiTM$ از آن، دیفرانسیل چارت
\begin{align}
	dx: \piTM^{-1}(U) \to TV
\end{align}
است که یک ایزومورفیسم کلاف برداری است که از ایزومورفیسم‌های فضای برداری $dx_p$ با عدم محدودیت به یک نقطه منفرد $p \in U$ متفاوت است.
برای ادامه، ما ایزومورفیسم کانونی $\iota_{\R^d}$ را در معادله~\eqref{eq:canonical_iso_TRk_Rk} از یک نقطه منفرد به تمام فضاهای مماس $T_xV \cong \R^d$ روی $V \subseteq \R^d$ تعمیم می‌دهیم، که منجر به \emph{تریویالیزاسیون محلی کانونی} زیر از $TV$ می‌شود:
\begin{align}
	\iota_{TV}: TV \to V\times\R^d,\ \ v \mapsto \big( \piTV(v),\, \iota_{\R^d}(v) \big) \,.
\end{align}
این اجازه می‌دهد تا $\hat{d}x_p$ را از یک نقطه منفرد به یک نگاشت
\begin{align}
	\hat{d}x\, :=\, \iota_{V\times\R^d} \circ dx \,:\ \piTM^{-1}(U) \to V \times \R^d \,,
\end{align}
تعمیم دهیم، که با این حال، هنوز تریویالیزاسیون محلی مورد نظر نیست.
با نگاشت عامل اول از طریق چارت وارون از $V$ به $U$ ما \emph{تریویالیزاسیون محلی کلاف القا شده توسط چارت} را به دست می‌آوریم:
\begin{align}
	\PsiTM\ :=\ \big(x^{-1} \times \id \big) \circ \hat{d}x
\end{align}
طبق معمول، ما تعاریف انجام شده را در یک نمودار جابجایی به تصویر می‌کشیم:
\begin{equation}\label{cd:coordinate_basis_bundle_trivialization}
	\begin{tikzcd}[row sep=3.5em, column sep=6em]
		V \times \R^d
		\arrow[rrr, rounded corners, to path={ 
			([xshift=-1ex]\tikztostart.north)
			|- node[above, pos=.75]{\small$\big( x^{-1} \times \id \big)$} ([yshift=10ex]\tikztotarget.north)
			-- (\tikztotarget.north)
		}]
		\arrow[dr, "\proj_1"']
		&
		TV  \arrow[d, "\piTV"]
		\arrow[l, "\iota_{V \times \R^d}"']
		&
		\piTM^{-1}(U)
		\arrow[d, "\piTM"']
		\arrow[r, "\PsiTM"]
		\arrow[l, "dx"']
		\arrow[ll, rounded corners, to path={ 
			|- node[above, pos=.75]{\small$\hat{d}x$} ([yshift=4ex, xshift=1ex]\tikztotarget.north)
			-- ([xshift=1ex]\tikztotarget.north)
		}]
		&
		U \times \R^d
		\arrow[ld, "\proj_1"]
		\\
		&
		V
		&
		U
		\arrow[l, "x"]
	\end{tikzcd}
\end{equation}

با در نظر گرفتن دو چارت همپوشان $x^A: U^A \to V^A$ و $x^B: U^B \to V^B$ و با نشان دادن $U^{AB} = U^A \cap U^B$ نگاشت‌های گذار
\begin{align}
	\hat{d}x^B \circ \big( \hat{d}x^A \big)^{-1} \,=\,
	\left( x^B \mkern-5mu\circ\mkern-3mu (x^A)^{-1} \times \frac{\partial x^B}{\partial x^A} \right)
	\,:\ x^A\big( U^{AB}\big) \times \R^d \to x^A\big( U^{AB}\big) \times \R^d
\end{align}
و
\begin{align}
	\PsiTM^B \circ \big( \PsiTM^A \big)^{-1} \,=\,
	\left( \id \times \frac{\partial x^B}{\partial x^A} \right)
	\,:\ U^{AB} \times \R^d \to U^{AB} \times \R^d \,.
\end{align}
به دست می‌آیند. این تعاریف و رابطه متقابل آنها در نمودار جابجایی زیر نشان داده شده است:
\begin{equation}\label{cd:coordinate_basis_bundle_trivialization_transition}
	\begin{tikzcd}[row sep=3.5em, column sep=4em]
		x^B\big(U^{AB}\big) \times \R^d
		\arrow[rr, "{\big( (x^B)^{-1} \times \id \,\big)}"]
		& &
		U^{AB} \times \R^d
		\\
		&
		\piTM^{-1}\big(U^{AB}\big)
		\arrow[ul, "{\hat{d}x^B}"]
		\arrow[dl, "{\hat{d}x^A}"']
		\arrow[ur, "{\PsiTM^B}"']
		\arrow[dr, "{\PsiTM^A}"]
		\\
		x^A\big(U^{AB}\big) \times \R^d
		\arrow[rr, "{\big( (x^A)^{-1} \times \id \,\big)}"']
		\arrow[uu, "{\pig( x^B \mkern-5mu\circ\mkern-3mu (x^A)^{-1} \times \frac{\partial x^B}{\partial x^A} \pig)}"]
		&&
		U^{AB} \times \R^d
		\arrow[uu, "{
			$\begin{array}{l}
				\hspace*{10pt}
				\big(\id \times g^{BA} \big)
				\\ \rule{0pt}{16pt}
				= \pig( \id \times \frac{\partial x^B}{\partial x^A} \pig)
			\end{array}$
		}"' align = left]
	\end{tikzcd}
\end{equation}




















\subsection{$G$-ساختارها و میدان‌های ویل‌باین}
\label{apx:vielbein_fields}


همانطور که در بخش‌های~\ref{sec:G_associated_bundles} و~\ref{sec:bundle_trivializations} بحث شد، هر $G$-اطلس $\{(\PsiTM^X, U^X)\}$ از تریویالیزاسیون‌های محلی کلاف مماس، یک $G$-ساختار متناظر را مشخص می‌کند، یعنی یک زیرکلاف~$\GM$ از چارچوب‌های مرجع متمایز که به یک ساختار هندسی روی~$M$ احترام می‌گذارند (یا آن را تعریف می‌کنند).
بنا به تعریف، نگاشت‌های گذار $g^{BA}$ از کلاف‌های همبسته $G$ مقادیری در یک گروه ساختاری کاهش یافته $G\leq\GL{d}$ می‌گیرند.
این سؤال را مطرح می‌کند که آیا می‌توان به طور مشابه «$G$-اطلس‌هایی از چارت‌ها» $\{(x^X, U^X)\}$ را یافت، که ژاکوبین‌های آنها $\frac{\partial x^B}{\partial x^A}$ مقادیری در یک گروه ساختاری کاهش یافته $G \leq \GL{d}$ بگیرند و بنابراین یک $G$-ساختار را کدگذاری کنند.
برای برخی گروه‌های ساختاری این قطعاً ممکن است؛ به عنوان مثال، یک جهت‌گیری از یک منیفلد جهت‌پذیر همیشه می‌تواند با مشخص کردن یک اطلس $\operatorname{GL}^+(d)$ از چارت‌های با جهت‌گیری مثبت، که ژاکوبین‌های گذار آنها مقادیری در $\operatorname{GL}^+(d)$ می‌گیرند، ثابت شود.
با این حال، به طور کلی، یافتن چارت‌های مختصاتی که پایه‌های مختصاتی را القا کنند که در یک $G$-ساختار داده شده قرار گیرند، غیرممکن است.
بنابراین به \emph{تبدیلات پیمانه صریح از پایه‌های مختصاتی به $G$-ساختار}، که به عنوان \emph{میدان‌های ویل‌باین} شناخته می‌شوند، متوسل می‌شویم~\cite{yepez2011einstein, zhou2016gauge, nakahara2003geometry, carroll2004spacetime}.
پس از تبدیل اولیه از پایه‌های مختصاتی به $G$-ساختار، آزادی پیمانه در داخل $G$-ساختار به تبدیلات پیمانه با مقادیر $G$ که $G$-ساختار را حفظ می‌کنند، اجازه می‌دهد.


یک مثال مهم در فیزیک، $\OO{d}$-ساختارها (یا $\OO{1,\, d-1}$-ساختارها برای فضازمان‌ها) هستند، که از چارچوب‌های مرجع راست‌هنجار نسبت به متریک (شبه) ریمانی $\eta$ از~$M$ تشکیل شده‌اند.%
\footnote{
	نماد $\eta$ در ادبیات فیزیک معمولاً برای متریک مینکوفسکی $\operatorname{diag}(+1,\, -1,\, \dots,\, -1)$ حفظ می‌شود در حالی که متریک (شبه) ریمانی~$M$ با~$g$ نشان داده می‌شود.
	در مقابل، ما اعضای گروه را در گروه ساختاری به صورت $g\in G$ می‌نویسیم و بنابراین از $\eta$ برای متریک (شبه) ریمانی~$M$ استفاده می‌کنیم.
}
چنین چارچوب‌های راست‌هنجاری، چارچوب‌های آزمایشگاهی ممکن یک ناظر لخت را نمایش می‌دهند.
آنها به عنوان مثال برای فرمول‌بندی نظریه‌های میدان کوانتومی نسبیتی، به ویژه معادله دیراک، در فضازمان‌های خمیده استفاده می‌شوند.
به یاد بیاورید که یک $G$-ساختار داده شده باید توسط تریویالیزاسیون‌های محلی کلاف مورد احترام قرار گیرد، که به این معنی است که نگاشت‌های پیمانه $\psiGMp$ باید $G$-ساختار $\GpM$ را در~$p\in M$ به $G$-ساختار استاندارد کانونی $G$ از $\R^d$ نگاشت دهند.
برای مورد خاص $\OO{d}$-ساختارها، این معادل این الزام است که تریویالیزاسیون‌های کلاف متریک را حفظ کنند، یعنی $\eta_p(v,w) = \langle \psiTMp(v), \psiTMp(w) \rangle$ برای هر $p\in M$ و $v,w \in \TpM$ که این در فرمالیسم کلاف بدون مشکل انجام می‌شود.
با توجه به یک چارت مختصاتی $x: U \to V$ پیمانه‌های القا شده در $p\in U$ در بخش‌های قبلی نشان داده شد که با $\psiTMp = \hat{d}x_p: \TpM \to \R^d$ داده می‌شوند.
بنابراین الزام بر روی آنها برای حفظ متریک به
\begin{align}
	\eta_p(v,w) = \pig\langle \hat{d}x_p(v) \,,\, \hat{d}x_p(w) \pig\rangle \,,
\end{align}
تبدیل می‌شود، که دقیقاً ویژگی تعریف‌کننده برای~$x$ به عنوان یک \emph{ایزومتری} است.
این نتیجه دلالت بر این دارد که \emph{پایه‌های مختصاتی فقط در صورتی یک $\OO{d}$-ساختار را تعریف می‌کنند که $U$ و $V$ ایزومتریک باشند} -- که فقط در صورتی است که $M$ به صورت محلی روی $U$ تخت باشد.
بنابراین برای هر ناحیه غیرتخت از $M$ توصیف مستقیم یک $\OO{d}$-ساختار از طریق پایه‌های مختصاتی غیرممکن است.
این ناسازگاری به عنوان مثال در این واقعیت بیان می‌شود که مؤلفه‌های~$\eta_{\mu\nu}$ از متریک ریمانی روی~$M$ نسبت به پایه مختصاتی انتخاب شده با $\delta_{\mu\nu}$ (یا $\operatorname{diag}(+1,-1,\dots,-1)_{\mu\nu}$) متفاوت هستند.

همانطور که قبلاً ذکر شد، چارچوب‌های راست‌هنجار یک $\OO{d}$-ساختار $\OM$ در ادبیات فیزیک معمولاً از طریق یک تبدیل پیمانه نسبت به یک میدان چارچوب القا شده توسط چارت $\big[ \frac{\partial}{\partial x_\mu} \big]_{\mu=1}^d$ تعریف می‌شوند.
با نشان دادن این تبدیل پیمانه، که \emph{میدان ویل‌باین} نامیده می‌شود، با
\begin{align}
	\mathfrak{e}^A: U \to \GL{d} \,,
\end{align}
میدان چارچوب راست‌هنجار با%
\footnote{
	در ادبیات فیزیک این رابطه به صورت
	$e^A_i = (\mathfrak{e}^A)^{\mu}_{\,\ i} \frac{\partial}{\partial x^\mu}$
	بیان می‌شود. وارون در اینجا صرفاً با موقعیت مخالف اندیس‌ها
	$(\mathfrak{e}^A)^\mu_{\,\ i} := (\mathfrak{e}^A)^{-1}_{\mu i}$ 
	در مقایسه با
	$(\mathfrak{e}^A)_\mu^{\,\ i} := \mathfrak{e}^A_{\mu i}$
	نشان داده می‌شود.
}
\begin{align}
	\big[e^A_i\big]_{i=1}^d
	\ :=\ \bigg[ \frac{\partial}{\partial x_\mu} \bigg]_{i=1}^d \lhd \big(\mathfrak{e}^A\big)^{-1}
	\  =\ \bigg[ \sum_\mu \frac{\partial}{\partial x_\mu} \big(\mathfrak{e}^A\big)^{-1}_{\mu i} \bigg]_{i=1}^d
	\ \ \in\ \Gamma(U, \OM) \,.
\end{align}
تعریف می‌شود. راست‌هنجاری میدان چارچوب حاصل معمولاً به صورت%
\footnote{
	در ادبیات فیزیک این رابطه معمولاً به صورت
	$\eta_{\mu\nu}\ (\mathfrak{e}^A)^\mu_{\,\ i}\ (\mathfrak{e}^A)^\nu_{\,\ j}\, =\, \delta_{ij}$ نوشته می‌شود.
}
\begin{align}
	\delta_{ij}
	\ &=\ \eta\big( e_i^A,\, e_j^A \big) \notag \\
	\ &=\ \eta\bigg( \sum_\mu \frac{\partial}{\partial x_\mu} \big(\mathfrak{e}^A\big)^{-1}_{\mu i}  \,,\; \sum_\nu \frac{\partial}{\partial x_\nu} \big(\mathfrak{e}^A\big)^{-1}_{\nu j} \bigg) \notag \\
	\ &=\ \sum_{\mu\nu} \eta\bigg( \frac{\partial}{\partial x_\mu} \,,\, \frac{\partial}{\partial x_\nu} \bigg)\, \big(\mathfrak{e}^A\big)^{-1}_{\mu i}\, \big(\mathfrak{e}^A\big)^{-1}_{\nu j} \notag \\
	\ &=\ \sum_{\mu\nu} \eta_{\mu\nu}\, \big(\mathfrak{e}^A\big)^{-1}_{\mu i}\, \big(\mathfrak{e}^A\big)^{-1}_{\nu j} \,,
\end{align}
بیان می‌شود، که توضیح می‌دهد چرا میدان ویل‌باین گاهی اوقات «ریشه دوم متریک» نامیده می‌شود.
طبق معمول، مؤلفه‌های برداری از طریق تبدیل پیمانه غیروارون ترجمه می‌شوند، یعنی:%
\footnote{
	دوباره، در نمادگذاری معمول در فیزیک این رابطه به صورت $(v^A)^i\, =\, (\mathfrak{e}^A)_\mu^{\,\ i} v^\mu$ خوانده می‌شود.
}
\begin{align}
	v^A_i\ =\ \sum_\mu \mathfrak{e}^A_{i\mu}\, v_\mu
\end{align}

یک استدلال ساده شمارش ابعاد، آزادی پیمانه را در $\OO{d}$-ساختار نشان می‌دهد:%
\footnote{
	در فیزیک، بیشتر تبدیلات لورنتس محلی $\Lambda \in \OO{1,3}$ را در نظر می‌گیرند، که دوران‌ها و بوست‌های چارچوب‌های مرجع محلی را توصیف می‌کنند.
}
یک ویل‌باین $\mathfrak{e}^A(p) \in \GL{d}$ به عنوان عضوی از گروه خطی عمومی دارای $d^2$ درجه آزادی است، در حالی که متریک $\eta$ به عنوان یک فرم دوخطی متقارن، دارای $d(d+1)/2$ درجه آزادی است.
$d(d-1)/2$ درجه آزادی گمشده دقیقاً متناظر با تبدیلات پیمانه با اعضای گروه ساختاری $g^{BA} \in \OO{d}$ است.
به طور جایگزین، از دیدگاه $G$-ساختارها، $\FpM \cong \GL{d}$ دارای $d^2$ درجه آزادی است در حالی که $\OpM \cong \OO{d}$ دارای $d(d-1)/2$ درجه آزادی است، که $d(d+1)/2$ درجه آزادی را که متناظر با انتخاب متریک است، ثابت می‌کند.


تمام ساختارها به وضوح به $G$-ساختارهای دلخواه با میدان‌های ویل‌باین با مقادیر $\GL{d}$ که پایه‌های مختصاتی را به $\GM$ نگاشت می‌دهند و آزادی اعمال تبدیلات پیمانه با مقادیر $G$ پس از آن، تعمیم می‌یابند.



\begin{landscape}
	\begin{table}[h!]
		\vspace*{8ex}
		\centering%
		\scalebox{1.}{%
			\def\arraystretch{2.75}% 1 is the default
\setlength\tabcolsep{2.8ex}
\small
\begin{tabular}{ @{\ \ } l r@{\,}r@{\ }c@{\ }l cc @{\ \ } }
	\toprule
	\\[-8.0ex]
	& \multicolumn{4}{c}{ایزومورفیسم}
	& فرمالیسم کلاف
	& فرمالیسم چارت \\
	\midrule[0.07em] % default width = 0.05
	%%%%%%%%%%%%%%%%%%%%%%%%%%%%%%%%%%%%%%%%%%%%%%%%%%%%%%%%%%%%%%%%%%%%%
	چارت
	& $x^A :\ $
	& $U^A$
	& $\xrightarrow{\sim}$
	& $V^A$
	& ـــ
	& هر دیفئومورفیسم دلخواه
	\\
	نگاشت گذار
	& $x^B \!\circ\! \big(x^A\big)^{-1} :\ $
	& $x^B \mkern-1mu\big( U^{\mkern-2muA\mkern-2muB} \big)$
	& $\xrightarrow{\sim}$
	& $x^A \mkern-1mu\big( U^{\mkern-2muA\mkern-2muB} \big)$
	& ـــ
	& القا شده توسط چارت‌ها
	\\
	\midrule[0.04em] % default width = 0.05
	%%%%%%%%%%%%%%%%%%%%%%%%%%%%%%%%%%%%%%%%%%%%%%%%%%%%%%%%%%%%%%%%%%%%%
	تریویالیزاسیون نقطه‌ای \hspace*{-3ex}
	& $\psiTMp^A :\ $
	& $\TpM$
	& $\xrightarrow{\sim}$
	& $\R^d$
	& ایزومورفیسم خطی از $G$-اطلس
	& $\hat{d}x^A_p = \big( \hat{d}x^A_1|_p,\, \dots,\, \hat{d}x^A_d|_p\, \big)^{\!\top}$
	\\
	نگاشت گذار
	& $\psiTMp^B \mkern-2mu\circ\! \big(\psiTMp^A\big)^{\mkern-2mu-1} \!:\ $
	& $\R^d$
	& $\xrightarrow{\sim}$
	& $\R^d$
	& عنصر گروه ساختار $g_p^{BA} \in G$
	& $\hat{d}x^B_p \circ \big(\hat{d}x^A_p\big)^{-1}
	=\, \displaystyle \frac{\partial x^B}{\partial x^A} \bigg|_{\mkern-1mu x^{\mkern-1mu A}\mkern-2mu(p)} $
	\\
	\midrule[0.04em] % default width = 0.05
	%%%%%%%%%%%%%%%%%%%%%%%%%%%%%%%%%%%%%%%%%%%%%%%%%%%%%%%%%%%%%%%%%%%%%
	تریویالیزاسیون محلی
	& $\PsiTM^A :\ $
	& $\piTM^{-1} \big(U^A\big)$
	& $\xrightarrow{\sim}$
	& $U^A \times \R^d$
	& $v \mapsto \big( \piTM\mkern-1mu(v),\ \psiTMpiv(v) \big)$
	& $\big((x^A)^{-1} \times \id\big) \circ \hat{d}x^A$
	\\
	نگاشت گذار
	& $\PsiTM^B \mkern-2mu\circ\! \big(\PsiTM^A\big)^{\mkern-2mu-1} :\ $
	& $U^{\mkern-2muA\mkern-2muB} \mkern-4mu\times\! \R^d$
	& $\xrightarrow{\sim}$
	& $U^{\mkern-2muA\mkern-2muB} \mkern-4mu\times\! \R^d$
	& $\big(\id \times g^{BA} \big)$
	& $\displaystyle \bigg( \id \times \frac{\partial x^B}{\partial x^A} \bigg)$
	\\
	\midrule[0.07em] % default width = 0.05
	%%%%%%%%%%%%%%%%%%%%%%%%%%%%%%%%%%%%%%%%%%%%%%%%%%%%%%%%%%%%%%%%%%%%%
	چارچوب عمومی
	& \multicolumn{4}{c}{$\big[ e^A_i \big]_{i=1}^d \in \FpM$}
	& $\Big[ \big(\psiTMp^A\big)^{-1} (\epsilon_i) \Big]_{i=1}^d$\ از $\GL{d}$-اطلس
	& $\displaystyle \bigg[\frac{\partial}{\partial x^A_\mu} \bigg|_p \,\bigg]_{\mu=1}^d = \Big[ \big(\hat{d}x^A_p \big)^{-1} (\epsilon_i) \Big]_{\mu=1}^d$
	\\
	چارچوب $G$-ساختار
	& \multicolumn{4}{c}{$\big[ e^A_i \big]_{i=1}^d \in \GpM$}
	& $\Big[ \big(\psiTMp^A\big)^{-1} (\epsilon_i) \Big]_{i=1}^d$\ از $G$-اطلس \kern16pt
	& $\displaystyle \bigg[\sum\nolimits_{\mu}\, \frac{\partial}{\partial x_\mu} \bigg|_p\, \big( \mathfrak{e}^A \big)^{-1}_{\!\mu i}\, \bigg]_{i=1}^d$
	\\
	\bottomrule
\end{tabular}
		}%
		\vspace*{4ex}%
		\captionsetup{width=.9\linewidth}
		\caption{
			یک نمای کلی از انواع مختلف مختصاتی‌سازی روی منیفلدها.
			فرمالیسم کلاف (ستون سوم)، که در این کار استفاده می‌شود، مختصات را مستقیماً به فضاهای مماس اختصاص می‌دهد، در حالی که به نقاط~$p$ از فضای پایه~$M$ به روشی مستقل از مختصات اشاره می‌کند.
			در مقابل، فرمالیسم چارت (ستون چهارم) مختصات را به زیرمجموعه‌های محلی $U^X \subseteq M$ از منیفلد اختصاص می‌دهد.
			تریویالیزاسیون‌های محلی کلاف مماس و نگاشت‌های گذار کلاف بین آنها به عنوان دیفرانسیل‌های چارت‌ها و نگاشت‌های گذار آنها، که دومی معمولاً به عنوان ژاکوبین‌ها نامیده می‌شوند، القا می‌شوند.
			ردیف ماقبل آخر عباراتی را برای چارچوب‌های مرجعی که به عنوان مقاطع همانی از تریویالیزاسیون‌های محلی $\TM$ (ستون سوم) یا به عنوان پایه‌های مختصاتی القا شده توسط چارت (ستون چهارم) القا می‌شوند، ارائه می‌دهد.
			به طور مشابه، ردیف آخر تعاریف $G$-ساختارها -- به عنوان مثال چارچوب‌های راست‌هنجار -- را از طریق یک $G$-اطلس برای $\TM$ (ستون سوم) و از طریق میدان‌های ویل‌باین به عنوان تبدیلات پیمانه نسبت به پایه‌های مختصاتی (ستون چهارم) مقایسه می‌کند.
			طبق معمول، ما $U^A \cap U^B$ را با $U^{AB}$ مخفف می‌کنیم و فرض می‌کنیم $p\in U^{AB}$.
		}
		\label{tab:coord_charts_gauge_trafos}
	\end{table}
\end{landscape}