%!TEX root=../GaugeCNNTheory.tex


\newpage
\setcounter{tocdepth}{2} % Keep tableofcontents depth
\tableofcontents

~ % <<=== newline

این کار در قالب یک مقدمه، سه بخش اصلی و یک پیوست سازماندهی شده است.

بخش~\ref{part:local_theory} تلاش می‌کند شبکه‌های عصبی مستقل از مختصات را با زبانی آسان معرفی کند.
میدان‌های ویژگی و لایه‌های شبکه نسبت به \emph{مختصات محلی} (تسهیم‌های بندل) بیان می‌شوند.
\emph{استقلال مختصات} مورد نیاز، مستلزم آن است که ویژگی‌ها با یک \emph{قانون تبدیل} خاصی مرتبط باشند.
لایه‌های شبکه ملزم به تضمین رفتار تبدیل صحیح ویژگی‌ها هستند.

بخش~\ref{part:bundle_theory} نظریه شبکه‌های عصبی مستقل از مختصات را بر حسب \emph{بندل‌های فیبر} رسمی‌سازی می‌کند.
این امر امکان فرمول‌بندی \emph{سراسری و مستقل از مختصات} را فراهم می‌کند، که به ویژه هنگام بررسی تناوب‌پذیری ایزومتری شبکه‌ها مفید است.
تعاریف از بخش~\ref{part:local_theory} با بیان عملیات مستقل از مختصات در تسهیم‌های بندل محلی (مختصات) بازیابی می‌شوند.

بخش~\ref{part:literature_review} نظریه ما را در \emph{کارهای مرتبط} جای می‌دهد.
این بخش بررسی‌های مفصلی از معماری‌های شبکه کانولوشنی بر روی هندسه‌های مختلف ارائه می‌دهد و آن‌ها را به عنوان شبکه‌های کانولوشنی مستقل از مختصات فرمول‌بندی مجدد می‌کند.
برای تسهیل توسعه معماری‌های شبکه جدید، ما ویژگی‌های مربوط به هندسه‌های خاص را قبل از بررسی شبکه‌هایی که بر روی آن‌ها عمل می‌کنند، مورد بحث قرار می‌دهیم.

خواننده می‌تواند بخش~\ref{part:bundle_theory} را در مرحله اول نادیده بگیرد
فرمول‌بندی از بخش~\ref{part:local_theory} برای خواندن مرور ادبیات در بخش~\ref{part:literature_review} کاملاً کافی است.

مروری بر مفاهیم و نتایج اصلی کار ما در بخش~\ref{sec:visual_intro} زیر ارائه شده است.
این مرور از معادلات پرهیز می‌کند و بر شهود هندسی از طریق بصری‌سازی‌ها تکیه دارد.
امیدواریم این بخش به مخاطبان غیرتخصصی کمک کند تا ایده‌ای از محتوای کار ما به دست آورند.


\subsubsection*{مرور کلی جزئیات}


\paragraph{بخش~\ref{part:local_theory}:}

هدف بخش~\ref{sec:gauge_cnns_intro_local} ابداع فضاهای ویژگی مستقل از مختصات است.
به طور خاص، بخش~\ref{sec:21_main} \emph{گیج‌ها، تبدیل‌های گیج و $G$-ساختارها} را معرفی می‌کند.
گیج‌ها روشی رسمی برای بیان بردارهای مماس (مستقل از مختصات) و توابع روی فضاهای مماس نسبت به چارچوب‌های مرجع هستند.
تبدیل‌های گیج بین این عبارات مختصاتی در گیج‌های مختلف ترجمه می‌کنند.
بخش~\ref{sec:feature_fields} \emph{میدان‌های بردار ویژگی} مستقل از مختصات را معرفی می‌کند.
همانند مورد بردارهای مماس، ضرایب عددی بردارهای ویژگی هنگام انتقال بین چارچوب‌های مرجع تغییر می‌کنند.
قوانین تبدیل بردارهای ویژگی به ویژه \emph{انتقال موازی} آن‌ها و پیش‌برداری آن‌ها هنگام عمل توسط \emph{ایزومتری‌ها} را تعیین می‌کنند، که به ترتیب در بخش‌های~\ref{sec:transport_local} و~\ref{sec:isometries_local} توضیح داده شده‌اند.

بخش~\ref{sec:gauge_CNNs_local} \emph{شبکه‌های عصبی} را که بین میدان‌های ویژگی نگاشت می‌کنند، توسعه می‌دهد.
\emph{عملیات نقطه‌ای}، مانند جمع بایاس، کانولوشن‌های $1\times1$ و غیرخطی‌ها، در بخش~\ref{sec:pointwise_operations} مورد بحث قرار می‌گیرند.
بخش~\ref{sec:gauge_conv_main} بر روی \emph{کانولوشن‌ها} با کرنل‌های گسترده فضایی تمرکز می‌کند.
هر یک از این عملیات در ابتدا بدون فرض اشتراک وزن معرفی می‌شوند، یعنی به عنوان مثال، اجازه یک کرنل متفاوت در هر نقطه از منیفولد را می‌دهند.
این کرنل‌ها (یا بایاس‌ها یا غیرخطی‌ها) به هیچ وجه محدود نمی‌شوند.
با این حال، هنگام نیاز به اشتراک وزن فضایی، آن‌ها مجبور می‌شوند که تناوب‌پذیر گیج باشند زیرا تنها مقادیر تناوب‌پذیر می‌توانند به صورت مستقل از مختصات به اشتراک گذاشته شوند.
بخش~\ref{sec:gauge_conv_isom_equiv} یک اثبات مختصر از تناوب‌پذیری ایزومتری کانولوشن‌های $\GM$ را بر حسب عبارات مختصاتی محلی ارائه می‌دهد.
ایده اصلی در اینجا این است که ایزومتری‌ها را می‌توان به عنوان القاکننده تبدیل‌های گیج (تفسیر منفعل) در نظر گرفت، که با تناوب‌پذیری گیج کرنل‌ها توضیح داده می‌شود.

بخش~\ref{sec:mobius_conv} پیاده‌سازی \emph{کانولوشن‌های مستقل از جهت‌گیری بر روی نوار موبیوس} را توصیف می‌کند.
پس از بررسی هندسه نوار موبیوس در بخش~\ref{sec:mobius_geometry}، انواع مختلفی از میدان‌های ویژگی در بخش~\ref{sec:mobius_representations} تعریف می‌شوند.
بخش~\ref{sec:mobius_cnn_ops_analytical} بعدی، شبکه‌های کانولوشنی مستقل از جهت‌گیری را به صورت تحلیلی توصیف می‌کند.
به طور خاص، ما کرنل‌های کانولوشنی تناوب‌پذیر گیج، بایاس‌ها و غیرخطی‌ها را برای هر یک از انواع میدان استخراج می‌کنیم.
بخش~\ref{sec:mobius_experiment_main} با یک پیاده‌سازی و ارزیابی عددی مدل‌های مربوطه به پایان می‌رسد.


\paragraph{بخش~\ref{part:bundle_theory}:}

بخش~\ref{sec:bundles_fields} محتوای بخش~\ref{sec:gauge_cnns_intro_local} را بازتاب می‌دهد، با این حال، به صورت سراسری و بر حسب \emph{بندل‌های فیبر}.
مقدمه‌ای کلی بر بندل‌های فیبر در بخش~\ref{sec:fiber_bundles_general} ارائه شده است.
بخش‌های~\ref{sec:GL_associated_bundles} و~\ref{sec:G_associated_bundles} بندل مماس $\TM$، بندل چارچوب $\FM$، $G$-ساختارها $\GM$ و بندل‌های بردار ویژگی مرتبط با $G$ ($\A$) را معرفی می‌کنند.
میدان‌های ویژگی به صورت سراسری به عنوان برش‌هایی از بندل‌های بردار ویژگی تعریف می‌شوند.
\emph{تسهیم‌های بندل محلی} (گیج‌ها)، که در بخش~\ref{sec:bundle_trivializations} مورد بحث قرار گرفته‌اند، این بندل‌ها را در مختصات بیان می‌کنند و بدین ترتیب تعاریف ما از بخش~\ref{sec:gauge_cnns_intro_local} را بازیابی می‌کنند.
ما به طور خاص نشان می‌دهیم که چگونه تسهیم‌های محلی بندل‌های مختلف یکدیگر را القا می‌کنند، به طوری که تبدیل‌های گیج (نگاشت‌های انتقال) آن‌ها همگام‌سازی می‌شوند.
بخش~\ref{sec:bundle_transport} \emph{انتقال‌دهنده‌های موازی} در $G$-بندل‌ها را مورد بحث قرار می‌دهد.

بخش~\ref{sec:gauge_CNNs_global} شبکه‌های مستقل از مختصات از بخش~\ref{sec:gauge_CNNs_local} را بر حسب بندل‌های فیبر بازنویسی می‌کند.
کانولوشن‌های $1\times1$ در بخش~\ref{sec:onexone} به عنوان مورفیسم‌های مشخص بندل برداری $M$ توصیف می‌شوند.
به طور جایگزین، آن‌ها را می‌توان به عنوان برش‌هایی از یک بندل همومورفیسم مشاهده کرد.
بخش~\ref{sec:global_conv} \emph{میدان‌های کرنل مستقل از مختصات} و \emph{تبدیل‌های میدان کرنل} را معرفی می‌کند.
این عملیات مشابه کانولوشن‌های $\GM$ هستند اما نیازی به اشتراک وزن ندارند، یعنی ممکن است در هر مکان فضایی یک کرنل متفاوت اعمال کنند.
یک \emph{میدان کرنل کانولوشنی} $\GM$ با اشتراک یک کرنل $G$-استیریبل (تناوب‌پذیر گیج) منفرد در سراسر منیفولد ساخته می‌شود.
سپس \emph{کانولوشن‌های مستقل از مختصات} $\GM$ به عنوان تبدیل‌های میدان کرنل با میدان‌های کرنل $\GM$-کانولوشنی تعریف می‌شوند.
هنگام بیان فرمول‌بندی مستقل از مختصات کانولوشن‌های $\GM$ نسبت به تسهیم‌های محلی (گیج‌ها)، عبارات مختصاتی کانولوشن‌های $\GM$ را از بخش~\ref{sec:gauge_conv_main} بازیابی می‌کنیم.

\emph{تناوب‌پذیری ایزومتری} کانولوشن‌های $\GM$ در بخش~\ref{sec:isometry_intro} بررسی می‌شود.
پس از معرفی ایزومتری‌ها، بخش~\ref{sec:isom_background} \emph{عمل پیش‌برنده} آن‌ها را بر روی بندل‌های فیبر مورد بحث قرار می‌دهد.
این عمل نیز می‌تواند در تسهیم‌های محلی بیان شود، که منجر به فرمول‌بندی از بخش~\ref{sec:isometries_local} می‌شود.
بخش~\ref{sec:isometry_equivariance} عمل ایزومتری‌ها بر روی میدان‌های کرنل را تعریف می‌کند و ثابت می‌کند که \emph{تناوب‌پذیری ایزومتری یک تبدیل میدان کرنل، ناوردایی ایزومتری میدان کرنل آن را نتیجه می‌دهد و بالعکس}.
کانولوشن‌های $\GM$ ثابت شده‌اند که تحت عمل آن ایزومتری‌هایی که اتومورفیسم‌های بندلی (تقارن‌ها) از $G$-ساختار~$\GM$ هستند، تناوب‌پذیر می‌باشند.
بخش~\ref{sec:quotient_kernel_fields} میدان‌های کرنل ناوردا با ایزومتری را با جزئیات بیشتری بررسی می‌کند و ثابت می‌کند که آن‌ها معادل \emph{میدان‌های کرنل در فضاهای خارج‌قسمتی} از عمل ایزومتری هستند
-- به طور شهودی، میدان‌های کرنل ناوردا با ایزومتری ملزم به اشتراک کرنل‌ها بر روی مدارات ایزومتری هستند.
این نتیجه به ویژه دلالت دارد که تبدیل‌های میدان کرنل تناوب‌پذیر ایزومتری در \emph{فضاهای همگن} لزوماً کانولوشن‌های $\GM$ هستند.


\paragraph{بخش~\ref{part:literature_review}:}

بخش سوم این کار نشان می‌دهد که تعداد زیادی از شبکه‌های کانولوشنی از ادبیات را می‌توان به عنوان اعمال کانولوشن‌های $\GM$ برای انتخاب خاصی از $G$-ساختار و انواع میدان تفسیر کرد.
این بخش با بحثی کلی در مورد انتخاب‌های طراحی شبکه‌های کانولوشنی مستقل از مختصات آغاز می‌شود.
جدول~\ref{tab:network_instantiations} مروری و طبقه‌بندی از مدل‌های مورد بررسی را ارائه می‌دهد.
خواننده دعوت می‌شود که به $G$-ساختارهای بصری‌سازی شده در بخش~\ref{part:literature_review} نگاهی بیندازد زیرا اینها ایده شهودی در مورد ویژگی‌های کانولوشن‌های $\GM$ مربوطه ارائه می‌دهند.

\emph{شبکه‌های کانولوشنی اقلیدسی} که نه تنها تناوب‌پذیر ایزومتری هستند بلکه به طور کلی‌تر تحت عمل \emph{گروه‌های آفین} تناوب‌پذیر هستند، در بخش~\ref{sec:instantiations_euclidean} بررسی می‌شوند.
این مدل‌ها اساساً معادل \emph{شبکه‌های کانولوشنی استیریبل} در فضاهای \emph{برداری} اقلیدسی $\R^d$ \cite{Cohen2017-STEER,3d_steerableCNNs,Weiler2019_E2CNN} هستند.
بخش~\ref{sec:steerable_cnns_in_coords} شبکه‌های کانولوشنی استیریبل را بررسی می‌کند و ارتباط آن‌ها با کانولوشن‌های $\GM$ را مورد بحث قرار می‌دهد.
این رویکرد تا حدی نامطلوب است زیرا $\R^d$ با یک میدان چارچوب کانونی (ساختار $\{e\}$) همراه است، که به طور ضمنی توسط مدل‌های تناوب‌پذیر نادیده گرفته می‌شود.
بخش~\ref{sec:euclidean_geometry} رویکردی اصولی‌تر را در پیش می‌گیرد و فضاهای \emph{آفین} اقلیدسی $\Euc_d$ را تعریف می‌کند که دقیقاً با $G$-ساختارهایی مجهز شده‌اند که منجر به کانولوشن‌های $\GM$ تناوب‌پذیر $\Aff(G)$ می‌شوند.
کانولوشن‌های واقعی $\GM$ در بخش~\ref{sec:euclidean_affine_equiv} تعریف می‌شوند.
بخش~\ref{sec:euclidean_literature} شبکه‌های کانولوشنی اقلیدسی تناوب‌پذیر آفین را که در ادبیات یافت می‌شوند، بررسی می‌کند.
آن‌ها عمدتاً در انتخاب‌های مفروض گروه‌های ساختار و نمایش‌های گروهی متفاوت هستند.

بخش~\ref{sec:instantiations_euclidean_polar} شبکه‌های کانولوشنی را در \emph{فضاهای اقلیدسی سوراخ‌دار} $\Euc_d\backslash\{0\}$، که مبدأ $\{0\}$ آن‌ها حذف شده است، پوشش می‌دهد.
این مدل‌ها نسبت به چرخش حول مبدأ تناوب‌پذیر هستند، با این حال، نسبت به انتقال تناوب‌پذیر نیستند.
آن‌ها بر اساس $G$-ساختارهایی هستند که با مختصات قطبی، مختصات لگاریتمی-قطبی یا مختصات کروی مطابقت دارند.

\emph{شبکه‌های کانولوشنی کروی} در بخش~\ref{sec:instantiations_spherical} پوشش داده می‌شوند.
بخش~\ref{sec:sphere_geometry} هندسه ۲-کره (جاسازی شده)~$S^2$ را مورد بحث قرار می‌دهد.
با تفسیر فضاهای مماس به عنوان زیرفضاهای دو بعدی یک فضای جاسازی~$\R^3$، عبارات فرم بسته نگاشت‌های نمایی و لگاریتمی، چارچوب‌ها، گیج‌ها، انتقال‌دهنده‌ها و عمل ایزومتری را استخراج می‌کنیم.
بخش~\ref{sec:spherical_CNNs_fully_equivariant} شبکه‌های کانولوشنی کروی تناوب‌پذیر $\SO{3}$ و $\OO{3}$ را بررسی می‌کند.
ما به طور خاص ثابت می‌کنیم که نظریه ما فرمول‌بندی عمومی کانولوشن‌های کروی توسط کوهن و همکاران~\cite{Cohen2019-generaltheory} را به عنوان یک حالت خاص شامل می‌شود.
شبکه‌های کانولوشنی کروی که فقط نسبت به چرخش $\SO{2}$ حول یک محور ثابت تناوب‌پذیر هستند در بخش~\ref{sec:spherical_CNNs_azimuthal_equivariant} توصیف می‌شوند.
بخش~\ref{sec:spherical_CNNs_icosahedral} شبکه‌های کانولوشنی بیست‌وجهی را بررسی می‌کند.
بیست‌وجهی کره را تقریب می‌زند اما از وجوه محلی مسطح تشکیل شده است که امکان پیاده‌سازی کارآمد عملیات کانولوشن را فراهم می‌کند.

یک بررسی از شبکه‌های کانولوشنی بر روی \emph{سطوح دو بعدی عمومی} در بخش~\ref{sec:instantiations_mesh} یافت می‌شود.
بخش~\ref{sec:surfaces_geom_main} مقدمه‌ای کوتاه بر هندسه دیفرانسیل کلاسیک سطوح جاسازی شده و گسسته‌سازی آن‌ها بر حسب شبکه‌های مثلثی ارائه می‌دهد.
کانولوشن‌های سطحی در ادبیات به دو دسته طبقه‌بندی می‌شوند:
دسته اول، که در بخش~\ref{sec:so2_surface_conv} پوشش داده شده است، بر اساس کرنل‌های $G=\SO{2}$-استیریبل است.
این مدل‌ها مستقل از انتخاب خاص چارچوب متعامد راست‌گرد هستند.
بخش~\ref{sec:e_surface_conv} دسته دوم مدل‌ها را بررسی می‌کند که بر اساس کرنل‌های $\{e\}$-استیریبل، یعنی غیرتناوب‌پذیر هستند.
این مدل‌ها به صراحت بر انتخاب یک میدان چارچوب تکیه دارند.
بنابراین آن‌ها عمدتاً در روش‌های اکتشافی که برای تعیین چارچوب‌های مرجع استفاده می‌شوند، تفاوت دارند.
توجه داشته باشید که چنین مدل‌هایی لزوماً در منیفلدهای غیرقابل موازی‌سازی مانند کره‌های توپولوژیکی ناپیوسته هستند.


\paragraph{پیوست:}

پیوست شامل اطلاعات اضافی و اثبات‌های طولانی است.

گیج‌ها به صورت رسمی تخصیص فوری چارچوب‌های مرجع به فضاهای مماس را انجام می‌دهند اما به نقاط روی منیفولد به صورت مستقل از مختصات اشاره دارند.
یک جایگزین محبوب انتخاب \emph{چارت‌های مختصاتی} است که به اصطلاح \emph{پایه‌های مختصاتی} (پایه‌های هولونومیک) فضاهای مماس را القا می‌کنند.
پیوست~\ref{apx:coordinate_bases} مقدمه‌ای بر فرمالیسم چارت‌ها ارائه می‌دهد و آن را با فرمالیسم گیج کلی‌تر مرتبط می‌کند.

پیوست~\ref{apx:coord_indep_weight_sharing} درباره \emph{استقلال مختصات} کرنل‌ها و \emph{اشتراک وزن} در امتداد چارچوب‌های مرجع توضیح می‌دهد.
اشتراک وزن مستقل از مختصات $\GM$ فقط برای کرنل‌های $G$-استیریبل امکان‌پذیر است.

کانولوشن‌های $\GM$ با بیان میدان‌های ویژگی در مختصات نرمال ژئودزیک محاسبه می‌شوند، جایی که آن‌ها با کرنل‌های کانولوشنی $G$-استیریبل مطابقت داده می‌شوند.
این فرآیند شامل یک \emph{انتگرال‌گیری بر روی فضاهای مماس} است که در پیوست~\ref{apx:tangent_integral} توصیف شده است.


کوندور و تریودی~\cite{Kondor2018-GENERAL}، کوهن و همکاران~\cite{Cohen2019-generaltheory} و بکرز~\cite{bekkers2020bspline} نظریه‌های نسبتاً عمومی از \emph{کانولوشن‌ها بر روی فضاهای همگن} را پیشنهاد کردند. از آنجا که این مدل‌ها وزن‌ها را از طریق عمل یک گروه تقارنی به اشتراک می‌گذارند، آن‌ها بسیار شبیه به تبدیل‌های میدان کرنل تناوب‌پذیر ایزومتری ما از بخش‌های~\ref{sec:isometry_equivariance} و~\ref{sec:quotient_kernel_fields} هستند. پیوست~\ref{apx:homogeneous_conv} این مدل‌ها را بررسی می‌کند و توضیح می‌دهد که چگونه آن‌ها با کانولوشن‌های $\GM$ ما مرتبط هستند.


پیوست~\ref{apx:lifting_iso_proof} ثابت می‌کند که میدان‌های کرنل ناوردا با ایزومتری در منیفولد معادل میدان‌های کرنل در فضاهای خارج‌قسمتی از عمل ایزومتری هستند.
حالت خاص فضاهای همگن، که در آن‌ها تبدیل‌های میدان کرنل تناوب‌پذیر ایزومتری معادل کانولوشن‌های $\GM$ هستند، در پیوست~\ref{apx:homogeneous_equivalence_proof} پوشش داده شده است.
کانولوشن‌های کروی کوهن و همکاران~\cite{Cohen2019-generaltheory} در پیوست~\ref{apx:spherical_conv_main} ثابت شده‌اند که یک حالت خاص از کانولوشن‌های $\GM$ کروی ما هستند.
بنابراین هر شبکه کانولوشنی کروی که توسط نظریه آن‌ها پوشش داده می‌شود، توسط نظریه ما نیز توضیح داده می‌شود.

پیوست~\ref{apx:smoothness_kernel_field_trafo} تأکید می‌کند که تبدیل‌های میدان کرنل و کانولوشن‌های $\GM$ ما به خوبی تعریف شده‌اند اگر میدان کرنل هموار باشد و شامل کرنل‌های با پشتیبانی فشرده باشد.
"به خوبی تعریف شده" در اینجا به این معنی است که انتگرال‌ها وجود دارند و میدان‌های ویژگی حاصل هموار هستند.

در نهایت، پیوست~\ref{apx:regular_field_scalar_GM} استدلال می‌کند که میدان‌های ویژگی که مطابق با نمایش منظم گروه ساختار $G$ تبدیل می‌شوند، معادل میدان‌های اسکالر بر روی $G$-ساختار هستند.
این موضوع از آن جهت اهمیت دارد که برخی مدل‌ها، به ویژه کانولوشن‌های گروهی، این دیدگاه را اتخاذ می‌کنند.