%!TEX root=../GaugeCNNTheory.tex


\section{میدان‌های کرنل نماینده خارج‌قسمتی -- اثبات‌ها}
\label{apx:lifting_iso_proof}


در این پیوست، ما اثبات‌هایی برای قضایای~\ref{thm:tangent_quotient_repr_kernel_fields} و~\ref{thm:GM_conv_homogeneous_equivalence} ارائه می‌دهیم.


\toclesslab\subsection{اثبات قضیه~\ref{thm:tangent_quotient_repr_kernel_fields} -- ایزومورفیسم بین میدان‌های کرنل ناوردای ایزومتری و میدان‌های کرنل نماینده خارج‌قسمتی}{apx:lifting_iso_proof}


قضیه~\ref{thm:tangent_quotient_repr_kernel_fields} ادعا می‌کند که فضاهای $\KIfull$ از میدان‌های کرنل ناوردای ایزومتری در معادله~\eqref{eq:KIfull_def} و $\KIquot$ از میدان‌های کرنل نماینده خارج‌قسمتی در معادله~\eqref{eq:KIquot_def} با یکدیگر ایزومورف هستند و این ایزومورفیسم با ارتقای $\Lambda$ داده می‌شود که وارون آن $\Lambda^{-1}$ تحدید به $\rTM(\ITM)$ است.
در اینجا ما اثباتی برای این گزاره ارائه می‌دهیم که شامل نشان دادن این است که
\textit{۱)} $\Lambda^{-1}$ واقعاً یک وارون برای $\Lambda$ است،
\textit{۲)} ویژگی‌های تعریف‌کننده $\KIfull$ و $\KIquot$ پس از ارتقا و تحدید برآورده می‌شوند و
\textit{۳)} ساختارها به انتخاب‌های دلخواه بستگی ندارند.


\begin{itemize}[leftmargin=0cm]
	
	\item[] {\textit{۱)} $\Lambda^{\!-1}$ در معادله~\eqref{eq:lifting_isomorphism_lambda_inv} یک وارون خوش‌تعریف برای $\Lambda$ در معادله~\eqref{eq:lifting_isomorphism_lambda} است: }
	
	\begin{itemize}[leftmargin=1.1cm]
		\setlength\itemsep{2ex}
		
		\item[\textit{1a)}]
		$\Lambda \circ \Lambda^{-1} = \id_{\KIfull}$،
		یعنی $\Lambda^{-1}$ یک وارون راست برای $\Lambda$ است:
		
		این ادعا برای هر $\K\in \KIfull$ و هر $v \in TM$ از
		\begin{align}
			\big[\Lambda \circ \Lambda^{-1} (\K) \big](v)
			\ =&\ \big[ \Lambda(\Krestr) \big](v) \notag \\
			\ =&\ \dPhirHom{v} \, \Krestr    \, \rTM \QTM (v) \notag \\
			\ =&\ \dPhirHom{v} \, \K         \, \rTM \QTM (v) \notag \\
			\ =&\ \K         \, \dPhirTM{v} \, \rTM \QTM (v) \notag \\
			\ =&\ \K(v) \,,
		\end{align}
		نتیجه می‌شود، که در آن ناوردایی (هموردایی) میدان کرنل در معادله~\eqref{eq:kernel_constraint_isom_full_1} اجازه داد تا ترتیب عمل ایزومتری و ارزیابی میدان کرنل را در مرحله ماقبل آخر جابجا کنیم.
		
		\item[\textit{1b)}]
		$\Lambda^{-1} \circ \Lambda = \id_{\KIquot}$،
		یعنی $\Lambda^{-1}$ یک وارون چپ برای $\Lambda$ است:
		
		فرض کنید $\Q\in \KIquot$ و $w \in \rTM(\ITM)$ باشد.
		توجه داشته باشید که $\rTM\QTM(w) = w$ زیرا $w$ یک نماینده مدار است.
		علاوه بر این، از آنجا که $w = \dPhirTM{w}\, \rTM\QTM(w) = \dPhirTM{w} (w)$ نتیجه می‌شود که $\Phir{w} \in \Stab{w}$ به طوری که، با محدودیت در معادله~\eqref{eq:KIquot_def} داریم $\dPhirHom{w} \Q(w) = \Q(w)$.
		در مجموع، این ادعا را اثبات می‌کند:
		\begin{align}
			\big[\Lambda^{-1} \circ \Lambda (\Q) \big](w)
			\ =&\ \Lambda(\Q) \big|_{\rTM(\ITM)}(w) \notag \\
			\ =&\ \Lambda(\Q)(w) \notag \\
			\ =&\ \dPhirHom{w} \Q\, \rTM \QTM(w) \notag \\
			\ =&\ \dPhirHom{w} \Q(w) \notag \\
			\ =&\ \Q(w)
		\end{align}
		
	\end{itemize}
	
	
	
	
	
	
	
	\item[] {\emph{۲)} ویژگی‌های تعریف‌کننده $\KIfull$ و $\KIquot$ پس از ارتقا و تحدید برآورده می‌شوند: }
	
	\begin{itemize}[leftmargin=1.1cm]
		\setlength\itemsep{2ex}
		\item[\textit{2a)}]
		$\piHom \mkern-5mu\circ\mkern-2mu \Lambda(\Q) = \piTM$ برای هر $\Q \in \KIquot$،
		یعنی ارتقای $\Lambda(\Q)$ یک $M$-مورفیسم کلاف است:
		
		برای هر $\Q\in \KIquot$ و برای هر $v\in TM$ این ادعا از
		\begin{align}
			\big[ \piHom \Lambda(\Q) \big](v)
			\ =&\ \piHom \dPhirHom{v} \Q\, \rTM \QTM(v) \notag \\
			\ =&\ \Phir{v} \piHom \Q\, \rTM \QTM(v) \notag \\
			\ =&\ \Phir{v} \piTM\, \rTM \QTM(v) \notag \\
			\ =&\ \Phir{v} \rM \piITM\, \QTM(v) \notag \\
			\ =&\ \Phir{v} \rM \QM \piTM(v) \notag \\
			\ =&\ \piTM(v) \,,
		\end{align}
		نتیجه می‌شود، که در آن مرحله آخر از معادله~\eqref{eq:reconstruction_isometry_basespace} استفاده شده است.
		
		\item[\textit{2b)}]
		${\piHom \mkern-5mu\circ\mkern-2mu \Lambda^{-1}(\K) = \piTM}$ برای هر $\K \in \KIfull$،
		یعنی $\Lambda^{-1}(\K)$ یک $\rM(\IM)$-مورفیسم کلاف است:
		
		این ویژگی بلافاصله از ویژگی متناظر $\K$ پس از تحدید به $\rTM(\ITM) \subseteq \piTM^{-1}\big(\rM(\IM)\big)$ نتیجه می‌شود.
		برای هر $w \in \rTM(\ITM)$:
		\begin{align}
			\piHom \big[ \Lambda^{-1}(\K) \big](w)
			\ =&\ \piHom \Krestr(w) \notag \\
			\ =&\ \piHom \K(w) \notag \\
			\ =&\ \piTM(w) \notag \\
		\end{align}
		
		\item[\textit{2c)}]
		$\dphiHom \Lambda(\Q)\, \dphiTMinv = \Lambda(\Q)\ \ \forall \phi \in \I$،
		یعنی $\Lambda(\Q)$ محدودیت کامل ناوردایی ایزومتری را برآورده می‌کند:
		
		فرض کنید $v\in TM$ و $\phi \in \I$ باشند.
		به دلیل ناوردایی نگاشت خارج‌قسمتی $\QTM$ تحت ایزومتری‌ها، ما $\QTM(\dphiTMinv v) = \QTM(v)$ را داریم.
		علاوه بر این توجه داشته باشید که
		\begin{align}
			& \big[\Phir{v}^{-1}\, \phi\; \Phir{\dphiTMinv v}\big]_{*,\scalebox{.58}{$TM$}} \rTM \QTM(v) \notag \\
			\ =\ & \big[\Phir{v}^{-1}\, \phi\; \Phir{\dphiTMinv v}\big]_{*,\scalebox{.58}{$TM$}} \rTM \QTM\big( \dphiTMinv v\big) \notag \\
			\ =\ & \big[\Phir{v}^{-1}\, \phi \big]_{*,\scalebox{.58}{$TM$}}\, \dphiTMinv\, v \notag \\
			\ =\ & \dPhirTM{v}^{-1}\, v \notag \\
			\ =\ & \rTM \QTM(v)
		\end{align}
		نتیجه می‌دهد
		\begin{align}
			\big[\Phir{v}^{-1}\, \phi\; \Phir{\dphiTMinv v}\big]\ \in\ \Stab{\rTM\QTM(v)} \,,
		\end{align}
		که از طریق محدودیت پایدارساز در معادله~\eqref{eq:KIquot_def} منجر به
		\begin{align}
			\big[\Phir{v}^{-1}\, \phi\; \Phir{\dphiTMinv v}\big]_{*,\scalebox{.58}{$\Hom$}} \Q\; \rTM \QTM(v)
			\ =\ \Q\; \rTM \QTM(v) \,.
		\end{align}
		می‌شود. با کنار هم گذاشتن این مشاهدات، ادعا اثبات می‌شود:
		\begin{align}
			\dphiHom \Lambda(\Q)\, \dphiTMinv(v)
			\ =&\ \dphiHom \dPhirHom{\dphiTMinv v} \Q\; \rTM \QTM \big(\dphiTMinv v\big) \notag \\
			\ =&\ \dphiHom \dPhirHom{\dphiTMinv v} \Q\; \rTM \QTM(v) \notag \\
			\ =&\ \big[\Phir{v}\, \Phir{v}^{-1}\big]_{*,\scalebox{.58}{$\Hom$}} \dphiHom\, \dPhirHom{\dphiTMinv v} \Q\; \rTM \QTM(v) \notag \\
			\ =&\ \dPhirHom{v}\, \big[\Phir{v}^{-1}\, \phi\; \Phir{\dphiTMinv v}\big]_{*,\scalebox{.58}{$\Hom$}} \Q\; \rTM \QTM(v) \notag \\
			\ =&\ \dPhirHom{v}\, \Q\; \rTM \QTM(v) \notag \\
			\ =&\ \Lambda(\Q)
		\end{align}
		
		
		\item[\textit{2d)}]
		$\dxiHom \big[\Lambda^{-1}(\K)\big](w) = \big[\Lambda^{-1}(\K)\big](w) \ \ \
		\forall\; w \mkern-2mu\in\mkern-1mu \rTM(\ITM),\ \xi \mkern-1mu\in\mkern-1mu \Stab{w}$،
		یعنی $\Lambda^{-1}(\K)$ محدودیت پایدارساز را برآورده می‌کند:
		
		این گزاره به راحتی اثبات می‌شود زیرا ویژگی‌های ناوردایی (هموردایی) $\K$ به تحدید آن $\Lambda^{-1}(\K)$ منتقل می‌شوند.
		ما برای $w\in \rTM(\ITM)$ و $\xi\in \Stab{w}$ دلخواه به دست می‌آوریم که:
		\begin{align}
			\dxiHom \big[\Lambda^{-1}(\K)\big](w)
			\ =&\ \dxiHom \Krestr (w) \notag \\
			\ =&\ \dxiHom \K(w) \notag \\
			\ =&\ \K \big(\dxiTM w\big) \notag \\
			\ =&\ \K(w) \notag \\
			\ =&\ \Krestr(w) \notag \\
			\ =&\ \big[\Lambda^{-1}(\K)\big](w)
		\end{align}
		
	\end{itemize}
	
	
	
	
	
	
	
	
	
	
	\item[] {\emph{۳)} تمام ساختارها و اثبات‌ها مستقل از انتخاب خاص $\PhirNoArg$ هستند: }
	\begin{itemize}[leftmargin=1.1cm]
		\setlength\itemsep{2ex}
		\item[]%
		تعریف
		\begin{align}
			\PhirNoArg: TM \to \I \quad \textup{such that}\quad \dPhirTM{v} \rTM \QTM(v) = v
		\end{align}
		از معادله~\eqref{eq:reconstruction_isometry} تا ضرب راست $\PhirNoArg$ در \emph{هر}
		\begin{align}
			\xirNoArg: TM \to \I \quad \textup{such that}\quad \xir{v} \in \Stab{\rTM\QTM(v)}
		\end{align}
		یکتا است، زیرا به وضوح $\dPhirTM{v}\, \dxirTM{v}\, \rTM\QTM(v)\ =\ \dPhirTM{v}\, \rTM\QTM(v)\ =\, v\,$ برای هر $v\in TM$.
		همانطور که در پاورقی~\ref{footnote:ambiguity_reconstruction_isometry} استدلال شد، این تمام درجات آزادی را در تعریف ایزومتری‌های بازسازی پوشش می‌دهد.
		از محدودیت پایدارساز در معادله~\eqref{eq:KIquot_def} نتیجه می‌شود که $\dxirHom{v} \Q\, \rTM\QTM(v) = \Q\, \rTM\QTM(v)$ به طوری که ارتقای $\Lambda$ نسبت به ابهام $\PhirNoArg$ ناوردا دیده می‌شود:
		\begin{align}
			\Lambda(\Q)
			\ =&\ \dPhirHom{v} \Q\, \rTM\QTM(v) \notag \\
			\ =&\ \dPhirHom{v}\, \dxirHom{v} \Q\, \rTM\QTM(v) \notag \\
		\end{align}
		به جز تعریف ایزومورفیسم ارتقا، $\PhirNoArg$ فقط (در یک زمینه کمی متفاوت) در مرحله \textit{۲c)} استفاده می‌شود، که در آن ابهام با استدلال‌های مشابه حذف می‌شود.
		
	\end{itemize}
	
\end{itemize}

\noindent در مجموع، این مراحل اثبات می‌کنند که $\Lambda: \KIquot \to \KIfull$ یک ایزومورفیسم است.
\hfill$\Box$







\toclesslab\subsection{اثبات قضیه~\ref{thm:GM_conv_homogeneous_equivalence} -- هم‌ارزی تبدیلات میدان کرنل هموردا و کانولوشن‌ها روی فضاهای همگن}{apx:homogeneous_equivalence_proof}

برای حفظ یک نمای کلی بهتر، ما اثبات را به دو بخش تقسیم می‌کنیم و ادعاهای مطرح شده در گزاره اول و دوم قضیه~\ref{thm:GM_conv_homogeneous_equivalence} را به ترتیب اثبات می‌کنیم.

\paragraph{بخش ۱) -- ساخت \textit{H}, \textit{HM} و Isom\textsubscript{\textit{HM}}:}
فرض کنید $r\in M$ یک نقطه نماینده دلخواه باشد و بدون از دست دادن کلیت، فرض کنید $\psiGMr^{\widetilde{A}}$ هر پیمانه ایزومتریک در $r$ باشد.
ما قرار می‌دهیم
\begin{align}
	H\ :=\ \psiGMr^{\widetilde{A}} \,\Stab{r} \big(\psiGMr^{\widetilde{A}} \big)^{-1} \,,
\end{align}
که فقط یک نمایش خاص از $\Stab{r}$ نسبت به مختصاتی‌سازی انتخاب شده است.
از آنجا که نگاشت‌های پیمانه ایزومورفیسم هستند، ما یک ایزومورفیسم بین این دو گروه به دست می‌آوریم:
\begin{align}\label{eq:stabr_H_iso}
	\alpha: \Stab{r} \to H,\ \ \ \xi \to \psiGMr^{\widetilde{A}} \;\dxiGM\, \big(\psiGMr^{\widetilde{A}} \big)^{-1} =: h_\xi^{\widetilde{A}\widetilde{A}}(r)
\end{align}
از آنجا که $\Stab{r} \leq \I \leq \IsomGM$ قضیه~\ref{thm:isom_GM_in_coords} تضمین می‌کند که $h_\xi^{\widetilde{A}\widetilde{A}}(r)$ برای هر $\xi \in \Stab{p}$ عضوی از $G$ است و بنابراین $H \leq G$.
ما علاوه بر این داریم که $H\leq\OO{d}$، که با محاسبه زیر دیده می‌شود، که برای هر $\mathscr{v},\mathscr{w} \in \R^d$ برقرار است:
\begin{align}
	\pig\langle h_\xi^{\widetilde{A}\widetilde{A}}(r) \cdot\mathscr{v} \,,\,\ h_\xi^{\widetilde{A}\widetilde{A}}(r) \cdot\mathscr{w} \pig\rangle
	\ \overset{(1)}{=}&\ \ \pig\langle \pig( \psiGMr^{\widetilde{A}} \;\dxiGM\, \big(\psiGMr^{\widetilde{A}} \big)^{-1} \pig) \cdot \mathscr{v} \,,\,\ \pig( \psiGMr^{\widetilde{A}} \;\dxiGM\, \big(\psiGMr^{\widetilde{A}} \big)^{-1} \pig) \cdot \mathscr{w} \pig\rangle \notag \\
	\ \overset{(2)}{=}&\ \ \pig\langle \psiTMr^{\widetilde{A}} \;\dxiTM\, \big(\psiTMr^{\widetilde{A}} \big)^{-1} \,\mathscr{v} \,,\,\ \psiTMr^{\widetilde{A}} \;\dxiTM\, \big(\psiTMr^{\widetilde{A}} \big)^{-1} \,\mathscr{w} \pig\rangle \notag \\
	\ \overset{(3)}{=}&\ \ \eta_r\pig( \dxiTM\, \big(\psiTMr^{\widetilde{A}} \big)^{-1} \,\mathscr{v} \,,\,\ \dxiTM\, \big(\psiTMr^{\widetilde{A}} \big)^{-1} \,\mathscr{w} \pig) \notag \\
	\ \overset{(4)}{=}&\ \ \eta_r\pig( \big(\psiTMr^{\widetilde{A}} \big)^{-1} \,\mathscr{v} \,,\,\ \big(\psiTMr^{\widetilde{A}} \big)^{-1} \,\mathscr{w} \pig) \notag \\
	\ \overset{(5)}{=}&\ \ \langle \mathscr{v} \,, \mathscr{w} \rangle
\end{align}
مرحله~$(۱)$ از معادله~\eqref{eq:stabr_H_iso} استفاده کرد.
در مرحله~$(۲)$ ما عبارت $h_\xi^{\widetilde{A}\widetilde{A}}(r)$ را از طریق $\psiGMr^{\widetilde{A}}$ با عبارت آن از طریق $\psiTMr^{\widetilde{A}}$ یکی گرفتیم، که با جابجایی نمودارهای معادلات~\eqref{cd:pushforward_GM_coord_extended} و~\eqref{cd:pushforward_TM_coord} توجیه می‌شود.
از آنجا که ما~$\psiTMr^{\widetilde{A}}$ را بدون از دست دادن کلیت، ایزومتریک فرض کردیم، می‌توانیم حاصلضرب داخلی $\langle\,\cdot,\cdot\,\rangle$ روی~$\R^d$ را در مرحله~$(۳)$ با متریک ریمانی~$\eta_r$ یکی بگیریم.
مرحله~$(۴)$ از این استفاده می‌کند که $\xi \in \Stab{r} \leq \I$ یک ایزومتری است، که بنا به تعریف متریک را حفظ می‌کند؛ به معادله~\eqref{eq:isometry_def} مراجعه کنید.
در آخر، ما در مرحله~$(۵)$ متریک را از طریق پیمانه ایزومتریک به حاصلضرب داخلی روی~$\R^d$ پول‌بک می‌کنیم.
برابری عبارت اولیه و نهایی نشان می‌دهد که $h_\xi^{\widetilde{A}\widetilde{A}}(r)$ حاصلضرب داخلی را روی $\R^d$ حفظ می‌کند -- این دقیقاً الزامی است که گروه متعامد را \emph{تعریف} می‌کند.
بنابراین ما داریم که $H\leq \OO{d}$ و به همراه $H\leq G$ که
\begin{align}
	H \,\leq\, G \cap \OO{d} \,.
\end{align}
این گزاره اول از بخش ۱) قضیه~\ref{thm:GM_conv_homogeneous_equivalence} را اثبات می‌کند.
ما به گزاره دوم از بخش ۱)، یعنی ساخت $\HM$ و $\IsomHM$ می‌پردازیم.

با توجه به اینکه $\Stab{r}$ یک زیرگروه از $\I$ است، ما نگاشت خارج‌قسمتی کانونی
\begin{align}
	\mathscr{q}: \I \to \I/\Stab{r},\ \ \ \phi \to \phi.\Stab{r}
\end{align}
را داریم که اعضای گروه $\phi \in \I$ را به هم‌دسته چپ $\phi.\Stab{r} := \{\phi\,\xi \,|\, \xi\in\Stab{r} \}$ از $\Stab{r}$ می‌فرستد.
به خوبی شناخته شده است که این نگاشت خارج‌قسمتی، $\I$ را به یک کلاف اصلی $\Stab{r}$ روی فضای پایه $\I/\Stab{r}$ تبدیل می‌کند، با عمل راست که با ضرب راست $\blacktriangleleft \,: \I \times \Stab{r} \to \I,\ (\phi,\xi) \mapsto \phi\,\xi$ با اعضای پایدارساز داده می‌شود~\cite{gallier2019diffgeom2,neeb2010differential}.
علاوه بر این، $\I/\Stab{r}$ با فضای همگن~$M$ ایزومورف است.
ایزومورفیسم با
\begin{align}
	\beta: \I/\Stab{r} \to M,\ \ \ \phi.\Stab{r} \mapsto \phi(r) \,,
\end{align}
داده می‌شود، که به وضوح مستقل از انتخاب نماینده هم‌دسته است زیرا نمایندگان مختلف با اعضای گروهی که $r$ را پایدار می‌کنند، متفاوت هستند.
توجه داشته باشید که ما می‌توانستیم به همان اندازه $\mathscr{q}: \I \to \I/\Stab{r}$ را به عنوان یک کلاف اصلی $H$ در نظر بگیریم زیرا تار نوعی فقط تا ایزومورفیسم تعریف شده است.


با این مقدمات، ما $H$-ساختار $\HM$ را به عنوان یک جایگذاری از کلاف اصلی $\I$ در $\GM$ (و در نتیجه در $\FM$) تعریف می‌کنیم.
ما نگاشت جایگذاری را به صورت
\begin{align}
	\mathscr{E}: \I \to \GM,\ \ \ \phi \mapsto \dphiGM\, \sigma^{\widetilde{A}}(r) \,,
\end{align}
تعریف می‌کنیم، که دوباره به انتخاب ما از پیمانه بستگی دارد زیرا $\sigma^{\widetilde{A}}(r) = \big(\psiGMr^{\widetilde{A}}\big)^{-1}(e)$.
می‌توان آن را به عنوان ردیابی یک کپی جایگذاری شده از $\I$ در $\GM$ با پوش‌فوروارد کردن چارچوب $\sigma^{\widetilde{A}}(r) \in \GrM$ در نظر گرفت.
اینکه این واقعاً یک جایگذاری معتبر را می‌دهد، تضمین شده است زیرا عمل ایزومتری‌ها روی کلاف چارچوب بدون نقطه ثابت است.
جایگذاری $\mathscr{E}$ یک نگاشت کلاف روی $\beta$ است، یعنی $\beta\circ\mathscr{q} = \piGM \circ \mathscr{E}$.
برای نشان دادن این، کافی است هر دو طرف را بر روی یک عضو دلخواه $\phi\in\I$ اعمال کنیم، که نتیجه یکسانی می‌دهد:
$\beta \circ \mathscr{q} (\phi) = \beta\big( \phi.\Stab{r} \big) = \phi(r)$ و
$\piGM \circ \mathscr{E} (\phi) = \piGM\, \dphiGM\, \sigma^{\widetilde{A}}(r) = \phi\; \piGM\, \sigma^{\widetilde{A}}(r) = \phi(r)$.
نگاشت جایگذاری علاوه بر این هموردای-راست است:
برای هر $\xi \in \Stab{r}$ و هر $\phi \in \I$ داریم
\begin{align}
	\mathscr{E}(\phi\,\xi)
	\ =&\ \ \dphiGM\, \dxiGM \sigma^{\widetilde{A}}(r) \notag \\
	\ =&\ \ \dphiGM\, \dxiGM \big(\psiGMr^{\widetilde{A}}\big)^{-1}(e) \notag \\
	\ =&\ \ \dphiGM\, \big(\psiGMr^{\widetilde{A}}\big)^{-1} \psiGMr^{\widetilde{A}}\, \dxiGM \big(\psiGMr^{\widetilde{A}}\big)^{-1}(e) \notag \\
	\ =&\ \ \dphiGM\, \big(\psiGMr^{\widetilde{A}}\big)^{-1} \big( h_\xi^{\widetilde{A}\widetilde{A}}(r) \big) \notag \\
	\ =&\ \ \dphiGM\, \big(\psiGMr^{\widetilde{A}}\big)^{-1}(e) \lhd h_\xi^{\widetilde{A}\widetilde{A}}(r) \notag \\
	\ =&\ \ \mathscr{E}(\phi) \lhd h_\xi^{\widetilde{A}\widetilde{A}}(r) \,,
\end{align}
که در آن ما از هموردایی-راست $G$ (و در نتیجه $H$) از $\psiGMr^{\widetilde{A}}$ (و در نتیجه $\big(\psiGMr^{\widetilde{A}}\big)^{-1}$) در مرحله ماقبل آخر استفاده کردیم.
در مجموع، این ویژگی‌ها نشان می‌دهند که $\mathscr{E}$ یک نگاشت کلاف اصلی است که نمودار زیر را جابجایی می‌کند:
\begin{equation}\label{cd:GM_def_embedding}
	\begin{tikzcd}[row sep=3.em, column sep=6.5em]
		\I \times \Stab{r}
		\arrow[r, "\mathscr{E}\times\alpha", hook]
		\arrow[d, "\blacktriangleleft\,"']
		& \GM \times H
		\arrow[d, "\,\lhd"]
		\\
		\I
		\arrow[r, pos=.55, "\mathscr{E}", hook]
		\arrow[d, "\mathscr{q}"']
		& \GM
		\arrow[d, "\piGM"]
		\\
		\I/\Stab{r}
		\arrow[r, pos=.45, "\beta"']
		& M
	\end{tikzcd}
\end{equation}
$H$-ساختار ادعا شده سپس به عنوان تصویر
\begin{align}
	\HM\ :=\ \mathscr{E}(\I)\ =\ \big\{ \dphiGM\, \sigma^{\widetilde{A}}(r) \,\big|\, \phi \in \I \big\}
\end{align}
از $\mathscr{E}$ به همراه عمل راست و نگاشت تصویر محدود شده از $\GM$ تعریف می‌شود.
از آنجا که جایگذاری‌ها لزوماً یک‌به‌یک هستند، ما به طور خاص داریم که $\I$ و $\HM$ به عنوان کلاف‌های اصلی ایزومورف هستند.


به عنوان نکته آخر، ما استدلال می‌کنیم که $\I$ و $\IsomHM = \{ \theta \in \IsomM \,|\, \dthetaGM \HM = \HM \}$ منطبق هستند.
برابری $\dthetaGM \HM = \HM$ برای یک $\theta \in \IsomM$ داده شده برقرار است اگر $\dthetaGM \HM$ همزمان یک زیرمجموعه و یک ابرمجموعه از~$\HM$ باشد.
مورد اول، $\dthetaGM \HM \subseteq \HM$ نیازمند این است که برای هر عضو $\dthetaGM\, \dphiGM \sigma^{\widetilde{A}}(r) \in \dthetaGM \HM$ یک $\dphiGM' \sigma^{\widetilde{A}}(r) \in \HM$ وجود داشته باشد به طوری که $\dthetaGM\, \dphiGM \sigma^{\widetilde{A}}(r) = \dphiGM' \sigma^{\widetilde{A}}(r)$.
از آنجا که عمل ایزومتری‌ها روی کلاف چارچوب بدون نقطه ثابت است، این نیازمند $\dthetaGM = \dphiGM' \dphiGM^{-1}$ است، که به نوبه خود $\theta = \phi' \phi^{-1}$ را نتیجه می‌دهد.
همانطور که به راحتی می‌توان بررسی کرد، مورد دوم منجر به همان الزام می‌شود.
هم $\phi'$ و هم $\phi$ اعضای $\I$ هستند به طوری که $\theta$ باید عضوی از $\I$ باشد.
این ادعای
\begin{align}\label{eq:IsomHM_I}
	\IsomHM = \I \,.
\end{align}
را اثبات می‌کند.














\paragraph{بخش ۲) -- هم‌ارزی تبدیلات میدان کرنل $\I$-هموردا و کانولوشن‌های \textit{HM}:}

برای اثبات گزاره دوم قضیه، ما یک تبدیل میدان کرنل $\I$-هموردا را روی $M$ می‌سازیم و نشان می‌دهیم که با یک کانولوشن $\HM$ معادل است.
قضیه~\ref{thm:isometry_equivariant_kernel_field_trafos} اثبات کرد که تبدیلات میدان کرنل $\I$-هموردا به میدان‌های کرنل $\I$-ناوردا نیاز دارند، که طبق قضیه~\ref{thm:manifold_quotient_repr_kernel_fields} می‌توانند به طور معادل بر حسب یک میدان از کرنل‌های نماینده $\Qhat: \piTM^{-1}(\rM(\IM)) \to \piHom^{-1}(\rM(\IM))$ کدگذاری شوند.
برای مورد یک فضای همگن $M$ فضای خارج‌قسمتی $\IM$ از یک عضو منفرد تشکیل شده است، که ما آن را با $r=\rM(\IM) \in M$ نمایش می‌دهیم.
بنابراین کل میدان کرنل ناوردا با یک کرنل منفرد $\Qhat|_r = \Qhat: \TrM \to \Hom(\Ainr,\Aoutr)$ توصیف می‌شود.
این کرنل باید محدودیت پایدارساز
$\dxiHom \Qhat\; \dxiTM^{-1} = \Qhat \quad \forall\ \xi \in \Stab{r}$
را برآورده کند و از طریق ایزومورفیسم ارتقا
$\widehat{\Lambda}(\Qhat)(v) = \dPhirHom{v} \Qhat\ \rTM \QTM (v) = \dPhirHom{v} \Qhat\ \dPhirTM{v}^{-1}(v)$ روی $M$ به اشتراک گذاشته می‌شود.
همانطور که در ادامه نشان داده می‌شود، کرنل نماینده منفرد با محدودیت $\Stab{r}$ دقیقاً متناظر با یک کرنل الگوی $H$-راهبری‌پذیر است، در حالی که اشتراک وزن از طریق ایزومورفیسم ارتقای $\widehat{\Lambda}$ از قضیه~\ref{thm:manifold_quotient_repr_kernel_fields} دقیقاً متناظر با اشتراک وزن کانولوشنی در تعریف~\ref{dfn:conv_kernel_field} است.


برای صریح کردن هم‌ارزی محدودیت‌های کرنل، ما کرنل $\Qhat$ را از طریق معادله~\eqref{eq:conv_kernel_field_def_ptwise} نسبت به همان پیمانه~$\widetilde{A}$ که قبلاً در نظر گرفته شد، به صورت
$K := \psiHomr^{\widetilde{A}}\, \Qhat\, \big(\psiTMr^{\widetilde{A}} \big)^{-1}$ بیان می‌کنیم.
ضریب حجم چارچوب $\sqrt{|\eta_r^{\widetilde{A}}|}$ در اینجا حذف می‌شود زیرا ما پیمانه را بدون از دست دادن کلیت، ایزومتریک فرض کردیم.
سپس محدودیت پایدارساز نسبت به این پیمانه منجر به
\begin{alignat}{3}
	K
	\ &=&\ \ \psiHomr^{\widetilde{A}}\, &\Qhat\ \big(\psiTMr^{\widetilde{A}} \big)^{-1} \notag \\
	\ &=&\ \ \psiHomr^{\widetilde{A}}\, \dxiHom\, &\Qhat\,\ \dxiTM^{-1}\, \big(\psiTMr^{\widetilde{A}} \big)^{-1} \notag \\
	\ &=&\ \ \psiHomr^{\widetilde{A}}\, \dxiHom\, \big(\psiHomr^{\widetilde{A}})^{-1}\, &K\ \psiTMr^{\widetilde{A}}\, \dxiTM^{-1}\, \big(\psiTMr^{\widetilde{A}} \big)^{-1} \notag \\
	\ &=&\ \ \rhoHom\big( h_\xi^{\widetilde{A}\widetilde{A}}(r) \big)\, &K\ \big( h_\xi^{\widetilde{A}\widetilde{A}}(r) \big)^{-1} \notag \\
	\ &=&\ \ \frac{1}{\big|\mkern-2mu \det h_\xi^{\widetilde{A}\widetilde{A}}(r) \big|} \;
	\rhoHom\big( h_\xi^{\widetilde{A}\widetilde{A}}(r) \big)\, &K\ \big( h_\xi^{\widetilde{A}\widetilde{A}}(r) \big)^{-1}
\end{alignat}
برای هر $\xi$ در $\Stab{r}$ می‌شود.
توجه داشته باشید که می‌توانیم ضریب دترمینان را در مرحله آخر اضافه کنیم زیرا $h_\xi^{\widetilde{A}\widetilde{A}}(r) \in \OO{d}$ همانطور که در بالا نشان داده شد.
بنابراین ایزومورفیسم بین $\Stab{r}$ و $H$ در معادله~\eqref{eq:stabr_H_iso} به ما اجازه می‌دهد تا محدودیت پایدارساز را به عنوان محدودیت $H$-راهبری‌پذیری
\begin{align}
	K\ =\ \frac{1}{|\mkern-2mu \det h \mkern1mu|}\; \rhoHom(h) \circ K \circ h^{-1} \qquad \forall\ h \in H \,.
\end{align}
روی کرنل‌های الگوی یک کانولوشن $\HM$ بازنویسی کنیم.%
\footnote{
	از آنجا که $h\in H \leq G\cap\OO{d}$، ضریب دترمینان همیشه حذف می‌شود و بنابراین می‌توان آن را نادیده گرفت.
}



آنچه باقی می‌ماند نشان دادن هم‌ارزی دو روش اشتراک وزن است.
اشتراک وزن از طریق $\widehat{\Lambda}$ که از طریق پیمانه $\widetilde{A}$ بر حسب $K$ بیان می‌شود، به صورت
\begin{alignat}{3}\label{eq:weight_sharing_lifting_homogeneous}
	\widehat{\Lambda}(\Qhat)(v)
	\ &=&\ \dPhirHom{v}\; &\Qhat\,\ \rTM\, \QTM (v) \notag \\
	\ &=&\ \dPhirHom{v}\; &\Qhat\,\ \dPhirTM{v}^{-1}(v) \notag \\
	\ &=&\ \dPhirHom{v}\, \big(\psiHomr^{\widetilde{A}}\big)^{-1}\, \psiHomr^{\widetilde{A}}\; &\Qhat\,\ \big(\psiTMr^{\widetilde{A}}\big)^{-1}\, \psiTMr^{\widetilde{A}}\; \dPhirTM{v}^{-1}(v) \notag \\
	\ &=&\ \Big( \psiHomr^{\widetilde{A}}\, \dPhirHom{v}^{-1} \Big)^{-1}\, &K\, \Big( \psiTMr^{\widetilde{A}}\, \dPhirTM{v}^{-1} \Big)(v) \,.
\end{alignat}
خوانده می‌شود. خط آخر در حال حاضر بسیار شبیه به تعریف میدان‌های کرنل کانولوشنی $\HM$ در تعریف~\ref{dfn:conv_kernel_field} است.
برای اثبات هم‌ارزی آنها، باید نشان دهیم
۱) که پیمانه‌های القا شده توسط ایزومتری 
$\psiTMr^{\widetilde{A}}\, \dPhirTM{v}^{-1}$ و $\psiHomr^{\widetilde{A}}\, \dPhirHom{v}^{-1}$ در~$\piTM(v)$
با پیمانه‌های اصلی $\psiTMr^{\widetilde{A}}$ و $\psiHomr^{\widetilde{A}}$، $H$-سازگار هستند و
۲) که پیمانه‌های القا شده متناظر با چارچوب‌های مرجع با حجم واحد هستند (برای توضیح ضریب حجم چارچوب گمشده در معادله~\eqref{eq:weight_sharing_lifting_homogeneous}).
برای نکته اول، توجه داشته باشید که هم‌دامنه ایزومتری بازسازی $\PhirNoArg: \TM \to \I$ با معادله~\eqref{eq:IsomHM_I} با~$\IsomHM$ منطبق است.
بنابراین قضیه~\ref{thm:isom_GM_in_coords} تأیید می‌کند که این پیمانه‌های القا شده با هر $H$-اطلس از~$\HM$ سازگار هستند.
نکته دوم بلافاصله نتیجه می‌شود زیرا $H \leq \OO{d}$ (یا زیرا $\Phir{v}$ یک ایزومتری است و $\widetilde{A}$ ایزومتریک است).
بنابراین دیده می‌شود که اشتراک وزن $\Qhat$ از طریق ایزومورفیسم ارتقای معادله~\ref{eq:weight_sharing_lifting_homogeneous} با اشتراک وزن کانولوشنی $\HM$ از کرنل $H$-راهبری‌پذیر~$K$ در تعریف~\ref{dfn:conv_kernel_field} منطبق است.
این به همراه نتیجه‌ای که محدودیت کرنل پایدارساز منجر به محدودیت $H$-راهبری‌پذیری می‌شود، دلالت بر این دارد که میدان کرنل ارتقا یافته معادل یک میدان کرنل کانولوشنی $\HM$ است، که بخش ۲) قضیه را اثبات می‌کند.







یک انتخاب متفاوت از پیمانه $\widetilde{A}$ ممکن است برای $G<\OO{d}$ منجر به یک زیرگروه مزدوج $\overline{H}$ به $H$ و یک جایگذاری $\overline{H}\mkern-2muM$ از $\I$ شود که با $\HM$ متفاوت است.
همانطور که به راحتی می‌توان بررسی کرد، محدودیت $\overline{H}$-راهبری‌پذیری اجازه می‌دهد تا همان کرنل را نسبت به $\overline{H}\mkern-2muM$ مانند محدودیت $H$-راهبری‌پذیری در رابطه با $\HM$ توصیف کنیم، زیرا تبدیل حذف می‌شود.