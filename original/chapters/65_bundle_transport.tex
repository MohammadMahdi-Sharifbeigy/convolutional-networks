%!TEX root=../GaugeCNNTheory.tex


\subsection{Parallel transporters on associated bundles}
\label{sec:bundle_transport}

Section~\ref{sec:transport_local} gave an intuitive introduction to the parallel transport of tangent vectors and feature vectors along a path~$\gamma$ from $q\in M$ to $p\in M$.
Here we briefly discuss how coordinate free parallel transporters on the fiber bundles induce each other and derive coordinate expressions relative to given trivializations for them.
We start by assuming coordinate free transporters
\begin{alignat}{2}
    \PTMgamma:&\ &\TqM &\to \TpM \\
\intertext{
    on the tangent bundle $\TM$ to be \emph{given} and explain how they \emph{induce} transporters
}
    \PFMgamma:&\ &\FqM &\to \FpM \\
\intertext{
    on the frame bundle $\FM$.
    If these transporters are $G$-\emph{compatible} with the chosen $G$-structure, as discussed below, they further induce transporters
}
    \PGMgamma:&\ &\GqM &\to \GpM \\
    \PAgamma:&\  &\A_q &\to \A_p
\end{alignat}
on the associated $G$-bundles $\GM$ and $\A$.
In practice, that is, in our literature review in Part~\ref{part:literature_review}, most convolutional networks assume either transporters that are based on the Levi-Civita connection or some trivial connection.

A more formal definition of bundle transporters might take a different route, starting by introducing a so called principal Ehresmann connection on the principal $G$-bundle $\GM$ (which would by definition be $G$-compatible).
Such an Ehresmann connection can either be defined by a choice of horizontal subbundle $HGM$ of the tangent bundle $TGM$ of $\GM$ or, equivalently, by a Lie algebra-valued connection 1-form $\omega:TGM\to\mathfrak{g}$ on $\GM$.
The transport on $\GM$ would subsequently be defined via the horizontal lift $\gamma^\uparrow:[0,1]\to \GM$ of curves $\gamma:[0,1]\to M$ on the base space such that the tangent vectors of the lift in $\GM$ are horizontal, i.e.~$\dot{\gamma}^\uparrow \in HGM$.
All transporters on $\TM$, $\FM$ and $\A$ as associated $G$-bundles would then be induced from the transporters on the $G$-structure.
Instead of following this formal approach, which would be rather technical and can be found in the literature~\cite{schullerGeometricalAnatomy2016,wendlLectureNotesBundles2008,husemollerFibreBundles1994a,nakahara2003geometry,marshGaugeTheoriesFiber2016,shoshichikobayashiFoundationsDifferentialGeometry1963},
we focus on how the different transporters interrelate by inducing each other.









\paragraph{Transport on \textit{TM}:}
To this end, we take a shortcut by assuming the coordinate free transporters $\PTMgamma$ on $\TM$ to be given.
Recall that, given gauges $\PsiTM^{\widetilde{A}}$ on a neighborhood $U^{\widetilde{A}}$ of $q$ and $\PsiTM^A$ on a neighborhood $U^A$ of $p$, the tangent vector transporter is coordinatized according to Eq.~\eqref{eq:transporter_gauge}, that is,
\begin{align}\label{eq:transporter_gauge_copy}
    g_\gamma^{A\widetilde{A}} \ :=\ \psiTMp^A \circ \PTMgamma \circ \Big( \psiTMq^{\widetilde{A}} \Big)^{-1} \ \in\, \GL{d} \,,
\end{align}
and that its coordinatizations transform under gauge transformations at $q$ and $p$ according to Eq.~\eqref{eq:transporter_gauge_trafo}:
\begin{align}\label{eq:transporter_gauge_trafo_copy}
    g_\gamma^{B\widetilde{B}} \ =\ g_p^{BA}\, g_\gamma^{A\widetilde{A}} \Big(g_q^{\widetilde{B}\widetilde{A}}\Big)^{-1}
\end{align}
We refer back to Eq.~\eqref{cd:transporter_trivialization} for a visualization of these definitions in terms of a commutative diagram.








\paragraph{Transport on \textit{FM}:}
Given the transporter on the tangent bundle, the transporter on the frame bundle follows immediately from the transport of individual frame axes.
In equations, let $[e_i]_{i=1}^d \in \FqM$ be a frame at $q$, then the individual axes $e_i$ for $i=1,\dots,d$ are tangent vectors in $\TqM$ which can be transported via $\PTMgamma$.
We thus define the transporter on the frame bundle as:%
\footnote{
    The transport of a frame along $\gamma$ describes a curve $\gamma^\uparrow$ (horizontal lift) in $\FM$.
    The space spanned by all tangent vectors $\dot{\gamma}^\uparrow$ in $TFM$ along such curves is the horizontal subbundle $HFM$ of $TFM$, mentioned above.
}
\begin{align}\label{eq:transporter_FM_def}
  \PFMgamma\!:\ \FqM \to \FpM, \quad
  [e_i]_{i=1}^d \mapsto \PFMgamma\big([e_i]_{i=1}^d\big) := \big[\PTMgamma(e_i)\big]_{i=1}^d
\end{align}
In order to derive the explicit form of its coordinatization
$\psiFMp^A \circ \PFMgamma \circ \big(\psiFMq^{\widetilde{A}}\big)^{-1}\! \in \GL{d}$,
consider its action on a group element $h\in \GL{d}$, representing a trivialized frame of $\R^d$ which is spanned by the matrix columns $h_{:,i}\in\R^d,\ i=1,\dots,d$\,:
\begin{alignat}{3}\label{eq:transporter_gauge_FM}
    \Big[ \psiFMp^A \circ \PFMgamma \circ \big(\psiFMq^{\widetilde{A}}\big)^{-1} \Big](h)
    \ &=\ \Big[ \psiFMp^A \circ \PFMgamma \Big] \Big(\! \big[\big(\psiTMq^{\widetilde{A}}\big)^{-1}(h_{:,i})\big]_{i=1}^d \Big)
        \qquad && \big( \text{\small def. of $\psiFMp^{\widetilde{A}}$, Eq.~\eqref{eq:trivialization_FM_p}} \big) \notag \\
    \ &=\ \psiFMp^A \Big( \big[\PTMgamma \circ \big(\psiTMq^{\widetilde{A}}\big)^{-1}(h_{:,i})\big]_{i=1}^d \Big)
        \qquad && \big( \text{\small def. of $\PFMgamma$, Eq.~\eqref{eq:transporter_FM_def}} \big) \notag \\
    \ &=\ \Big( \psiTMp^A \circ \PTMgamma \circ \big(\psiTMq^{\widetilde{A}}\big)^{-1}(h_{:,i}) \Big)_{i=1}^d
        \qquad && \big( \text{\small def. of $\psiFMp^{\widetilde{A}}$, Eq.~\eqref{eq:trivialization_FM_p}} \big) \notag \\
    \ &=\ \Big( g_\gamma^{A\widetilde{A}} (h_{:,i}) \Big)_{i=1}^d
        \qquad && \big( \text{\small triv. of $\PTMgamma$, Eq.~\eqref{eq:transporter_gauge_copy}} \big) \notag \\
    \ &=\ g_\gamma^{A\widetilde{A}} \, h
\end{alignat}
The coordinatizations of the frame transporters are therefore equivalent to those of the tangent vector transporters in Eq.~\eqref{eq:transporter_gauge_copy} but act on trivialized frames in $\GL{d}$ instead of acting on coefficient vectors in $\R^{d}$.
Their gauge transformations are from the commutative diagram
\begin{equation}\label{cd:FM_transport_trivialization}
\begin{tikzcd}[column sep=53pt, row sep=30, font=\normalsize]
    \GL{d}
        \arrow[dd, "g_q^{\widetilde{B}\widetilde{A}}\cdot\ "']
        \arrow[rrr, "g_\gamma^{A\widetilde{A}}\cdot"]
    & &[-3ex] &
    \GL{d}
        \arrow[dd, "\ g_p^{BA}\cdot"]
    \\
    &
    \FqM
        \arrow[ul, "\psiFMq^{\widetilde{A}}", pos=.45]
        \arrow[dl, "\psiFMq^{\widetilde{B}}"', pos=.45]
        \arrow[r, "\PFMgamma"]
    &
    \FpM
        \arrow[ur, "\psiFMp^A"', pos=.45]
        \arrow[dr, "\psiFMp^B", pos=.45]
    \\
    \GL{d}
        \arrow[rrr, "g_\gamma^{B\widetilde{B}}\cdot"']
    & & &
    \GL{d}
\end{tikzcd}
\quad
\end{equation}
seen to coincide with those of the coordinatized transporters on $\TM$ in Eq.~\eqref{eq:transporter_gauge_trafo_copy}.





\paragraph{Compatibility of connections and \textit{G}-structures:}

Not any choice of connection or definition of transporters on the $\GL{d}$-bundles $\TM$ and $\FM$ is compatible with any $G$-structure.
Specifically, a $G$-structure might not be closed under the transport of frames, that is,
while a frame in $\GqM \subseteq \FqM$ will by $\PFMgamma$ be transported to some frame in $\FpM$, this frame is \emph{not} necessarily contained in $\GpM$.%
\footnote{
    In terms of a principal Ehresmann connection on $\FM$, this is the case if the horizontal subbundle $HFM \subseteq TFM$ is not contained in $TGM \subseteq TFM$.
    An immediate definition of parallel transport in terms of a choice of horizontal subbundle $HGM$ on the $G$-structure will always (by definition) lead to a well defined transport on $\GM$.
}
Relative to trivializations of $\GM$, such an incompatibility would reflect in coordinatized transporters ${g_\gamma^{A\widetilde{A}} \notin G}$, whose left multiplication is well defined on the fibers $\R^d$ and $\GL{d}$ of the $\GL{d}$-bundles $\TM$ and $\FM$, but not on the fiber $G$ of~$\GM$.
If the subbundle $\GM$ is not closed under the parallel transport on $\FM$, this means that no well defined corresponding transport on $\GM$ -- and thus on any associated $G$-bundles $\A$ -- exists.

As an example, consider the Levi-Civita connection on Euclidean spaces, whose transporters keep tangent vectors and frames parallel in the usual sense on~$\Euc_d$.
The $\{e\}$-structure (frame field) in Fig.~\ref{fig:frame_field_automorphism_1} is closed under this transport, and therefore compatible.
The $\{e\}$-structure in Fig.~\ref{fig:frame_field_automorphism_2}, on the other hand, is not closed under the transport, and thus incompatible with the Levi-Civita connection.
Similarly, the $\SO2$-structure on $S^2$ in Fig.~\ref{fig:G_structure_S2_1} is compatible with the Levi-Civita connection on the sphere, while the $\{e\}$-structure in Fig.~\ref{fig:G_structure_S2_2} is not.

The reader might wonder which general statements about the compatibility of connections (or transporters) and $G$-structures can be made.
In general, the Levi-Civita connection, or any other metric connection, are compatible with the $\O{d}$-structure $\OM$ that corresponds to the metric.%
\footnote{
    This statement holds by definition since metric connections preserve angles and lengths between vectors and thus the orthonormality of frames.
    One can furthermore \emph{define} metric connections as principal Ehresmann connections on~$\OM$.
}
If the manifold is orientable, the Levi-Civita connection is furthermore compatible with any $\SO{d}$-structure that corresponds to the metric.
An example is the $\SO2$-structure on $S^2$ in Fig.~\ref{fig:G_structure_S2_1}.
A necessary (but not sufficient) condition for a $G$-structure to be compatible with a given connection is that the holonomy group of the connection is a subgroup of the structure group~$G$.

An important special case is that of $\{e\}$-structures, since they imply a \emph{unique trivial connection}.%
\footnote{
    A connection is \emph{trivial} if its holonomy group, i.e. its parallel transport around any closed loop, is trivial~\cite{craneTrivialConnectionsDiscrete2010}.
}%
\footnote{
    Only one principal Ehresmann connection $H\eM = T\eM$ can be chosen on $\eM$ since the vertical subbundle $V\eM$ is the zero-section of $T\eM$.
}
The corresponding transporters move frames in such a way that the stay parallel with the frames of the $\{e\}$-structure.
Trivial connections might not seem to be of particular importance for the theory of $\GM$-convolutions, however, they are actually utilized by many convolutional networks.
Specifically, any network that relies on an $\{e\}$-structure is implicitly assuming a trivial connection.
This includes all of the models in Table~\ref{tab:network_instantiations} with $G=\{e\}$, specifically those which are reviewed in Sections~\ref{sec:spherical_CNNs_azimuthal_equivariant} and~\ref{sec:e_surface_conv}.%
\footnote{
    These models are \emph{implicitly} assuming a trivial connection by not modeling non-trivial transporters of feature vectors:
    they accumulate feature vector coefficients without transforming them.
}
Note that these models assume the trivial connection only for their feature vector transport but compute geodesics for the transporter pullback, Eq.~\eqref{eq:transporter_pullback_in_coords}, based on the original Levi-Civita connection.



\paragraph{Transport on \textit{GM}:}
Assuming that $\GM$ is compatible with (i.e. closed under) the transport on $\FM$, a well defined transporter is given by restricting the frame bundle transporter to the $G$-structure:
\begin{align}\label{eq:transporter_GM_def}
  \PGMgamma := \PFMgamma \big|_{\scalebox{.62}{$\GM$}}:\ \ \GqM \to \GpM
\end{align}
The transition functions between different coordinatizations of $\PGMgamma$ do then agree with those of $\PFMgamma$ and thus also $\PTMgamma$.
We obtain the following commutative diagram, which visualizes the restriction of the diagram in Eq.~\eqref{cd:FM_transport_trivialization} from $\FqM$, $\FpM$ and $\GL{d}$ to $\GqM$, $\GpM$ and $G$:
\begin{equation}\label{cd:transport_GM_triv}
\begin{tikzcd}[column sep=60pt, row sep=30, font=\normalsize]
    G
        \arrow[dd, "g_q^{\widetilde{B}\widetilde{A}}\cdot\ "']
        \arrow[rrr, "g_\gamma^{A\widetilde{A}}\cdot"]
    & &[-3ex] &
    G
        \arrow[dd, "\ g_p^{BA}\cdot"]
    \\
    &
    \GqM
        \arrow[ul, "\psiGMq^{\widetilde{A}}", pos=.45]
        \arrow[dl, "\psiGMq^{\widetilde{B}}"', pos=.45]
        \arrow[r, "\PGMgamma"]
    &
    \GpM
        \arrow[ur, "\psiGMp^A"', pos=.45]
        \arrow[dr, "\psiGMp^B", pos=.45]
    \\
    G
        \arrow[rrr, "g_\gamma^{B\widetilde{B}}\cdot"']
    & & &
    G
\end{tikzcd}
\quad
\end{equation}
We will in the remainder of this work assume that the transport on $\GM$ is well defined.





\paragraph{Transport on $\boldsymbol{\A}$:}
If the transporters of a connection are well defined on $\GM$, they induce transporters on any associated $G$-bundle, including the feature vector bundles $\A=(\GM\times\R^c)/\!\sim_\rho$.
Let $f_q := \big[[e_i]_{i=1}^d,\,\mathscr{f}\:\!\big]$ be a coordinate free feature vector in $\A_q$.
Its parallel transport is given by that equivalence class defined by keeping some representative coefficients $\mathscr{f}\in \R^c$ fixed and transporting the corresponding frame $[e_i]_{i=1}^d$:
\begin{align}\label{eq:transporter_A_def}
    \PAgamma: \A_q &\to \A_p, \quad
    f_q \,\mapsto\, \PAgamma(f_q) \,:=\, \pig[\PGMgamma\big([e_i]_{i=1}^d),\: \mathscr{f} \,\pig]
\end{align}
In Section~\ref{sec:transport_local} we claimed that the transporter of numerical feature vector coefficients is given by $\rho\big(g_\gamma^{A\widetilde{A}}\big)$ provided that $g_\gamma^{A\widetilde{A}}\in G$, which is the case if the transport on $\GM$ is well defined.
This coordinate expression of $\PAgamma$ can be derived by evaluating the action of
$\psiAp^A \circ \PAgamma \circ \big(\psiAq^{\widetilde{A}}\big)^{-1} \in\, \rho(G)\, \leq\, \GL{c}$
on a feature coefficient vector $\mathscr{f}\in\R^c$ step by step:

\begin{alignat}{3}\label{eq:transporter_gauge_A}
    \Big[\psiAp^A \circ \PAgamma \circ \big(\psiAq^{\widetilde{A}}\big)^{-1}\Big] (\mathscr{f})
    \ &=\ \Big[\psiAp^A \circ \PAgamma\Big] \big(\big[\sigma^{\widetilde{A}}(q),\, \mathscr{f}\,\big]\big)
        \qquad && \big( \text{\small def. of $\big(\psiAp^{\widetilde{A}}\big)^{-1}$, Eq.~\eqref{eq:trivialization_A_p_inv}} \big) \\
    \ &=\ \psiAp^A  \Big(\big[\PGMgamma\big(\sigma^{\widetilde{A}}(q)\big),\, \mathscr{f}\,\big]\Big)
        \qquad && \big( \text{\small def. of $\PAgamma$, Eq.~\eqref{eq:transporter_A_def}} \big) \notag \\
    \ &=\ \rho\Big(\psiGMp^A \circ \PGMgamma \circ \sigma^{\widetilde{A}}(q)\Big) \cdot \mathscr{f}
        \qquad && \big( \text{\small def. of $\psiAp^A$, Eq.~\eqref{eq:trivialization_A_p}} \big) \notag \\
    \ &=\ \rho\Big(\psiGMp^A \circ \PGMgamma \circ \big(\psiGMq^{\widetilde{A}}\big)^{-1}(e)\Big) \cdot \mathscr{f}
        \qquad && \big( \text{\small def. of identity section $\sigma^{\widetilde{A}}$, Eq.~\eqref{eq:identity_section_def}} \big) \notag \\
    \ &=\ \rho\big(g_\gamma^{A\widetilde{A}}\big) \!\cdot\! \mathscr{f}
        \qquad && \big( \text{\small $\PGMgamma$ in coordinates Eq.~\eqref{cd:transport_GM_triv}} \big) \notag
\end{alignat}
The commutative diagram
\begin{equation}
\begin{tikzcd}[column sep=65pt, row sep=30, font=\normalsize]
    \R^c
        \arrow[dd, "\rho\big(g_q^{\widetilde{B}\widetilde{A}}\big)\ "']
        \arrow[rrr, "\rho\big(g_\gamma^{A\widetilde{A}}\big)"]
    & &[-3ex] &
    \R^c
        \arrow[dd, "\ \rho\big(g_p^{BA}\big)"]
    \\
    &
    \A_q
        \arrow[ul, "\psiAq^{\widetilde{A}}"]
        \arrow[dl, "\psiAq^{\widetilde{B}}"']
        \arrow[r, "\PAgamma"]
    &
    \A_p
        \arrow[ur, "\psiAp^A"']
        \arrow[dr, "\psiAp^B"]
    \\
    \R^c
        \arrow[rrr, "\rho\big(g_\gamma^{B\widetilde{B}}\big)"']
    & & &
    \R^c
\end{tikzcd}
\end{equation}
implies that the gauge transformations of the coordinatized feature vector transporters read:
\begin{align}
    \rho\big(g_\gamma^{B\widetilde{B}}\big)
    \ =\
    \rho\big(g_p^{BA}\big)
    \rho\big(g_\gamma^{A\widetilde{A}}\big)
    \rho\big(g_q^{\widetilde{B}\widetilde{A}}\big)^{-1}
\end{align}
Note that this transformation law is in agreement with that in Eq.~\eqref{eq:transporter_gauge_trafo_copy}.
