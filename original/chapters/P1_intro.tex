%!TEX root=../GaugeCNNTheory.tex


\mypart{An introduction to coordinate independent CNNs}
\label{part:local_theory}


Convolutional networks extract a hierarchy of feature fields from an input signal on a manifold.
Features are thereby computed via kernels, optimized to detect characteristic spatial patterns in lower level features.
We demand that this inference process should be solely based on the relative arrangement of features but independent from the particular choice of coordinatization.
Features are therefore required to be coordinate independent geometric quantities, similar to scalars, vectors or tensors.
While such geometric quantities exist independently of coordinates, a (non-symbolic) computer implementation requires them to be expressed in terms of numerical coefficients in \emph{some} gauge, i.e. relative to some choice of reference frame.
The specific choice of coordinates is irrelevant -- it represents just one of multiple equivalent descriptions.
The appropriate mathematical framework to regulate such redundant degrees of freedom are \emph{gauge theories}.
A~gauge theory accounts for the equivalence of different gauges by consistently relating them to each other via \emph{gauge transformations}.
Fields of coordinate independent features are therefore associated with a certain transformation law, i.e. a group action of the structure group which describes how features transform under gauge transformations.
Any neural network layer processing such feature fields is required to respect their transformation laws in order to preserve their coordinate independence.


The aim of this first part of our work is to introduces coordinate independent CNNs in an easily accessible language.
Geometric intuition and visualizations are therefore favored over mathematical formalism.
A more formal exposition of the presented definitions and results is provided in Part~\ref{part:bundle_theory}.


\etocsettocdepth{2}
\etocsettocstyle{}{} % from now on only local tocs
\localtableofcontents

\vspace*{2.ex}

Section~\ref{sec:gauge_cnns_intro_local} introduces gauges and gauge transformations, based on which fields of coordinate independent feature vectors are defined.
Neural networks that map between such feature fields are developed in Section~\ref{sec:gauge_CNNs_local}.
Section~\ref{sec:mobius_conv} presents an exemplary instantiation of feature fields and network layers on the M\"obius strip.
