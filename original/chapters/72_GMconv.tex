%!TEX root=../GaugeCNNTheory.tex


\subsection{Kernel field transforms and \textit{GM}-convolutions}
\label{sec:global_conv}

We now turn to kernel field transforms and $\GM$-convolutions with spatially extended (convolution) kernels.
Section~\ref{sec:kernel_fields} introduces general, unconstrained kernel fields and more specific $\GM$-convolutional kernel fields, which are defined in terms of a shared, $G$-steerable template kernel.
General kernel field transforms and $\GM$-convolutions are introduced in Section~\ref{sec:KFTs_GM-conv_global}.
As both are defined \emph{globally}, their formulation is necessarily \emph{coordinate free}.
Section~\ref{sec:KFTs_GM-conv_local} expresses both operations relative to local trivializations, recovering our local definitions from Section~\ref{sec:gauge_conv_main}.





\subsubsection[Kernel fields]{Coordinate free kernels fields and \textit{G}-steerable kernels}
\label{sec:kernel_fields}

To detect spatial pattens in feature fields, convolutional networks apply spatially extended kernels which linearly accumulate features from a local neighborhood around each point.
In Eq.~\eqref{eq:conv_kernel_unrestricted} we defined (unconstrained) template kernels for a $d$-dimensional manifold and $\cin$- and $\cout$-dimensional input and output feature fields as maps
$K: \R^d \to \R^{\cout\times\cin}$
which assign a $\cout\times\cin$ matrix to each point of their domain.
The definition of convolution kernels as maps with domain $\R^d\cong \TpM$ and codomain $\R^{\cout\times\cin}\cong \Hom(\Ainp,\Aoutp)$ suggests a coordinate free definition of kernels
as maps between the tangent spaces and the corresponding homomorphism spaces:

\begin{dfn}[Kernel field]
\label{dfn:kernel_field_general}
    We define (unconstrained) \emph{kernel fields} of type $\rhoin,\rhoout$ on a manifold $M$ as smooth bundle $M$-morphisms between the tangent bundle $\TM$ and the feature vector homomorphism bundle $\Hom(\Ain,\Aout)$.
    By its definition as an $M$-morphism, a kernel field $\K$ lets the following diagram commute:
    \begin{equation}\label{eq:kernel_bundle_map}
        \begin{tikzcd}[row sep=3.5em, column sep=2.5em]
            % ROW 1
            \TM  \arrow[rd, start anchor={[xshift=-1ex]}, "\piTM"']
                \arrow[rr, "\K"]
            & &
            \mkern-3mu
            \Hom(\Ain,\Aout)
                \arrow[ld, start anchor={[xshift=-3ex]}, shorten >=.1ex, shorten <=-.75ex, "\piHom"] \\
            % ROW 2
            & M &
        \end{tikzcd}
    \end{equation}
    Despite smoothly mapping between two vector bundles, $\K$ is \emph{not} assumed to be a \emph{vector} bundle morphism, that is, the restrictions $\Kp: \TpM \to \Hom(\Ainp, \Aoutp)$ are not assumed to be linear.%
    \footnote{
        This reflects that convolution kernels are in general not linear as maps $K:\R^d \to \R^{\cout\times\cin}$.
        Note that this does not interfere with the linearity of $K(\mathscr{v}) \in \R^{\cout\times\cin}$ (as map $\R^{\cin} \to \R^{\cout}$) for any $\mathscr{v}\in\R$ or, here, the linearity of $\Kp(v) \in \Hom(\Ainp,\Aoutp)$ (as map $\Ainp \to \Aoutp$) for any $v\in\TpM$.
    }
\end{dfn}
The name \emph{kernel field} is motivated by the fact that such defined bundle maps $\K$ assign a (potentially different) coordinate free kernel $\Kp: \TpM \to \Hom(\Ainp,\Aoutp)$ to each point $p$ of the manifold.%
\footnote{
    We expect that it is possible to work out a well defined notion of \emph{kernel bundles} whose sections are in one to one correspondence to our definition of kernel fields as bundle maps (this reformulation would mirror the transition from Eq.~\eqref{eq:bundle_morphism_onexone} to Eq.~\eqref{eq:hom_bdl_section_unconstrained}).
}
In practice, kernels $\Kp$ are often designed to detect local patterns around $p$ and are therefore assumed to be compactly supported around the origin of $\TpM$.

A coordinate free kernel $\Kp$ at $p$ is relative to gauges $\psiTMp^A$ and $\psiHomp^A$ of the $G$-atlases given by the map
\begin{align}\label{eq:kernel_field_general_coord_expression}
    \Kp^A: \R^d \to \R^{\cout\times\cin}, \qquad
    \Kp^A\: :=\: \psiHomp^A \circ \Kp \circ \big(\psiTMp^A\big)^{-1} \,.
\end{align}
Fig.~\ref{fig:kernel_coordinatization} visualizes a coordinate free kernel on $\TpM$ and its coordinatizations on $\R^d$ relative to different gauges.
From the commutative diagram
\begin{equation}
    \begin{tikzcd}[column sep=45pt, row sep=30, font=\normalsize]
        \R^d    \arrow[rrr, pos=.4, "\Kp^A"]
                \arrow[dd, "g_p^{BA}\cdot"']
        & & &[-5pt]
        \R^{\cout\times\cin}
                \arrow[dd, "\,\rhoHom\big(g_p^{BA}\big)"]
        \\
        &
        \TpM    \arrow[r, "\Kp"]
                \arrow[lu, pos=.4, "\psiTMp^A"']
                \arrow[ld, pos=.4, "\psiTMp^B"]
        &
        \Hom\!\big(\Ain|_p,\Aout|_p\big)
                \arrow[ru, start anchor={[xshift=2.em]}, pos=.4, "\psiHomp^A"]
                \arrow[rd, start anchor={[xshift=2.em]}, pos=.4, "\psiHomp^B"']
        &
        \\
        \R^d    \arrow[rrr, pos=.4, "\Kp^B"']
        & & &
        \R^{\cout\times\cin}
    \end{tikzcd}
\end{equation}
it follows that different kernel coordinatizations are related by
\begin{align}\label{eq:kernel_field_unconstrained_gaugetrafo}
    \Kp^B \ =\ \rhoHom\big(g_p^{BA}\big) \circ \Kp^A \circ \big(g_p^{BA}\big)^{-1} \,.
\end{align}
Note that this relation only implies $\GM$-coordinate independence but does not constrain the coordinate free kernel in any way.
As before, the situation changes when sharing weights over spatial positions.


In order for a kernel field $\KK$ to correspond to a convolution, it needs to be fully specified by a single template kernel $K: \R^d \to \R^{\cout\times\cin}$ which is shared over all spatial positions.
We are again forced to share weights with all gauges $X \in \mathfrak{X}$ simultaneously in order to preserve their equivalence and thus $\GM$-coordinate independence.
As argued in Section~\ref{sec:gauge_conv}, the appropriate way of sharing $K$ with kernel coordinatizations $\KKp^X$ involves a normalization by the reference frame volume $\sqrt{|\eta_p^X|}$ and is defined by
\begin{align}\label{eq:weight_sharing_kernel_62}
    \KKp^X\ =\ \frac{K}{\sqrt{|\eta_p^X|}}
    \qquad \textup{for \emph{any} gauge}\,\ X \in \mathfrak{X}\,\ \textup{with}\,\ p\in U^X \,.
\end{align}
The reason for the frame normalization factor is that convolutions will later be defined in terms of integrals over the tangent spaces.
We are therefore actually required to share the integral operator itself in different coordinatizations,
which is equivalent to identifying the matrix-valued integration measures $\KKp^X(\mathscr{v})\, \sqrt{|\eta_p^X|}\; d\mathscr{v}$ for any gauge $X \in \mathfrak{X}$ at $p\in M$ with a template measure $K(\mathscr{v})\, d\mathscr{v}$.
The form of the kernel sharing in Eq.~\eqref{eq:weight_sharing_kernel_62} follows by equating both expressions.


Together with the relation $\sqrt{|\eta_p^A|} = \big|\!\det(g_p^{BA})\big| \mkern1.5mu \sqrt{|\eta_p^B|}$ between different frame volumes, the kernel transformation law in Eq.~\eqref{eq:kernel_field_unconstrained_gaugetrafo} and the weight sharing in Eq.~\eqref{eq:weight_sharing_kernel_62} imply the \emph{$G$-steerability kernel constraint}
\begin{align}\label{eq:kernel_constraint_rhohom}
    \frac{1}{\detg} \, \rhoHom(g) \circ K \circ g^{-1}\ =\ K \qquad\forall\: g\in G \,.
\end{align}
Valid template kernels are thus given by the invariants under the simultaneous gauge action of $\detg^{-1}$, $\rhoHom(g)$ and $g^{-1}$.
Writing out the representation $\rhoHom$, acting on $\R^{\cout\times\cin}$ via multiplication with $\rhoout$ and $\rhoin^{-1}$ from the left and right, respectively, the constraint in Eq.~\eqref{eq:kernel_constraint_rhohom} is seen to be equivalent to that in Eq.~\eqref{eq:kernel_constraint}, i.e.
$K(g\mkern1mu\mathscr{v}) = \detg^{-1} \rho_\text{out}(g)\, K(\mathscr{v})\, \rho_\text{in}(g)^{-1} \ \ \forall\, g\in G,\ \mathscr{v}\in\R^d$.

We cast these insights into definitions:
\begin{dfn}[$G$-steerable kernel]
\label{dfn:G-steerable_kernel_def_43}
    $G$-steerable kernels are characterized by their \emph{invariance under the gauge action}.
    The vector space of smooth $G$-steerable kernels that map between field types $\rhoin$ and $\rhoout$ is defined by
    \begin{align}
        \KG \!:=&\,
        \Big\{ K\!: \R^d \to \R^{\cout\times\cin}\ \text{smooth} \,\Big|\,
        \frac{1}{\detg}\, \rhoHom(g) \circ K \circ g^{-1} =\, K \ \ \ \forall g\in G \Big\} \,,
        \label{eq:G_steerable_space_in_dfn_Hom} \\[1ex]
        =&\ 
        \Big\{ K\!: \R^d \to \R^{\cout\times\cin}\ \text{smooth} \,\Big|\,
        \frac{1}{\detg}\mkern1mu \rhoout(g) K(g^{-1}\mathscr{v}) \rhoin(g)^{-1} \mkern-2mu= K(\mathscr{v}) \ \ \ \forall\, g\in G,\ \mathscr{v} \in \R^d \Big\} ,
        \label{eq:G_steerable_space_in_dfn_classical}
    \end{align}
    where $\rhoHom(g)H := \rhoout(g)H \rhoin(g)^{-1}$ for any $H\in \R^{\cout\times\cin}$ and $G\leq\GL{d}$.
    The gauge invariance of $G$-steerable kernels allows for $\GM$-coordinate independent weight sharing.
\end{dfn}
$G$-steerable kernels were in~\cite{Cohen2017-STEER} introduced to equivariant deep learning, where finite groups were assumed.
The current formulation in Def.~\ref{dfn:G-steerable_kernel_def_43} was proposed in~\cite{3d_steerableCNNs}.
A complete solution for the $G$-steerable kernel spaces for arbitrary representations $\rhoin$ and $\rhoout$ of structure groups $G\leq\O2$ has been derived in~\cite{Weiler2019_E2CNN}, an implementation is publicly available at \url{https://quva-lab.github.io/e2cnn/api/e2cnn.kernels.html}.
Mathematically, steerable kernel are equivalent to \emph{representation operators} like for instance the spherical tensor operators from quantum mechanics.
A generalization of the \emph{Wigner-Eckart theorem} describes $G$-steerable kernels as being composed from harmonic basis functions, Clebsch-Gordan coefficients and endomorphisms of irreducible representations~\cite{lang2020WignerEckart}.



\begin{dfn}[$\GM$-convolutional kernel field]
\label{dfn:conv_kernel_field}
    A \emph{$\GM$-convolutional kernel field} $\KK$ of type $\rhoin,\rhoout$ is a kernel field which is determined by a shared, \emph{$G$-steerable} template kernel $K \in \KG$.
    It is in \emph{arbitrary gauges} $\psiTMp^X$ and $\psiHomp^X\,$ from the considered $G$-atlas pointwise defined by:
    \begin{align}\label{eq:conv_kernel_field_def_ptwise}
        \KKp \,:=\ \big(\psiHomp^X\big)^{\mkern-2mu-1} \circ \frac{K}{\sqrt{|\eta_p^X|}} \circ \psiTMp^X
    \end{align}
    The smoothness of $\KK$ follows from the smoothness of the gauges, the metric and the template kernel.
\end{dfn}
As in the case of \onexoneGMs, the arbitrariness of the particular choice of gauge in Eq.~\eqref{eq:conv_kernel_field_def_ptwise} -- and therefore the $\GM$-coordinate independence of the definition -- is guaranteed by the $G$-steerability of $K\in\KG\!$.
To show this explicitly, one may define the kernel field relative to some gauge $B$ and then apply a transformation to any other gauge $A$, which cancels out and therefore leads to an equivalent expression:
\begin{align}\label{eq:arbitrariness_gauge_GM_kernel_field_def}
    \KKp
    \,\ =&\,\ \big(\psiHomp^B \big)^{-1} \,\circ \,\frac{K}{\sqrt{|\eta^B|}} \circ \psiTMp^B \notag \\
    \,\ =&\,\ \big( \rhoHom\big(g_p^{BA}\big)\, \psiHomp^A \big)^{-1} \,\circ \,\frac{K}{\sqrt{|\eta^A|} \,/\, |\! \det(g_p^{BA})| } \circ \big( g_p^{BA}\cdot \psiTMp^A \big) \notag \\
    \,\ =&\,\ \big(\psiHomp^A \big)^{-1} \,\circ \frac{\,\big|\! \det(g_p^{BA})\big|\, \rhoHom\big(g_p^{BA}\big)^{-1} \circ K \circ g_p^{BA}}{\sqrt{|\eta^A|}} \circ \psiTMp^A \notag \\
    \,\ =&\,\ \big(\psiHomp^A \big)^{-1} \,\circ \,\frac{K}{\sqrt{|\eta^A|}} \circ \psiTMp^A
\end{align}
Fig.~\ref{fig:triv_kernel_bundle_morphism} gives an overview of the local trivializations of $\GM$-convolutional kernel fields in terms of a commutative diagram.

\begin{figure}
    \centering
    \begin{tikzcd}[row sep=4.5em, column sep=5.2em]
        % ROW 1
          U\times\R^d
                        \arrow[rrrr, pos=.48, rounded corners, to path={ 
                                -- ([yshift=2.5ex]\tikztostart.north) 
                                --node[above]{\small$
                                    \KK^B = \big(\id\times K/\mkern-2mu \sqrt{|\eta^B|} \,\big)
                                $} ([yshift=2.5ex]\tikztotarget.north) 
                                -- (\tikztotarget.north)
                                }]
        & &[-3.25em] &[-3.25em] &
        U\times\R^{c_\text{out}\times c_\text{in}}
        \\
        % ROW 2
        U\times\R^d
                        \arrow[u, "\big(\id\times g^{BA}\!\!\cdot\big)"]
                        \arrow[rrd, "\proj_1"']
                        \arrow[rrrr, pos=.5, rounded corners, to path={ 
                                -- ([yshift=-16.ex]\tikztostart.south) 
                                --node[below]{\small$
                                    \KK^A = \big(\id\times K/\mkern-2mu \sqrt{|\eta^A|} \,\big)
                                $} ([yshift=-16.ex]\tikztotarget.south) 
                                -- (\tikztotarget.south)
                                }]
        &
        \piTM^{-1}(U)   
                        \arrow[rd, "\piTM\!\!"', pos=0.3]
                        \arrow[rr, "\KK"]
                        \arrow[lu, "\PsiTM^B"']
                        \arrow[l,  "\PsiTM^A"']
        & &
        \piHom^{-1}(U)
                        \arrow[ld, "\!\piHom", pos=0.3]
                        \arrow[r,  "\PsiHom^A"]
                        \arrow[ru, "\PsiHom^B"]
        &
        U\times\R^{c_\text{out}\times c_\text{in}}
                        \arrow[u, "\big(\id\times \rhoHom\big(g^{BA}\big)\!\cdot\big)"']
        \arrow[lld, "\proj_1"] \\
        % ROW 3
        & &
        U
        & &
    \end{tikzcd}
    \caption{\small
        Commutative diagram showing local coordinatizations of a $\GM$-\emph{convolutional kernel field} $\KK$ as defined in Def.~\ref{dfn:conv_kernel_field}.
        Convolutional weight sharing requires the coordinate expression of the kernel field $\KK$ at any point $p\in M$ and any gauge~$X$ at~$p$ to be determined by the shared template kernel ${K: \R^d \to \R^{\cout\times\cin}}$ as
         $\KKp^X = K/\sqrt{|\eta_p^X|}$.
        The commutativity of the diagram then implies the $G$-steerability constraint ${\detg^{-1} \rhoHom(g) \circ K \circ g^{-1} = K} \ \ \forall g\in G$ on the space~$\KG$ of template kernels.
        We want to emphasize that, despite looking similar to the diagram in Fig.~\ref{fig:triv_bundle_morphism_onexone}, the diagram in the current figure should be seen as analog to that in Fig.~\ref{fig:trivialization_hom_onexone_section}.
        The difference between the current diagram and that in Fig.~\ref{fig:trivialization_hom_onexone_section} is that the linear maps in the homomorphism bundle are via $\KK: \TM\to \Hom(\Ain,\Aout)$ determined by an element of the tangent bundle $\TM$ instead of the section $\sigma_{K_{1\!\times\!1}}: M\to \Hom(\Ain,\Aout)$.
    }
    \label{fig:triv_kernel_bundle_morphism}
\end{figure}


Note that the $G$-steerability constraint in Eq.~\eqref{eq:G_steerable_space_in_dfn_classical} or~\eqref{eq:G_steerable_space_in_dfn_Hom} reduces to the constraint on \onexoneGM\ kernels in Eq.~\eqref{eq:onexone_kernel_constraint} or~\eqref{eq:onexone_intertwiner_constraint_rhoHom} when being evaluated at the origin $\mathscr{v}=0$ of $\R^d$, which is invariant under the action of any~$g\in G$.
The results on \onexoneGMs, derived in the previous section, are therefore seen to be a special case for the choice of point-like kernels.%
\footnote{
    To make this statement precise, one would have to generalize Def.~\ref{dfn:G-steerable_kernel_def_43} to operator-valued distributions and define \onexoneGM\ kernels as operator-valued Dirac deltas.
    We omit this generalization here for brevity.
}
We further want to mention that the constraint on spatially extended kernels does in general not require their codomain to be restricted to $\Hom_G(\rhoin,\rhoout)$, i.e. the space of intertwiners.
In contrast to \onexoneGMs, this allows $\GM$-convolutions with spatially extended kernels to map between fields that transform according to non-isomorphic irreducible representations.










\subsubsection{Kernel field transforms and \textit{GM}-convolutions}
\label{sec:KFTs_GM-conv_global}

Having defined both feature fields and kernel fields, we are ready to introduce kernel field transforms and $\GM$-convolutions.
They are pointwise defined in terms of integral operators which compute output feature vectors $\fout(p)$ at points~$p\in M$ by matching the kernel $\Kp$ at~$p$ with the feature field~$\fin$ ``as seen from~$p$''.

The local representation of an input field ``as seen from~$p$'' is formally given by its \emph{transporter pullback}, which is visualized in Fig.~\ref{fig:pullback_field_exp_TpM}.
It~is defined as the usual pullback from $M$ to $\TM$ via the Riemannian exponential map%
\footnote{
    We define the exponential map on the full tangent bundle as
    $\exp: \TM \!\to M,\ \ v \mapsto \exp_{\scalebox{.85}{$\pi_{\overset{}{\protect\scalebox{.6}{$T\mkern-2muM$}}}\mkern-1mu(v)$}}(v)$.
    Recall that we assumed the manifold to be geodesically complete, such that the exponential map is well defined on the whole tangent bundle (and resort to zero-padding if this assumption fails to hold).
}
with the additional application of a parallel transporter (Eq.~\eqref{eq:transporter_A_def}), which is necessary in order to express the pulled back features in $\mathcal{A}_{\textup{in},\exp(v)}$ as features in~$\Ainp$.
Denoting this parallel transporter along the geodesic path $\gamma_v(t) := \exp((1-t) \,v)$ between $\gamma(0) = \exp(v)$ and $\gamma(1) = \pi(v) =: p$ by
\begin{align}
    \mathcal{P}_{\mkern-4mu\overset{}{\protect\scalebox{.75}{$\!\A$},\protect\scalebox{.75}{$\, p\!\leftarrow\!\exp(v)$}}}
    : \A_{\exp(v)} \to \A_p \,,
\end{align}
we thus define the pulled back feature field representations on the tangent spaces as follows:
\begin{dfn}[Transporter pullback of feature field to \textit{TM}]
\label{dfn:Expf_pullback_field}
    Given a feature field $f \in \Gamma(\A)$, we define its (redundant) representation on the tangent bundle as
    \begin{align}
        \Expsf :\ \TM \to \A, \quad\ 
        v \,\mapsto\, \PAexpv \!\circ f \circ \exp(v) \,.
    \end{align}
    The Riemannian exponential map $\exp$ corresponds hereby to the Levi-Civita connection, while the transporter $\PAexpv$ relies on some $G$-compatible connection; see Sections~\ref{sec:transport_local} and~\ref{sec:bundle_transport}.

    From the construction it is clear that $\Expsf(v) \in \A_p\,$ for any $v \in \TpM$, that is, $\Expsf$ is a bundle $M$-morphism, satisfying the following commutative diagram:
    \begin{equation}\label{eq:pullback_field_bundle_map}
        \begin{tikzcd}[row sep=3.5em, column sep=2.5em]
            % ROW 1
            \TM  \arrow[rd, start anchor={[xshift=.6ex]}, "\piTM"']
                \arrow[rr, "\Expsf"]
            & &
            \mkern-3mu
            \A
                \arrow[ld, "\piA"] \\
            % ROW 2
            & M &
        \end{tikzcd}
    \end{equation}
    Despite smoothly mapping between two vector bundles, $\Expsf$ is \emph{not} assumed to be a \emph{vector} bundle morphism, that is, the restrictions $\Expspf: \TpM \to \A_p$ are usually not linear.
\end{dfn}
The restriction $\Expspf := \Expsf\big|_{\TpM}$ of the transporter pullback's domain to $\TpM$ captures the feature field from the perspective of an observer at~$p$ as shown in Fig.~\ref{fig:pullback_field_exp_TpM}.
Note that this definition resembles a local representation of the feature field in terms of \emph{geodesic normal coordinates}, with the difference that it is not restricted to the injectivity radius of the exponential map.%
\footnote{
    Any feature vector $f(q)$ might therefore be represented multiple times on the same tangent space $\TpM$, once for each $v\in\TpM$ with $\exp(v)=q$.
    If this is not desired, one may restrict the kernel support to the injectivity radius of the exponential map, such that only the geodesically nearest occurrence will be measured.
}
We furthermore want to mention that the transporter may be replaced with any other isomorphism between $\A_{\exp(v)}$ and $\A_p$, as done for instance in~\cite{sommer2019horizontal}.


As stated before, kernel field transforms and $\GM$-convolutions are defined as matching the local feature field representations on the tangent spaces with kernels.
Working towards these definitions, note that the bundle $M$-morphisms of kernels $\K: \TM \to \Hom(\Ain,\Aout)$ and local field representations $\Expsfin: \TM \to \Ain$, can be combined to yet another (nonlinear) $M$-morphism from $\TM$ to $\Aout$,
\begin{equation}\label{eq:integrand_bundle_map}
    \quad
    \begin{tikzcd}[row sep=3.5em, column sep=6.5em]
        % ROW 1
        \TM  \arrow[rd, start anchor={[xshift=-1ex]}, "\piTM"']
            \arrow[r, "\K \mkern-1mu\times\mkern-1mu \Expsfin"]
        &
        \mkern-3mu
        \Hom(\Ain,\Aout) \!\times\! \Ain
            \arrow[r, "\ev"]
        &
        \Aout
            \arrow[ld, "\piAout"] \\
        % ROW 2
        & M &
    \end{tikzcd}
    \quad,
\end{equation}
where $\ev: \big(\K(v),\, \Expsfin(v)\big) \mapsto \K(v) \Expsfin(v)$ is the evaluation map on $\Hom(\Ain,\Aout) \!\times\! \Ain$.
Kernel field transforms compute output feature vectors at~$p$ by integrating this product of kernels and input fields over the respective tangent space $\TpM$:
\begin{dfn}[Kernel field transform]
\label{dfn:kernel_field_trafo}
    Let $\K$ be any smooth kernel field.
    The corresponding \emph{kernel field transform} is a smooth integral transform
    \begin{align}\label{eq:kernel_field_trafo_def_signature}
        \TK: \Gamma(\Ain)\to \Gamma(\Aout)
    \end{align}
    which is pointwise defined by%
    \footnote{
        The integration over $\TpM$ via the Riemannian volume density $dv$ is discussed in Appendix~\ref{apx:tangent_integral}.
    }
    \begin{align}\label{eq:kernel_field_trafo_def_ptwise}
        \big[ \TK (\fin)\big] (p)
        \,\ :=\, \int\limits_{\TpM}\!\!
            \K(v) \,
            \Expsfin (v)
            \ dv
        \,\ =\, \int\limits_{\TpM}\!\!
            \K(v) \ 
            \PAinexppv \; \fin(\exp_p\!v)
            \ dv \,.
    \end{align}
    In order to be well defined, the integral needs to exist and the resulting output field $\TK(f)$ needs to be smooth.
    This requires $\K$ to be chosen suitably, e.g. by assuming it to decay rapidly or to be compactly supported.
\end{dfn}
Note that general kernel field transforms do not necessarily model convolutions as they do not assume weights (kernels) to be shared between spatial positions.
Such general kernel field transforms will become handy in Section~\ref{sec:isometry_intro}, where we derive a requirement for spatial weight sharing from the requirement for isometry equivariance.


Appendix~\ref{apx:smoothness_kernel_field_trafo} discusses the existence and smoothness of kernel field transforms.
A sufficient condition for kernel field transforms to be well defined is the restriction of kernel supports to balls of a fixed radius $R>0$:
\begin{thm}[Kernel field transform existence for compactly supported kernels]
\label{thm:existence_kernel_field_trafo_compact_kernels}
    Let $\K$ be a kernel field whose individual kernels $\Kp$ at any $p\in M$ are (at most) supported on a closed ball of radius $R>0$ around the origin of $\TpM$, that is,
    \begin{align}
        \supp\!\big(\Kp\big)\ \subseteq\ \big\{ v\in\TpM \,\big|\, \lVert v\rVert \leq R \big\} \quad \forall p\in M \,.
    \end{align}
    The corresponding kernel field transform $\TK$ is then guaranteed to be well defined, i.e. the integral in Eq.~\eqref{eq:kernel_field_trafo_def_ptwise} exists and the output field $\TK(f) \in \Gamma(\Aout)$ is smooth for any smooth input field $f\in\Gamma(\Ain)$.
\end{thm}
\begin{proof}
    See Appendices~\ref{apx:smoothness_kernel_field_trafo} and~\ref{apx:proof_sufficiency_ball_kernel_support}.
\end{proof}
The requirement to restrict the kernel support to a closed ball of certain radius is common practice in deep learning.
Note, however, that a compactly supported kernel is at odds with scale equivariant convolutions, which, by the corresponding $G$-steerability kernel constraints, require infinitely far extending kernels.
Current implementations of scale equivariant convolutions usually approximate scale equivariant kernel spaces by restricting their support~\cite{marcos2018scale,Worrall2019DeepScaleSpaces,ghosh2019scale,zhu2019scale,bekkers2020bspline,Sosnovik2020scale,naderi2020scalesteerable} and are therefore covered by Theorem~\ref{thm:existence_kernel_field_trafo_compact_kernels}.


Based on general kernel field transforms, we define \emph{coordinate free $\GM$-convolutions} by adding the assumption of spatial weight sharing, i.e. by assuming $\GM$-\emph{convolutional kernel fields}:
\begin{dfn}[$\GM$-convolution]
\label{dfn:coord_free_conv}
    Let $\Ain$ and $\Aout$ be $G$-associated feature vector bundles with types $\rhoin$ and $\rhoout \,$, respectively.
    We define the \emph{$\GM$-convolution} with a $G$-steerable kernel $K\in\KG$ as the kernel field transform with the corresponding $\GM$-\emph{convolutional kernel field}~$\KK$:
    \begin{align}\label{eq:coord_free_conv_def_signature}
        K\,\star\,:\, \Gamma(\Ain)\to \Gamma(\Aout), \quad
        \fin \mapsto K\star \fin \,:=\, \TKK(\fin)
        \,=\! \int\limits_{\TpM}\! \KK(v)\ \Expsfin(v)\ dv
    \end{align}
\end{dfn}
As $\GM$-convolutions do not prefer any reference frame in the $G$-structure, they are guaranteed to generalize their inference over all ``poses'' of patterns which are related by the action of the structure group~$G$; see Eq.~\eqref{eq:active_local_gauge_trafo} and Fig.~\ref{fig:active_TpM_equivariance}.












\subsubsection{Kernel field transforms and \textit{GM}-convolutions in local coordinates}
\label{sec:KFTs_GM-conv_local}

What is left to show is that the coordinate free definitions of transporter pullbacks, kernel field transforms and $\GM$-convolutions introduced in this section reduce to the coordinate expressions from Section~\ref{sec:gauge_conv_main} when being expressed relative to some local trivialization.

The local coordinate expression of the transporter pullback $\Expsf$ of a feature field $f$ is, as usual, defined by pre- and post-composing it with local trivializations of the corresponding bundles, that is:
\begin{align}\label{eq:kft_coordinate_expression_62}
    \big[\Expsf \big]^A: U\times \R^d \to U\times \R^c,\ \ \ 
    (p,\mathscr{v}) \mapsto&\phantom{=} \PsiA^A \circ \Expsf \circ \big(\PsiTM^A \big)^{-1} (p, \mathscr{v}) \notag \\
                           &= \pig(p,\,\ \psiAp^A \circ \Expspf \circ \big(\psiTMp^A \big)^{-1} (\mathscr{v}) \pig)
\end{align}
Local gauge transformations at $p\in M$ are from this definition seen to be given by
\begin{align}
    \big[\Expspf \big]^B \ =\ \rho\big( g_p^{BA}\big) \circ \big[\Expspf \big]^A \circ \big(g_p^{BA}\big)^{-1} \,.
\end{align}
We visualize these coordinate expressions in terms of a commutative diagram, which is very similar to that for the local trivializations of kernel fields in Fig.~\ref{fig:triv_kernel_bundle_morphism}:
\begin{equation}
\begin{tikzcd}[row sep=4.5em, column sep=5.2em]
    % ROW 1
      U\times\R^d
                    \arrow[rrrr, pos=.48, rounded corners, to path={ 
                            -- ([yshift=2.5ex]\tikztostart.north) 
                            --node[above]{\small$
                                \big[\Expsf \big]^B
                            $} ([yshift=2.5ex]\tikztotarget.north) 
                            -- (\tikztotarget.north)
                            }]
    & &[-3.25em] &[-3.25em] &
    U\times\R^c
    \\
    % ROW 2
    U\times\R^d
                    \arrow[u, "\big(\id\times g^{BA}\!\!\cdot\big)"]
                    \arrow[rrd, "\proj_1"']
                    \arrow[rrrr, pos=.5, rounded corners, to path={ 
                            -- ([yshift=-16.ex]\tikztostart.south) 
                            --node[below]{\small$
                                \big[\Expsf \big]^A
                            $} ([yshift=-16.ex]\tikztotarget.south) 
                            -- (\tikztotarget.south)
                            }]
    &
    \piTM^{-1}(U)   
                    \arrow[rd, "\piTM\!\!"', pos=0.3]
                    \arrow[rr, "\Expsf"]
                    \arrow[lu, "\PsiTM^B"']
                    \arrow[l,  "\PsiTM^A"']
    & &
    \piA^{-1}(U)
                    \arrow[ld, "\!\piA", pos=0.3]
                    \arrow[r,  "\PsiA^A"]
                    \arrow[ru, "\PsiA^B"]
    &
    U\times\R^c
                    \arrow[u, "\big(\id\times \rho\big(g^{BA}\big)\!\cdot\big)"']
    \arrow[lld, "\proj_1"] \\
    % ROW 3
    & &
    U
    & &
\end{tikzcd}
\end{equation}

For an implementation it is useful to further resolve the coordinate expression of the transporter pullback into those of its individual components, i.e. of the transporter
$\mathcal{P}_{\mkern-4mu\overset{}{\protect\scalebox{.75}{$\!\A$},\protect\scalebox{.75}{$\, p\!\leftarrow\!\exp(v)$}}}$,
the feature field $f$ and the exponential map $\exp$.
This is achieved by expanding it with an identity of the form
$\id_{\A_{\exp(v)}} = \big(\psiAexpnop^{\widetilde{A}} \big)^{-1} \circ \psiAexpnop^{\widetilde{A}}$,
where the choice of gauge $\widetilde{A}$ at $\exp(v)$ is irrelevant as it ultimately drops out:
\begin{align}\label{eq:coordinate_expression_transporter_pullback_62}
    \big[\Expspf \big]^A (\mathscr{v})
    \ &=\ \pig[ \psiAp^A \circ \Expspf \circ \big(\psiTMp^A \big)^{-1} \pig] (\mathscr{v}) \notag \\
    \ &=\ \psiAp^A \circ 
        \mathcal{P}_{\mkern-4mu\overset{}{\protect\scalebox{.75}{$\!\A$},\protect\scalebox{.75}{$\, p\!\leftarrow\!\exp(v)$}}}
     \circ f \big( \exp \circ \big(\psiTMp^A\big)^{-1} (\mathscr{v})\big) \notag \\
    \ &=\ \psiAp^A \circ 
        \mathcal{P}_{\mkern-4mu\overset{}{\protect\scalebox{.75}{$\!\A$},\protect\scalebox{.75}{$\, p\!\leftarrow\!\exp(v)$}}}
     \circ \big(\psiAexpnop^{\widetilde{A}_v} \big)^{-1} \circ \psiAexpnop^{\widetilde{A}_v}
     \circ f \big( \exp \circ \big(\psiTMp^A\big)^{-1} (\mathscr{v})\big) \notag \\
    \ &=\ \rho\pig( g^{A\widetilde{A}}_{p\leftarrow\exp\circ (\psiTMp^A)^{-1}(\mathscr{v})} \pig) \mkern1mu
     \cdot f^{\widetilde{A}} \big( \exp \circ \big(\psiTMp^A\big)^{-1} (\mathscr{v})\big) \notag \\
\end{align}
As expected, we recover our definition from Eq.~\eqref{eq:transporter_pullback_in_coords} in Section~\ref{sec:observers_view}, which approves that Def.~\ref{dfn:Expf_pullback_field} is indeed its coordinate free counterpart.


The coordinate expression of a kernel field transform, which coincides with Eq.~\eqref{eq:kft_coord_expression} in Section~\ref{sec:observers_view}, is given by the following theorem:
\begin{thm}[Kernel field transform in coordinates]
\label{thm:kernel_field_trafo_in_coords}
    Relative to some gauge $A$ at $p\in U^A$, the kernel field transform is given by the coordinate expression
    \begin{align}\label{eq:kernel_field_trafo_in_coords}
        \!\big[ \TK(\fin) \big]^A (p)
        \ &=
        \int\limits_{\R^d}
        \Kp^A (v^A) \ 
        \big[\Expspfin\big]^A (v^A)
        \ \sqrt{|\eta_p^A|}\,\ dv^A
        \notag \\
        \ &=
        \int\limits_{\R^d}
        \Kp^A (v^A) \ 
        \rho\pig( g^{A\widetilde{A}}_{p\leftarrow\exp\circ (\psiTMp^A)^{-1}(v^A)} \pig)
         \cdot \fin^{\widetilde{A}} \big( \exp \circ \big(\psiTMp^A\big)^{-1} (v^A) \big)
        \ \sqrt{|\eta_p^A|}\ dv^A ,\!
    \end{align}
    where the gauges $\widetilde{A}$ at $\exp(v)$ are chosen arbitrarily as they cancel out.%
    \footnote{
        Note that the gauges at $\exp(v)$ might differ for different $v\in\TpM$ and should more correctly be labeled by $\widetilde{A}_v$.
        We suppress this dependency for brevity.
    }
\end{thm}
\begin{proof}
    The first expression is derived by a simple calculation which translates all involved quantities into their corresponding coordinate expressions:
    \begin{align}
        % ROW 1
        & \ 
            \big[ \TK(\fin) \big]^A (p) \notag \\[.5ex]
        % ROW 2
        \overset{(1)}{=} & \ \ 
            \psiAoutp^A \big[ \TK(\fin) \big] (p) \notag \\[.5ex]
        % ROW 3
        \overset{(2)}{=} & \ \ 
            \psiAoutp^A
            \int\limits_{\TpM}\mkern-4mu
            \Kp(v) \,
            \big[\Expspfin\big] (v)
            \,\ dv
        \notag \\[.5ex]
        % ROW 4
        \overset{(3)}{=} & \ \ 
            \psiAoutp^A
            \int\limits_{\R^d}
            \Kp \pig(\! \big(\psiTMp^A\big)^{-1}(v^A) \pig) \;
            \big[\Expspfin\big] \pig(\!\big( \psiTMp^A \big)^{-1} (v^A) \pig)
            \ \sqrt{|\eta_p^A|}\,\ dv^A
        \notag \\[.5ex]
        % ROW 5
        \overset{(4)}{=} & \ \ 
            \int\limits_{\R^d}
            \Big[ \psiAoutp^A \circ
            \Kp \pig(\! \big(\psiTMp^A\big)^{-1}(v^A) \pig) \circ
            \big( \psiAinp^A \big)^{-1} \Big]
            \Big[ \psiAinp^A \circ
            \big[\Expspfin\big] \circ \big( \psiTMp^A \big)^{-1} \Big] (v^A)
            \ \sqrt{|\eta_p^A|}\,\ dv^A
        \notag \\[.5ex]
        % ROW 6
        \overset{(5)}{=} & \ \ 
            \int\limits_{\R^d}
            \Big[ \psiHomp^A \circ \Kp \circ \big(\psiTMp^A\big)^{-1} \Big] (v^A) \,\ 
            \Big[ \psiAinp^A \circ \big[\Expspfin\big] \circ \big( \psiTMp^A \big)^{-1} \Big] (v^A)
            \,\ \sqrt{|\eta_p^A|}\,\ dv^A
        \notag \\[.5ex]
        % ROW 7
        \overset{(6)}{=} & \ \ 
            \int\limits_{\R^d}
            \Kp^A (v^A) \ 
            \big[\Expspfin\big]^A (v^A)
            \ \sqrt{|\eta_p^A|}\,\ dv^A
    \end{align}
    Step~(1) expresses the output feature vector at $p$ explicitly in terms of gauge $\psiAoutp^A$, acting on the coordinate free kernel field transform.
    This coordinate free expression is in step~(2) expanded as defined in Def.~\ref{dfn:kernel_field_trafo}.
    Step~(3) pulls the integral over $\TpM$ via the chosen gauge back to $\R^d$, which is in more detail described in Appendix~\ref{apx:tangent_integral}.
    Step~(4) inserts an identity map of the form $\id = \big(\psiAinp^A \big)^{-1} \circ \psiAinp^A$ and pulls $\psiAoutp^A$ into the integral
    while step~(5) identifies the definition of $\psiHomp^A$ from Eq.~\eqref{eq:Hom_bdl_triv_ptwise}.
    Lastly, we identify the coordinate expressions of $\Kp$ and $\Expspfin$ from Eqs.~\eqref{eq:kernel_field_general_coord_expression} and~\eqref{eq:kft_coordinate_expression_62}.

    The second expression follows from the first one by expanding the coordinate expression of the transporter pullback according to Eq.~\eqref{eq:coordinate_expression_transporter_pullback_62}.
\end{proof}



The coordinate expression for the coordinate free $\GM$-convolutions follows immediately:
\begin{thm}[$\GM$-convolutions in coordinates]
\label{thm:gauge_equiv_conv_from_coordinate_free}
    A coordinate free $\GM$-convolution
    $K\star:\, \Gamma(\Ain) \to \Gamma(\Aout)$
    with a $G$-steerable kernel $K\in\KG$ is relative to some gauge $A$ at $p\in U^A$ given by
    \begin{align}
        &\mkern-40mu
        \big[ K\star f \big]^A (p)
        \ =\ \big[ \TKK(f) \big]^A(p)
        \ =\ \int\limits_{\R^d}
            K \mkern-1mu (v^A) \,\ 
            \big[ \Expspf \big]^A (v^A)
            \,\ dv^A \,,
    \end{align}
    that is, by the coordinate expression that was introduced in Eq.~\eqref{eq:gauge_conv_coord_expression}.
    This expression may be written out further as done for general kernel field transforms in Eq.~\eqref{eq:kernel_field_trafo_in_coords}.
\end{thm}
\begin{proof}
    The result follows from Theorem~\ref{thm:kernel_field_trafo_in_coords} by observing that the coordinate free $\GM$-convolution $K\star$ is just a kernel field transform with the corresponding $\GM$-convolutional kernel field $\KK$; see Def.~\ref{dfn:coord_free_conv}.
    Specifically, the coordinate expression of a $\GM$-convolutional kernel field $\KK$ is according to Def.~\ref{dfn:conv_kernel_field} given by the frame volume normalized $G$-steerable kernel $K$, that is, $\KKp^A = K/ \sqrt{|\eta_p^A|}$.
    Inserting this identity in Eq.~\eqref{eq:kernel_field_trafo_in_coords} leads to the claimed coordinate expression for $\GM$-convolutions.
\end{proof}

This result assures that a global, coordinate free $\GM$-convolution can be implemented in terms of its local coordinate expressions relative to some $G$-atlas of local trivializations that cover $M$.


