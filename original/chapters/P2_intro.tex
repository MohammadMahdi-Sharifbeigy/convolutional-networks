%!TEX root=../GaugeCNNTheory.tex


\mypart{Theory of coordinate independent CNNs}
\label{part:bundle_theory}

Part~\ref{part:local_theory} introduced feature fields and network layers in terms of their coordinate expressions relative to some choice of gauge on \emph{local} neighborhoods $U\subseteq M$.
As the existence of \emph{global} gauges is in general topologically obstructed, global coordinate representations of feature fields do in general not exist.
Part~\ref{part:local_theory} addressed this issue by assembling the global content of feature fields from their local coordinate expressions relative to an atlas of gauges that cover~$M$.
A more elegant alternative is to define global feature fields in an abstract, \emph{coordinate free} formalism in terms of fiber bundles.
Bundle trivializations allow to recover the local coordinate expressions of feature fields and network layers.

\etocsettocdepth{2}
\etocsettocstyle{}{} % from now on only local tocs
\localtableofcontents

~

The following sections develop a global, coordinate free description of the neural networks and feature spaces from Part~\ref{part:local_theory}.
Section~\ref{sec:bundles_fields} introduces fiber bundles, in particular the tangent bundle, $G$-structures and $G$-associated feature vector bundles.
Neural network operations like kernel field transforms and $\GM$-convolutions are defined in Section~\ref{sec:gauge_CNNs_global}.
Section~\ref{sec:isometry_intro} investigates the isometry equivariance of these operations.
