%!TEX root=../GaugeCNNTheory.tex


\section{Coordinate independent feature fields}
\label{sec:gauge_cnns_intro_local}

The feature spaces of coordinate independent neural networks are spaces of feature vector fields.
The goal of this section is to define such feature fields and their geometric properties.

\etocsettocdepth{3}
\etocsettocstyle{}{} % from now on only local tocs
\localtableofcontents

Feature vectors are in practice represented relative to some reference frame and are characterized by their transformation laws when transitioning between different frames.
Section~\ref{sec:21_main} begins therefore with a discussion of the coordinatization of tangent spaces.
In particular, Section~\ref{sec:gauges_gauge_trafos} introduces gauges and gauge transformations of the tangent spaces as a formal way of describing choices of local reference frames and transformations between them.
Section~\ref{sec:gauges_TpM_functions} explains how functions on tangent spaces are represented relative to different coordinatizations
-- this introduces the idea of coordinate independent mappings, which we use later to define coordinate independent network layers.
Section~\ref{sec:local_G-structure_G-atlas} defines $G$-structures and $G$-atlases.
Coordinate independent feature fields and their gauge transformations are introduced in Section~\ref{sec:feature_fields}.
While Section~\ref{sec:individual_fields} describes the construction of individual feature fields,
Section~\ref{sec:stacked_fields} defines full feature spaces, consisting of multiple independent feature fields.
Parallel transporters of feature vectors and their representation relative to different coordinatizations are introduced in Section~\ref{sec:transport_local}.
Section~\ref{sec:isometries_local} discusses isometries and their action on geometric quantities like tangent vectors and feature vectors.
