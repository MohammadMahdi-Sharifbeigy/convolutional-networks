%!TEX root=../GaugeCNNTheory.tex


\subsection%
    [\textit{G}-structures \textit{GM} and associated feature vector bundles \texorpdfstring{$\A$}{A}]%
    {\textit{G}-structures \textit{GM} and associated feature vector bundles $\boldsymbol{\A}$}
\label{sec:G_associated_bundles}


We will now introduce $G$-structures $\GM$ as distinguished subsets of frames in $\FM$, which encode additional geometric structure on $M$ that is to be respected by coordinate independent CNNs.
The tangent bundle is via a similar associated bundle construction to that in the last section reintroduced as an associated $G$-bundle.
This approach can be generalized to construct any other associated $G$-bundle, which we use to define the feature vector bundles $\A$.
All such constructed bundles are associated to each other, that is, they differ only in their fiber $F$ but share the same base space $M$, structure group $G$ and transition functions $g^{BA}$ between trivializing neighborhoods.
The local trivializations of the bundles and their mutual gauge transformations are discussed in detail in the next Section~\ref{sec:bundle_trivializations}.






\paragraph{\textit{G}\hspace{.5pt}-\hspace{.5pt}structures \textit{GM}:}
As discussed in Section~\ref{sec:21_main} and Table~\ref{tab:G_structures}, it is often possible to work with a \emph{distinguished subset of reference frames} which are related by the action of a \emph{reduced structure group} $G \leq \GL{d}$.
This is best understood by discussing a few examples before coming to a technical definition below.
For instance, a restriction to orthonormal frames 
\begin{align}
    \OpM\ :=\ \pig\{ \big[e_{1},\dots,e_{d}\big]\, \pig|\ 
    \{e_{1},\dots,e_{d}\}\ \text{is an \emph{orthonormal} basis of } \TpM\ \text{w.r.t.}\ \eta \pig\}\ \cong\ \O{d}
\end{align}
gives rise to a principal subbundle $\OM$ of $\FM$ with structure group $\O{d}$.
Note that the orthonormality of reference frames is judged by the metric $\eta$ on $M$ -- different choices of metrics on a manifold therefore correspond to different subsets of preferred reference frames for the same structure group $\O{d}$.
As a second example, consider a choice of orientation on an orientable manifold, which allows to specify a preferred notion of frames%
\footnote{
    Conversely, non-orientable manifolds do not allow for a reduction of structure group to $\operatorname{GL}^{\!+}\!(d)$.
}
\begin{align}
    \operatorname{GL}^{\!+}_p\!M\ :=\ \pig\{ \big[e_{1},\dots,e_{d}\big]\, \pig|\ \{e_{1},\dots,e_{d}\}\ \text{is a \emph{positively oriented} basis of } \TpM \pig\}\ \cong\ \operatorname{GL}^{\!+}\!(d)
\end{align}
and a corresponding principal subbundle $\operatorname{GL}^{\!+}\!(d)M$ of $\FM$ with structure group $\operatorname{GL}^{\!+}\!(d)$.
Again, the two different choices of orientations correspond to two different choices of subbundles of accordingly oriented frames.
Combining both requirements for the orthonormality and right-handedness of frames results in an $\SO{d}$-structure with fibers
\begin{align}
    \SOpM\ :=\ \pig\{ \big[e_{1},\dots,e_{d}\big]\, \pig|\ 
    \{e_{1},\dots,e_{d}\}\ \text{is a \emph{positively oriented}, \emph{orthonormal} basis of } \TpM \pig\}\ \cong\ \SO{d} \,,
\end{align}
Fig.~\ref{fig:frame_bundle} can be thought of as showing an $\SO{2}$-structure since only right-handed, orthonormal frames are shown (the typical fiber $\GL{d}$ should then be labeled $\SO2$).
Different choices of $\SO{d}$-structures correspond either to an opposite handedness of frames, sticking to the same notion of orthonormality, or to a different choice of metric (or both).
The exact same pattern repeats for volume forms $\omega$ (on orientable manifolds $M$):
they allow to specify a preferred notion of frames
\begin{align}
    \operatorname{SL}_p\!M\ :=\ \pig\{ \big[e_{1},\dots,e_{d}\big]\, \pig|\ \{e_{1},\dots,e_{d}\}\ \text{is a basis of }\, \TpM\ \text{with \emph{unit volume} w.r.t.}\ \omega \pig\}\ \cong\ \operatorname{SL}(d)
\end{align}
and thus principal subbundles $\operatorname{SL}\!M$ of $\FM$ with structure group $\operatorname{SL}(d)$.
The specific set of frames which are preferred depends here on the specific choice of volume form.
As a last example, consider $\{e\}$-structures, corresponding to a trivial structure group $G=\{e\}$ and therefore consisting of one single frame at each point~$p$.
By definition, $\{e\}$-structures are equivalent to global (smooth) frame fields $\sigma \in \Gamma(\FM)$:
\begin{align}
    \epM\ :=\ \pig\{ \big[e_{1},\dots,e_{d}\big] = \sigma(p) \pig\}\ \cong\ \{e\}
\end{align}
They do therefore only exist on trivial manifolds.
Figs.~\ref{fig:frame_field_automorphism_1} and~\ref{fig:frame_field_automorphism_2} visualize two different choices of $\{e\}$-structures~$\eM$ on~$M=\R^2$.


All of these examples represent specific choices of $G$\emph{-structures} $\GM$ on $M$.
In general, a $G$-structure on~$M$ is a principal $G$-\emph{subbundle} of $\FM$, that is, a ``smoothly varying'' choice of \emph{subsets} $\GpM \subseteq \FpM$ which are right $G$-torsors w.r.t. $\lhd$ for any $p\in M$ \cite{sternberg1999lectures,piccione2006theory,crainic2013GStructuresExamples}.%
\footnote{\label{footnote:GpM_G_orbit_in_FpM}
    As $\FpM$ is a right $\GL{d}$-torsor, any $G$-orbit $\GpM$ in $\FpM$ is automatically guaranteed to be a right $G$-torsor.
}
The smoothness can hereby be formalized by requiring that around each frame $[e_i]_{i=1}^d \in \GpM$ there exists a neighborhood $U$ of $p$ on which a smooth section $\sigma: U \mapsto \piGM^{-1}(U) \subseteq GM$ with $\sigma(p) = [e_i]_{i=1}^d$ exists.
The projection
\begin{align}
    \piGM :=\, \piFM\mkern-2mu \big|_{\scalebox{.6}{$\GM$}}:\ \ \GM \to M
\end{align}
of $\GM$ is hereby simply given by the restriction of the projection map of $\FM$ to $\GM$.
Together with the restriction
\begin{align}\label{eq:rightaction_GM}
    \lhd:\ \GM\times G \to \GM, \quad
    \big( [e_i]_{i=1}^d,\ g \big)
    \ \mapsto\ 
    [e_i]_{i=1}^d\! \lhd g \ :=\ 
    \left[ \sum\nolimits_j e_j\, g_{ji} \right]_{i=1}^d
\end{align}
of the right action of $\GL{d}$ on $\FM$ in Eq.~\eqref{eq:rightaction_FM} to an action of $G \leq \GL{d}$ on $\GM \subseteq \FM$, this makes the $G$-structure to a \emph{principal $G$-bundle} $\GM\!\xrightarrow{\piGM}\!M$.
However, it is important to note that there are \emph{multiple choices} of such subbundles, corresponding to different $G$-structures for the same structure group~$G$; compare this claim with the examples above.
As discussed earlier, the topology of a bundle might obstruct the reduction to a structure group~$G$, and thus the existence of a corresponding $G$-structure $\GM$.


While the above definition of $G$-structures would be sufficient, it is instructive to briefly review some alternative, equivalent definitions.
The claim that $\GM$ is a principal $G$-\emph{subbundle} of $\FM$ is made precise by defining it as a tuple $(P, \mathscr{E})$ consisting of a choice of an (also non-unique) principal $G$-bundle $P$ over $M$ together with a smooth, right $G$-equivariant embedding $\mathscr{E}: P \to \FM$ (over $M$).%
\footnote{
    The embedding is a principal $G$-bundle $M$-morphism as introduced in Section~\ref{sec:fiber_bundles_general} with the group homomorphism $\theta:G\to\GL{d}$ being the canonical inclusion of the subgroup $G\leq\GL{d}$ into $\GL{d}$.
}
This is visualized by the following diagram, which is required to commute for any $g\in G$:
\begin{equation}\label{cd:GM_def_embedding}
\begin{tikzcd}[row sep=3.em, column sep=2.5em]
    P
        \arrow[rr, "\mathscr{E}", hook]
    && \FM
    \\
    P
        \arrow[rr, "\mathscr{E}", hook]
        \arrow[u, "\lhd_{\overset{}{\mkern-2muP}}\mkern2mu g\:"]
        \arrow[dr, "\pi_P"']
    && \FM
        \arrow[u, "\:\lhd\mkern2mu g"']
        \arrow[dl, "\piFM"]
    \\
    & M
\end{tikzcd}
\end{equation}
Different subsets of preferred frames correspond in this viewpoint to different choices of embeddings ${\GM = \mathscr{E}(P)}$ of $P$ in $\FM$.
$G$-structures are furthermore equivalent to sections of the form $s: M \mapsto \FM/G$ with $\GM = s(M)$, which emphasizes that $\GpM = s(p) \in \FpM/G$ is indeed a choice of $G$-orbit in $\FpM$ as stated in footnote~\ref{footnote:GpM_G_orbit_in_FpM}.
Yet another definition of $G$-structures is in terms of (equivalence classes of) $G$-atlases~\cite{wendlLectureNotesBundles2008}.
As this is the viewpoint which might be taken in an implementation of $\GM$-convolutions, we discuss it in more detail in the following Section~\ref{sec:bundle_trivializations}.
For the interested reader we want to mention that $G$-structures are a specific case of the more general concept of a \emph{reduction (or lift) of structure groups}~\cite{sternberg1999lectures,piccione2006theory,crainic2013GStructuresExamples}.



$G$-structures are of pivotal importance for the theory of $\GM$-convolutions.
The particular choice of $G$-structure determines the specific set of reference frames over which the $G$-steerable template kernel is shared.
By the gauge equivariance of the kernels, $\GM$-convolutions are guaranteed to respect the $G$-structure, i.e. to be $\GM$-\emph{coordinate independent}.
As derived in Section~\ref{sec:isometry_intro}, the isometries with respect to which a $\GM$-convolution is equivariant are exactly those which preserve the $G$-structure (i.e. those which induce automorphisms of~$\GM$).








\paragraph{\textit{TM} as \textit{G}-associated vector bundle (\textit{GM}$\boldsymbol{\times \mathds{R}^d)/}$\textit{G}\,:}

Given a $G$-structure $\GM$, one can adapt the associated $\GL{d}$-bundle construction of $\TM$ from $\FM$ in Section~\ref{sec:GL_associated_bundles} to a similar associated $G$-bundle construction of $\TM$ based on $\GM$.
Instead of expressing tangent vectors relative to general frames in $\FM$, they will thereby be expressed relative to the distinguished frames in~$\GM$ and the quotient is taken w.r.t. the reduced structure group~$G$ instead of~$\GL{d}$.
The resulting bundle is by design associated to $\GM$ (or to $\FM$ with a $G$-atlas, which is equivalent as explained in the next section) and therefore has transition functions which take values in~$G$.
The restriction of $\chi$ in Eq.~\eqref{eq:A_TM_isomorphism} to $(\GM\times\R^d)/G$ yields a vector bundle isomorphism
\begin{align}
    \TM\, \cong\, (\GM\times\R^d)/G \,.
\end{align}
While all three bundles $\TM$, $(\FM\times\R^d)/\GL{d}$ and $(\GM\times\R^d)/G$ are thus isomorphic \emph{as vector bundles}, they are only isomorphic as associated $G$-bundles if $\TM$ and $(\FM\times\R^d)/\GL{d}$ are endowed with a $G$-structure (or $G$-atlas), which is a-priori not the case.
In contrast, the bundle $(\GM\times\R^d)/G$ comes with a $G$-structure by design.
For a precise definition of associated $G$-bundle isomorphisms we refer to \cite{schullerGeometricalAnatomy2016}.









\paragraph{Associated feature vector bundles $\boldsymbol{\A}$:}
The associated $G$-bundle construction $(\GM\times\R^d)/G$ can be generalized to attach other fibers with other group actions to the $G$-structure $\GM$.
Indeed, \emph{any} bundle associated to $\GM$ can be constructed in this way.
Important examples in differential geometry are the cotangent bundle $\TsM$ with its typical fiber being the dual $\R^{d*}$ of $\R^d$, acted on by the dual action, or the $(r,s)$ tensor bundles $T^r_s\!M$ with fibers $\big(\R^d\big)^{\otimes r}\! \otimes\! \big(\R^{d*}\big)^{\otimes s}$ being acted on by the corresponding tensor product representation of~$G$.

In the following we consider \emph{associated feature vector bundles} with feature vector coefficients $\R^c$ as typical fibers.
Under gauge transformations, these fibers are acted on from the left by a multiplication with a group representation $\rho:G\to\GL{c}$, that is, Eq.~\eqref{eq:gauge_trafo_leftaction} is instantiated by $\blacktriangleright_\rho:\ {G\times\R^c \to \R^c,}\ \ {(g,\mathscr{f}) \mapsto \rho(g)\mathscr{f}}$.
Similar to before, feature vector bundles are then constructed as a quotient
\begin{align}\label{eq:associated_bundle_def}
    \A\ :=\ (\GM\times\R^c)/\!\sim_{\!\rho}
\end{align}
with the equivalence relation $\sim_{\!\rho}$ here given by
\begin{align}\label{eq:equiv_relation_A}
    \big([e_i]_{i=1}^d,\, \mathscr{f} \mkern1mu\big)\ \sim_{\!\rho}\ 
    \big([e_i]_{i=1}^d\!\lhd g^{-1},\ \rho(g) \mathscr{f} \mkern1mu\big) \qquad\forall g\in G.
\end{align}
The elements of $\A$ are the equivalence classes $\big[[e_i]_{i=1}^d,\ \mathscr{f}\big]$ of feature vector coefficients relative to reference frames and are therefore \emph{coordinate free}.
A (well defined) projection map is again induced from the projection of the $G$-structure:
\begin{align}\label{eq:associated_A_proj}
    \piA: \A \to M,\ \ 
    \big[[e_i]_{i=1}^d,\,\mathscr{f}\big] \mapsto \piGM \big( [e_i]_{i=1}^d \big)
\end{align}
Linear combinations on the fibers are defined in analogy to Eq.~\eqref{eq:associated_bdl_linear_combination}.
Since such defined feature vector bundles are associated to $\GM$, their structure group is $G\leq\GL{d}$, as we will explicitly derive in the next Section~\ref{sec:bundle_trivializations}.%
\footnote{
    Strictly speaking, the transition functions will take values in $\rho(G) \leq \GL{c}$ instead of $G \leq \GL{d}$, however, since the resulting transitions are still $G$\emph{-compatible}, the term ``$G$-valued'' is usually adapted to include such cases~\cite{wendlLectureNotesBundles2008}.
}
Note that this definition includes tangent vector fields and scalar fields, which can of course be processed as feature fields, for $\rho(g)=g$ and $\rho(g)=1$, respectively.


The construction of $\A$ as an associated $G$-bundle models $\GM$-coordinate independent feature vectors on~$M$:
features $f_p \in \A$ are equivalently expressed relative to arbitrary frames in $\GM$, with feature coefficients in different coordinatizations being related via Eq.~\eqref{eq:equiv_relation_A}, but do not have a well defined coordinate expression relative to other frames.
From an engineering viewpoint, this is reflected in the $G$-steerability of convolution kernels which ensures that measurements of features are performed \emph{relative} to frames in $\GM$ but allows to discriminate between patterns whose poses are not related by a $G$-valued gauge transformation in an \emph{absolute} sense.



\paragraph{Associated feature vector field and feature spaces:}
Smooth, coordinate free feature fields are defined as smooth global sections $f\in\Gamma(\A)$ of the feature vector bundles, that is, as smooth maps $f:M\to\A$ satisfying $\piA\circ f=\id_M$.
As discussed before, such feature fields are guaranteed to exist since vector bundles always admit smooth global sections.
In the following Section~\ref{sec:bundle_trivializations} we show how a local bundle trivialization over $U^A$ allows to represent $f$ by a field $f^A:U^A\to\R^c$ of feature vector coefficients.
A different trivialization over $U^B$ will lead to a different coefficient field $f^B:U^B\to\R^c$ representing $f$ locally.
From the transition maps between bundle trivializations it will follow that both coefficient fields are on the overlap $U^{AB}=U^A\cap U^B$ of their domains related by $f^B(p)=\rho\big(g_p^{BA}\big) f^A(p)$.
The commutative diagram in Fig.~\ref{fig:trivialization_A_sections} visualizes the relations between feature vector fields and their local trivializations.


The feature spaces of coordinate independent CNNs usually consist of multiple independent feature fields over the same base space.
The bundle describing a feature space as a whole is the \emph{Whitney sum} $\bigoplus_i\A_i$ of the feature vector bundles $\A_i\xrightarrow{\scalebox{.85}{$\pi_{\scalebox{.6}{$\!\A_i$}}$}}M$ underlying its individual fields.
As such it has the same base space $M$, a typical fiber $\bigoplus_i\R^{c_i} \cong \R^{\sum_i\!c_i}$ defined as direct sum of the individual fields' fibers and is equipped with the obvious projection map.
It is associated to $\TM$, $\FM$, $\GM$ and the $\A_i$ as $G$-bundles and can therefore equivalently be defined as
\begin{align}
    \scalebox{1.1}{$\bigoplus$}_i\,\A_i\ \ \cong\,\ \left(\GM\times\R^{\sum_i\!c_i}\right)\!\!\Big/\!\!\sim_{\oplus_i\rho_i}
\end{align}
Note that the direct sum $\bigoplus_i\rho_i$ of representations $\rho_i$ defining $\A_i$ guarantees that the transition maps of $\bigoplus_i\A_i$ transform each individual field independently.
The feature spaces are then defined as the spaces $\Gamma\!\left(\bigoplus_i\A_i\right)$ of global sections of the Whitney sum bundle.
