%!TEX root=../GaugeCNNTheory.tex


\subsection%
    [Classical formulation of \texorpdfstring{$G$}{G}-steerable CNNs on \texorpdfstring{$\R^d$}{R^d}]%
    {Classical formulation of \textit{G}-steerable CNNs on $\fakebold{\R^d}$}
\label{sec:steerable_cnns_in_coords}


In this section we review the usual notion of convolutions (or cross-correlations%
\footnote{
    In deep learning it became common to use the terms ``convolution'' and ``cross-correlation'' synonymously.
})
on $\R^d$. 
When convolving with a $G$-steerable kernel, the convolutions become equivariant under the action of the affine group $\Aff(G)$.
The following Sections~\ref{sec:euclidean_geometry} and~\ref{sec:euclidean_affine_equiv} will identify these operations on $\R^d$ as coordinate expressions of coordinate free $\GM$-convolutions on Euclidean spaces~$\Euc_d$.


\paragraph{Euclidean steerable CNNs:}
Conventional CNNs consider \emph{feature maps} on $\R^d$, which are functions of the form
\begin{align}
    F: \R^d \to \R^c \,.
\end{align}
A convolution (actually a correlation) with a matrix-valued kernel $K: \R^d \to \R^{\cout\times\cin}$ is then defined as the integral transform
\begin{align}
    \Fout(\mathscr{x}) \,:=\, [K * \Fin] (\mathscr{x}) \,:= \int_{\R^d} K(\mathscr{v})\, \Fin(\mathscr{x} + \mathscr{v})\ d\mathscr{v} \,,
\end{align}
which maps an input feature map with $\cin$ channels to an output feature map with $\cout$ channels.
It can be shown that this operation is the most general linear and translation equivariant mapping between feature maps~\cite{Cohen2019-generaltheory}.%
\footnote{
    For full generality, one would actually have to allow for kernels in the distributional sense (generalized functions).
}


Euclidean steerable CNNs~\cite{Cohen2017-STEER,3d_steerableCNNs,Weiler2019_E2CNN} generalize conventional CNNs to convolutional networks that are equivariant under the action of affine groups on feature fields.
The affine groups of $\R^d$ are hereby defined as semidirect products of the form
\begin{align}\label{eq:AffG_def}
    \Aff(G)\ :=\ \Trans_d \rtimes G \,,
\end{align}
where $G\leq\GL{d}$.
Affine groups include the isometries of $\R^d$, visualized in Fig.~\ref{fig:isometries_plane}, as a special case for $G\leq\O{d}$ but allow for more general point groups (structure groups)~$G$, for instance uniform scaling~$\Scale$.
The following equations give an overview of the most common affine groups in the literature (up to discretizations) and alternative ways of writing them (assuming $\IsomGM$ to be determined by $\Aff(G)$-invariant $G$-structures; see below):
\begin{alignat}{5}
    &\Aff(\{e\}) \ &&=\ \Trans_d                \ &&=\ (\R^d,+)\ &&=\ \IsomeM \notag \\
    &\Aff(\Flip) \ &&=\ \Trans_d \rtimes \Flip    &&           \ &&=\ \IsomRM \notag \\
    &\Aff(\SO{d})\ &&=\ \Trans_d \rtimes \SO{d} \ &&=\ \SE{d}  \ &&=\ \IsomSOM\ &&=\ \Isom_+\!(\R^d) \\
    &\Aff(\O{d}) \ &&=\ \Trans_d \rtimes \O{d}  \ &&=\  \E{d}  \ &&=\ \IsomOM \ &&=\ \Isom(\R^d) \notag \\
    &\Aff(\Scale)\ &&=\ \Trans_d \rtimes \Scale \notag
\end{alignat}
The group $\Aff(\GL{d})$ comprises \emph{all} affine transformations of~$\R^d$.
Since affine groups are defined as semidirect products, any of their elements $tg \in \Aff(G)$ can be uniquely decomposed into a translation $t\in \Trans_d$ and a point group element $g\in G$.
Their (canonical) action on $\R^d$ is given~by
\begin{align}
    \Aff(G)\times\R^d \to \R^d, \quad (tg,\, \mathscr{x})\ \mapsto\ g\mkern1mu \mathscr{x} + t \,.
\end{align}
The action of an inverse group element $(tg)^{-1}$ follows to be given by
\begin{align}
    \big( (tg)^{-1},\, \mathscr{x}\big)\ \mapsto\ g^{-1} (\mathscr{x} - t) \,.
\end{align}


A feature field of type $\rho$ on $\R^d$ transforms according to the \emph{induced representation} $\Ind_G^{\Aff(G)} \rho$ of $\rho$ as specified by
\begin{align}\label{eq:induced_rep_affine}
    (tg) \mkern2mu\rhd_\rho F \ :=\ \big[\Ind_G^{\Aff(G)} \rho\big](tg)\, F \ :=\ \rho(g)\, F\, (tg)^{-1} \,,
\end{align}
which can be seen as the analog of the coordinate free action on sections in Eq.~\eqref{eq:pushforward_section_A}.%
\footnote{
    Induced representations act in a similar way on fields as shown in Fig.~\ref{fig:active_TpM_equivariance}.
    In contrast to the transformation in this Figure, induced representations allow additionally for translations (the transformation law in Section~\ref{sec:gauge_conv} is the restriction $\Res_G^{\Aff(G)}\Ind_G^{\Aff(G)}\!\rho$ of the induced representation back to $G$, i.e. the induced representation without translations).
}
A convolution with a $G$-steerable kernel $K \in \KG$ is equivariant w.r.t. these actions on the input and output field, that is,
\begin{align}\label{eq:Euclidean_conv_equiv_in_coords_Rd}
     K *\, \big(tg \,\rhd_{\rhoin}\! \Fin \big)\ =\ tg \,\rhd_{\rhoout} \big( K*\Fin \big) \qquad \forall\ \ tg \,\in\, \Aff(G) \,.
\end{align}
This is easily checked by an explicit computation,
\begin{align}
     \pig[ K * (tg \rhd_{\rhoin}\! \Fin) \pig] (\mathscr{x})
    \ =&\ \pig[ K * \big(\rhoin(g)\, \Fin\, (tg)^{-1}\big) \pig] (\mathscr{x}) \notag \\
    \ =&\ \int_{\R^d} K(\mathscr{v})\; \rhoin(g)\, \Fin \big((tg)^{-1} (\mathscr{x} + \mathscr{v}) \big)\ d\mathscr{v} \notag \\
    \ =&\ \int_{\R^d} K(\mathscr{v})\; \rhoin(g)\, \Fin \big(g^{-1}( \mathscr{x} + \mathscr{v} - t)\big)\ d\mathscr{v} \notag \\
    \ =&\ \int_{\R^d} K(g \tilde{\mathscr{v}})\; \rhoin(g)\, \Fin \big(g^{-1}(\mathscr{x} - t) + \tilde{\mathscr{v}}\big)\ \detg\ d\tilde{\mathscr{v}} \notag \\
    \ =&\ \int_{\R^d} \rhoout(g)\, K(\tilde{\mathscr{v}})\; \Fin \big(g^{-1}(\mathscr{x} - t) + \tilde{\mathscr{v}}\big)\ d\tilde{\mathscr{v}} \notag \\
    \ =&\ \rhoout(g)\, \big[ K * \Fin \big] \big(g^{-1}(\mathscr{x}-t)\big) \notag \\
    \ =&\ tg \rhd_{\rhoout} \big[K * \Fin \big] (\mathscr{x}) \,,
\end{align}
which used the $G$-steerability of $K$ in the fifth step and which holds for any $\mathscr{x} \in \R^d$ and any $tg\in \Aff(G)$.
As~proven in~\cite{3d_steerableCNNs}, such
\emph{$G$-steerable convolutions are the most general $\Aff(G)$-equivariant linear maps between Euclidean feature fields}.%
\footnote{
    Assuming that the feature fields transform according to the induced representation, Eq.~\eqref{eq:induced_rep_affine}, which is required to end up with a convolution.
}%
\footnote{
    This generalizes the well known statement that conventional Euclidean convolutions are the most general translation equivariant linear maps between functions (or feature maps) on Euclidean spaces, which is recovered for $G=\{e\}$.
}




\paragraph{Relation to Euclidean \textit{GM}-convolutions:}
How do these steerable convolutions on $\R^d$ relate to $\GM$-convolutions on Euclidean spaces ${M = \Euc_d}$?
The fact that steerable convolutions rely on $G$-steerable kernels suggests that they are not only globally $\Aff(G)$-equivariant but (more generally) locally $G$-equivariant.
To draw the connection between classical $G$-steerable CNNs on $\R^d$ and our coordinate free $\GM$-convolutions, we need to identify the geometric structure that is implicitly being considered by the former.


In general, $\R^d$ comes canonically equipped with an $\{e\}$-structure, visualized in Fig.~\ref{fig:G_structure_R2_1}.%
\footnote{
    Formally, the canonical $\{e\}$-structure of $\R^d$ arises as follows:
    the vector space $M=\R^d$ comes itself with a canonical basis, given by the basis vectors $e_i \in \R^d$ with elements $(e_i)_j = \delta_{ij}$.
    The canonical reference frames of tangent spaces $T_p\R^d$ follow from this basis via the canonical isomorphisms
    $\iota_{\R^d,p}: T_p{\R^d} \xrightarrow{\sim} \R^d$ from Eq.~\eqref{eq:canonical_iso_TRk_Rk}.
    Intuitively, the local frames of the tangent spaces $T_p\R^d$ are ``aligned'' with the global frame of $\R^d$.
    This is equivalent to introducing the identity map as global coordinate chart $x = \id_{\R^d} : M=\R^d \to \R^d$ and then defining the canonical $\{e\}$-structure as the field of induced coordinate bases $\big[\frac{\partial}{\partial x_i} \big]_{i=1}^d$.
}
It furthermore comes with a Riemannian metric corresponding to the standard inner product of~$\R^d$.%
\footnote{
    This standard metric $\eta$ is defined as the pullback of the standard inner product
    $\langle\cdot,\cdot\rangle_{\R^d}: \R^d\times\R^d \to \R$
    on~$\R^d$ via the canonical isomorphisms
    $\iota_{\R^d,p}: T_p{\R^d} \xrightarrow{\sim} \R^d$ from Eq.~\eqref{eq:canonical_iso_TRk_Rk}
    to the tangent spaces.
    It is thus for any $v,w\in T_p\R^d$ given by
    ${\eta_p(v,w) := \langle \iota_{\R^d,p}(v) ,\, \iota_{\R^d,p}(w) \rangle_{\R^d}}$.
}
The corresponding Levi-Civita connection gives rise to the parallel transporters in Fig.~\ref{fig:transport_flat}, which keep vectors parallel in the usual sense on Euclidean spaces.
When being expressed relative to the frames of the canonical $\{e\}$-structure, the parallel transporters become trivial and drop therefore out.
The exponential maps reduce to a mere summation (after applying some isomorphisms, see below).


While we are given an $\{e\}$-structure on $\R^d$, $\GM$-convolutions rely on less specific $G$-structures.
These $G$-structures could be seen as the (canonical) $G$-\emph{lifts}
\begin{align}\label{eq:G_lifted_G_structure_Rd}
    \GM\ =\ \eM\lhd G\ :=\ \pig\{\, [e_i]_{i=1}^d \lhd g \;\pig|\; [e_i]_{i=1}^d \in \eM,\ g\in G \,\pig\}
\end{align}
of the canonical $\{e\}$-structure~$\eM$ of~$\R^d$.
Intuitively, these lifted $G$-structures are defined by augmenting every canonical reference frame in $\eM$ with any other $G$-related frame (its right $G$-orbit in $\FM$).
Fig.~\ref{fig:G_structures_R2_main} shows such lifted $G$-structures for different structure groups.
As proven in Theorem~\ref{thm:Aff_GM_in_charts} below, they are invariant under the action of $\Aff(G)$ -- which is in Theorem~\ref{thm:affine_equivariance_Euclidean_GM_conv} shown to explain the $\Aff(G)$-equivariance of the convolutions.
They are furthermore $G$-compatible with the Levi-Civita connection.


The claims made here are more rigorously discussed in the following two Sections.
This formalizations is, however, not strictly necessary to understand our classification of Euclidean CNNs in the literature, such that the reader may skip them and jump immediately to Section~\ref{sec:euclidean_literature}.

