%!TEX root=../GaugeCNNTheory.tex


\section{Isometry equivariance}
\label{sec:isometry_intro}


A main characteristic of the convolution operation and its various generalizations is their equivariance w.r.t. symmetries of the underlying manifold.
For instance, the conventional convolution on Euclidean spaces is translation equivariant while spherical convolutions are rotation equivariant.
More generally, any locally compact group and their homogeneous spaces admit group convolutions%
~\cite{gurarie1992symmetries,kowalski2010introduction,chirikjian2001engineering,gallier2019harmonicRepr},
which were recently picked up by the deep learning community to generalize convolutional networks to such spaces~\cite{Cohen2016-GCNN,Kondor2018-GENERAL,Cohen2019-generaltheory,bekkers2020bspline}.
However, as these approaches rely fundamentally on the global, \emph{transitive} symmetries of the homogeneous space, they do not immediately apply to general Riemannian manifolds.

$\GM$-convolutions on the other hand shift the focus from \emph{global symmetries of the space itself} to \emph{local symmetries in the coordinatization of the space}.
As it turns out, the local gauge equivariance of $\GM$-convolutions, together with convolutional weight sharing, induces their equivariance under \mbox{the action of global symmetries.}
Stated more precisely, $\GM$-convolutions are equivariant under the action of \emph{$G$-structure preserving isometries} (Def.~\ref{dfn:IsomGM}), which form a subgroup $\IsomGM \leq \IsomM$ of the full isometry group.
The requirement on the symmetry to be an isometry (i.e. to preserve the metric) comes hereby from the use of exponential maps, which rely on the Levi-Civita connection and thus Riemannian metric.
The additional requirement on these isometries to preserve the $G$-structure is a consequence of the definition of feature vector bundles as associated $G$-bundles, whose elements have a well defined meaning only relative to those reference frames that are contained in~$\GM$.
Note that the latter is not really a restriction, as one may always choose structure groups $G\geq\O{d}$, for which \emph{any} isometry respects the corresponding $G$-structure.
On the contrary, this design allows for a precise control of the level of isometry equivariance.
For instance, the conventional convolution on Euclidean vector spaces relies on the canonical $\{e\}$-structure of $\R^d$, visualized in Fig.~\ref{fig:frame_field_automorphism_1}, and is therefore solely translation equivariant.
An $\SO{d}$-structure on $\R^d$, visualized in Fig.~\ref{fig:SO2_structure_SE2}, is additionally preserved by rotations, and thus corresponds to $\SE{d}$-equivariant convolutions.
Equivariance under the full isometry group $\E{d}$ of $\R^d$ is implied when choosing an $\O{d}$-structure on $\R^d$.


\etocsettocdepth{3}
\etocsettocstyle{}{} % from now on only local tocs
\localtableofcontents


The goal of this section is to derive theorems which formally characterize the isometry equivariance of $\GM$-convolutions and kernel field transforms.
Section~\ref{sec:isom_background} lays the foundations of this investigation by introducing isometry groups of Riemannian manifolds and discussing a range of well known relation and constructions which they induce.
Specifically, Section~\ref{sec:isometry_groups} introduces isometries and isometry groups while Section~\ref{sec:isom_action_bundles} defines their induced action (``pushforwards'') on the associated bundles in a coordinate free setting.
In Section~\ref{sec:isom_coordinatization} we express these actions on bundles relative to local trivializations and discuss their passive interpretation as isometry induced gauge transformations, visualized in Figs.~\ref{fig:intro_gauge_isom_induction} (right) and~\ref{fig:pushforward_vector_components}.
Section~\ref{sec:isom_expmap_transport} briefly states how the quantities involved in kernel field transforms behave under the action of isometries.


Based on these properties, we study the isometry equivariance of kernel field transforms and $\GM$-convolutions in Section~\ref{sec:isometry_equivariance}.
After defining the term ``isometry equivariance'' formally, Section~\ref{sec:isometry_constraint} proves a central result, which asserts that the demand for \emph{isometry equivariance requires the invariance of the kernel field under isometries}; see Fig.~\ref{fig:isom_invariant_kernel_field_multiple_orbits}.
Section~\ref{sec:isom_equiv_GM_conv} considers the more specific $\GM$-convolutions and proves that they are by design equivariant under any isometry that preserves the $G$-structure.
This result implies in particular, that $\OM$-convolutions are equivariant w.r.t. any isometry.


The invariance constraint on kernel fields enforces that they share weights over the orbits of the isometry group.
This suggests that invariant kernel fields can equivalently be described by representative kernels on orbit representatives, which we formalize in Section~\ref{sec:quotient_kernel_fields}.
Section~\ref{sec:isom_quotients} discussed isometry induced quotient spaces and their representatives.
In Section~\ref{sec:quotient_kernels_stabilizers} we use these mathematical definitions to prove that the space of isometry invariant kernel fields is indeed isomorphic to kernel fields on quotient representatives.
This implies in particular that isometry equivariant kernel field transforms on homogeneous spaces are necessarily convolutions, which closes the loop to prior work.
