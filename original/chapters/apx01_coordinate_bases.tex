%!TEX root=../GaugeCNNTheory.tex


\section{Relation to the coordinate chart formalism of differential geometry}
\label{apx:coordinate_bases}


This appendix serves the purpose of drawing connections between the \emph{bundle formalism}, underlying the theory of coordinate independent CNNs, and the \emph{coordinate chart formalism}, which one likely encounters in a first study of differential geometry.
The main difference between both is that the bundle formalism refers to points~$p$ of the base space $M$ in a \emph{coordinate free} way.
If required, coordinates are directly assigned to the fibers (e.g. tangent spaces) via local bundle trivializations.
In contrast, the chart formalism relies on \emph{coordinate charts} (diffeomorphisms)
\begin{align}
    x: M \supseteq U \to V \subseteq \R^d \,,
\end{align}
which assign coordinates to local patches $U$ of the manifold.
Local bundle trivializations and gauge transformations between them are \emph{induced as differentials of charts and chart transition functions}.
In this section we work out the connection between both formalisms.
An overview of the results is given in Table~\ref{tab:coord_charts_gauge_trafos}.


\etocsettocdepth{3}
\etocsettocstyle{}{} % from now on only local tocs
\localtableofcontents


We start in Section~\ref{apx:tangent_cotangent_dual_bases} by briefly introducing tangent spaces $\TpM$ as spaces of directional derivative operators, from which the cotangent spaces $\TspM$ follow as dual spaces.
Section~\ref{apx:differentials_gradients_jacobians} defines general differentials and the more specific gradients and Jacobians.
Based on these preparations, we will in Section~\ref{apx:coord_basis_def} define \emph{coordinate bases} (holonomic bases)
${\pig[ \frac{\partial}{\partial x_1}\big|_p, \ \dots,\ \frac{\partial}{\partial x_d}\big|_p \pig]\ \in \FpM}$
of the tangent spaces $\TpM$, which are spanned by directional derivative operators along the coordinate grid that is pulled by the chart from $V$ to~$U$.
The dual bases
${\pig[ \hat{d}x_\mu|_p, \ \dots,\ \hat{d}x_\mu|_p \pig]}$
of the cotangent spaces $\TspM$ are given by the gradients of the chart components $x_\mu$.
Transition maps between charts induce covariant and contravariant gauge transformations between the corresponding bases, which are derived in Section~\ref{apx:chart_transition_induced_gauge_trafos}.
Section~\ref{apx:correspondences_bundle_trivializations} interprets the coordinate bases as local bundle trivializations and makes the connection between the bundle formalism and the chart formalism precise.
The bases and trivializations induced from coordinate charts do not cover all possible trivializations, such that one distinguishes between coordinate bases and non-coordinate bases (the bundle formalism allows with general non-coordinate bases).
In the physics literature, non-coordinate bases are usually introduced via \emph{vielbein fields}.
Section~\ref{apx:vielbein_fields} argues that these vielbein fields are just $\GL{d}$-valued gauge transformations from general frames in $\FM$ into a given $G$-structure~$\GM$, within which one can subsequently apply $G$-valued gauge transformations that preserve the $G$-structure.

Comprehensive introductions to the chart formalism are given in~\cite{nakahara2003geometry,schullerGeometricalAnatomy2016,carroll2004spacetime}.
A more rigorous exposition is found in~\cite{schullerGeometricalAnatomy2016}.

We want to remind the reader that we are \emph{not} making use of covariant and contravariant indices.
Indices will always appear as subscripts, with Greek letters $\mu,\nu,\dots$ signaling coordinate chart related indices and Latin letters $i,j,\dots$ signaling indices of general gauges.
Superscripts $A,B,\dots$ are preserved for labeling different charts or gauges.











\subsection{Tangent spaces, cotangent spaces and dual bases}
\label{apx:tangent_cotangent_dual_bases}

\subsubsection{Tangent spaces in terms of directional derivatives}
A common definition of the tangent spaces $\TpM$ of a manifold $M$ is as vector spaces of directional derivative operators at $p \in M$, which we will briefly motivate here.
Let $f\in C^\infty(M)$, that is, $f:M\to\R$ is a smooth map, and, for some interval $I\subseteq\R$ containing $0$, let $\gamma: I \to M$ be a smooth curve which passes at time $t = 0$ through~$p$, i.e. satisfies $\gamma(0) = p$.
One then defines the \emph{directional derivative operator} at~$p$ along~$\gamma$ as the linear operator
\begin{align}\label{eq:tangent_vector_directional_derivative}
    v_\gamma: C^\infty(M) \to \R,\ \ \ f \mapsto \big(f \circ \gamma \big)'(0) \,.
\end{align}
As the derivative is taken along the direction of $\gamma$, that is, tangential to it, $v_\gamma$ is called \emph{tangent vector}.
It can be thought of as the velocity of a particle with trajectory $\gamma$ at time $t=0$.
For later reference we give the following simple commutative diagram, which shows the pullback $f\circ\gamma$ of $f$ from $M$ to $\R$ via $\gamma$, in terms of which the directional derivative is defined:
\begin{equation}\label{cd:directional_derivative}
    \begin{tikzcd}[row sep=2.5em, column sep=4em]
          \R \supset I
                \arrow[r, "\gamma"]
                \arrow[rr, rounded corners, to path={ 
                        |- node[below, pos=.75]{\small$f \circ \gamma$} ([yshift=-3.5ex]\tikztotarget.south)
                        -- (\tikztotarget.south)
                        }]
        & M     \arrow[r, "f"]
        & \R
    \end{tikzcd}
\end{equation}

One can show that the space of all tangent vectors to curves at $p$ forms a $d$-dimensional vector space
\begin{align}
    \TpM\ :=\ \big\{ v_\gamma \,\big|\, \gamma\ \text{is a smooth curve through}\ p \big\} \,,
\end{align}
known as the tangent space at~$p$.
For more details on the definition of tangent vectors and the vector space structure of the tangent spaces we refer to~\cite{schullerGeometricalAnatomy2016}.

Having defined the tangent spaces as vector spaces, one might choose to treat tangent vectors as abstract geometric vectors, thereby ``forgetting'' about their definition via directional derivatives (or any alternative definition made).
We do this at most places, but refer back to the definition via directional derivatives in the following sections to derive differentials of smooth maps and coordinate bases.



\subsubsection{Cotangent spaces}
\label{apx:cotangent_spaces}
As real vector spaces, the tangent spaces $\TpM$ have corresponding \emph{dual spaces} $\TspM := (\TpM)^*$, the \emph{cotangent spaces}.
By the definition of dual spaces, they consist of linear functionals
\begin{align}
    \omega: \TpM \to \R \,,
\end{align}
which are in differential geometry usually called \emph{covectors} or \emph{1-forms}.
Together with the (co)vector addition
$(\omega + \widetilde{\omega})(v) = \omega(v) + \widetilde{\omega}(v)$
and scalar multiplication 
$(\lambda \cdot \omega)(v) = \lambda\cdot (\omega(v))$,
the cotangent spaces are vector spaces themselves.

As finite-dimensional duals of each other, $\TpM$  and $\TspM$ are isomorphic and are thus in particular of the same dimensionality $d = \dim(M) = \dim(\TpM) = \dim(\TspM)$.
The isomorphism between both is, however, not canonical.
A vector space isomorphism can be specified via a (non-degenerate) bilinear form $\eta_p: \TpM \times \TpM \to \R$ on $\TpM$, for instance a Riemannian metric, via
\begin{align}
    \widehat{\eta}_p:\ \TpM \to \TspM,\ \ v \mapsto \eta_p(v, \cdot) \,,
\end{align}
which determines the linear functional
$\widehat{\eta}_p(v):\ \TpM \to \R,\ \ w \mapsto \eta_p(v, w)$.



\subsubsection{Dual bases}
Any basis $\big[e_i\big]_{i=1}^d$ of $\TpM$ canonically induces a \emph{dual basis} $\big[e_i^*\big]_{i=1}^d$ of $\TspM$, defined to satisfy the relations
\begin{align}
    e^*_i e_j = \delta_{ij} \quad \textup{for any }\ i,j \in 1,\dots,d \,.
\end{align}
Let $\big[e^A_i\big]_{i=1}^d$ and $\big[e^B_i\big]_{i=1}^d = \big[e^A_i\big]_{i=1}^d \lhd \big(g^{BA}\big)^{-1}$ be two bases of $\TpM$, which are related by the right action~$\lhd$ of the (inverse) structure group element $\big(g^{BA}\big)^{-1} \in \GL{d}$ in Eq.~\eqref{eq:frame_rightaction}, that is, for $j=1,\dots,d$\ :
\begin{align}\label{eq:general_tangent_basis_gauge_trafo}
    e^B_j\ =\ \sum_l e^A_l \big(g^{BA}\big)^{-1}_{lj}
\end{align}
The dual basis $\big[e^{A,*}_i\big]_{i=1}^d$ transforms accordingly under that left action which sends $e_i^{A,*}$ to
\begin{align}\label{eq:general_cotangent_basis_gauge_trafo}
    e_i^{B,*}\ =\ \sum_k g^{BA}_{ik} e_k^{A,*} \,.
\end{align}
This is affirmed by pairing
\begin{align}
    e_i^{B,*} e_j^B
    \ &=\ \sum_{k,l} g^{BA}_{ik} e_k^{A,*} e_l^A \big(g^{BA}\big)^{-1}_{lj}  \notag \\
    \ &=\ \sum_{k,l} g^{BA}_{ik} \delta_{kl} \big(g^{BA}\big)^{-1}_{lj} \notag \\
    \ &=\ \sum_{k} g^{BA}_{ik} \big(g^{BA}\big)^{-1}_{kj} \notag \\
    \ &=\ \delta_{ij} \,.
\end{align}
The inverse transformation behavior of bases and dual bases is usually referred to as \emph{covariant} and \emph{contravariant} transformation.
Note the similarity of the dual basis transformation to the contravariant transformations $\psi^B = g^{BA} \psi^A$ of gauges in Eq.~\eqref{eq:gauge_trafo_local_def_21} and $v^B = g^{BA} v^A$ of vector components in \eqref{eq:components_leftaction}.
Indeed, gauges are just choices of a cotangent basis as further discussed below.








\subsection{Differentials, gradients and Jacobians}
\label{apx:differentials_gradients_jacobians}

In vector calculus one considers functions $\phi: \R^m \to \R^n$, which can at any point $p\in \R^m$ be linearly approximated by their Jacobian matrix (or total derivative or differential) $d\phi_p = \big(\frac{\partial\phi_i}{\partial x_j} \big|_p \big)_{ij}$.
Here we introduce the generalization of this concept to differentials of smooth functions between smooth manifolds.

\paragraph{Differentials in general:}
Let $\phi: M \to N$ be a smooth map between smooth manifolds $M$ and $N$.
At any point $p\in M$, such a map induces a differential (or pushforward)
\begin{align}
    d\phi_p : \TpM \to \TphipN,\ \ v \mapsto d\phi_p(v)
\end{align}
which linearly maps tangent vectors at~$p$ to tangent vectors at $\phi(p)$.
For the definition of tangent spaces in terms of directional derivatives in Eq.~\eqref{eq:tangent_vector_directional_derivative}, the pushforward of $v \in \TpM$ along $\phi$ is explicitly given by
\begin{align}
    d\phi_p(v): C^\infty(N) \to \R,\ \ \ f \mapsto \big( d\phi_p(v) \big)(f)\ :=\ v(f \circ \phi) \,,
\end{align}
that is, by the application of~$v$ on the pullback $f \circ \phi: M \to \R$ of $f: N\to \R$ via $\phi$.
These definitions are clarified by the following two commutative diagrams:
\begin{equation}
\begin{tikzcd}[column sep=70pt, row sep=30, font=\normalsize]
    M
        \arrow[r, "\phi"]
        \arrow[dr, "f \circ \phi"']
    &
    N
        \arrow[d, "\ f"]
    \\
    & \R
\end{tikzcd}
\qquad\qquad
\begin{tikzcd}[column sep=60pt, row sep=30, font=\normalsize]
    C^\infty(M)
        \arrow[d, "v\ "']
    &
    C^\infty(N)
        \arrow[l, "(\,\cdot\,) \circ \phi"']
        \arrow[dl, "d\phi(v)"]
    \\
    \R
\end{tikzcd}
\end{equation}

From this definition it follows immediately that the differential of the composition of smooth maps equals the composition of their individual differentials, which is just the chain rule:
\begin{align}
    d(\phi \circ \psi)_p\ =\ d\phi_{\psi(p)} \circ d\psi_p
\end{align}
If $\phi$ is invertible (a diffeomorphism) it furthermore follows that its differential is a vector space isomorphism whose inverse equals the differential of $\phi^{-1}$, that is,
\begin{align}\label{eq:differential_inverse}
    \big( d\phi_{p} \big)^{-1}\ =\ d\big( \phi^{-1} \big)_{\phi(p)} \,.
\end{align}

Together, the differentials $d\phi_p$ at individual points $p\in M$ imply a vector bundle morphism (a fiber-wise linear bundle map, see Sections~\ref{sec:fiber_bundles_general}) between the tangent bundles of $M$ and $N$:
\begin{equation}
\begin{tikzcd}[column sep=60pt, row sep=35, font=\normalsize]
    TM
        \arrow[r, "d\phi"]
        \arrow[d, "\piTM"']
    &
    TN
        \arrow[d, "\piTM"]
    \\
    M
        \arrow[r, "\phi"']
    &
    N
\end{tikzcd}
\end{equation}

Note that we are in this appendix using a different notation, namely $d\phi$, than in the main paper, where we instead write $\dphiTM$.
We decided for the former to connect to the usual notation $dx_\mu$ for the chart induced bases of cotangent spaces.
The latter is used in the main text to emphasize the similarity to the bundle maps $\dphiFM$, $\dphiGM$ and $\dphiA$, which are induced on the associated bundles $\FM$, $\GM$ and $\A$.



\paragraph{Gradients:}
In the case of smooth real-valued functions $\phi: M \to \R$, i.e. $\phi \in C^\infty(M)$, the differential $d\phi_p: \TpM \to T_{\phi(p)}\R$ pushes vectors $v$ in $\TpM$ to vectors $d\phi(v): C^\infty(\R) \to \R,\ f \mapsto v(f \circ \phi)$ in $T_{\phi(p)}\R$.
By leveraging the canonical isomorphism
\begin{align}\label{eq:canon_isom_TR_R}
    \iota_{\R}: T_{\phi(p)}\R \xrightarrow{\sim} \R,\ \ v \mapsto v(\id_{\R})
\end{align}
one defines the \emph{gradient operator}
\begin{align}
    \hat{d}_p: C^\infty(M) \to \TspM,\ \ \phi \mapsto  \hat{d}\phi_p := \iota_{\R} \circ d\phi_p = \big( d\phi_p(\,\cdot\,) \big)(\id_{\R}) \,,
\end{align}
which sends smooth functions $\phi$ to covectors%
\footnote{
    The gradient field is often defined as a \emph{vector} field $\nabla f := (\hat{d}f)^{\sharp^\eta}$ which is computed from the \emph{covector} field $\hat{d}f$ via the musical isomorphism $\sharp^\eta: \TsM \to \TM$ corresponding to the metric (``raising indices'').
}
$\hat{d}\phi$, which in turn act on vectors as
\begin{align}\label{eq:gradient_vector_action}
    \hat{d}\phi_p: \TpM \to \R,\ \ v \mapsto \hat{d}\phi_p(v) = \big( d\phi_p(v) \big)(\id_{\R}) = v(\id_{\R} \circ \phi) = v(\phi) \,.
\end{align}
By an \emph{abuse of notation} one usually drops the ``hat'' on $\hat{d}$ and immediately defines $d\phi_p(v) := v(\phi)$.
While this notation is very common, we stick in the following with the ``hat'' to make the requirement for the canonical isomorphism $\iota_{\R}$ explicit.

In Section~\ref{apx:coord_basis_def} below we will see that the bases of $\TspM$ which are dual to coordinate bases of $\TpM$ are given by the gradient 1-forms $\hat{d}x_\mu|_p$, where $x_\mu$ are the components of the coordinate chart.



\paragraph{Jacobians:}
Specifically for functions $\phi: \R^n \to \R^m$ between (subsets of) Euclidean spaces the differential
$d\phi_{x_0}: {T_{x_0}\R^n \to T_{\phi(x_0)}\R^m}$
is easily seen to coincide with the \emph{Jacobian} $\frac{\partial \phi}{\partial x} \big|_{x_0}: \R^n \to \R^m$
after canonically identifying $T_p\R^k \cong \R^k$ in both the domain and codomain.
The canonical isomorphism is here given by
\begin{align}\label{eq:canonical_iso_TRk_Rk}
    \iota_{\R^k}: v \mapsto \big( v(\proj_1), \dots, v(\proj_k) \big) \,,
\end{align}
which generalizes $\iota_{\R}$ from Eq.~\eqref{eq:canon_isom_TR_R} to multiple dimensions.
As the calculation is mostly similar as in the case of gradients, we will not repeat it here but visualize the idea via a commutative diagram:
\begin{equation}\label{cd:jacobian_def}
    \begin{tikzcd}[row sep=2.5em, column sep=4em]
          \R^n
                \arrow[rrr, rounded corners, to path={ 
                        |- node[below, pos=.75]{$\frac{\partial \phi}{\partial x} \Big|_{x_0}$} ([yshift=-3.5ex]\tikztotarget.south)
                        -- (\tikztotarget.south)
                        }]
        & T_{x_0}\R^n
                \arrow[l, "\iota_{\R^n}"']
                \arrow[r, "d\phi |_{x_0}"]
        & T_{\phi(x_0)}\R^m
                \arrow[r, "\iota_{\R^m}"]
        & \R^m
    \end{tikzcd}
\end{equation}

If $\phi$ is invertible, the identity in Eq.~\eqref{eq:differential_inverse} becomes
\begin{align}\label{eq:inv_fct_thm_jacobian}
    \frac{\partial \phi}{\partial x} \bigg|_{x_0}^{-1} \ =\ 
    \frac{\partial \phi^{-1}}{\partial x} \bigg|_{\phi(x_0)} \,,
\end{align}
which is just the inverse function theorem.
We will use this identity later on to invert gauge transformations between different coordinate bases which are induced as Jacobians of chart transition maps.











\subsection{Chart induced coordinate bases}
\label{apx:chart_induced_bases_main}

In this Section we consider \emph{coordinate charts} of the form
\begin{align}
  x: U \to V \,,
\end{align}
which diffeomorphically assign coordinates $x(p) \in V \subseteq \R^d$ to each point $p \in U \subseteq M$.
Any such chart induces a natural choice of bases for the tangent spaces $\TpM$ over~$U$, known as \emph{coordinate bases}.
The dual spaces $\TspM$ of the tangent spaces over~$U$ are accordingly endowed with dual coordinate bases of cotangent vectors.
Transition maps between the coordinates of two charts induce gauge transformations which translate between the corresponding coordinate bases.
These gauge transformations are given by the Jacobians of the transition maps.




\subsubsection{Charts and induced coordinate bases}
\label{apx:coord_basis_def}

\paragraph{Coordinate bases for $\bm\TpM$:}
To motivate the definition of coordinate bases, observe that~$x$ implies a \mbox{``coordinate grid''} on~$U$ by pulling the canonical coordinate grid on $V$ back to the manifold.
The coordinate basis at a specific point $p \in U$ can then be thought of as consisting of those $d$ many \emph{directional derivative operators} which are going \emph{along the coordinate grid lines of~$x$ on~$U$}.

To make this more precise, consider first the curves
\begin{align}
    \widetilde{\gamma}_\mu: I \to V,\ \ \ t \mapsto x(p) + t \epsilon_\mu \quad\qquad \mu = 1,\dots,d
\end{align}
which pass at time $t=0$ with unit velocity in $\mu$-direction through~$x(p) \in V$.
Mapping those $\widetilde{\gamma}_\mu$ via the chart to~$U$ defines the above mentioned curves
\begin{align}
    \gamma_\mu: I \to U,\ \ \ t \mapsto\,
    x^{-1}\mkern-2mu \circ \widetilde{\gamma}_\mu (t) \ =\ 
    x^{-1} \big( x(p) + t \epsilon_\mu \big)
\end{align}
which pass at time $t=0$ along the coordinate grid of~$x$ on~$U$ through~$p$.
The $d$-dimensional coordinate basis of~$\TpM$ induced by~$x$ is then given by the directional derivative operators in Eq.~\eqref{eq:tangent_vector_directional_derivative} along the paths~$\gamma_\mu$.
Denoting the $\mu$-th basis vector by the usual abuse of notation as $\frac{\partial}{\partial x_\mu}\big|_p$ one therefore defines:
\begin{align}\label{eq:coord_basis_def}
    \frac{\partial}{\partial x_\mu} \bigg|_p \!:\ \ f\, \mapsto\,
    \frac{\partial}{\partial x_\mu} \bigg|_p f
    \ \ :=&\ \ \big(f \circ \gamma_\mu \big)'(0) \notag \\
    \ \  =&\ \ \big(f \circ x^{-1} \circ \widetilde{\gamma}_\mu \big)'(0) \notag \\
    \ \  =&\ \ \big(f \circ x^{-1}\big( x(p) + t\epsilon_\mu \big) \big)'(0) \notag \\
    \ \  =&\ \ \pig[\mkern1.5mu \partial_\mu \big(f \circ x^{-1} \big)\pig] \big(x(p)\big)
\end{align}
In the last step we identified the usual $\mu$-th partial derivative of the pullback $f\circ x^{-1}: V \to \R$, which motivates the notation $\frac{\partial}{\partial x_\mu}\big|_p$.
These definitions are visualized in the following commutative diagram which extends the diagram in Eq.~\eqref{cd:directional_derivative}:
\begin{equation}
    \begin{tikzcd}[row sep=4.em, column sep=6.em]
        & V     \arrow[rd, "f\circ x^{-1}"]
        \\
          \R \supset I
                \arrow[r, "\gamma_\mu"]
                \arrow[ru, "\widetilde{\gamma}_\mu"]
                \arrow[rr, rounded corners, to path={ 
                        |- node[below, pos=.75]{\small$f \circ \gamma_\mu$} ([yshift=-3.5ex]\tikztotarget.south)
                        -- (\tikztotarget.south)
                        }]
        & U     \arrow[r, pos=.4, "f"]
                \arrow[u, pos=.4, "x"]
        &[1.4em] \R
    \end{tikzcd}
\end{equation}




\paragraph{Dual coordinate bases for $\bm\TspM$:}

As stated in Section~\ref{apx:tangent_cotangent_dual_bases}, any basis of $\TpM$ induces a \emph{dual basis} of $\TspM$.
Specifically for coordinate bases, spanned by vectors $\frac{\partial}{\partial x_\mu} \big|_p$, the dual basis elements are given by the \emph{gradients $\hat{d}x_\mu|_p = \hat{d}(x_\mu)_p \in \TspM$ of the chart components} $x_\mu = \proj_\mu \circ x: U \to \R$.
That these gradients do indeed make up the dual basis, is easily seen by acting on the basis vectors as defined in Eq.~\eqref{eq:gradient_vector_action}:
\begin{align}
    \hat{d}x_\mu \big|_p\ \frac{\partial}{\partial x_\nu} \bigg|_p
    \ &=\ \frac{\partial}{\partial x_\nu} \bigg|_p x_\mu \notag \\
    \ &=\ \pig[ \partial_\nu \big( x_\mu \circ x^{-1} \big) \pig] \big(x(p)\big) \notag \\
    \ &=\ \pig[ \partial_\nu \big( \proj_\mu \big) \pig] \big(x(p)\big) \notag \\
    \ &=\ \delta_{\mu\nu} \,.
\end{align}





\paragraph{Chart differentials as canonical local trivialization:}

Given that the chart maps from $U \subseteq M$ to $V \subseteq \R^d$, its differentials at $p\in U$ are maps of the form
\begin{align}
    dx_p: \TpM \to T_{x(p)}\R^d \,.
\end{align}
Employing the canonical isomorphism $\iota_{\R^d}$ from $T_{x(p)}\R^d$ to $\R^d$ from Eq.~\eqref{eq:canonical_iso_TRk_Rk} once again, we obtain a map
\begin{align}\label{eq:chart_differential_via_gradients}
    \qquad
    \hat{d}x_p: \TpM \to \R^d,\ \ \ v\ \mapsto\ \hat{d}x_p (v)
    :=\ &\iota_{\R^d} \circ dx_p (v) \notag \\
     =\ & \Big( \big(dx_p(v) \big)(\proj_1) \,,\,\dots,\, \big(dx_p(v) \big)(\proj_d) \Big)^\top \notag \\
     =\ & \Big( v\big(\proj_1 \circ x \circ x^{-1}\big)(x(p)) \,,\,\dots,\, v\big(\proj_1 \circ x \circ x^{-1}\big)(x(p)) \Big)^\top \notag \\
     =\ & \Big( v(x_1(p)) \,,\,\dots,\, v(x_d(p)) \Big)^\top \notag \\
     =\ & \Big( \hat{d}x_1 |_p(v) \,,\,\dots,\, \hat{d}x_d |_p(v) \Big)^\top
\end{align}
after identifying the individual chart component gradients in the last step.
Note that the action of this chart differential on the $\mu$-th coordinate basis yields
\begin{align}
    \hat{d}x_p \: \frac{\partial}{\partial x_\mu} \bigg|_p
     \ &=\ \bigg( \hat{d}x_1|_p \: \frac{\partial}{\partial x_\mu} \bigg|_p \,,\,\dots,\, \hat{d}x_d|_p \: \frac{\partial}{\partial x_\mu} \bigg|_p \bigg)^\top \notag \\
     \ &=\ \big( \delta_{\mu1} \,,\,\dots,\, \delta_{\mu d} \big)^\top \notag \\
     \ &=\ \epsilon_\mu \,,
\end{align}
that is, the $\mu$-th unit vector $\epsilon_\mu$ of $\R^d$.
This implies that $\hat{d}x_p: \TpM \to \R^d$ plays the role of a \emph{gauge} $\psi_p$ at~$p$.
One could therefore equally well have started by defining a cotangent basis and setting
\begin{align}\label{eq:coord_basis_vector_via_chart_differential}
    \frac{\partial}{\partial x_\mu} \bigg|_{x(p)}\ =\ \hat{d}x_p^{-1} (\epsilon_\mu) \,,
\end{align}
which is the analog of Eq.~\eqref{eq:framefield_gauge_equivalence} in the chart formalism.











\subsubsection{Chart transition maps and induced gauge transformations}
\label{apx:chart_transition_induced_gauge_trafos}

Different charts induce different coordinate bases.
Chart transitions therefore induce gauge transformations, i.e. transformations of bases and vector coefficients, which we derive in this section.

In the following we consider two arbitrary, overlapping charts $x^A: U^A \to V^A$ and $x^B: U^B\to V^B$.
The different coordinates which they assign to the overlap $U^A \cap U^B \neq \varnothing$ are then related via \emph{chart transition maps}
\begin{align}\label{eq:chart_transition_fct}
  x^B\circ\left(x^A\right)^{-1} \!:\ x^A\big(U^A\cap U^B\big)\to x^B\big(U^A\cap U^B\big) \,.
\end{align}



\paragraph{Transformation of tangent coordinate bases:}
The coordinate bases of $\TpM$ which are induced by the two charts are according to the last line of Eq.~\eqref{eq:coord_basis_def} by their action on $f \in C^\infty(M)$ defined as
\begin{align}
    \frac{\partial}{\partial x^A_\mu} \bigg|_p f
    \ =\ \Big[\mkern1.5mu \partial_\mu \pig(f \circ \big(x^A\big)^{-1} \pig)\Big] \big(x^A(p)\big)
    \qquad \text{and} \qquad
    \frac{\partial}{\partial x^B_\mu} \bigg|_p f
    \ =\ \Big[\mkern1.5mu \partial_\mu \pig(f \circ \big(x^B\big)^{-1} \pig)\Big] \big(x^B(p)\big) \ ,
    \quad
\end{align}
which is visualized by the following commutative diagram:
\begin{equation}\label{cd:scalar_field_chart_expressions}
    \begin{tikzcd}[row sep=4.em, column sep=6.em] %,
        V^A \supset x^A \big( U^A \cap U^B \big)
                \arrow[rd, "
                    f\circ \big(x^A\big)^{-1}
                    "]
                \arrow[dd, rounded corners, to path={ 
                        -| node[left, pos=.75]{\small$x^B \circ \big(x^A\big)^{-1}$} ([xshift=-3.5ex]\tikztotarget.west)
                        -- (\tikztotarget.west)
                        }]
        \\
        U^A \cap U^B
                \arrow[r, pos=.4, "f"]
                \arrow[u, "x^A"]
                \arrow[d, "x^B"']
        &
        \R
        \\
        V^B \supset x^B \big( U^A \cap U^B \big)
                \arrow[ru, "
                    f\circ \big(x^B\big)^{-1}
                    "']
    \end{tikzcd}
\end{equation}

Via the chart transition maps, the different coordinate bases relate by
\begin{align}\label{eq:coord_basis_trafo_action_f}
    \frac{\partial}{\partial x^B_\mu} \bigg|_p f
    \ &=\ \Big[\mkern1.5mu \partial_\mu \pig(f \circ \big(x^B\big)^{-1} \pig) \Big] \big(x^B(p)\big) \notag \\
    \ &=\ \Big[\mkern1.5mu \partial_\mu \pig(f \circ \big(x^A\big)^{-1} \circ x^A \circ \big(x^B\big)^{-1} \pig) \Big] \big(x^B(p)\big) \,, \notag
\intertext{
which, making use of the multivariate chain rule, further leads to:
}
    \frac{\partial}{\partial x^B_\mu} \bigg|_p f
    \ &=\ \sum_{\nu=1}^d
        \Big[\mkern1.5mu \partial_\nu \pig(f \circ \big(x^A\big)^{-1} \pig)\Big] \big(x^A(p)\big) \cdot
        \Big[\mkern1.5mu \partial_\mu \pig(x^A_\nu \circ \big(x^B\big)^{-1} \pig)\Big] \big(x^B(p)\big) \notag \\
    \ &=\ \sum_{\nu=1}^d \,
        \frac{\partial f}{\partial x^A_\nu} \bigg|_p \ 
        \frac{\partial x^A_\nu}{\partial x^B_\mu} \bigg|_{x^B(p)}
\end{align}
In the last step we introduced the usual abuse of notation%
\footnote{
    The ``abuse'' is that $x^A$ is interpreted as a function of $x^B(p)$, and should therefore rather be written $x^A \circ \big(x^B\big)^{-1}$ as made precise on the right-hand side.
}
\begin{align}\label{eq:abuse_of_notation_jacobian}
    \frac{\partial x^A_\nu}{\partial x^B_\mu} \bigg|_{x^B(p)}
    :=\ \partial_\mu \pig(x^A_\nu \circ \big(x^B\big)^{-1} \pig) \big( x^B(p) \big)
\end{align}
for the components of the \emph{Jacobian}
\begin{align}
    \frac{\partial x^A}{\partial x^B} \bigg|_{x^B(p)}
    =\ \hat{d}x^A_p \circ \hat{d}(x^B_p)^{-1}
\end{align}
\emph{of the transition maps}.
Dropping~$f$ from Eq.~\eqref{eq:coord_basis_trafo_action_f}, we identify the transformation law
\begin{align}\label{eq:coord_bases_trafo_law}
    \frac{\partial}{\partial x^B_\mu} \bigg|_p
    \ =\ \sum_{\nu=1}^d \,
        \frac{\partial}{\partial x^A_\nu} \bigg|_p \ 
        \frac{\partial x^A_\nu}{\partial x^B_\mu} \bigg|_{x^B(p)}
\end{align}
of tangent coordinate bases.
We did hereby choose to write the Jacobian on the right of the basis vector to emphasize that the change of basis is to be understood as a \emph{right action}.
Doing so, we need to warn the reader that $\frac{\partial}{\partial x_\nu}\big|_p$ is just an abuse of notation for the basis vector but does not imply an action of a differential operator on the Jacobian on the right.




\paragraph{Transformation of cotangent coordinate bases:}
The contravariant transformation law of cotangent space coordinate bases follows from the inverse transformation of dual bases in Eq.~\eqref{eq:general_cotangent_basis_gauge_trafo} relative to~\eqref{eq:general_tangent_basis_gauge_trafo}.
To apply this relation, we first adapt Eq.~\eqref{eq:coord_bases_trafo_law} to our convention that bases transform according to a right action with an \emph{inverse} group element.
This is achieved by applying Eq.~\eqref{eq:inv_fct_thm_jacobian} to invert the Jacobian (remember the abuse of notation)
\begin{align}
    \frac{\partial x^A}{\partial x^B} \bigg|_{x^B(p)} \ =\ 
    \frac{\partial x^B}{\partial x^A} \bigg|_{x^A(p)}^{-1}
\end{align}
which implies:
\begin{align}\label{eq:coord_bases_trafo_law_with_inv}
    \frac{\partial}{\partial x^B_\mu} \bigg|_p
    \ =\ \sum_{\nu=1}^d \,
        \frac{\partial}{\partial x^A_\nu} \bigg|_p \ 
        \frac{\partial x^A_\nu}{\partial x^B_\mu} \bigg|_{x^B(p)}
    \ =\ \sum_{\nu=1}^d \,
        \frac{\partial}{\partial x^A_\nu} \bigg|_p \ 
        \bigg( \frac{\partial x^A}{\partial x^B} \bigg|_{x^B(p)} \bigg)_{\nu\mu}
    \ =\ \sum_{\nu=1}^d \,
        \frac{\partial}{\partial x^A_\nu} \bigg|_p \ 
        \bigg( \frac{\partial x^B}{\partial x^A} \bigg|_{x^A(p)}^{-1} \bigg)_{\nu\mu}
\end{align}
The cotangent basis elements therefore transform according to Eqs.~\eqref{eq:general_tangent_basis_gauge_trafo} and ~\eqref{eq:general_cotangent_basis_gauge_trafo} like
\begin{align}\label{eq:chart_component_gradient_trafo_law}
    \hat{d}x^B_\mu|_p \ =\ 
    \sum_{\nu=1}^d\ 
        \frac{\partial x^B_\mu}{\partial x^A_\nu} \bigg|_{x^A(p)}
        \hat{d}x^A_\nu|_p \,.
\end{align}



\paragraph{Transformation of chart differentials:}
The expression of chart differentials $\hat{d}x^A|_p$ in terms of chart component gradients $\hat{d}x^A_\mu|_p$ in Eq.~\eqref{eq:chart_differential_via_gradients} allows to deduce their transformation law from that in Eq.~\eqref{eq:chart_component_gradient_trafo_law}.
Alternatively, one obtains the transformation law by right multiplying with the identity in the form $\id_{\TpM} = \hat{d}x^A|_p \circ \big( \hat{d}x^A|_p \big)^{-1}$ and identify a left multiplication with the Jacobian of the chart transition maps:
\begin{align}\label{eq:chart_differential_trafo_law}
    \hat{d}x^B|_p
    \ &=\ \hat{d}x^B|_p \circ \big( \hat{d}x^A|_p \big)^{-1} \circ \hat{d}x^A|_p \notag \\
    \ &=\ \frac{\partial x^B}{\partial x^A} \bigg|_{x^A(p)} \hat{d}x^A|_p
\end{align}
Note that this result is simply the matrix expression of Eq.~\eqref{eq:chart_component_gradient_trafo_law}.


\paragraph{Transformation of vector coefficients:}
Vectors $v \in \TpM$ are relative to a coordinate basis
$\big[\frac{\partial}{\partial x^B_\mu} \big|_p \big]_{\mu=1}^d$
expressed by coefficients $v^A \in \R^d$:
\begin{align}
    v\ =\
    \sum_{\mu=1}^d \,
    v_\mu^A \frac{\partial}{\partial x^A_\mu} \bigg|_p
\end{align}
The individual coefficients are recovered by the action of the cotangent basis:
\begin{align}
    \hat{d}x^A_\mu \big|_p(v)
    \ =\ \hat{d}x^A_\mu \big|_p\ \sum_{\nu=1}^d \, v_\nu^A \frac{\partial}{\partial x^A_\nu} \bigg|_p
    \ =\ \sum_{\nu=1}^d \, v_\nu^A \delta_{\mu\nu}
    \ =\ v^A_\mu
\end{align}
This implies that the coefficients transform contravariantly, just as the cotangent coordinate basis:
\begin{align}
    v^B_\mu
    \ =\ \hat{d}x^B_\mu \big|_p (v)
    \ =\ \sum_{\nu=1}^d\ 
        \frac{\partial x^B_\mu}{\partial x^A_\nu} \bigg|_{x^A(p)}
        \hat{d}x^A_\nu|_p (v)
    \ =\ \sum_{\nu=1}^d\ 
        \frac{\partial x^B_\mu}{\partial x^A_\nu} \bigg|_{x^A(p)}
        v^A_\nu
\end{align}
It is easily asserted that this transformation law does indeed lead to a coordinate independent representation of coordinate free vectors $v\in\TpM$:
\begin{align}
    \sum_\mu \frac{\partial}{\partial x^B_\mu} \bigg|_p v_\mu^B
    \ =\ \sum_{\mu,\nu,\rho} \frac{\partial}{\partial x^A_\nu} \bigg|_p \,
        \frac{\partial x^A_\nu}{\partial x^B_\mu} \bigg|_{x^B(p)} \,
        \frac{\partial x^B_\mu}{\partial x^A_\rho} \bigg|_{x^A(p)} \,
        v^A_\rho
    \ =\ \sum_{\nu,\rho} \frac{\partial}{\partial x^A_\nu} \bigg|_p \,
        \delta_{\nu\rho} \,
        v^A_\rho
    \ =\ \sum_\nu \frac{\partial}{\partial x^A_\nu} \bigg|_p v_\nu^A
\end{align}



















\subsection{Coordinate bases as local bundle trivializations}
\label{apx:correspondences_bundle_trivializations}

The chart transition map induced transformation laws in Section~\ref{apx:chart_transition_induced_gauge_trafos} coincide the gauge transformations as formulated in Section~\ref{sec:21_main} when identifying the Jacobians
$\frac{\partial x^B}{\partial x^A} \big|_{x^A(p)}$ with $g_p^{BA}$.
In Section~\ref{apx:correspondences_chart_gauge_ptwise} we make these connections precise by listing all correspondences.
Section~\ref{apx:correspondences_chart_gauge_local} extends these results by deriving expressions for chart induced bundle trivializations on extended domains~$U \subseteq M$ as introduced in Section~\ref{sec:bundles_fields}.
A dictionary which summarizes the correspondences is given in Table~\ref{tab:coord_charts_gauge_trafos}.



\subsubsection[Correspondences to pointwise trivializations of \texorpdfstring{$   \TpM$}{TpM}]%
              {Correspondences to pointwise trivializations of \texorpdfstring{$\bm\TpM$}{TpM}}
\label{apx:correspondences_chart_gauge_ptwise}


\paragraph{Gauges and chart differentials:}
The bundle formalism relies on the definition of gauges (Eq.~\eqref{eq:gauge_definition})
\begin{align}
    \psiTMp^A: \TpM \to \R^d \,,
\end{align}
which are vector bundle isomorphisms, assigning coordinates to tangent spaces with $p\in U^A$.
In the chart formalism, gauges over $U^A$ are \emph{induced} as chart differentials (Eq.~\eqref{eq:chart_differential_via_gradients}):
\begin{align}
    \hat{d}x^A_p: \TpM \to \R^d
\end{align}
Different gauges are related by gauge transformations (Eq.~\eqref{eq:gauge_trafo_local_def_21})
\begin{align}
    \psiTMp^B\ =\ g^{BA}_p\, \psiTMp^A \,\ \qquad
    \qquad &\textup{with} \qquad
    g^{BA}_p\ :=\ \psiTMp^B \circ \big(\psiTMp^A\big)^{-1} \ \ \in\ G \,.
\intertext{
The same definition holds for the chart induced gauges, where gauge transformations turn out to coincide with the Jacobian of the chart transition maps (Eq.~\eqref{eq:chart_differential_trafo_law}):
}
    \hat{d}x^B_p\ =\ \frac{\partial x^B}{\partial x^A} \bigg|_{x^A(p)} \hat{d}x^A_p
    \qquad &\textup{with} \qquad
    \frac{\partial x^B}{\partial x^A} \bigg|_{x^A(p)}
    \!=\ \hat{d}x^B_p \circ \big( \hat{d}x^A_p \big)^{-1}
    \ \  \in\ \GL{d}
\end{align}



\paragraph{Vector components:}
As vector components $v^A = \psiTMp^A(v)$ or $v^A = \hat{d}x^A|_p(v)$ are given by the action of gauges, they show the same covariant transformation behavior
\begin{align}
    v^B = g_p^{BA} v^A
    \qquad &\textup{and} \qquad
    v^B = \frac{\partial x^B}{\partial x^A} \bigg|_{x^A(p)} v^A \,.
\intertext{In terms of components, these relations are written as}
    v^B_i = \sum_{j=1}^d \big(g_p^{BA}\big)_{ij}\, v^A_j
    \qquad &\textup{and} \qquad
    v^B_\mu = \sum_{\nu=1}^d \frac{\partial x^B_\mu}{\partial x^A_\nu} \bigg|_{x^A(p)} v^A_\nu \,.
\end{align}




\paragraph{Induced reference frames:}

Reference frames are in the bundle formalism induced by mapping the vectors~$\epsilon_i$ of the standard frame $e\in G$ of $\R^d$ through the gauge map back to $\TpM$ (Eq.~\ref{eq:framefield_gauge_equivalence}):
\begin{align}
    \big[ e_i^A \big]_{i=1}^d\ =\ \Big[ \big(\psiTMp^A \big)^{-1} (\epsilon_i) \Big]_{i=1}^d
\end{align}
The corresponding relation in the chart formalism is according to Eq.~\eqref{eq:coord_basis_vector_via_chart_differential} given by 
\begin{align}
    \bigg[ \frac{\partial}{\partial x^A_\mu} \bigg|_p \bigg]_{\mu=1}^d\ =\ \Big[ \big(\hat{d}x_p^A \big)^{-1} (\epsilon_\mu) \Big]_{\mu=1}^d
\end{align}
Eq.~\eqref{eq:frame_rightaction} shows that the transformation laws of reference frames is given by the right action
\begin{align}\label{eq:trafo_law_comparison_basis_gauge}
    \left[e_{i}^B\right]_{i=1}^d
    \  =\ \left[ e_{i}^A \right]_{i=1}^d \!\lhd \left(g_p^{BA}\right)^{-1}
    \ :=\ \left[ \sum\nolimits_{j=1}^d e_{j}^A\, \big(g_p^{BA}\big)^{-1}_{ji} \right]_{i=1}^d
    \ =\ \left[ \sum\nolimits_{j=1}^d e_{j}^A\, \big(g_p^{AB}\big)_{ji} \right]_{i=1}^d \,.
\end{align}
In analogy, the transformation law of coordinate bases is from Eq.~\eqref{eq:coord_bases_trafo_law_with_inv} seen to be given by
\begin{align}\label{eq:trafo_law_comparison_basis_chart}
    \bigg[\frac{\partial}{\partial x^B_\mu} \bigg|_p \bigg]_{\mu=1}^d
    =\ \bigg[\frac{\partial}{\partial x^A_\mu} \bigg|_p \bigg]_{\mu=1}^d \mkern-6mu\lhd \frac{\partial x^B}{\partial x^A} \bigg|_{x^{\mkern-1muA}\mkern-1mu(p)}^{\;-1}
    &=\ \Bigg[ \sum_{\nu=1}^d \,
            \frac{\partial        }{\partial x^A_\nu} \bigg|_p \ 
            \bigg( \frac{\partial x^B}{\partial x^A} \bigg|_{x^{\mkern-1muA}\mkern-1mu(p)} \bigg)^{-1}_{\nu\mu}
          \Bigg]_{\mu=1}^d 
    \notag \\
    &=\ \Bigg[ \sum_{\nu=1}^d \,
            \frac{\partial        }{\partial x^A_\nu} \bigg|_p \ 
            \frac{\partial x^A_\nu}{\partial x^B_\mu} \bigg|_{x^{\mkern-1muB}\mkern-1mu(p)}
          \Bigg]_{\mu=1}^d 
\end{align}











\subsubsection[Chart induced local trivializations of \texorpdfstring{$    \pi_{\TM}^{-1}(U) $}{TU}]%
              {Chart induced local trivializations of \texorpdfstring{$\bm{\pi_{\TM}^{-1}(U)}$}{TU}}
\label{apx:correspondences_chart_gauge_local}

The correspondences laid out in the last section were relating \emph{pointwise} trivializations $\psiTMp$ of $\TpM$ to chart differentials $\hat{d}x_p$.
In order to complete this picture, this section adds expressions for local trivializations
\begin{align}
    \PsiTM: \piTM^{-1}(U) \to U \times \R^d
\end{align}
which are induced by charts.

A good candidate to construct $\PsiTM$ from is the chart differential
\begin{align}
    dx: \piTM^{-1}(U) \to TV
\end{align}
which is a vector bundle isomorphism that differs from the vector space isomorphisms $dx_p$ by not being restricted to a single point $p \in U$.
To proceed, we generalize the canonical isomorphism $\iota_{\R^d}$ in Eq.~\eqref{eq:canonical_iso_TRk_Rk} from a single point to all the tangent spaces $T_xV \cong \R^d$ over $V \subseteq \R^d$, resulting in the following \emph{canonical local trivialization} of $TV$:
\begin{align}
    \iota_{TV}: TV \to V\times\R^d,\ \ v \mapsto \big( \piTV(v),\, \iota_{\R^d}(v) \big) \,.
\end{align}
This allows to generalize $\hat{d}x_p$ from a single point to a map
\begin{align}
    \hat{d}x\, :=\, \iota_{V\times\R^d} \circ dx \,:\ \piTM^{-1}(U) \to V \times \R^d \,,
\end{align}
which is, however, still not the local trivialization sought for.
By mapping the first factor via the inverse chart from $V$ to $U$, we obtain the \emph{chart induced local bundle trivialization}:
\begin{align}
    \PsiTM\ :=\ \big(x^{-1} \times \id \big) \circ \hat{d}x
\end{align}
As usual, we visualize the definitions made in a commutative diagram:
\begin{equation}\label{cd:coordinate_basis_bundle_trivialization}
    \begin{tikzcd}[row sep=3.5em, column sep=6em]
        V \times \R^d
            \arrow[rrr, rounded corners, to path={ 
                    ([xshift=-1ex]\tikztostart.north)
                    |- node[above, pos=.75]{\small$\big( x^{-1} \times \id \big)$} ([yshift=10ex]\tikztotarget.north)
                    -- (\tikztotarget.north)
                    }]
            \arrow[dr, "\proj_1"']
        &
        TV  \arrow[d, "\piTV"]
            \arrow[l, "\iota_{V \times \R^d}"']
        &
        \piTM^{-1}(U)
            \arrow[d, "\piTM"']
            \arrow[r, "\PsiTM"]
            \arrow[l, "dx"']
            \arrow[ll, rounded corners, to path={ 
                    |- node[above, pos=.75]{\small$\hat{d}x$} ([yshift=4ex, xshift=1ex]\tikztotarget.north)
                    -- ([xshift=1ex]\tikztotarget.north)
                    }]
        &
        U \times \R^d
            \arrow[ld, "\proj_1"]
        \\
        &
        V
        &
        U
            \arrow[l, "x"]
    \end{tikzcd}
\end{equation}

Considering two overlapping charts $x^A: U^A \to V^A$ and $x^B: U^B \to V^B$ and denoting $U^{AB} = U^A \cap U^B$, one obtains transition maps
\begin{align}
    \hat{d}x^B \circ \big( \hat{d}x^A \big)^{-1} \,=\,
    \left( x^B \mkern-5mu\circ\mkern-3mu (x^A)^{-1} \times \frac{\partial x^B}{\partial x^A} \right)
    \,:\ x^A\big( U^{AB}\big) \times \R^d \to x^A\big( U^{AB}\big) \times \R^d
\end{align}
and
\begin{align}
    \PsiTM^B \circ \big( \PsiTM^A \big)^{-1} \,=\,
    \left( \id \times \frac{\partial x^B}{\partial x^A} \right)
    \,:\ U^{AB} \times \R^d \to U^{AB} \times \R^d \,.
\end{align}
These definitions and their mutual relation is shown in the following commutative diagram:
\begin{equation}\label{cd:coordinate_basis_bundle_trivialization_transition}
    \begin{tikzcd}[row sep=3.5em, column sep=4em]
        x^B\big(U^{AB}\big) \times \R^d
            \arrow[rr, "\big( (x^B)^{-1} \times \id \,\big)\ "]
        & &
        U^{AB} \times \R^d
        \\
        &
        \piTM^{-1}\big(U^{AB}\big)
            \arrow[ul, "\hat{d}x^B"]
            \arrow[dl, "\hat{d}x^A"']
            \arrow[ur, "\PsiTM^B"']
            \arrow[dr, "\PsiTM^A"]
        \\
        x^A\big(U^{AB}\big) \times \R^d
            \arrow[rr, "\big( (x^A)^{-1} \times \id \,\big)"']
            \arrow[uu, "\pig( x^B \mkern-5mu\circ\mkern-3mu (x^A)^{-1} \times \frac{\partial x^B}{\partial x^A} \pig)\ "]
        &&
        U^{AB} \times \R^d
            \arrow[uu, "
                    \hspace*{10pt}
                    $\big(\id \times g^{BA} \big)$
                    \\ \rule{0pt}{16pt}
                    $ = \pig( \id \times \frac{\partial x^B}{\partial x^A} \pig)$
                    "' align = left]
    \end{tikzcd}
\end{equation}




















\subsection{\textit{G}-structures and vielbein fields}
\label{apx:vielbein_fields}


As discussed in Sections~\ref{sec:G_associated_bundles} and~\ref{sec:bundle_trivializations}, any $G$-atlas $\{(\PsiTM^X, U^X)\}$ of local tangent bundle trivializations specifies a corresponding $G$-structure, that is, a subbundle~$\GM$ of distinguished reference frames which respect (or define) some geometric structure on~$M$.
By definition, the transition maps $g^{BA}$ of associated $G$-bundles take values in a reduced structure group $G\leq\GL{d}$.
This raises the question whether one can similarly find ``$G$-atlases of charts'' $\{(x^X, U^X)\}$, whose Jacobians $\frac{\partial x^B}{\partial x^A}$ take values in a reduced structure group $G \leq \GL{d}$ and therefore encode a $G$-structure.
For some structure groups this is certainly possible; for instance, an orientation of an orientable manifold can always be fixed by specifying some $\operatorname{GL}^+(d)$-atlas of positively oriented charts, whose transition Jacobians take values in $\operatorname{GL}^+(d)$.
In general, it is, however, impossible to find coordinate charts which induce coordinate bases that lie in a given $G$-structure.
One therefore resorts to \emph{explicit gauge transformation from coordinate bases into the $G$-structure}, known as \emph{vielbein fields}~\cite{yepez2011einstein, zhou2016gauge, nakahara2003geometry, carroll2004spacetime}.
After initially transforming from coordinate bases to the $G$-structure, the gauge freedom within the $G$-structure allows for further $G$-valued gauge transformations.


An important example in physics are $\O{d}$-structures (or $\O{1,\, d-1}$-structures for spacetimes), which consist of orthonormal reference frames relative to the (pseudo) Riemannian metric $\eta$ of~$M$.%
\footnote{
    The symbol $\eta$ is in the physics literature commonly preserved for the Minkowski metric $\operatorname{diag}(+1,\, -1,\, \dots,\, -1)$ while the (pseudo) Riemannian metric of~$M$ is denoted by~$g$.
    In contrast, we are writing group elements in the structure group as $g\in G$ and thus use $\eta$ for the (pseudo) Riemannian metric of~$M$.
}
Such orthonormal frames represent the possible laboratory frames of an inertial observer.
They are for instance used to formulate relativistic quantum field theories, specifically the Dirac equation, in curved spacetimes.
Recall that a given $G$-structure is to be respected by local bundle trivializations, which means that the gauge maps $\psiGMp$ need to map the $G$-structure $\GpM$ at~$p\in M$ to the canonical standard $G$-structure $G$ of $\R^d$.
For the specific case of $\O{d}$-structures this is equivalent to the requirement on bundle trivializations to preserve the metric, i.e. $\eta_p(v,w) = \langle \psiTMp(v), \psiTMp(w) \rangle$ for any $p\in M$ and $v,w \in \TpM$, which is accomplished without problems in the bundle formalism.
Given a coordinate chart $x: U \to V$, the induced gauges on $p\in U$ were in the previous sections shown to be given by $\psiTMp = \hat{d}x_p: \TpM \to \R^d$.
The requirement on them to preserve the metric therefore becomes
\begin{align}
    \eta_p(v,w) = \pig\langle \hat{d}x_p(v) \,,\, \hat{d}x_p(w) \pig\rangle \,,
\end{align}
which is exactly the defining property for~$x$ being an \emph{isometry}.
This result implies that \emph{coordinate bases only define an $\O{d}$-structures if $U$ and $V$ are isometric} -- which is only the case if $M$ is locally flat on $U$.
For any non-flat region of $M$ it is therefore impossible to describe an $\O{d}$-structures via coordinate bases directly.
This incompatibility expresses itself for instance in the fact that the components~$\eta_{\mu\nu}$ of the Riemannian metric on~$M$ relative to the chosen coordinate basis differ from $\delta_{\mu\nu}$ (or $\operatorname{diag}(+1,-1,\dots,-1)_{\mu\nu}$).

As mentioned before, the orthonormal frames of an $\O{d}$-structure $\OM$ are in the physics literature typically defined via a gauge transformation relative to some chart induced frame field $\big[ \frac{\partial}{\partial x_\mu} \big]_{\mu=1}^d$.
Denoting this gauge transformation, which is called \emph{vielbein field}, by
\begin{align}
    \mathfrak{e}^A: U \to \GL{d} \,,
\end{align}
the orthonormal frame field is defined by%
\footnote{
    In the physics literature this relation is expressed as
    $e^A_i = (\mathfrak{e}^A)^{\mu}_{\,\ i} \frac{\partial}{\partial x^\mu}$
    The inverse is here merely signaled by the opposite position of the indices
    $(\mathfrak{e}^A)^\mu_{\,\ i} := (\mathfrak{e}^A)^{-1}_{\mu i}$ 
    in comparison to
    $(\mathfrak{e}^A)_\mu^{\,\ i} := \mathfrak{e}^A_{\mu i}$.
}
\begin{align}
    \big[e^A_i\big]_{i=1}^d
    \ :=\ \bigg[ \frac{\partial}{\partial x_\mu} \bigg]_{i=1}^d \lhd \big(\mathfrak{e}^A\big)^{-1}
    \  =\ \bigg[ \sum_\mu \frac{\partial}{\partial x_\mu} \big(\mathfrak{e}^A\big)^{-1}_{\mu i} \bigg]_{i=1}^d
    \ \ \in\ \Gamma(U, \OM) \,.
\end{align}
The orthonormality of the resulting frame field is usually expressed as%
\footnote{
    In the physics literature this relation is usually written
    $\eta_{\mu\nu}\ (\mathfrak{e}^A)^\mu_{\,\ i}\ (\mathfrak{e}^A)^\nu_{\,\ j}\, =\, \delta_{ij}$.
}
\begin{align}
    \delta_{ij}
    \ &=\ \eta\big( e_i^A,\, e_j^A \big) \notag \\
    \ &=\ \eta\bigg( \sum_\mu \frac{\partial}{\partial x_\mu} \big(\mathfrak{e}^A\big)^{-1}_{\mu i}  \,,\; \sum_\nu \frac{\partial}{\partial x_\nu} \big(\mathfrak{e}^A\big)^{-1}_{\nu j} \bigg) \notag \\
    \ &=\ \sum_{\mu\nu} \eta\bigg( \frac{\partial}{\partial x_\mu} \,,\, \frac{\partial}{\partial x_\nu} \bigg)\, \big(\mathfrak{e}^A\big)^{-1}_{\mu i}\, \big(\mathfrak{e}^A\big)^{-1}_{\nu j} \notag \\
    \ &=\ \sum_{\mu\nu} \eta_{\mu\nu}\, \big(\mathfrak{e}^A\big)^{-1}_{\mu i}\, \big(\mathfrak{e}^A\big)^{-1}_{\nu j} \,,
\end{align}
which explains why the vielbein field is sometimes called ``square root of the metric''.
As usual, vector components are translated via the non-inverted gauge transformation, that is:%
\footnote{
    Again, in the usual notation in physics this relation reads $(v^A)^i\, =\, (\mathfrak{e}^A)_\mu^{\,\ i} v^\mu$.
}
\begin{align}
    v^A_i\ =\ \sum_\mu \mathfrak{e}^A_{i\mu}\, v_\mu
\end{align}

A simple dimension counting argument illustrates the gauge freedom in the $\O{d}$-structure:%
\footnote{
    In physics, one rather considers local Lorentz transformations $\Lambda \in \O{1,3}$, which describe rotations and boosts of local reference frames.
}
Being an element of the general linear group, a vielbein $\mathfrak{e}^A(p) \in \GL{d}$ has $d^2$ degrees of freedom, while the metric $\eta$, as a symmetric, bilinear form, has $d(d+1)/2$ degrees of freedom.
The missing $d(d-1)/2$ degrees of freedom correspond exactly to gauge transformations by structure group elements $g^{BA} \in \O{d}$.
Alternatively, from the viewpoint of $G$-structures, $\FpM \cong \GL{d}$ has $d^2$ degrees of freedom while $\OpM \cong \O{d}$ has $d(d-1)/2$ degrees of freedom, fixing $d(d+1)/2$ degrees of freedom which correspond to the choice of metric.


All constructions are obviously generalized to arbitrary $G$-structures with $\GL{d}$-valued vielbein fields mapping coordinate bases into $\GM$ and the freedom to apply $G$-valued gauge transformation afterwards.



\begin{landscape}
\begin{table}[h!]
    \vspace*{8ex}
    \centering%
    \scalebox{1.}{%
        \def\arraystretch{2.75}% 1 is the default
\setlength\tabcolsep{2.8ex}
\small
\begin{tabular}{ @{\ \ } l r@{\,}r@{\ }c@{\ }l cc @{\ \ } }
	\toprule
	\\[-8.0ex]
	& \multicolumn{4}{c}{ایزومورفیسم}
	& فرمالیسم کلاف
	& فرمالیسم چارت \\
	\midrule[0.07em] % default width = 0.05
	%%%%%%%%%%%%%%%%%%%%%%%%%%%%%%%%%%%%%%%%%%%%%%%%%%%%%%%%%%%%%%%%%%%%%
	چارت
	& $x^A :\ $
	& $U^A$
	& $\xrightarrow{\sim}$
	& $V^A$
	& ـــ
	& هر دیفئومورفیسم دلخواه
	\\
	نگاشت گذار
	& $x^B \!\circ\! \big(x^A\big)^{-1} :\ $
	& $x^B \mkern-1mu\big( U^{\mkern-2muA\mkern-2muB} \big)$
	& $\xrightarrow{\sim}$
	& $x^A \mkern-1mu\big( U^{\mkern-2muA\mkern-2muB} \big)$
	& ـــ
	& القا شده توسط چارت‌ها
	\\
	\midrule[0.04em] % default width = 0.05
	%%%%%%%%%%%%%%%%%%%%%%%%%%%%%%%%%%%%%%%%%%%%%%%%%%%%%%%%%%%%%%%%%%%%%
	تریویالیزاسیون نقطه‌ای \hspace*{-3ex}
	& $\psiTMp^A :\ $
	& $\TpM$
	& $\xrightarrow{\sim}$
	& $\R^d$
	& ایزومورفیسم خطی از $G$-اطلس
	& $\hat{d}x^A_p = \big( \hat{d}x^A_1|_p,\, \dots,\, \hat{d}x^A_d|_p\, \big)^{\!\top}$
	\\
	نگاشت گذار
	& $\psiTMp^B \mkern-2mu\circ\! \big(\psiTMp^A\big)^{\mkern-2mu-1} \!:\ $
	& $\R^d$
	& $\xrightarrow{\sim}$
	& $\R^d$
	& عنصر گروه ساختار $g_p^{BA} \in G$
	& $\hat{d}x^B_p \circ \big(\hat{d}x^A_p\big)^{-1}
	=\, \displaystyle \frac{\partial x^B}{\partial x^A} \bigg|_{\mkern-1mu x^{\mkern-1mu A}\mkern-2mu(p)} $
	\\
	\midrule[0.04em] % default width = 0.05
	%%%%%%%%%%%%%%%%%%%%%%%%%%%%%%%%%%%%%%%%%%%%%%%%%%%%%%%%%%%%%%%%%%%%%
	تریویالیزاسیون محلی
	& $\PsiTM^A :\ $
	& $\piTM^{-1} \big(U^A\big)$
	& $\xrightarrow{\sim}$
	& $U^A \times \R^d$
	& $v \mapsto \big( \piTM\mkern-1mu(v),\ \psiTMpiv(v) \big)$
	& $\big((x^A)^{-1} \times \id\big) \circ \hat{d}x^A$
	\\
	نگاشت گذار
	& $\PsiTM^B \mkern-2mu\circ\! \big(\PsiTM^A\big)^{\mkern-2mu-1} :\ $
	& $U^{\mkern-2muA\mkern-2muB} \mkern-4mu\times\! \R^d$
	& $\xrightarrow{\sim}$
	& $U^{\mkern-2muA\mkern-2muB} \mkern-4mu\times\! \R^d$
	& $\big(\id \times g^{BA} \big)$
	& $\displaystyle \bigg( \id \times \frac{\partial x^B}{\partial x^A} \bigg)$
	\\
	\midrule[0.07em] % default width = 0.05
	%%%%%%%%%%%%%%%%%%%%%%%%%%%%%%%%%%%%%%%%%%%%%%%%%%%%%%%%%%%%%%%%%%%%%
	چارچوب عمومی
	& \multicolumn{4}{c}{$\big[ e^A_i \big]_{i=1}^d \in \FpM$}
	& $\Big[ \big(\psiTMp^A\big)^{-1} (\epsilon_i) \Big]_{i=1}^d$\ از $\GL{d}$-اطلس
	& $\displaystyle \bigg[\frac{\partial}{\partial x^A_\mu} \bigg|_p \,\bigg]_{\mu=1}^d = \Big[ \big(\hat{d}x^A_p \big)^{-1} (\epsilon_i) \Big]_{\mu=1}^d$
	\\
	چارچوب $G$-ساختار
	& \multicolumn{4}{c}{$\big[ e^A_i \big]_{i=1}^d \in \GpM$}
	& $\Big[ \big(\psiTMp^A\big)^{-1} (\epsilon_i) \Big]_{i=1}^d$\ از $G$-اطلس \kern16pt
	& $\displaystyle \bigg[\sum\nolimits_{\mu}\, \frac{\partial}{\partial x_\mu} \bigg|_p\, \big( \mathfrak{e}^A \big)^{-1}_{\!\mu i}\, \bigg]_{i=1}^d$
	\\
	\bottomrule
\end{tabular}
    }%
    \vspace*{4ex}%
    \captionsetup{width=.9\linewidth}
    \caption{
        An overview of different types of coordinatizations on manifolds.
        The bundle formalism (3rd column), which is used in this work, directly assigns coordinates to the tangent spaces, while referring to the points~$p$ of the base space~$M$ in a coordinate free fashion.
        In contrast, the chart formalism (4th column) assigns coordinates to local subsets $U^X \subseteq M$ of the manifold.
        Local trivializations of the tangent bundle and bundle transition maps between them are induced as differentials of the charts and their transition maps, the latter usually referred to as Jacobians.
        The second last row gives expressions for the reference frames which are induced as identity sections of local trivializations of $\TM$ (3rd column) or as chart induced coordinate bases (4th column).
        Similarly, the last row compares definitions of $G$-structures -- for instance orthonormal frames -- via an $G$-atlas for $\TM$ (3rd column) and via vielbein fields as gauge transformations relative to coordinate bases (4th column).
        As usual, we abbreviate $U^A \cap U^B$ by $U^{AB}$ and assume $p\in U^{AB}$.
    }
    \label{tab:coord_charts_gauge_trafos}
\end{table}
\end{landscape}
