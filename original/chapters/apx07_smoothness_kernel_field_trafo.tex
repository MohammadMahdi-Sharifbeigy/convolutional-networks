%!TEX root=../GaugeCNNTheory.tex


%%%%%%%%%%%%%%%%%%%%%%%%%%%%%%%%%%%%%%%%%%%%%%%%%%%%%%%%%%%%%%%%%%%%%%%%%%%%%%%%%%%%%%%%%%%%%%%%%%%%%%%%%%%%%%%%%%%
\filbreak
%%%%%%%%%%%%%%%%%%%%%%%%%%%%%%%%%%%%%%%%%%%%%%%%%%%%%%%%%%%%%%%%%%%%%%%%%%%%%%%%%%%%%%%%%%%%%%%%%%%%%%%%%%%%%%%%%%%

\section{Existence and smoothness of kernel field transforms}
\label{apx:smoothness_kernel_field_trafo}


In Def.~\ref{dfn:kernel_field_trafo} we proposed \emph{kernel field transforms} $\TK$ as smooth integral transforms
\begin{align}
    \TK: \Gamma(\Ain)\to \Gamma(\Aout)
\end{align}
which are parameterized by some kernel field $\K$ (Def.~\ref{dfn:kernel_field_general}) and are pointwise given by
\begin{align}\label{eq:APX_kernel_field_trafo_def_ptwise}
    \big[ \TK (f)\big] (p)
    \ \ :=\ 
    \int\limits_{\TpM}\!\!
    \K(v) \,
    \Expspf(v)
    \ dv
    \ \ =\ 
    \int\limits_{\TpM}\!\!
    \K(v) \ 
    \PAinexppv \,
    f(\exp_p\!v)
    \ dv \,.
\end{align}
Kernel field transforms include $\GM$-convolutions from Def.~\ref{dfn:coord_free_conv} as a special cases for $\GM$-convolutional kernel fields.

Here we briefly discuss the well-definedness of kernel field transforms.
It is clear that the integrand of Eq.~\eqref{eq:APX_kernel_field_trafo_def_ptwise} lies for any $p\in M$ and $v \in \TpM$ in $\Aoutp$.
What remains to be shown is the \emph{existence} of the integral and the \emph{smoothness} of the resulting feature field.
In the following we will first give some general remarks on how to approach these questions.
We will then prove Theorem~\ref{thm:existence_kernel_field_trafo_compact_kernels}, i.e. the well-definedness of kernel field transforms for the specific case of fields of kernels which are compactly supported on a ball of fixed radius around the origin.







\paragraph{Existence:}
The existence and smoothness of kernel field transforms requires a suitable choice of kernel field~$\K$.
Similar to the case of conventional convolutions on $M=\R$, the requirements on $\K$ in order for the kernel field transform to exist depend on the specific properties of the input feature field $f\in\Gamma(\Ain)$.%
\footnote{
    See the discussion at \url{https://en.wikipedia.org/wiki/Convolution\#Domain_of_definition}.
}
In~general, $\K$ needs to decay sufficiently rapidly in order to make the integrand in Eq.~\eqref{eq:APX_kernel_field_trafo_def_ptwise} integrable.


A special case of great practical importance is that of kernels $\Kp: \TpM \to \Hom(\Ainp,\Aoutp)$ which are at any $p\in M$ \emph{compactly supported}.
In this case the integral is always guaranteed to exist.
To see this, note that (input) feature fields and kernel fields are defined to be smooth.
The smoothness of the metric further implies that the
Riemannian volume density, the exponential map and the parallel transport are smooth~\cite{gallier2019diffgeom1}.
In combination, the whole integrand in Eq.~\eqref{eq:APX_kernel_field_trafo_def_ptwise} is seen to be a smooth and thus continuous function from~$TM$ to~$\Aout$.
If~$\Kp$ is in addition compactly supported, the integrand becomes continuous and compactly supported on $\TpM$, which, by a generalization of the extreme value theorem, implies that its image is compact (and in a local trivialization $\R^{c}$ of $\Aoutp$ bounded)~\cite{rudin1976analysis}.
This guarantees the existence of the integral~\cite{forster2012analysis3,spivak2019calculus}.


Depending on the application, the requirement on the support of $\K$ might be relaxed.
For instance, images on $M=\R^d$ are usually compactly supported themselves, such that no additional properties of $\K$ except for its smoothness are required.







\paragraph{Smoothness:}

We turn to discuss the smoothness of kernel field transforms, that is, their property to map smooth input fields $\fin\in\Gamma(\Ain)$ to smooth output fields $\fout := \TK(\fin) \in \Gamma(\Aout)$.
By definition, a map $\fout: M\to \Aout$ \emph{between manifolds} $M$ and $\Aout$ is said to be smooth if its coordinate representations are smooth.
In equations, $\fout$ is smooth if for any $p\in M$ there exist smooth charts $(U,\phi)$ about $p$ in $M$ and $(\widetilde{U}, \widetilde{\phi})$ about $\fout(p)$ in $\Aout$ with $\fout(U) \subseteq \widetilde{U}$ such that $\widetilde{\phi} \circ \fout \circ \phi^{-1}: \phi(U) \to \widetilde{\phi}(\widetilde{U})$ is smooth as a map between (subsets of) Euclidean spaces.
Given $(U,\phi)$, a convenient choice for $(\widetilde{U}, \widetilde{\phi})$ would be
$ \widetilde{\phi} := \big(\phi \times \id\big) \circ \PsiAout\!:\ 
  \piAout^{-1}(U) \mapsto \phi(U) \times \R^c\, \subseteq\, \R^d \times \R^c \,, $
however, the following discussion is independent from this choice.%
\footnote{
    Note that $\piAout^{-1}(U)$ is guaranteed to be trivializable given that $(\phi,U)$ is a chart of $M$.
    This is clear since the coordinate bases $\big[ \frac{\partial}{\partial \phi_\mu} \big]_{\mu=1}^d$ of $(\phi,U)$ yields a trivialization of $\piTM^{-1}(U)$ (see Appendix~\ref{apx:coordinate_bases}) and since the local trivializations of $FM$ and $\A$ were in Section~\ref{sec:bundle_trivializations} induced from those of $TM$.
}
A map between (subsets of) Euclidean spaces is smooth if it is smooth in each component of its image, here in each of the $d+c$ dimensions of $\phi(U)\times\R^c$.
We are therefore interested in the smoothness of the maps
\begin{align}
    F_i: \phi(U) \to \R,\ \ x \mapsto \Big[\, \widetilde{\phi} \circ \fout \circ \phi^{-1} \Big]_i
\end{align}
for any $i = 1,\dots,d+c$.
By writing out $\fout$ and expressing the integral over $\TpM$ by an integral over $\R^d$ as discussed in Appendix~\ref{apx:correspondences_bundle_trivializations}, the $F_i$ are seen to be of the form
\begin{align}\label{eq:integral_smoothness_component}
    F_i(x) = \int_{\R^d} I_i(\mathscr{v},x)\,\ d\mathscr{v} \,.
\end{align}
The coordinate expressions of the integrands $I_i$ are hereby for any $i = 1,\dots,d+c$ given by
\begin{align}\label{eq:integrand_smoothness_full}
    I_i\!: \R^d \mkern-2mu\times\mkern-2mu \phi(U) &\to \R, \\
    (\mathscr{v},x) &\mapsto
    \bigg[ \widetilde{\phi} \circ
    \K\pig(\mkern-1mu \psi_{\protect\scalebox{.65}{$T\!M\mkern-1mu,\mkern2mu$}\protect\scalebox{.75}{$\phi^{\mkern-1mu\shortminus1}\mkern-1mu(x)$}}^{-1} \mkern-2mu(\mathscr{v}) \mkern-1mu\pig) \circ
    \mathcal{P}_{\mkern-4mu\protect\scalebox{.8}{$\!\Ain$},\protect\scalebox{.83}{$\,\phi^{\mkern-1mu\shortminus1}(x)\!\leftarrow\!\exp \circ\mkern1mu \psi_{\protect\scalebox{.7}{$T\!M\mkern-1mu,\mkern1mu$}\protect\scalebox{.9}{$\phi^{\mkern-1mu\shortminus1}(x)$}}\!(\mathscr{v}) $}} \mkern-1mu\circ
    \fin \circ \exp \circ \,
    \psi_{\protect\scalebox{.65}{$T\!M\mkern-1mu,$}\protect\scalebox{.7}{$\phi^{\mkern-1mu\shortminus1}(x)$}}^{-1} \mkern-2mu(\mathscr{v})
    \bigg]_i
    \nonumber \,,
\end{align}
where we assumed, for convenience and without loss of generality, that
$\psi_{\protect\scalebox{.65}{$T\!M\mkern-1mu,\mkern2mu$}\protect\scalebox{.75}{$\phi^{\mkern-1mu\shortminus1}\mkern-1mu(x)$}}$
is an \emph{isometric} gauge of $T_{\phi^{\mkern-1mu\shortminus1}\mkern-1mu(x)}M$, such that the volume scaling factor $\sqrt{|\! \det\eta_p|} = 1$ drops out.
Note that the integrands $I_i$ are composed of smooth maps and are therefore smooth as well.


From the previous discussion it is clear that the smoothness of $\fout$ holds if all $F_i$ are smooth, i.e. infinitely often partially differentiable.
To prove the smoothness of the $F_i$, it is sufficient to show that the partial differentiations and the integration in Eq.~\eqref{eq:integral_smoothness_component} commute -- which is not always the case.
If they do commute, partial derivatives of arbitrary orders $(n_1,\dots,n_d) \in \N^d$ are given by
\begin{align}\label{eq:APX_smoothness_partial_diff_swapping}
    \Big[ \partial_{x_1}^{n_1} \dots \partial_{x_d}^{n_d}\: F_i \Big](x)
    \ =\ \int_{\R^d} \Big[ \partial_{x_1}^{n_1} \dots \partial_{x_d}^{n_d}\: I_i\Big] (\mathscr{v},x)\,\ d\mathscr{v}
\end{align}
where
1) the partial derivatives $\pig[ \partial_{x_1}^{n_1} \dots \partial_{x_d}^{n_d}\: I_i\pig]$ of the integrand exist (due to the smoothness of $I_i$ their existence is guaranteed) and
2) their integral exists.
Whether or not the differentiations commute with the integral can be investigated by making use of the following lemma
from~\cite{forster2012analysis3},%
\footnote{
    Similar versions of this lemma in English language can be found in \cite{klenke2006probability} or at
    \url{https://en.wikipedia.org/wiki/Leibniz_integral_rule\#Measure_theory_statement}.
    In contrast to those versions, the version from~\cite{forster2012analysis3} allows for $T$ being any non-degenerate interval, including closed intervals, which saves us some additional steps below.
}
which is a consequence of the dominated convergence theorem.
\begin{thm}[Differentiation lemma~\cite{forster2012analysis3}]
\label{thm:differentiation_lemma}
    Let $\mathscr{V}$ be a measure space, let $T\subset\R$ be a non-degenerate interval and let $I: \mathscr{V} \times T \to \R$ be a map with the following properties:
    \begin{itemize}
        \item[(i)]  For any fixed $t\in T$ the map $\mathscr{v} \mapsto I(\mathscr{v},t)$ is Lebesgue integrable on $\mathscr{V}$
        \item[(ii)] For any fixed $\mathscr{v} \in \mathscr{V}$ the map $t\mapsto I(\mathscr{v},t)$ is differentiable in $T$
        \item[(iii)] There exists a Lebesgue integrable function $\mathscr{B}: \mathscr{V} \to \R$ such that
                    $\big| \frac{\partial}{\partial t} I(\mathscr{v},t) \big| \leq \mathscr{B}(\mathscr{v})$
                    for any $(\mathscr{v},t) \in \mathscr{V} \times T$
    \end{itemize}
     Then the function $F: T \to \R,\ t \mapsto \int_\mathscr{V} I(\mathscr{v},t)\, d\mathscr{v}$ is differentiable
     with derivative
    \begin{align*}
        \frac{\partial}{\partial t} F(t)\ =\ \int_\mathscr{V} \frac{\partial}{\partial t} I(\mathscr{v},t)\; d\mathscr{v} \,.
    \end{align*}
\end{thm}
The applicability of this lemma (repeatedly for every single partial differentiation) depends on the properties of the integrand, which in turn depends on the specific properties of the kernel field $\K$ and the input feature field $\fin$.
For the case of a kernel field which is compactly supported on balls of fixed radius around the origin of each tangent space, the lemma applies.
Based on this, we give a proof of Theorem~\ref{thm:existence_kernel_field_trafo_compact_kernels} in the remainder of this appendix.










\toclesslab\subsection{Proof of Theorem~\ref{thm:existence_kernel_field_trafo_compact_kernels} (Sufficiency of compact kernel support on tangent balls)}{apx:proof_sufficiency_ball_kernel_support}

Denote by
$B_{\TpM}^{\mkern1.5mu\textup{closed}}(0,R) := \big\{ v\in\TpM \,\big|\, \lVert v\rVert \leq R \big\}$
the closed ball of radius $R>0$ around the origin of $\TpM$ and by
$B_{\R^d}^{\mkern1.5mu\textup{closed}}(0,R) := \big\{ v\in\R^d \,\big|\, \lVert v\rVert \leq R \big\}$
the corresponding ball around the origin of $\R^d$.
Note that any isometric gauge satisfies $\psiTMp\big(B_{\TpM}^{\mkern1.5mu\textup{closed}}(0,R)\big) = B_{\R^d}^{\mkern1.5mu\textup{closed}}(0,R)$.
Let $\widetilde{\K}$ be a kernel field whose support falls within balls of the same radius $R$ in each tangent space, i.e. which satisfies
\begin{align}
    \supp\!\big(\widetilde{\K}_p\big) \subseteq B_{\TpM}^{\mkern1.5mu\textup{closed}}(0,R) \quad \forall p\in M
\end{align}
and thus, for any isometric gauge $\psiTMp$:
\begin{align}
    \supp\! \pig(\widetilde{\K}_p \circ \big(\psiTMp\big)^{-1} \pig) \subseteq B_{\R^d}^{\mkern1.5mu\textup{closed}}(0,R) \quad \forall p\in M
\end{align}
According to Theorem~\ref{thm:existence_kernel_field_trafo_compact_kernels} this property is sufficient to guarantee that the corresponding kernel field transform $\mathscr{T}_{\mathcal{K}_{\mkern-1muR}}$ is well defined.
A proof of this statement is given in the following.

\begin{proof}
    As already stated in the beginning of this appendix, the existence of the integral is guaranteed given that the kernel supports are compact:
    The compactness of the kernels carries over to the integrands of the kernel field transform.
    Their smoothness further implies their continuity and integrals of compactly supported continuous functions always exists.

    To prove the smoothness of the resulting output feature field $\fout$, we proceed with the discussion earlier in this section.
    We aim to apply the differentiation lemma~\ref{thm:differentiation_lemma} to swap partial derivatives $\frac{\partial}{\partial x_\mu}$ for any $\mu=1,\dots,d$ in Eq.~\eqref{eq:APX_smoothness_partial_diff_swapping} at any $x_0 \in \phi(U)$ with the integration over $\R^d$.
    For this purpose, we introduce the auxiliary functions
    \begin{align}
        I_{i,x_0,\mu}: \R^d \times [-\varepsilon,\varepsilon] \to \R,\ \ (\mathscr{v},t) \mapsto I(\mathscr{v}, x_0 + t\epsilon_\mu)
    \end{align}
    and
    \begin{align}
        F_{i,x_0,\mu}: [-\varepsilon,\varepsilon] \to \R,\ \ t \mapsto F(x_0 + t\epsilon_\mu) = \int_{\R^d} I_{i,x_0,\mu}(\mathscr{v},t)\ d\mathscr{v} \,,
    \end{align}
    where $\epsilon_\mu \in \R^d$ is the unit vector in $\mu$-direction and
    $\varepsilon > 0$ is chosen such that $\big\{ x_0 + t\epsilon_\mu \,\big|\, t\in [-\varepsilon,\varepsilon] \big\} \subset \phi(U)$, which is always possible since $\phi(U)$ is open.
    Then $I_{i,x_0,\mu}$ is with the identifications $\mathscr{V}=\R^d$ and $T=[-\varepsilon,\varepsilon]$ of the form required by lemma~\ref{thm:differentiation_lemma}.
    It satisfies property $(i)$ by the assumption that the kernel field transform exists as discussed earlier.
    Property $(ii)$ holds due to the smoothness of the full integrand in Eq.~\eqref{eq:integrand_smoothness_full}.
    For property $(iii)$, observe that both $I_{i,x_0,\mu}$ and its derivative are smooth such that the absolute value $\big| \frac{\partial}{\partial t} I_{i,x_0,\mu} \big|$ is continuous.
    Since it is in addition compactly supported on $B_{\R^d}^{\mkern1.5mu\textup{closed}}(0,R) \times [-\varepsilon,\varepsilon]$, it is by (a generalization of) the extreme value theorem bounded by some number $b\geq0$.
    We therefore set $\mathscr{B}(\mathscr{v}) = b\cdot \mathbb{I}_{B_{\R^d}^{\mkern1.5mu\textup{closed}}(0,R) \times [-\varepsilon,\varepsilon]}$ where $\mathbb{I}$ is the indicator function.
    This choice satisfies 
    $\big| \frac{\partial}{\partial t} I(\mathscr{v},t) \big| \leq \mathscr{B}(\mathscr{v})$
    for any $(\mathscr{v},t) \in \mathscr{V} \times T$
    and is integrable such that property $(iii)$ is fulfilled as well.
    We can therefore swap the order of differentiation and integration for arbitrary choices of $x_0$ and $\mu$, which we use to pull arbitrary partial derivatives into the integral:
    \begin{align}
        \bigg[ \frac{\partial}{\partial x_\mu} F_i \bigg](x_0)
        \,=\, \bigg[ \frac{\partial}{\partial t} F_{i,x_0,\mu} \bigg](0)
        \,=\, \int_{\R^d} \frac{\partial}{\partial t} I_{i,x_0,\mu}(\mathscr{v},t) \Big|_{t=0}\ d\mathscr{v}
        \,=\, \int_{\R^d} \frac{\partial}{\partial x_\mu} I_i(\mathscr{v},x) \Big|_{x=x_0}\ d\mathscr{v}
    \end{align}

    Due to the smoothness and compact support of the integrand $I_i$, its partial derivatives $\frac{\partial}{\partial x_\mu} I$ are smooth and compactly supported as well.
    They do therefore satisfy properties $(i)$, $(ii)$ and $(iii)$ as well (with a potentially adapted bound $b$).
    It is thus possible to repeat the partial differentiation of $F_i$ infinitely often, which proves its smoothness.
    Since the derivations were independent from the particular choices for the point $p\in M$, charts $(U,\phi)$ and $(\widetilde{U},\widetilde{\phi})$, points $x_0\in\phi(U)$ and indices $i$ and $\mu$, this result proves the smoothness of the whole output feature field $\fout = \TK(\fin)$.

\NoEndMark
$~\hfill\Box$
\end{proof}
