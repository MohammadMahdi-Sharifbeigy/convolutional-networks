%!TEX root=../GaugeCNNTheory.tex


\section{Equivariant convolutions on homogeneous spaces}
\label{apx:homogeneous_conv}

The works by \citet{Kondor2018-GENERAL}, \citet{Cohen2018-intertwiners}\cite{Cohen2019-generaltheory} and \citet{bekkers2020bspline} are in spirit quite similar to ours in that they are defining group equivariant convolutions in a fairly general setting.
These papers have in common that they operate on \emph{feature maps on homogeneous spaces}~$\I/H$ of a \emph{global symmetry group}~$\I$, where $H\leq\I$.%
\footnote{
    We use $\I$ here to denote arbitrary global symmetries, not necessarily isometries.
}%
\footnote{
    \cite{Kondor2018-GENERAL,Cohen2019-generaltheory,bekkers2020bspline} use $G$ instead of $\I$ to refer to global symmetries.
    We use $\I$ since we reserve $G$ for the structure group.
}
They differ in the types of groups $\I$ which they cover and in the definition of their feature spaces, specifically the linear group actions on them.
The main theorems of the papers assert that \emph{the most general equivariant linear maps between such feature spaces are convolutions} (or correlations) with \emph{symmetry constrained kernels}.
The specific details on these generalized convolutions depend on the particular feature spaces and group actions which the models consider.

This appendix examines these theories and their relation to our coordinate independent convolutions.
The most important similarities and differences are summarized in the following list:
\begin{itemize}
    \item[{\rule[2.2pt]{2pt}{2pt}}]
        Not any \emph{homogeneous space} is a \emph{Riemannian manifold} and not any Riemannian manifold is a homogeneous space of its isometry group.%
        \footnote{
            For instance, $\I/H = \O2/\SO2 \cong \{\pm1\}$ is a set but not a Riemannian manifold.
            Another example are $(\Z^d,+)$ group convolutions on the discrete pixel grid~$\Z^d$.
        }
        There is, however, a significant overlap, for instance for Euclidean CNNs on $\Euc_d \cong \E{d}/\O{d}$ or spherical CNNs on $S^d \cong \O{d+1}/\O{d}$.
    \item[{\rule[2.2pt]{2pt}{2pt}}]
        The authors consider \emph{compact}~\cite{Kondor2018-GENERAL}, \emph{locally compact, unimodular}~\cite{Cohen2018-intertwiners}\cite{Cohen2019-generaltheory}, and \emph{Lie} groups~\cite{bekkers2020bspline}, respectively.
        The global symmetry groups in our theory are \emph{isometries} of~$M$ or, specifically for Euclidean spaces, \emph{affine} groups~$\Aff(G)$.
        Note that affine groups are not compact and only for $G\leq\O{d}$ unimodular -- general affine groups are therefore not covered in the respective theories.
    \item[{\rule[2.2pt]{2pt}{2pt}}]
        Coordinate independent CNNs shift the focus from \emph{global} to \emph{local symmetries}.
        On homogeneous spaces $\I/H$ these local symmetries correspond to the stabilizer subgroups $\Stab{p} \cong H$ of~$\I$.
        Our Section~\ref{sec:isometry_intro} works out the relations between global and local symmetries in detail --
        the models' local equivariance induces their global equivariance.
    \item[{\rule[2.2pt]{2pt}{2pt}}]
        The models assume different \emph{types} of feature fields and \emph{group actions} on them:
        \citet{Kondor2018-GENERAL} and \citet{bekkers2020bspline} assume scalar fields on homogeneous spaces, i.e. real-valued functions ${f: \I/H \to \R}$ which transform according to
        $\phi.f (\zeta.H) = f\big( \phi^{-1} \zeta.H \big)$.%
        \footnote{
            Multi-channel feature maps are constructed by stacking multiple such functions.
            In contrast to the case of feature fields, the individual channels of such feature maps transform independently from each other.
        }
        \citet{Cohen2018-intertwiners}\cite{Cohen2019-generaltheory} consider feature fields of more general types $\rho$
        which are defined as sections of $H$-associated feature vector bundles.
        Their transformation laws are given by induced representations $\Ind_H^\I \rho$.
        This setting covers the real-valued functions from \cite{Kondor2018-GENERAL,bekkers2020bspline} as a special case when choosing trivial field representations
        (or, as made precise below, more general quotient representations $\rho_\textup{quot}^{G/H}$ where $H\leq G\leq\I$).
        Our theory models feature fields as sections of associated bundles as well.
        Their transformation is given by pushforwards $\phi \rhd f := \dphiA \circ f \circ \phi^{-1}$, which generalize induced representations.
    \item[{\rule[2.2pt]{2pt}{2pt}}]
        The works by \citet{Kondor2018-GENERAL}, \citet{Cohen2018-intertwiners}\cite{Cohen2019-generaltheory} and \citet{bekkers2020bspline} derive \emph{convolutional weight sharing} from the requirement on the models to be globally equivariant.
        Our $\GM$-convolutions, on the other hand, share weights by definition over the $G$-structure.
        We adopted the idea of deriving weight sharing from global symmetries (isometries) in Section~\ref{sec:quotient_kernel_fields}.
        The requirement for isometry equivariance implies weight sharing over the isometry orbits and a stabilizer constraint on the kernels; see e.g. Fig.~\ref{fig:isom_invariant_kernel_field_quotient}.
        Theorem~\ref{thm:GM_conv_homogeneous_equivalence} asserts that \emph{isometry equivariant kernel field transforms on homogeneous spaces are $\GM$-convolutions} -- this result mirrors those of \citet{Kondor2018-GENERAL}, \citet{Cohen2018-intertwiners}\cite{Cohen2019-generaltheory} and \citet{bekkers2020bspline}
        closely.
    \item[{\rule[2.2pt]{2pt}{2pt}}]
        All of the theories derive some \emph{linear symmetry constraint on the kernel spaces}.
        In the case of \citet{Kondor2018-GENERAL} and \citet{bekkers2020bspline}, the kernels are essentially scalar functions on double quotient spaces $\Hout\backslash \I/\Hin$ (assuming correlations, for convolutions $\Hin$ and $\Hout$ are swapped; see below).
        The kernels of \citet{Cohen2018-intertwiners}\cite{Cohen2019-generaltheory} and in our theory are satisfying a steerability constraint which depends on the particular choice of field types $\rhoin$ and~$\rhoout$.
        Note that the determinant factor is missing in the $G$-steerability constraint of \citet{Cohen2018-intertwiners}\cite{Cohen2019-generaltheory} since the authors restrict to unimodular groups.
        The factor does appear in the kernel constraint by \citet{bekkers2020bspline}.
    \item[{\rule[2.2pt]{2pt}{2pt}}]
        While \citet{Kondor2018-GENERAL} and \citet{Cohen2018-intertwiners}\cite{Cohen2019-generaltheory} describe kernels immediately on the group or homogeneous space,
        \citet{bekkers2020bspline} and our $\GM$-convolutions define kernels on the tangent spaces and project them subsequently via the exponential map.
        These approaches are in general inequivalent, for instance since the exponential map is on a non-connected manifold non-injective.
        On Euclidean spaces both approaches are obviously equivalent since the exponential map becomes trivial; see Section~\ref{sec:euclidean_affine_equiv}.
        Our Theorem~\ref{thm:spherical_kernel_space_iso} in Section~\ref{sec:spherical_CNNs_fully_equivariant} bridges this gap furthermore for spherical kernels by providing an isomorphism between kernels of the two approaches.
        In practice, the general incompatibility is irrelevant since kernels of convolutional networks are usually compactly supported within the injectivity radius of the exponential map.
\end{itemize}

We will in the following elaborate on the theories of
\citet{Kondor2018-GENERAL}, \citet{bekkers2020bspline} and \citet{Cohen2018-intertwiners}\cite{Cohen2019-generaltheory}
in more detail.
As a preparation, we will first discuss homogeneous spaces, group convolutions and group correlations.
For an alternative review of the topic we refer the reader to~\citet{esteves2020theoretical}.
We furthermore want to point to the work by \citet{chakraborty2018H-CNNs}, which also defines convolutions on homogeneous spaces.
It is not covered in more detail in this section since their models assume $\Hout=\{e\}$, that is, their convolution kernels are unconstrained and always lift the input signal to a scalar field on~$\I$.






\toclesslab\subsection{General remarks on homogeneous spaces, group convolutions and group correlations}{apx:homogeneous_preliminaries}

\paragraph{Homogeneous spaces:}
Let $\I$ be some group which acts on some space~$X$.
The space is said to be \emph{homogeneous} if the group action is \emph{transitive}, i.e. if any two points $p,q\in X$ are related by the $\I$-action.
In equations, $X$ is homogeneous if and only if for any $p,q\in X$ there exists an element $\phi\in\I$ such that~$q=\phi(p)$.
Note that the action on~$X$ is not required to be fixed point free, that is, each point $p\in X$ has a potentially non-trivial stabilizer subgroup $\Stab{p} = {\{\xi\in\I \,|\, \xi(p)=p \}} \leq\I$.
It can be shown that the homogeneous space can be identified with the quotient space $\I/H$ where $H = \Stab{p}$ for some $p\in X$.%
\footnote{
    Other choices of points yield other realizations of the non-canonical isomorphism $\I/H \cong X$.
    Any choice is equally valid since $\Stab{p} \cong \Stab{q}$ for homogeneous spaces.
}

Since any homogeneous space arises as a quotient, we consider in the following always some subgroup $H$ of~$\I$.
This subgroup has \emph{left cosets}, i.e. subsets of the form
\begin{align}
    \phi.H\ =\ \big\{ \phi h \,\big|\, h\in H \big\}
\end{align}
which are elements of the (homogeneous) quotient space
\begin{align}
    \I/H\ =\ \big\{ \phi.H \,\big|\, \phi\in \I \big\} \,.
\end{align}
A natural left action of $\I$ on $\I/H$ is given by
\begin{align}
    \I \times \I/H \to \I/H,\ \ \ \big(\widetilde{\phi},\, \phi.H\big) \mapsto \widetilde{\phi}\phi.H \,.
\end{align}
This action is easily seen to be transitive, making $\I/H$ a homogeneous space of~$\I$.
The canonical quotient map
\begin{align}
    \mathscr{q}^{\I}_{\I/H}:\ \I\to \I/H,\ \ \phi \mapsto \phi.H
\end{align}
turns $\I$ into a principal $H$-bundle over~$\I/H$.
Analogous definitions can be made for \emph{right cosets}
\begin{align}
    H.\phi\ \ &\in\ \ H\backslash\I \,.
\intertext{and \emph{double cosets}}
    \widetilde{H}.\phi.H\ \ &\in\ \ \widetilde{H} \backslash\I/H
\end{align}
and their respective quotient spaces.


An universal property of the quotient maps $\mathscr{q}^{\I}_{\I/H}$, which will become important in our discussion below, is the following.
Let $f^\uparrow: \I \to \R$ be a continuous, right $H$-invariant function, i.e. a function which satisfies $f^\uparrow(\phi h) = f^\uparrow(\phi)$ for any $\phi\in\I$ and~$h\in H$.
Then there exists a unique continuous function $f: \I/H \to \R$ such that $f^\uparrow = f \circ \mathscr{q}^{\I}_{\I/H}$.
Conversely, one may lift any continuous map $f: \I/H \to \R$ uniquely to a right $H$-invariant map $f^\uparrow: \I \to \R$, which is used by \citet{Kondor2018-GENERAL} to generalize group convolutions to homogeneous spaces.
The relation between both functions is visualized in the following commutative diagram:
\begin{equation}
\label{cd:left_cosets_lift}
\begin{tikzcd}[column sep=55pt, row sep=35pt, font=\normalsize]
    \I
        \arrow[rd, "\!f^\uparrow := f\circ \mathscr{q}^{\I}_{\overset{}{\I/H}}"]
        \arrow[d, "\mathscr{q}^{\I}_{\overset{}{\I/H}}\,"']
    \\
    \I/H
        \arrow[r, "f"']
    & \R
\end{tikzcd}
\end{equation}
An analogous construction can obviously be made for right quotient spaces $H\backslash\I$ and left $H$-invariant maps.
The following commutative diagram visualizes the case of double quotient spaces $\widetilde{H} \backslash \I/H$ and maps $f^\uparrow$ which are simultaneously left $\widetilde{H}$-invariant and right $H$-invariant, i.e. which satisfy ${f^\uparrow\big(\widetilde{h}\phi h\big) = f^\uparrow(\phi)}$ for any ${\phi\in\I}$, \ ${\widetilde{h}\in\widetilde{H}}$, and ${h\in H}$:
\begin{equation}
\label{cd:double_cosets_lift}
\begin{tikzcd}[column sep=55pt, row sep=35pt, font=\normalsize]
    \I
        \arrow[rd, "\!f^\uparrow := f\circ \mathscr{q}^{\I}_{\overset{}{\widetilde{H} \backslash \I/H}}"]
        \arrow[d, "\mathscr{q}^{\I}_{\overset{}{\widetilde{H} \backslash \I/H}}\,"']
    \\
    \widetilde{H} \backslash \I/H
        \arrow[r, "f"']
    & \R
\end{tikzcd}
\end{equation}




\paragraph{Group convolutions and group correlations:}
Convolutions are naturally generalized from Euclidean spaces (or translation groups) to arbitrary locally compact groups.
Let $\I$ be a locally compact group and let $d\zeta$ be a left Haar measure on~$\I$.
The \emph{group convolution} $(f \ast_{\overset{}{\protect\scalebox{.64}{$\mkern-.5mu \I$}}} \kappa): \I \to \R$ of two integrable functions $f:\I\to\R$ and $\kappa:\I\to\R$ is then defined by the following equivalent expressions, taken from \cite{gallier2019harmonicRepr}:
\begin{align}\label{eq:group_conv_def}
    \big(f \ast_{\overset{}{\protect\scalebox{.64}{$\mkern-.5mu \I$}}} \kappa \big)(\phi)
    \ :&=\ \int_{\I} f(\zeta)\, \kappa\big( \zeta^{-1} \phi \big) \; d\zeta \notag \\
    \ &=\ \int_{\I} f(\phi\, \zeta)\, \kappa\big( \zeta^{-1}\big) \; d\zeta \notag \\
    \ &=\ \int_{\I} f\big(\zeta^{-1} \big)\, \kappa( \zeta\phi)\: \Delta(\zeta^{-1}) \; d\zeta \notag \\
    \ &=\ \int_{\I} f\big(\phi\, \zeta^{-1} \big)\, \kappa( \zeta)\: \Delta(\zeta^{-1}) \; d\zeta \,,
\end{align}
The group homomorphism $\Delta: \I \to (\R^+_{>0},*)$, appearing in the last two expressions, is the modular function of~$\I$.
\citet{Kondor2018-GENERAL} define group convolutions as in the last line, however, without the modular function.
This is valid since the authors assume compact groups, which are unimodular, i.e. satisfy $\Delta(\phi) = 1$ for any $\phi \in\I$.


Closely related to group convolutions are \emph{group correlations}
\begin{align}\label{eq:group_corr_def}
    \big(f \star_{\overset{}{\protect\scalebox{.64}{$\mkern-.5mu \I$}}} \kappa \big)(\phi)
    \ :=\ \big\langle f,\, \phi.\kappa \big\rangle_{L^1(\I)}
    \ =\ \int_{\I} f(\zeta)\, \kappa\big( \phi^{-1} \zeta \big) \; d\zeta\ ,
\end{align}
which are defined as the inner product of a function~$f$ with a shifted kernel~$\phi.\kappa$.
A comparison with Eq.~\eqref{eq:group_conv_def} reveals that group convolutions and group correlations are equivalent up to an inversion of the kernel argument, that is,
\begin{align}\label{eq:group_conv_corr_kernel_inversion}
    \big(f \star_{\overset{}{\protect\scalebox{.64}{$\mkern-.5mu \I$}}} \kappa \big)
    \ =\ \big(f \ast_{\overset{}{\protect\scalebox{.64}{$\mkern-.5mu \I$}}} \big[\kappa \circ (\,\cdot\,)^{-1} \big] \big) \,.
\end{align}
While \citet{Kondor2018-GENERAL} consider (generalized) group convolutions, \citet{bekkers2020bspline} and \citet{Cohen2019-generaltheory} assume correlations -- to reconcile the theories one has to invert the kernel arguments.


Group convolutions and group correlations are by definition \emph{equivariant} w.r.t. left actions $\alpha.f(\phi) = f\big( \alpha^{-1}\phi \big)$ of group elements $\alpha\in\I$ on the first factor.
For the case of convolutions, this is shown by
\begin{align}\label{eq:group_conv_equivariance}
    \big( [\alpha.f] \ast_{\overset{}{\protect\scalebox{.64}{$\mkern-.5mu \I$}}} \kappa \big)(\phi)
    \ &=\ \int_{\I} \big[\alpha.f \big](\zeta)\, \kappa\big( \zeta^{-1} \phi \big) \; d\zeta \notag \\
    \ &=\ \int_{\I} f\big( \alpha^{-1} \zeta\big)\, \kappa\big( \zeta^{-1} \phi \big) \; d\zeta \notag \\
    \ &=\ \int_{\I} f\big(\widetilde{\zeta} \big)\, \kappa\big( \widetilde{\zeta}^{-1} \alpha^{-1} \phi \big) \; d\big( \alpha \widetilde{\zeta} \big) \notag \\
    \ &=\ \big( f \ast_{\overset{}{\protect\scalebox{.64}{$\mkern-.5mu \I$}}} \kappa \big) (\alpha^{-1} \phi) \notag \\
    \ &=\ \big[\alpha.( f \ast_{\overset{}{\protect\scalebox{.64}{$\mkern-.5mu \I$}}} \kappa )\big] (\phi) \,,
\end{align}
where we substituted $\widetilde{\zeta} = \alpha^{-1}\zeta$ in the third step and made use of the fact that $d\widetilde{\zeta}$ is a left Haar measure, i.e. satisfies $d\big(\alpha \widetilde{\zeta}\big) = d\widetilde{\zeta}$.
The case of correlations follows trivially by Eq.~\eqref{eq:group_conv_corr_kernel_inversion}.


The majority of equivariant CNNs rely on group convolutions or group correlations.
In particular, the models in rows~(1-3), (5), (11), (15), (19), (21), (24), (25) and (32) of Table~\ref{tab:network_instantiations}, all of which are (or could equivalently be) labeled by regular representations, are group convolutional CNNs.
Prior to their use in equivariant CNNs, group convolutions have been widely applied in robotics~\cite{chirikjian1998numerical} or for image analysis~\cite{mallat2012group,sifre2012combined,Sifre2013-GSCAT,bruna2013invariant,sifre2014rigid,oyallon2015scattering}.
\citet{Cohen2016-GCNN} showed that group convolutions (or rather correlations) naturally generalize conventional CNNs.
Since the feature maps of convolutional networks comprise multiple channels, they are not given by real-valued functions on $\I$ but by vector-valued functions $f: \I \to \R^c$.
Kernels are accordingly defined to be (unconstrained) matrix-valued functions on the group, i.e. $\kappa: \I \to \R^{\cout\times\cin}$.
The works of \citet{Kondor2018-GENERAL}, \citet{bekkers2020bspline} and \citet{Cohen2018-intertwiners}\cite{Cohen2019-generaltheory}, which we review in the following, generalize such group convolutional networks to arbitrary homogeneous spaces.







\toclesslab\subsection{Scalar field convolutions on homogeneous spaces}{apx:homogeneous_scalar_field_convs}

We start with the $\I$-equivariant convolutional (or correlational) networks on homogeneous spaces by \citet{Kondor2018-GENERAL} and \citet{bekkers2020bspline}.
Both theories define feature maps as \emph{scalar fields on homogeneous spaces}, that is, each channel is given by a real-valued function
\begin{align}\label{eq:scalar_field_homogeneous_space}
    f: \I/H \to \R \,.
\end{align}
Individual channels transform independently under the action of the global symmetry group $\I$ as specified by
\begin{align}\label{eq:group_action_homogeneous_space_scalar_field}
    \big[\mkern1mu \widetilde{\phi}.f\big] (\phi.H)\ :=\ f\big( \widetilde{\phi}^{-1} \phi.H \big)
    \qquad\ \widetilde{\phi}\in H,\ \ \ \phi.H\in \I/H \,.
\end{align}

Each layer $l = 1,\dots, L$ may be assigned a different subgroup $H_l \leq\I$ and thus homogeneous space $\I/H_l$ on which its feature maps live.
This allows for instance to model lifting convolutions from the sphere $S^2 \cong \SO3/\SO2$ to the group $\SO3 \cong \SO{3}/\{e\}$ when choosing subgroups $\SO2$ and $\{e\}$, respectively.
The choices of subgroups correspond in some sense to the choices of group representations in our theory, which we will explain further below.

The results of the two papers are to large parts equivalent, however,
\citet{Kondor2018-GENERAL} consider compact groups $\I$ and convolutions
while \citet{bekkers2020bspline} assume $\I$ to be a Lie group and use correlations.



\paragraph{\citet{Kondor2018-GENERAL} :}

In a nutshell, \citet{Kondor2018-GENERAL} investigate the most general $\I$-equivariant linear maps between scalar field features on homogeneous spaces $\I/\Hin$ and $\I/\Hout$, assuming the transformation law in Eq.~\eqref{eq:group_action_homogeneous_space_scalar_field}.
They prove that this operation is given by a generalized group convolution with a kernel
\begin{align}
    \kappa: \Hin\backslash \I/\Hout \to \R
\end{align}
on the double quotient space specified by $\Hin$ and $\Hout$.
Formulated for finite groups, as done by the authors, this generalized convolution operation is shown to be given by
\begin{align}\label{eq:quotient_space_conv_Kondor}
    \big(f \ast_{\overset{}{\scalebox{.64}{$\I/\Hin$}}} \kappa \big) (\phi.\Hout)
    \ \ :=\ \ |\Hin| \sum_{\Hin.\zeta \,\in\, \Hin\mkern-2.5mu\backslash\mkern-.5mu\I}
        f\big( \phi\mkern2mu \zeta^{-1}\!. \Hin\big)\ \kappa\big( \Hin.\zeta.\Hout \big) \,.
\end{align}
A comparison with the last line of Eq.~\eqref{eq:group_conv_def} suggests that this operation is indeed closely related to group convolutions -- the modular function $\Delta$ drops out since $\I$ is compact and therefore unimodular.
The generalized convolution is in fact equivalent to a group convolution
\begin{align}
    \big(f \ast_{\overset{}{\scalebox{.64}{$\I/\Hin$}}} \kappa \big) (\phi.\Hout)
    \ =\ \big(f^\uparrow \ast_{\overset{}{\protect\scalebox{.64}{$\mkern-.5mu \I$}}} \kappa^\uparrow \big)(\phi)
\end{align}
with features and kernels that are lifted according to the diagrams in Eqs.~\eqref{cd:left_cosets_lift} and~\eqref{cd:double_cosets_lift}.
Note that the convolution kernel on $\Hin\backslash \I/\Hout$ corresponds to a correlation kernel on $\Hout\backslash \I/\Hin$ since convolutions and correlations are according to Eq.~\eqref{eq:group_conv_corr_kernel_inversion} related by an inversion of the kernel argument.
One could therefore view the kernels by \citet{Kondor2018-GENERAL} as left $\Hout$-invariant correlation kernels on the input space~$\I/\Hin$.


To give an intuition on these results, we come back to our spherical CNN example from above.
Let therefore $\I=\SO3$, $\Hin=\SO2$ and, for now, $\Hout=\{e\}$.
This setting describes lifting convolutions from the 2-sphere $\I/\Hin = \SO3/\SO2 \cong S^2$ to the rotation group manifold $\I/\Hout = \SO3/\{e\} \cong \SO3$.
Considering correlations instead of convolutions, the kernels are real-valued functions on $\Hout\backslash \I/\Hin = \{e\}\backslash \SO3/\SO2 \cong S^2$.
If we let instead $\Hout=\SO2$, the convolution maps from scalar fields on the 2-sphere to scalar fields on the 2-sphere $\I/\Hout = \SO3/\SO2$.
In this case the correlation kernels are given by real-valued functions on $\SO2\backslash \SO3/\SO2$.
Equivalently, the correlation kernels are given by left $\SO2$-invariant functions on~$S^2$, i.e. zonal kernels as visualized in Fig.~\ref{fig:zonal_kernel}.
When assuming $\Hin=\Hout=\{e\}$, one has $\I/\Hin = \I/\Hout \cong \SO3$ and unconstrained kernels on $\Hout\backslash \I/\Hin \cong \SO3$, corresponding to conventional group convolutions (or correlations).
These results are in line with our discussion in Section~\ref{sec:spherical_CNNs_fully_equivariant}.


For completeness, we mention that \citet{Kondor2018-GENERAL} explain their results additionally from a representation theoretic perspective, i.e. with features and kernels in Fourier space.
The fact that features and kernels live on quotient spaces is in this formulation reflected in sparsity patterns of the Fourier coefficients.








\paragraph{\citet{bekkers2020bspline} :}

Instead of considering compact groups, \citet{bekkers2020bspline} assumes $\I$ to be a general Lie group.
The feature maps of layer~$l$ are defined as real-valued square integrable functions in $L^2(\I/H_l)$
which transform according to Eq.~\eqref{eq:group_action_homogeneous_space_scalar_field} when being acted on by~$\I$.

\citet{bekkers2020bspline} models the layers of his convolutional (or rather correlational) networks as linear bounded operators
\begin{align}
    \mathfrak{K}:\ L^2(\I/\Hin) \to L^2(\I/\Hout)
\end{align}
between feature maps on homogeneous spaces $\I/\Hin$ and~$\I/\Hout$.
Such operators are in general given by integral operators of the form
\begin{align}
    \big[\mathfrak{K}f\big] (\phi.\Hout)\ =\ 
    \int_{\I/\Hin} \widehat{\kappa} \big(\phi.\Hout,\, \zeta.\Hin \big)\ f(\zeta.\Hin)\,\ \dmuIHin \,,
\end{align}
where $\dmuIHin$ is some Radon measure on $\I/\Hin$ and
\begin{align}
    \widehat{\kappa}:\ \I/\Hout \times \I/\Hin \to \R
\end{align}
is an integrable 2-argument kernel.

The requirement on the operator to be equivariant, that is,
\begin{align}
    \mathfrak{K}\big( \phi.f \big)\ =\ \phi \mkern1mu.\mkern1.5mu \mathfrak{K}(f)
    \qquad \forall\,\ \phi\in\I,\ \ f\in L^2(\I/\Hin) \ ,
\end{align}
is shown to imply that the 2-argument kernel reduces to a single argument kernel
\begin{align}
    \widehat{\kappa} \big(\phi.\Hout,\, \zeta.\Hin \big)
    \ =\ \frac{\dmuIHin(\phi^{-1} \zeta.\Hin)}{\dmuIHin(\zeta.\Hin)}\ \kappa\big(\phi^{-1} \zeta.\Hin \big) \,.
\end{align}
The group element $\phi\in \phi.\Hout \subset \I$ is hereby an arbitrary representative of the coset in which it is contained.
This 1-argument kernel is -- up to a measure dependent scale factor -- constrained to be left $\Hout$-invariant:
\begin{align}
    \kappa(\zeta.\Hin)\ =\ \frac{\dmuIHin(\xi^{-1} \zeta.\Hin)}{\dmuIHin(\zeta.\Hin)}\ \kappa\big(\xi^{-1} \zeta.\Hin \big)
    \qquad \forall\ \ \zeta.\Hin \in \I/\Hin,\ \ \xi\in\Hout
\end{align}
Note that this result is very similar to that of \citet{Kondor2018-GENERAL} since a left $\Hout$-invariant kernel on $\I/\Hin$ is equivalent to an element of $\Hout\backslash \I/\Hin$ (again assuming correlation kernels instead of convolution kernels).
The main difference is the additional scale factor, which appears since the Radon measure $\dmuIHin$ is not necessarily left $\I$-invariant.


One of the practically relevant cases is that of group correlations, for which $\Hin = \{e\}$ and $\I/\{e\} = \I$.
In this case $d\mu_{\overset{}{\I}}$ is a left (invariant) Haar measure on $\I$, such that the scale factor drops out.
A second relevant case is that of affine equivariant convolutions on Euclidean spaces, i.e. the choices $\I = \Aff(G)$ and $\Hin=G$, for which $\I/\Hin \cong \R^d$.
Assuming $\dmuIHin$ to be the Lebesgue measure on $\R^d$ and denoting $\phi=tg \in \I$, \citet{bekkers2020bspline} prove that the scale factor is in this case given by:
\begin{align}
    \frac{\dmuIHin\!( (tg)^{-1}x )}{\dmuIHin\!(x)}\ =\ \frac{1}{\detg} \qquad \forall\ x\in\R^d
\end{align}
This is exactly the determinant factor which appears in our $G$-steerability kernel constraint, Eq.~\eqref{eq:kernel_constraint}, as well.


Since $\SO3$ is a Lie group, the spherical CNN examples that we gave after discussing the theory by \citet{Kondor2018-GENERAL} apply without changes (assuming the standard left-invariant measure on $S^2$).


\citet{bekkers2020bspline} defines kernels in close analogy to our $\GM$-convolutions on the tangent spaces and projects them via exponential maps to the homogeneous spaces.
The kernels on the tangent spaces are hereby modeled via B-splines.
A difference is that \citet{bekkers2020bspline} does not need to consider parallel transporters since he is assuming scalar feature maps on the homogeneous spaces.





\paragraph{Relation to \emph{GM}-convolutions:}

Due to the quite different formulation it is not immediately obvious how the results of \citet{Kondor2018-GENERAL} and \citet{bekkers2020bspline} relate to our theory.
Instead of considering different quotient spaces $\I/H_l$ in each layer~$l$, we consider a fixed manifold~$M$.
To see how both approaches connect, assume another subgroup~$G$ to be given such that $H_l \leq G \leq \I$ for all layers~$l=1,\dots,L$ and satisfying that $M := \I/G$ is a manifold.
The scalar features on $\I/H_l$ can in this case be viewed as $G$-associated feature fields on $M$ which transform according to \emph{quotient representations}~$\rho_\textup{quot}^{G/H_l}$.
To see this, note that the group action in Eq.~\eqref{eq:group_action_homogeneous_space_scalar_field} is nothing but the induced representation $\Ind_{H_l}^{\I} \rho_\textup{triv}^{H_l} = \rho_\textup{quot}^{\I/H_l}$ from the trivial representation of $H_l$, which describes the transformation law of scalar fields on~$\I/H_l$.
This representation can via induction in stages (see~\cite{ceccherini2009induced}) be decomposed into
\begin{align}
    \Ind_{H_l}^{\I} \rho_\textup{triv}^{H_l}
    \ =\ \Ind_G^{\I} \Ind_{H_l}^G \rho_\textup{triv}^{H_l}
    \ =\ \Ind_G^{\I} \rho_\textup{quot}^{G/H_l} \,,
\end{align}
that is, into the induction of the quotient representation $\rho_\textup{quot}^{G/H_l}$ from~$G$ to~$\I$.
The real-valued functions on $\I/H_l$ are therefore equivalent to $\rho_\textup{quot}^{G/H_l}$-fields on~$M=\I/G$.


Interesting special cases are $G=H_l$ and $G=\{e\}$.
For the former one has $\rho_\textup{quot}^{G/H_l} = \rho_\textup{triv}^G$, describing scalar fields on~$M = \I/G = \I/H_l$.
For the latter, $\rho_\textup{quot}^{G/H_l} = \rho_\textup{reg}^G$ is the regular representation, corresponding to conventional group convolutions.


These insights imply that the theory of \citet{Kondor2018-GENERAL} explains all models in Table~\ref{tab:network_instantiations} which operate on homogeneous spaces of compact groups $\I$ and are labeled by either trivial, regular or more general quotient representations -- these are essentially the spherical CNNs in rows~(32) and~(33).
A minor generalization of the theory to locally compact, unimodular groups would additionally describe some of the isometry equivariant Euclidean CNNs.
As \citet{bekkers2020bspline} is assuming arbitrary Lie groups, his models additionally describe the $\Aff(G)$-equivariant CNNs in Table~\ref{tab:network_instantiations} which are labeled by trivial, regular or more general quotient representations.
They cover in particular scale equivariant Euclidean CNNs ($G=\Scale$) for which the determinant factor $\detg$ is non-trivial.


Other types of feature fields and non-homogeneous spaces like punctured Euclidean spaces $\Euc_d\backslash\{0\}$ and spheres $S^2\backslash\{n,s\}$, the icosahedron, general surfaces and the M\"obius strip are not covered.














\toclesslab\subsection{Steerable CNNs on homogeneous spaces}{apx:homogeneous_steerable_convs}

Motivated by Kondor and Trivedi's \cite{Kondor2018-GENERAL} generalization of group convolutions to homogeneous spaces, \citet{Cohen2018-intertwiners}\cite{Cohen2019-generaltheory} generalized steerable CNNs to homogeneous spaces of locally compact unimodular groups.%
\footnote{
    Note that there are a preprint version~\cite{Cohen2018-intertwiners} and a conference version~\cite{Cohen2019-generaltheory} of this paper.
}
Instead of restricting to scalar fields, \citet{Cohen2018-intertwiners}\cite{Cohen2019-generaltheory} assume more general \emph{$H_l$-associated feature fields} on $I/H_l$ which transform according to \emph{induced representations}~$\Ind_{H_l}^\I \rho_l$ of~$\I$.
The network layers implement linear equivariant maps between such fields, i.e. they are \emph{intertwiners between induced representations}.
As~expected, these layers are parameterized by -- and are thus isomorphic to -- spaces of steerable kernels.
\citet{Cohen2018-intertwiners}\cite{Cohen2019-generaltheory} show that these kernels can be described on $\I$, on $\I/\Hin$ or on $\Hout\backslash \I/\Hin$, in each case still satisfying a linear steerability constraint.%
\footnote{
    As we will argue below, the constructions on $\I/\Hin$ and $\Hout\backslash \I/\Hin$ depend on \emph{local} sections and are therefore only possible for trivial bundles.
    We adapt the former to nontrivial bundles by defining kernels on an open cover of~$\I/\Hin$.
}
The following three paragraphs will
1) introduce feature fields and their transformation laws on a global and local level,
2) review the spaces of intertwiners and steerable kernels which map between such fields, and
3) discuss how these results relate to ours.

Our formulation and notation in this section is adapted to be more similar to that which was chosen to develop our theory.
It differs therefore slightly from that of \citet{Cohen2018-intertwiners}\cite{Cohen2019-generaltheory}.
Most notably, we do not assume a single local trivialization (section) which is defined almost everywhere on $\I/H_l$ but consider an atlas of local trivializations which cover the homogeneous space.%
\footnote{
    This is only necessary if the homogeneous space is a (non-trivial) manifold.
    If it is discrete, one may always choose a global section $\I/H \to \I$ which selects coset representatives.
    One would in this case usually not talk about ``atlases'' and ``local trivializations'', however, we will do so for simplicity.
}
The notation of local, coordinatized quantities is therefore augmented with gauge labels $A,B,\dots$ \,.





\paragraph{Feature fields and induced representations:}

Let $\I$ be a locally compact unimodular group and let $H_l\leq\I$ be any subgroup of it.
As stated above, the quotient map
\begin{align}
    \mathscr{q}^{\I}_{\overset{}{\I/H_l}}: \I \to \I/H_l, \quad \phi \mapsto \phi.H_l
\end{align}
implies a \emph{principal $H_l$-bundle}; see Section~\ref{sec:fiber_bundles_general}.
The right $H_l$-action on the total space $\I$ is given by the usual right multiplication
\begin{align}
    \I \times \I/H_l \to \I/H_l, \quad (\phi,h) \mapsto \phi h
\end{align}
of group elements.
It preserves the fibers $\I_{\phi.H_l} = \big( \mathscr{q}^{\I}_{\overset{}{\I/H_l}} \big)^{-1} (\phi.H_l) \,\subset\, \I$ since it satisfies
\begin{align}
    \mathscr{q}^{\I}_{\overset{}{\I/H_l}} (\phi h)
    \ =\ \phi h.H_l\ =\ \phi.H_l
    \ =\ \mathscr{q}^{\I}_{\overset{}{\I/H_l}} (\phi)
\end{align}
for any $\phi\in\I$ and $h\in H_l$ and is easily seen to be both transitive and free.
Abbreviating $U := U^A \cap U^B$, local trivializations $\PsiI^A,\ \PsiI^B$ of this bundle and the transition maps $h^{BA}$ between them are defined via the following commutative diagram:
\begin{equation}
\begin{tikzcd}[row sep=4.em, column sep=5.5em]
    % ROW 1
    &[-6.5em]
    & U\times H_l
    \\
    % ROW 2
      \I\ \: \supseteq
    & \big(\mathscr{q}^{\I}_{\overset{}{\I/H_l}} \big)^{\mkern-2mu-1}(U)
                    \arrow[d, swap, "\mathscr{q}^{\I}_{\overset{}{\I/H_l}}"]
                    \arrow[r, "\PsiI^A"]
                    \arrow[ru, "\PsiI^B"]
    & U\times H_l   \arrow[u, swap, "(\id\times h^{BA}\cdot)"]
                    \arrow[ld, "\proj_1"]
    \\
    % ROW 3
      \I/H_l\ \ \supseteq \mkern30mu
    & U
\end{tikzcd}
\qquad
\end{equation}
As usual, the principal bundle trivializations imply local identity sections
\begin{align}
    \sigma^A: U^A \to \big(\mathscr{q}^{\I}_{\overset{}{\I/H_l}} \big)^{\mkern-2mu-1}(U^A) \,, \quad 
    \phi.H_l \,\mapsto\, \sigma^A (\phi.H_l) \,:=\, \big(\PsiI^A \big)^{-1} (\phi.H_l,\, e) \,,
\end{align}
which were introduced in Section~\ref{sec:bundle_trivializations}.
The identity sections labeled by $\widetilde{A}$ at $\zeta.H_l$ and $A$ at $\phi\,\zeta.H_l$ are related by
\begin{align}\label{eq:homogeneous_induce_gauge_trafo}
    \phi\; \sigma^{\widetilde{A}} (\zeta.H_l) \ =\ \sigma^A (\phi\,\zeta.H_l)\,\ h_\phi^{A\widetilde{A}} (\zeta.H_l) \,,
\end{align}
which defines the $\I$-induced gauge transformations $h_\phi^{A\widetilde{A}} (\zeta.H_l) \in H_l$; see Eq.~\eqref{eq:pushfwd_section_right_action}.%
\footnote{
    To avoid confusion, note that \citet{Cohen2018-intertwiners}\cite{Cohen2019-generaltheory} denote $h_\phi^{A\widetilde{A}} (\zeta.H_l)$ by $\operatorname{h} (\zeta.H_l,\, \phi)$, omitting the gauge labels.
}




The feature fields of steerable CNNs on homogeneous spaces are defined as sections $f\in \Gamma(\A_l)$ of associated $H_l$-bundles
\begin{align}\label{eq:equiv_rel_rhol}
    \A_l\ :=\ (\I\times\R^{c_l})/\!\sim_{\!\rho_l} \,,
\end{align}
which were introduced in Section~\ref{sec:G_associated_bundles}.
The equivalence relation
\begin{align}
    (\phi,\, \mathscr{f})\ \sim_{\!\rho_l}\, (\phi h^{-1},\, \rho_l(h) \mathscr{f})
\end{align}
is determined by a choice of field representation
\begin{align}
    \rho_l: H_l \to \R^{c_l}
\end{align}
of the layer's subgroup; compare this to our analogous definition in Eq.~\eqref{eq:equiv_relation_A}.
Being an associated $H_l$-bundle, the local feature vector bundle trivializations transform covariantly with those of the corresponding principal bundle:
\begin{equation}
\begin{tikzcd}[row sep=3.5em, column sep=5.em]
    % ROW 1
    & U\times \R^{c_l}
    \\
    % ROW 2
      \piAl^{-1}(U)  \arrow[d, swap, "\piAl"]
                    \arrow[r, "\PsiAl^A"]
                    \arrow[ru, "\PsiAl^B"]
    & U\times \R^ {c_l} \arrow[u, swap, "\big(\id\times \rho_l\big(h^{BA}\big)\cdot\big)"]
                    \arrow[ld, "\proj_1"]
    \\
    % ROW 3
    U
\end{tikzcd}
\end{equation}
The precise construction of associated bundle trivializations from principal bundle trivializations was given in Eq.~\eqref{eq:trivialization_A}.


\citet{Cohen2018-intertwiners}\cite{Cohen2019-generaltheory} use two different approaches to describe feature fields.
Globally, feature fields are represented as functions
\begin{align}\label{eq:mackey_functions}
    F: \I \to \R^{c_l}
    \quad \textup{such that} \quad
    F(\phi h^{-1}) = \rho_l(h) F(\phi)
    \quad \forall\ \phi\in\I,\ h\in H_l \,,
\end{align}
whose definition is consistent with the equivalence relation from Eq.~\eqref{eq:equiv_rel_rhol}.
On trivializing neighborhoods $U^A \subseteq \I/H_l$, the fields are furthermore given by feature vector coefficient fields
\begin{align}
    f^A: U^A \to \R^c
\end{align}
relative to some gauge $\PsiAl^A$.
While the former is more convenient for algebraic manipulations, the latter is non-redundant, and therefore more suitable for numerical implementations.
The local field representation may at any time be computed from the global one by setting
\begin{align}
    f^A( \phi.H_l)\ =\ F\big( \sigma^A (\phi.H_l) \big)
    \qquad \textup{for}\ \ \phi.H_l \in U^A \,.
\end{align}
Here $\sigma^A: U^A \to \I$ is that local section of the principal $H_l$ bundle which corresponds to the chosen trivialization $\PsiI^A$ (``identity section'') and is analogously defined to Eq.~\eqref{eq:GM_section_psi_inverse_def}.
Note that the global field representation can in general not be recovered from a (single) local one.
It is, however, locally over $U^A$ given by
\begin{align}
    \quad
    F(\phi)\ =\ \rho_l\big( \psiI{\phi.H_l}^A(\phi) \big)^{-1}\, f^A(\phi.H_l)
    \qquad \textup{for}\ \ \phi \in \big(\mathscr{q}^{\I}_{\overset{}{\I/H_l}} \big)^{\mkern-2mu-1} (U^A)\ \subseteq\,\I \,,
\end{align}
which is closely related to Eq.~\eqref{eq:trivialization_A_p}.


The global, active transformations of feature fields are formalized by \emph{induced representations} $\Ind_{H_l}^{\I}\mkern-2mu \rho_l$ of~$\I$, which are conceptually similar to our isometry pushforwards from Def.~\ref{dfn:isometry_pushforward}.
For the global field representations, this action is simply defined as a shift on~$\I$:
\begin{align}
    \pig[ \big[ \Ind_{H_l}^{\I}\mkern-2mu \rho_l\big] (\zeta)\, F \pig] (\phi)\ =\ F\big( \zeta^{-1} \phi\big)
\end{align}
Since the $\I$-action is global, it is more difficult to describe for local field representations.
Let $U^A$ be a trivializing neighborhood around $\phi.H_l$ and $U^{\widetilde{A}}$ around $\zeta^{-1}\phi.H_l$.
The action of the induced representation is relative to gauges on these neighborhoods given by
\begin{align}
    \pig[ \big[\Ind_{H_l}^{\I}\mkern-2mu \rho_l\big] (\zeta)\, f\pig]^{\raisebox{2.5pt}{$\scriptstyle A$}} (\phi.H_l)
    \ =\ \rho_l\big( h^{A\widetilde{A}}_{\zeta} \big) f^{\widetilde{A}} \big( \zeta^{-1} \phi.H_l \big) \,,
\end{align}
where $h^{A\widetilde{A}}_{\zeta}$ is the $\zeta$-induced gauge transformation, which is analogously defined to that in Eq.~\eqref{cd:pushforward_GM_coord_extended}.
Note the similarity of this definition to our isometry pushforward of feature fields in coordinates from Eq.~\eqref{eq:feature_field_trafo_in_coords}.
We furthermore identify the transformation law of scalar fields on homogeneous spaces from Eq.~\eqref{eq:group_action_homogeneous_space_scalar_field} as a special case for trivial representations~$\rho_l$.
Steerable CNNs on homogeneous spaces cover therefore the homogeneous scalar field convolutions of \citet{Kondor2018-GENERAL} and \citet{bekkers2020bspline} as a special case (ignoring the different assumptions made on the type of group $\I$).






\paragraph{Intertwiners between induced representations and steerable kernels:}
The main endeavor of \citet{Cohen2018-intertwiners}\cite{Cohen2019-generaltheory} is to characterize the space
\begin{align}
    \Hom_{\I}\! \big( \Gamma(\Ain), \Gamma(\Aout) \big)
    \ :=\ \big\{ \mathfrak{K}: \Gamma(\Ain) \to \Gamma(\Aout)\,\ \textup{linear} \ \big|\ 
        \mathfrak{K}\circ \Ind_{\Hin}^{\I}(\phi) = \Ind_{\Hout}^{\I}(\phi) \circ\mathfrak{K}\ \ \ \forall\ \phi\in\I \big\}
\end{align}
of intertwiners between induced representations, i.e. the space of linear equivariant maps between feature fields.
Diagrammatically, this space consists of those linear maps $\mathfrak{K}$ which let the following diagram commute for any $\phi\in\I$:
\begin{equation}
\begin{tikzcd}[column sep=60pt, row sep=35, font=\normalsize]
    \Gamma(\Ain)
        \arrow[r, "\mathfrak{K}"]
        \arrow[d, "\Ind_{\Hin}^{\I} \rhoin(\phi)"']
    &
    \Gamma(\Aout)
        \arrow[d, "\Ind_{\Hout}^{\I} \rhoout(\phi)"]
    \\
    \Gamma(\Ain)
        \arrow[r, "\mathfrak{K}"']
    &
    \Gamma(\Aout)
\end{tikzcd}
\end{equation}
These maps are the analog to our isometry equivariant kernel field transforms, which were defined in Def.~\ref{dfn:isometry_equivariance}.
\citet{Cohen2018-intertwiners}\cite{Cohen2019-generaltheory} prove that these maps are given by correlations with steerable kernels.
We will in the following briefly review these results for both global and local field representations.


When working with the global field representation from Eq.~\eqref{eq:mackey_functions}, \citet{Cohen2018-intertwiners}\cite{Cohen2019-generaltheory} start with a general bounded linear operator $\mathfrak{K}$ of the form
\begin{align}\label{eq:general_linear_map_mackey}
    \big[\mathfrak{K} F \big](\phi) \ =\ 
    \int_{\I} \widehat{\kappa} (\phi,\zeta)\, F(\zeta)\ d\zeta
\end{align}
where $d\zeta$ is a left Haar measure on $\I$ and
\begin{align}
    \widehat{\kappa}: \I\times\I \to \R^{\cout\times\cin}
\end{align}
is a matrix-valued 2-argument kernel.
The equivariance constraint is shown to require the kernels to satisfy the relation
$\widehat{\kappa} (\widetilde{\phi}\phi, \widetilde{\phi}\zeta) = \widehat{\kappa} (\phi, \zeta)$
for any choice of group elements $\widetilde{\phi},\, \phi,\, \zeta \in \I$.
This result is resembling our Theorem~\ref{thm:isometry_equivariant_kernel_field_trafos}, which states that isometry equivariant kernel field transforms imply kernel fields which are invariant under the action of isometries, Def.~\ref{dfn:isometry_invariant_kernel_fields}.
Given this constraint, the 2-argument kernel can be replaced by a 1-argument kernel which is defined as
\begin{align}\label{eq:one_arg_kernel_global}
    \kappa: \I \to \R^{\cout\times\cin},\ \ \ \phi \mapsto \kappa(\phi) := \widehat{\kappa}(e,\phi) \,.
\end{align}
We therefore have
$\widehat{\kappa}(\phi,\zeta) = \widehat{\kappa}(\phi^{-1} \phi,\, \phi^{-1}\zeta) = \kappa\big( \phi^{-1} \zeta\big)$,
implying that the linear operator is given by a \emph{group correlation} (Eq.~\eqref{eq:group_corr_def}), that is:
\begin{align}
    \big[\mathfrak{K} F \big](\phi)
    \ =\ \int_{\I} \kappa\big( \phi^{-1}\zeta \big)\, F(\zeta)\ d\zeta
    \ =\ \big(F \star_{\overset{}{\protect\scalebox{.64}{$\mkern-.5mu \I$}}} \kappa \big)(\phi)
\end{align}
The correlation kernel is furthermore required to satisfy the linear $\Hout$-$\Hin$-steerability constraint
\begin{align}\label{eq:double_steerability}
    \kappa(h_\textup{out}\, \phi\, h_\textup{in})
    \ =\ \rhoout(h_\textup{out})\, \kappa(\phi)\, \rhoin(h_\textup{in})
    \qquad \forall\,\ \phi\in\I,\ h_\textup{in}\in\Hin,\ h_\textup{out}\in\Hout \,.
\end{align}
This constraint is reminiscent of that found by \citet{Kondor2018-GENERAL} and \citet{bekkers2020bspline}.
Instead of enforcing kernels to be left $\Hout$- and right $\Hin$-\emph{invariant},
which would correspond to trivial representations $\rhoout=\rho_\textup{triv}^{\Hout}$ and $\rhoin=\rho_\textup{triv}^{\Hin}$,
the constraint of \citet{Cohen2018-intertwiners}\cite{Cohen2019-generaltheory} allows for more general steerable kernels.
The vector space $\mathscr{K}^{\I}_{\rhoin\mkern-1mu,\rhoout}$ of such steerable correlation kernels is argued to be isomorphic to the intertwiner space $\Hom_{\I}\! \big( \Gamma(\Ain), \Gamma(\Aout) \big)$.





Since the global field representations $F$ on $\I$ are redundant they are not the best choice for numerical implementations.
\citet{Cohen2018-intertwiners}\cite{Cohen2019-generaltheory} are therefore additionally investigating intertwiners which operate on local field representations.
The authors approach this problem by assuming one \emph{single local trivialization} to be given, which is \emph{defined almost everywhere} on the homogeneous space~$\I/\Hin$.
They are therefore effectively operating on a trivial bundle.
Our following review adapts their results slightly to the more general case of a set of field representations relative to an \emph{atlas of local trivializations}.
The formulation of \citet{Cohen2018-intertwiners}\cite{Cohen2019-generaltheory} is retrieved by restricting the integration to one single trivialization.
We explicitly write out all gauge labels to make the coordinate dependencies transparent.
To give an overview on the local trivializations that will play a role in the following, we mention that we will need to consider
trivializing neighborhoods $U^A, U^{\widetilde{A}}, U^H \subseteq \I/\Hin$ such that
\begin{align}
    \zeta.\Hin                   \in U^A \,, \qquad
    \widetilde{\phi}\zeta.\Hin   \in U^{\widetilde{A}} \quad \textup{and}\ \ \
    h\zeta.\Hin                  \in U^H
\end{align}
and trivializing neighborhoods $U^P, U^{\widetilde{P}}, U^E \subseteq \I/\Hout$ such that
\begin{align}
    \phi.\Hout                   \in U^P \,, \qquad
    \widetilde{\phi}\phi.\Hout   \in U^{\widetilde{P}} \quad \textup{and}\ \ \
    e.\Hout                  \in U^E \,.
\end{align}
We will furthermore assume any partition of unity
$\{ \mathscr{P}_{U^X} \}_{X\in\mathfrak{X}}$
subordinate to the open cover
underlying the atlas
$\mathscr{A}_\textup{in} = \{( U^X, \Psi^X )\}_{X\in\mathfrak{X}}$
of local trivializations on $\I/\Hin$.
This means that we are given maps $\mathscr{P}_{U^X}: \I/\Hin \to [0,1]$ with the properties
\begin{align}
    \supp\big( \mathscr{P}_{\!\overset{}{U^X}} \big) \subseteq U^X
    \qquad \textup{and} \qquad
    \sum_{U^X\in \mathscr{A}_\textup{in}} \mathscr{P}_{\!\overset{}{U^X}} (\phi.\Hin) = 1
    \quad \forall\ \phi.\Hin \in \I/\Hin \,.
\end{align}


Eq.~\eqref{eq:general_linear_map_mackey} stated the general form of a bounded linear operator between global field representations~$F$.
Its local analog, which makes use of the partition of unity, is given by
\begin{align}
    \big[ \mathfrak{K} f \big]^P (\phi.\Hout)\ =\ 
    \sum_{U^{\!A} \in \mathscr{A}_\textup{in}} \int_{U^{\!A}} \mathscr{P}_{\!\overset{}{U^{\!A}}} (\zeta.\Hin)\ \ 
        \widehat{\overleftarrow{\kappa}} \rule{0pt}{0pt}^{\mkern-1mu PA}(\phi.\Hout,\, \zeta.\Hin)\,\ 
        f^A (\zeta.\Hin)\ \ d(\zeta.\Hin) \,,
\end{align}
where $P$ and $A$ label local trivializations as stated above and $d(\zeta.\Hin)$ is a measure on~$\I/\Hin$.
We furthermore have 2-argument kernels
\begin{align}
    \widehat{\overleftarrow{\kappa}} \rule{0pt}{0pt}^{\mkern-1mu PA}:\, U^P \!\times U^A \to \R^{\cout\times\cin}
    ,\qquad (\phi.\Hout,\, \zeta.\Hin) \,\mapsto\,
    \widehat{\kappa} \big( \sigma^P(\phi.\Hout),\, \sigma^A(\zeta.\Hin) \big)
\end{align}
which are inherently \emph{locally defined} on ${U^P \!\times U^A} \,\subseteq\, \I/\Hout \times \I/\Hin$.
The global 2-argument kernel can be recovered from a set of local kernels on the open covers.
\citet{Cohen2018-intertwiners}\cite{Cohen2019-generaltheory} prove that these local kernels are required to satisfy
\begin{align}\label{eq:intertwiner_constraint_local}
    \widehat{\overleftarrow{\kappa}} \rule{0pt}{0pt}^{\mkern-1mu PA}(\phi.\Hout,\, \zeta.\Hin)
    \ =\ 
    \rhoout\big( h_{\widetilde{\phi}}^{\widetilde{P}P} (\phi.\Hout) \big)^{\!-1}\,\ 
    \widehat{\overleftarrow{\kappa}} \rule{0pt}{0pt}^{\mkern-1mu \widetilde{P}\widetilde{A}}( \widetilde{\phi}\phi.\Hout,\, \widetilde{\phi}\zeta.\Hin)
    \,\ \rhoin\big( h_{\widetilde{\phi}}^{\widetilde{A}A} (\zeta.\Hin) \big)
\end{align}
for any $\widetilde{\phi} \in\I$.
Note that $h_{\widetilde{\phi}}^{\widetilde{P}P} (\phi.\Hout)$ is hereby an induced gauge transformation on $\I/\Hout$ while $h_{\widetilde{\phi}}^{\widetilde{A}A} (\zeta.\Hin)$ is an induced gauge transformation on $\I/\Hin$.
In order to reduce these local 2-argument kernels to local 1-argument kernels, \citet{Cohen2018-intertwiners}\cite{Cohen2019-generaltheory} consider the unique group element $\widetilde{\phi} \in\I$ which satisfies
1) $\widetilde{\phi} \phi.\Hout = e.\Hout$ and
2) $\widetilde{\phi} \sigma^P(\phi.\Hout) = \sigma^E(e.\Hout) = e$,
where the last equality fixes a specific gauge at the ``origin'' $e.\Hout$, which is always possible.
The first point allows us to identify the gauges $\widetilde{P}$ and~$E$ without loss of generality.
The relations imply furthermore $\widetilde{\phi} = \sigma^P (e.\Hout)^{-1}$ and, by Eq.~\eqref{eq:homogeneous_induce_gauge_trafo}, $h_{\widetilde{\phi}}^{EP} (\phi.\Hout) = e$.
Plugging these choices into Eq.~\eqref{eq:intertwiner_constraint_local} yields
\begin{align}
    \widehat{\overleftarrow{\kappa}} \rule{0pt}{0pt}^{\mkern-1mu PA}(\phi.\Hout,\, \zeta.\Hin)
    \ =\ 
    \id_{\R^\cout}^{PE}\ 
    \underbrace{ \vphantom{\Big(}
    \widehat{\overleftarrow{\kappa}} \rule{0pt}{0pt}^{\mkern-1mu E\widetilde{A}}\big( e.\Hout,\, \sigma^P \mkern-2mu (\phi.\Hout)^{-1} \zeta.\Hin\big)}_{ \rule{0pt}{14pt} \displaystyle
    =: \overleftarrow{\kappa}^{E\widetilde{A}} \big( \sigma^P \mkern-2mu (\phi.\Hout)^{-1} \zeta.\Hin \big) }
    \,\ \rhoin\big( h_{\sigma^P \mkern-2mu (\phi.\Hout)^{-1}}^{\widetilde{A}A} (\zeta.\Hin) \big) \,,
\end{align}
where the identity map is kept explicit to explain the gauge labels.
We furthermore introduced the local 1-argument kernels
\begin{align}\label{eq:local_one_arg_kernel}
    \overleftarrow{\kappa}^{E\widetilde{A}} :\, U^{\widetilde{A}} \to \R^{\cout\times\cin} \,,
\end{align}
whose responses are always given in the specific gauge $E$ at $e.\Hout$.
These kernels are still required to satisfy the $\Hout$-steerability constraints
\begin{align}\label{eq:single_steerability}
    \overleftarrow{\kappa}^{EH} (h_\textup{out} \zeta.\Hin)
    \ =\ \rhoout(h_\textup{out})\, \overleftarrow{\kappa} (\zeta.\Hin)^{EA} \,
    \rhoin\big( h_{h_\textup{out}}^{HA} (\zeta.\Hin) \big)^{-1}
    \qquad \forall\,\ \zeta.\Hin\in \I/\Hin,\ h_\textup{out}\in\Hout \,.
\end{align}
Putting everything together, the equivariant correlation becomes
\begin{align}
    & \big[ \mathfrak{K} f \big]^P (\phi.\Hout) \ = \\
    & \id_{\R^\cout}^{PE} \mkern-6mu
    \sum_{U^{\!A} \in \mathscr{A}_\textup{in}} \int_{U^{\!A}} \mathscr{P}_{\!\overset{}{U^{\!A}}} (\zeta.\Hin)\ \ 
        \overleftarrow{\kappa} \rule{0pt}{0pt}^{\mkern-1mu E\widetilde{A}} \pig( \sigma^P \mkern-2mu (\phi.\Hout)^{-1} \zeta.\Hin \pig)\,\ 
        \rhoin\pig( h_{\sigma^P \mkern-2mu (\phi.\Hout)^{-1}}^{\widetilde{A}A} (\zeta.\Hin) \pig)\,\ 
        f^A (\zeta.\Hin)\ \ d(\zeta.\Hin) \,. \notag
\end{align}
Adding the assumption that a single gauge $A = \widetilde{A}$ covers $\I/\Hin$ almost everywhere, we can drop the partition of unity and retrieve the formulation of \citet{Cohen2018-intertwiners}\cite{Cohen2019-generaltheory}:
\begin{align}
    \big[ \mathfrak{K} f \big]^P (\phi.\Hout)
    \ =\ \id_{\R^\cout}^{PE} \int_{U^{\!A}}
        \overleftarrow{\kappa} \rule{0pt}{0pt}^{\mkern-1mu EA} \pig( \sigma^P \mkern-2mu (\phi.\Hout)^{-1} \zeta.\Hin \pig)\,\ 
        \rhoin\pig( h_{\sigma^P \mkern-2mu (\phi.\Hout)^{-1}}^{AA} (\zeta.\Hin) \pig)\,\ 
        f^A (\zeta.\Hin)\ \ d(\zeta.\Hin)
\end{align}
We comment on the relation of this operation to our $\GM$-convolutions further below.


Instead of defining the \emph{local} 1-argument kernels in coordinates from Eq.~\eqref{eq:local_one_arg_kernel} on local subsets $U^{\widetilde{A}}$, \citet{Cohen2018-intertwiners}\cite{Cohen2019-generaltheory} define them \emph{globally} on~$\I/\Hin$.
Since their construction relies on a continuous section, this is only possible if the bundles are trivial.
Our adaptation to local kernel representations on an open covering is bridging this gap.


\citet{Cohen2018-intertwiners}\cite{Cohen2019-generaltheory} claim an isomorphism between the global kernels on $\I$,
satisfying the steerability constraint in Eq.~\eqref{eq:double_steerability},
and their kernels on $\I/\Hin$,
satisfying the steerability constraint in Eq.~\eqref{eq:single_steerability}.
Note that this isomorphism can only hold if
either the bundle is trivial
or the continuity assumption on the sections (and therefore network inference) is dropped.
It should, however, be possible to prove an isomorphism between the global kernel and a collection of local kernels on a covering of $\I/\Hin$, satisfying the relations in Eq.~\eqref{eq:single_steerability}.

The authors furthermore claim that the steerable kernels can be described on the double quotient space $\Hout\backslash \I/\Hin$, still satisfying a steerability constraint.






\paragraph{Relation to \textit{GM}-convolutions:}

The steerable CNNs on homogeneous spaces by \citet{Cohen2018-intertwiners}\cite{Cohen2019-generaltheory} are conceptually quite similar to our $\GM$-convolutions on Riemannian manifolds, however, there are some important differences which we discuss in the following.
Most importantly, the theories differ in
1) being based on different spaces $\I/H_l$ in each layer $l$ vs. assuming a fixed manifold $M$,
2) modeling kernels on the space $\I/\Hin$ itself or on tangent spaces $\TpM$ of it,
3) the way of how weights are shared, and
4) the types of global symmetry group $\I$ and spaces $\I/H_l$ or $M$, which they cover.
Despite these differences, many of the results of \citet{Cohen2018-intertwiners}\cite{Cohen2019-generaltheory} have analogs in our theory.

Both theories share the idea to define feature fields as sections of associated vector bundles.
While \citet{Cohen2018-intertwiners}\cite{Cohen2019-generaltheory} consider a global symmetry group $\I$ as a set of multiple principal $H_l$-bundles over homogeneous spaces $\I/H_l$, we work with some $G$-structure $\GM$ over a fixed Riemannian manifold~$M$.
All of our feature vector bundles are defined as $G$-bundles and are associated to each other, while the feature bundles of \citet{Cohen2018-intertwiners}\cite{Cohen2019-generaltheory} may not be associated to each other if their structure groups $H_l$ do not agree.
As already claimed at the end of the last Section~\ref{apx:homogeneous_scalar_field_convs}, these differences can be mitigated if a structure group $G$ can be chosen such that $H_l \leq G \leq \I$ for every layer $l$ and $M := \I/G$ is a Riemannian manifold.
One can then replace all homogeneous spaces $\I/H_l$ with $M$ and all $H_l$-representations $\rho_l$ with induced $G$-representations
\begin{align}
    \rho_l^G\ :=\ \Ind_{H_l}^G \rho_l \,.
\end{align}
The global field transformation laws are preserved by this reinterpretation since
\begin{align}
    \Ind_{H_l}^{\I} \rho_l
    \ =\ \Ind_G^{\I} \Ind_{H_l}^G \rho_l
    \ =\ \Ind_G^{\I} \rho_l^G
\end{align}
holds by induction in stages~\cite{ceccherini2009induced}.


Another main difference lies in the definition of convolution kernels and weight sharing.
On the global, coordinate free level and prior to the isometry assumption, \citet{Cohen2018-intertwiners}\cite{Cohen2019-generaltheory} start in Eq.~\eqref{eq:general_linear_map_mackey} with a bounded linear operator which is parameterized by an unconstrained kernel
\begin{align}
    \widehat{\kappa}: \I\times\I \to \R^{\cout\times\cin} \,.
\end{align}
This operator corresponds in our theory to a general kernel field transform, Def.~\ref{dfn:kernel_field_trafo}, which is parameterized by an unconstrained kernel field
\begin{align}
    \K: \TM \to \Hom(\Ain,\Aout) \,,
\end{align}
see Def.~\eqref{dfn:kernel_field_general}.
The 2-argument kernels $\widehat{\kappa}$ can be thought of as representing a kernel field as well.
Their two arguments are thereby thought of as addressing
1)~a specific (1-argument) kernel, yielding a response at the corresponding point in the output bundle $\I \to \I/\Hout$ and
2)~the spatial dependency of this 1-argument kernel on the input bundle $\I \to \I/\Hin$.
The analog in our kernel fields $\K$ is that elements $v \in \TM$ encode
1)~the location $p = \piTM(v)$ of the kernel and
2)~its spatial dependency via $v\in\TpM$.


When requiring the bounded linear operator to be $\I$-equivariant, the 2-argument kernel $\widehat{\kappa}$ becomes constrained to satisfy
\begin{align}
    \widehat{\kappa} \big( \widetilde{\phi}\phi, \widetilde{\phi}\zeta \big) = \widehat{\kappa} (\phi, \zeta)
    \qquad \forall\ \widetilde{\phi} \in\I \,.
\end{align}
Isometry equivariant kernel field transforms were in Theorem~\ref{thm:isometry_equivariant_kernel_field_trafos} shown to require the isometry invariance of the kernel field, i.e.
\begin{align}
    \widetilde{\phi}_{\overset{}{\mkern-2mu*\mkern-1mu\scalebox{.55}{$,\mkern-2mu\mathscr{K}\mkern2mu$}}} \K = \K
    \qquad \forall\ \widetilde{\phi} \in\I \,;
\end{align}
see Def.~\ref{dfn:isometry_invariant_kernel_fields} and Fig.~\ref{fig:isom_invariant_kernel_field_multiple_orbits}.


The invariance constraint on 2-argument kernels $\widehat{\kappa}$ allows to replace them with 1-argument kernels
\begin{align}
    \kappa: \I \to \R^{\cout\times\cin},
\end{align}
defined in Eq.~\eqref{eq:one_arg_kernel_global}.
They are still required to satisfy the steerability constraint in Eq.~\eqref{eq:double_steerability}.
Our isometry invariant kernel fields were in Theorem~\ref{thm:manifold_quotient_repr_kernel_fields} shown to be equivalent to a field of kernels
\begin{align}
    \Qhat \mkern-2mu: \piTM^{-1}\big( \rM(\IM) \big) \to \piHom^{-1}\big( \rM(\IM) \big)
\end{align}
whose support is restricted to the tangent spaces over representatives $\rM(\IM) \subseteq M$ of the quotient~$\IM$.%
\footnote{
    Theorem~\ref{thm:tangent_quotient_repr_kernel_fields} proves another isomorphism to a space of kernels $\Q \mkern-2mu: \rTM(\ITM) \to \rHom(\IHom)$ whose support is even further restricted to representatives of the tangent bundle quotient~$\ITM$.
}
These kernels are required to satisfy a stabilizer subgroup steerability constraint as well.
For the specific case that $M$ is a homogeneous space of its isometry group, the quotient $\IM$ reduces to a single element.
Theorem~\ref{thm:manifold_quotient_repr_kernel_fields} implies in this case a single (1-argument) kernel
\begin{align}\label{eq:Qhat_homogeneous_rel_work_section}
    \Qhat \mkern-2mu: \TpM \to \Hom(\Ainp, \Aoutp)
\end{align}
at $p = \rM(\IM)$, which is the direct analog to the 1-argument kernel of \citet{Cohen2018-intertwiners}\cite{Cohen2019-generaltheory}.


Note that the full kernel fields can via the action of $\I$ be reconstructed from the single 1-argument kernels.
The theories derive therefore both a form of convolutional weight sharing from the requirement of global symmetry equivariance.
While kernels can for transitive symmetries be shared over the whole homogeneous space, they can in general only be shared over the orbits of the symmetry group.
If the manifold is asymmetric in such a way that the orbits are single points no weights can be shared with this definition.
As this is the default case for Riemannian manifolds, $\GM$-convolutions resort to the sharing of $G$-steerable kernels by placing them relative to frames of the $G$-structure.
This definition does not have a counterpart in steerable CNNs on homogeneous spaces.
Our Theorem~\ref{thm:GM_conv_homogeneous_equivalence} shows, however, that the global symmetry induced weight sharing is for the specific case of homogeneous spaces equivalent to our process of sharing $G$-steerable kernels.
In other words, isometry equivariant kernel field transforms on homogeneous spaces are necessarily convolutions
-- this mirrors the central results of \citet{Kondor2018-GENERAL}, \citet{bekkers2020bspline} and \citet{Cohen2018-intertwiners}\cite{Cohen2019-generaltheory}.


After investigating the analogies for the global, coordinate free kernels of both theories, we compare the definition of their coordinate representations relative to local trivializations.
Given some choice of trivializing neighborhoods $U^P \subseteq \I/\Hout$ and $U^A \subseteq \I/\Hin$, the unconstrained global 2-argument kernels of \citet{Cohen2018-intertwiners}\cite{Cohen2019-generaltheory} are locally represented by unconstrained functions
\begin{align}
    \widehat{\overleftarrow{\kappa}}:\, U^P \times U^A \to \R^{\cout\times\cin} \,.
\end{align}
In our theory, we instead have a single trivializing neighborhood $U^P = U^A \subseteq M$ relative to which a kernel field is given by an unconstrained map
\begin{align}
    \K^A:\, U^A \times \R^d \to \R^{\cout\times\cin} \,.
\end{align}
Investigating the \emph{global} $\I$-equivariance of the operator $\mathfrak{K}$ based on \emph{local} kernels is on non-trivial bundles necessarily difficult as it involves multiple trivializations.
The equivariance requirement implies for steerable CNNs on homogeneous spaces the constraints between different local kernels in Eq.~\eqref{eq:intertwiner_constraint_local}.
They leads to the 1-argument kernels
\begin{align}
    \overleftarrow{\kappa}^{EA} :\, U^A \to \R^{\cout\times\cin} \,,
\end{align}
from Eq.~\eqref{eq:local_one_arg_kernel}, which are still subject to the steerability constraint in Eq.~\eqref{eq:single_steerability}.
Single kernels $\Kp: \TpM \to \Hom(\Ainp,\Aoutp)$ (like e.g. $\Qhat$ from Eq.~\eqref{eq:Qhat_homogeneous_rel_work_section}) are according to Eq.~\eqref{eq:kernel_field_general_coord_expression} in coordinates given by functions
\begin{align}
    \Kp^A:\, \R^d \to \R^{\cout\times\cin} \,,
\end{align}
whose domains are tangent space coordinates $\R^d$ instead of of a open subset $U^A$ of the manifold.
A particular important example are $G$-steerable kernels, which correspond to the $\GM$-convolutional kernel fields from Def.~\ref{dfn:conv_kernel_field}.


While our kernels are globally defined in a single gauge $\psiTMp^A$ of $\TpM$, the local 1-argument kernels of \citet{Cohen2018-intertwiners}\cite{Cohen2019-generaltheory} need to be defined on an open cover of $\I/\Hin$.
As this is significantly more complicated, they propose therefore to represent the kernels on a single gauge which is defined almost everywhere.%
\footnote{
    In practice, one might anyways work with compactly supported kernels on a single trivializing neighborhood, which would render this choice unproblematic.
}
Note that this still requires that this single trivializing neighborhood is closed under the left action of $\Hout$ in order for the constraint in Eq.~\eqref{eq:single_steerability} to make sense.
We investigated this approach in Section~\ref{sec:spherical_CNNs_fully_equivariant} for the specific example of spherical CNNs, defining kernels on the trivializing neighborhood $U^A = S^2\backslash -n$.
Theorem~\ref{thm:spherical_kernel_space_iso} proved that Cohen \mbox{et al.'s}~\cite{Cohen2018-intertwiners}\cite{Cohen2019-generaltheory} steerable kernels on $S^2\backslash -n \subset S^2$ are in this case isomorphic to our $G$-steerable kernels on $B_{\R^2}(0,\pi) \subset \R^2$.
The equivalence of the corresponding convolutions was established in Theorem~\ref{thm:spherical_conv_GM_conv}.



Finally, we discuss which class of models steerable CNNs on homogeneous spaces cover.
Obviously, the theory does not describe convolutions on non-homogeneous spaces like punctured Euclidean spaces $\Euc_d\backslash\{0\}$, the sphere without poles $S^2 \backslash \{n,s\}$, whose isometries $\O2$ are non-transitive, the icosahedron, general surfaces or the M\"obius strip.
However, in contrast to $\GM$-convolutions, the base spaces $\I/H_l$ are not required to be Riemannian manifolds.
While \citet{Kondor2018-GENERAL} and \citet{bekkers2020bspline} cover only those convolutions whose feature fields transform according to scalar fields on $\I/H_l$, the associated bundle formulation of \citet{Cohen2018-intertwiners}\cite{Cohen2019-generaltheory} allows for general field representations~$\rho_l$.
Restricting to unimodular groups, steerable CNNs on homogeneous spaces do, however, only include those $\Aff(G)$-equivariant Euclidean convolutions for which the structure groups are subgroups of~$\O{d}$.
This reflects in the fact that the steerability constraints of \citet{Cohen2018-intertwiners}\cite{Cohen2019-generaltheory} do not include the determinant factor in the constraint of \citet{bekkers2020bspline} and of our $G$-steerable kernels.
