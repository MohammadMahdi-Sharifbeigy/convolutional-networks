%!TEX root=../GaugeCNNTheory.tex


\mypart{A literature review on coordinate independent CNNs}
\label{part:literature_review}


The formulation of coordinate independent CNNs in terms of associated $G$-bundles over Riemannian manifolds is quite general and covers a wide range of possible model instantiations.
To substantiate this claim, we review a large body of convolutional models from the literature and explain them from the unifying viewpoint of coordinate independent CNNs.
Most of the papers in the literature do not explicitly formulate their models in terms of $G$-structures and associated $G$-bundles.
The \emph{implicitly assumed} $G$-structures and group representations are therefore \emph{deduced} from the models' weight sharing patterns, kernel symmetries and equivariance properties; see for instance Fig.~\ref{fig:G_structure_R3_no_origin}.
Table~\ref{tab:network_instantiations} summarizes the resulting taxonomy of coordinate independent CNNs.
The following sections discuss the covered models and their properties in detail.


\etocsettocdepth{2}
\etocsettocstyle{}{} % from now on only local tocs
\localtableofcontents


Section~\ref{sec:instantiations_euclidean} covers $\Aff(G)$ (affine group) equivariant convolutions on Euclidean spaces $\Euc_d$.
They rely on $\Aff(G)$-invariant $G$-structures as shown in Fig.~\ref{fig:G_structures_R2_main}.
The models in Section~\ref{sec:instantiations_euclidean_polar} operate on punctured Euclidean spaces $\Euc_d\backslash\{0\}$; see Figs.~\ref{fig:G_structures_R2_no_origin}, \ref{fig:G_structure_R2_no_origin_logpolar} or~\ref{fig:G_structure_R3_no_origin}.
They are equivariant w.r.t. rotations around the chosen origin $\{0\}$ but are not translation equivariant.
Spherical and icosahedral CNNs are discussed in Section~\ref{sec:instantiations_spherical}.
Most of these models assume the $G$-structures that are visualized in Figs.~\ref{fig:G_structure_S2_1} and~\ref{fig:G_structure_S2_2} and are therefore $\SO3$ or $\SO2$-equivariant, respectively.
Section~\ref{sec:instantiations_mesh} reviews $\GM$-convolutions on general surfaces, which are mostly discretized as meshes.

The next few pages discuss various design choices of coordinate independent CNNs and give a first overview of the models in Table~\ref{tab:network_instantiations}.



%%%%%%%%%%%%%%%%%%%%%%%%%%%%%%%%%%%%%%%%%%%%%%%%%%%%%%%%%%%%%%%%%%%%%%%%%%%%%%%%%%%%%%%%%%%%%%%%%%%%%%%%%%
\afterpage{ % execute argument of this command *after* end of current page
\clearpage % clear any pending floats
%%%%%%%%%%%%%%%%%%%%%%%%%%%%%%%%%%%%%%%%%%%%%%%%%%%%%%%%%%%%%%%%%%%%%%%%%%%%%%%%%%%%%%%%%%%%%%%%%%%%%%%%%%
    \begin{table}
        \begin{center}
        %%%%%%%%%%%%%%%%%%%%%%%%%%%%%%%%%%%%%%%%%%%%%%%%%%%%%%%%%%%%%%%%%%%%%%%%%%%%%%%%%%%%%%%%%%%%%%%%%%
        % center overwide table
        \addtolength{\leftskip} {-2cm} % increase (absolute) value if needed
        \addtolength{\rightskip}{-2cm}
        %%%%%%%%%%%%%%%%%%%%%%%%%%%%%%%%%%%%%%%%%%%%%%%%%%%%%%%%%%%%%%%%%%%%%%%%%%%%%%%%%%%%%%%%%%%%%%%%%%
            \vspace*{-6.5ex}
            %!TEX root=../GaugeCNNTheoryfa.tex

\setcounter{magicrownumbers}{0}
\small
\renewcommand\arraystretch{1.25}%% row spacing, default is 1
\setlength{\aboverulesep}{0pt}
\setlength{\belowrulesep}{0pt}
\rowcolors{2}{gray!12.5}{white}
\begin{tabular}{>{\tiny\color{gray}}rccclc}
	\toprule
	& منیفلد      & گروه ساختاری           & تقارن سراسری           & نمایش    & منابع \\
	& $M$           & $G$                       & $\AffGM$ \lr{or} $\IsomGM$     & $\rho$            &          \\
	\bottomrule
	\rownumber&
	$\Euc_d$        & $\{e\}$                   & $\Trans_d$                & \lr{trivial}               & \cite{LeCun1990CNNs,
		zhang2019CNNsShiftInvariant}
	+ \lr{any conventional CNN} \\
	% 1-dim
	\cmidrule(lr){2-6}
	\cmidrule(lr){2-6}
	\rownumber&
	$\Euc_1$        & $\Scale$                  & $\Trans_1 \rtimes \Scale$ & \lr{regular}         & \cite{romero2020wavelet} \\
	% 2-dim
	\cmidrule(lr){2-6}
	\cmidrule(lr){2-6}
	\rownumber&
	& $\Flip$                   & $\Trans_2 \rtimes \Flip$  & \lr{regular}           & \cite{Weiler2019_E2CNN} \\
	\cmidrule(lr){3-6}
	\cmidrule(lr){3-6}
	\rownumber&
	&                           &                           & \lr{irreps}            & \cite{Worrall2017-HNET,
		Weiler2019_E2CNN,
		walters2020trajectory} \\
	\rownumber&
	&                           &                           & \lr{regular}           & 
	\makecell{
		\cite{Dieleman2016-CYC,
			Cohen2016-GCNN,
			zhou2017oriented,
			Cohen2017-STEER,
			Weiler2018SFCNN,
			bekkers2018roto,
			Hoogeboom2018-HEX,
			scaife2021RadioGalaxy}
		\\
		\cite{Weiler2019_E2CNN,
			graham2020dense,
			lafarge2020rototranslation,
			smets2020pde,
			wang2020incorporating,
			romero2020attentive,
			mohamed2020data}
		\\
		\cite{shen2020PDOeConvs,
			bekkers2020bspline,
			finzi2020generalizing,
			vanderPol2020MDP2,
			gupta2020rotation,
			mondal2020group,
			walters2020trajectory,
			holderrieth2020steerableCNP}
		\\
		\cite{dey2020groupGANs,
			sifre2012combined,
			bruna2013invariant,
			Sifre2013-GSCAT,
			sifre2014rigid,
			oyallon2015scattering,
			chavan2021rescaling,
			han2021ReDet}
	} \\
	\rownumber&
	&                           &                           & \lr{quotients}         & \cite{Cohen2017-STEER,
		Weiler2019_E2CNN} \\
	\rownumber&
	&                           &                           & \lr{regular}$\xrightarrow{\textup{pool}}$\lr{trivial}
	& \cite{Cohen2016-GCNN,
		marcos2016learning,
		Weiler2019_E2CNN} \\
	\rownumber&
	& \multirow{-7.5}{*}{$\SO2$}   & \multirow{-7.5}{*}{$\SE2$} & \lr{regular}$\xrightarrow{\textup{pool}}$\lr{vector}
	& \cite{Marcos2017-VFN,
		Weiler2019_E2CNN} \\
	\cmidrule(lr){3-6}
	\cmidrule(lr){3-6}
	\rownumber&
	&                           &                           & \lr{trivial}           & \cite{khasanova2018isometric,
		Weiler2019_E2CNN} \\
	\rownumber&
	&                           &                           & \lr{irreps}            & \cite{Weiler2019_E2CNN} \\
	\rownumber&
	&                           &                           & \lr{regular}           & 
	\makecell{
		\cite{Dieleman2016-CYC,
			Cohen2016-GCNN,
			Hoogeboom2018-HEX,
			Cohen2017-STEER,
			Weiler2019_E2CNN}
		\\
		\cite{mondal2020group,
			graham2020dense,
			shen2020PDOeConvs}
	} \\
	\rownumber&
	&                           &                           & \lr{quotients}         & \cite{Cohen2017-STEER} \\
	\rownumber&
	&                           &                           & \lr{regular}$\xrightarrow{\textup{pool}}$\lr{trivial}    & \cite{Weiler2019_E2CNN} \\
	\rownumber&
	& \multirow{-6.2}{*}{$\O2$}   & \multirow{-6.2}{*}{$\E2$} & \lr{induced} $\SO2$-\lr{irreps} \hspace*{-2.ex}          & \cite{Weiler2019_E2CNN} \\
	\cmidrule(lr){3-6}
	\cmidrule(lr){3-6}
	\rownumber&
	\multirow{-15.35}{*}{$\Euc_2$}
	&                           &                           & \lr{regular}           & \cite{Worrall2019DeepScaleSpaces,
		Sosnovik2020scale,
		bekkers2020bspline,
		zhu2019scale} \\
	\rownumber&
	& \multirow{-2}{*}{$\Scale$}& \multirow{-2}{*}{$\Trans_2\rtimes\Scale$}      & \lr{regular}$\xrightarrow{\textup{pool}}$\lr{trivial}    & \cite{ghosh2019scale} \\
	% 3-dim
	\cmidrule(lr){2-6}
	\cmidrule(lr){2-6}
	\rownumber&
	&                           &                           & \lr{irreps}            & \cite{3d_steerableCNNs,
		Thomas2018-TFN,
		miller2020relevance,
		Kondor2018-NBN,
		anderson2019cormorant,
		batzner2021se3equivariant} \\
	\rownumber&
	&                           &                           & \lr{quaternion}        & \cite{zhang2019quaternion} \\
	\rownumber&
	&                           &                           & \lr{regular}           & \cite{finzi2020generalizing,
		winkels3DGCNNsPulmonary2018,
		Worrall2018-CUBENET} \\
	\rownumber&
	& \multirow{-4}{*}{$\SO3$}  & \multirow{-4}{*}{$\SE3$}  & \lr{regular}$\xrightarrow{\textup{pool}}$\lr{trivial}
	& \cite{andrearczyk2019exploring} \\
	\cmidrule(lr){3-6}
	\cmidrule(lr){3-6}
	\rownumber&
	&                           &                           & \lr{regular}           & \cite{winkels3DGCNNsPulmonary2018} \\
	\rownumber&
	&                           &                           & \lr{quotient} $\O3/\O2$ \hspace*{-2ex}
	& \cite{janssen2018design} \\
	\rownumber&
	& \multirow{-3}{*}{$\O3$}   & \multirow{-3}{*}{$\E3$}   & \lr{irrep}$\xrightarrow{\textup{norm}}$\lr{trivial} \hspace*{-2ex}
	& \cite{poulenard2019effective} \\
	\cmidrule(lr){3-6}
	\cmidrule(lr){3-6}
	\rownumber&
	& $\C4$                     & $\Trans_3 \rtimes \C4$    & \lr{regular}           & \cite{su2020dv} \\
	\cmidrule(lr){3-6}
	\cmidrule(lr){3-6}
	\rownumber&
	\multirow{-7}{*}{$\Euc_3$}
	& $\D4$                     & $\Trans_3 \rtimes \D4$    & \lr{regular}           & \cite{su2020dv} \\
	% Minkowski
	\cmidrule(lr){2-6}
	\cmidrule(lr){2-6}
	\rownumber&
	$\Euc_{d-1,1}$& $\SO{d\minus1,1}$       & $\Trans_d\rtimes\SO{d\minus1,1}$& \lr{irreps}    & \cite{shutty2020learning} \\
	
	\bottomrule
	
	\rownumber&
	&                           & $\SO2$                    &                &    \cite{chidester2019rotation,
		finzi2020generalizing} \\
	\rownumber&
	\multirow{-2}{*}{$\Euc_2\backslash\{0\}$} & \multirow{-2}{*}{$\{e\}$}
	& $\SO2 \times \Scale$      & \multirow{-2}{*}{\lr{trivial}}
	&    \cite{esteves2017polar,
		finzi2020generalizing} \\
	% B3
	\cmidrule(lr){2-6}
	\cmidrule(lr){2-6}
	\rownumber&
	& $\O2$                 & $\O3$                      & \lr{trivial}           & \cite{ramasinghe2019representation} \\
	\rownumber&
	\multirow{-2}{*}{$\Euc_3\backslash\{0\}$}
	& $\{e\}$                   & $\{e\}$                    & \lr{trivial}           & \cite{boomsma2017spherical} \\
	\bottomrule
	
	%%%%%%%%%%%%%%%%%%%%%%%%%%%%%%%%%%%%%%%%%%%%%%%%%%%%%%%%%%%%%%%%%%%%%%%%%%%%%%%%%%%%%%%%%%%%%%%%%%%%%%%%%%%%%%%%%%%%%%%%%%%
	
	\rownumber&
	&                           &                           & \lr{irreps}            & 
	\cite{kondor2018ClebschGordan,
		esteves2020spinweighted} \\
	\rownumber&
	& \multirow{-2}{*}{$\SO2$}  & \multirow{-2}{*}{$\SO3$}  & \lr{regular}           & \cite{Cohen2018-S2CNN,
		kicanaoglu2019gaugeSphere} \\
	\cmidrule(lr){3-6}
	\cmidrule(lr){3-6}
	\rownumber&
	\multirow{-3}{*}{$S^2$}
	& $\O2$                     & $\O3$                     & \lr{trivial}           & \cite{esteves2018zonalSpherical,
		perraudin2018DeepSphere,
		yang2020rotation} \\
	\cmidrule(lr){2-6}
	\cmidrule(lr){2-6}
	\rownumber&
	$S^2 \,\backslash\,$\lr{poles} & $\{e\}$     & $\SO2$                    & \lr{trivial}           & 
	\makecell{
		\cite{coors2018spherenet,
			tateno2018distortion,
			zhao2018distortion,
			martin2020panoramic,
			jiang2019spherical}
		\\
		\cite{
			su2017spherical,
			su2019kernel,
			eder2019convolutions,
			lee2019spherephd}
	} \\
	\cmidrule(lr){2-6}
	\cmidrule(lr){2-6}
	\rownumber&
	\lr{icosahedron} & $\C6$                     & $\operatorname{I}$ {\color{gray}$(\approx\SO3)$} & \lr{regular}        & \cite{gaugeIco2019} \\
	\cmidrule(lr){2-6}
	\cmidrule(lr){2-6}
	\rownumber&
	\lr{ico}$\,\backslash\,$\lr{poles} & $\{e\}$     & $\C5$ {\color{gray}$(\approx\SO2)$}           & \lr{trivial}           & \cite{zhang2019orientation,
		liu2018icoAltAz} \\
	\bottomrule
	
	%%%%%%%%%%%%%%%%%%%%%%%%%%%%%%%%%%%%%%%%%%%%%%%%%%%%%%%%%%%%%%%%%%%%%%%%%%%%%%%%%%%%%%%%%%%%%%%%%%%%%%%%%%%%%%%%%%%%%%%%%%%
	
	\rownumber&
	&                           &                           & \lr{irreps}            & \cite{Wiersma2020} \\
	\rownumber&
	&                           &                           & \lr{regular}           & \cite{poulenard2018multi,
		sun2018zernet,
		Yang2020parallelFrameCNN,
		deHaan2020meshCNNs} \\
	\rownumber&
	& \multirow{-3}{*}{$\SO2$}  & \multirow{-3}{*}{$\IsomplusM$} & \lr{regular}$\xrightarrow{\textup{pool}}$\lr{trivial} & 
	\cite{masci2015geodesic,
		masci2015shapenet,
		monti2017geometric,
		sun2018zernet} \\
	\cmidrule(lr){3-6}
	\cmidrule(lr){3-6}
	\rownumber&
	\multirow{-4}{*}{\lr{surface} ($d\!=\!2$)}
	& $\D4$                     & $\Isom_{\D4\mkern-5muM}$            & \lr{trivial}           & \cite{huang2019texturenet} \\
	\cmidrule(lr){3-6}
	\cmidrule(lr){3-6}
	\rownumber&
	\multirow{-4}{*}{(\lr{e.g. meshes})}
	& $\{e\}$                   & $\IsomeM$                & \lr{trivial}           & \cite{
		monti2017geometric,
		schonsheck2018parallel,
		jin2018learning,
		tatarchenko2018tangent,
		li2019crossAtlas} \\
	\bottomrule
	
	\rownumber&
	&                       &                            & \lr{irreps}           &                            \\
	\rownumber&
	\multirow{-2}{*}{\lr{M\"obius strip}} & \multirow{-2}{*}{$\Flip$}               & \multirow{-2}{*}{$\SO2$}   & \lr{regular}          & \multirow{-2}{*}{\lr{Section}~\ref{sec:mobius_conv}} \\
	\bottomrule
	
\end{tabular}
            \vspace*{10pt}
            \captionsetup{width=1.06\columnwidth}
            \caption{
                Classification of convolutional networks in the literature from the viewpoint of coordinate independent CNNs.
                Bold lines separate different geometries.
                The affine group equivariant convolutions on Euclidean spaces $\Euc_d$
            (rows 1-26)
                are reviewed in Section~\ref{sec:instantiations_euclidean}.
                Section~\ref{sec:instantiations_euclidean_polar} discusses $\GM$-convolutions on punctured Euclidean spaces
                ${\Euc_d \backslash \{0\}} \cong {S^{d-1} \mkern-4mu\times\! \R^+}$
            (rows 27-30).
                Details on spherical CNNs
            (rows 31-36)
                are found in Section~\ref{sec:instantiations_spherical}.
                The models in
            rows (37-41)
                operate on general surfaces, mostly represented by triangle meshes; see Section~\ref{sec:instantiations_mesh}.
                The last two lines list our M\"obius convolutions from Section~\ref{sec:mobius_conv}.
                $\Trans_d$,~$\Flip$ and $\Scale$ denote the translation, reflection and scaling group, respectively, while $\CN$ and $\DN$ are cyclic and dihedral groups.
                Infinite-dimensional representations are in implementations discretized or sampled.
                For instance, the regular representations of $\SO2$ or $\O2$ are typically approximated by the regular representations of cyclic or dihedral groups $\CN$ or $\DN$.
            }
            \label{tab:network_instantiations}
        \end{center}
    \end{table}
%%%%%%%%%%%%%%%%%%%%%%%%%%%%%%%%%%%%%%%%%%%%%%%%%%%%%%%%%%%%%%%%%%%%%%%%%%%%%%%%%%%%%%%%%%%%%%%%%%%%%%%%%%
\thispagestyle{empty}
\clearpage % force a page break
}
%%%%%%%%%%%%%%%%%%%%%%%%%%%%%%%%%%%%%%%%%%%%%%%%%%%%%%%%%%%%%%%%%%%%%%%%%%%%%%%%%%%%%%%%%%%%%%%%%%%%%%%%%%

% !!!!!!!!!!!!!!!!!!!!!!!!!!!!!!!!!!!!!!!!!!!!!!!!!!!!!!!!!!!!!!!!!!!!
\newpage % necessary to prevent compile error 
% !!!!!!!!!!!!!!!!!!!!!!!!!!!!!!!!!!!!!!!!!!!!!!!!!!!!!!!!!!!!!!!!!!!!








\subsection*{Design choices and overview}
\label{sec:instantiations_taxonomy}

A coordinate independent CNN is \emph{in theory} fully specified by
\begin{itemize}
    \item[1)] a choice of \emph{Riemannian manifold} $(M,\eta)$
    \item[2)] its \emph{$G$-structure} $\GM$,
    \item[3)] a \emph{$G$-compatible connection} which specifies feature transporters $\PAgamma$,
    \item[4)] the \emph{field types} or $G$-representations $\rho$ of each feature space, and
    \item[5)] a choice of $G$-equivariant \emph{nonlinearities}.
\end{itemize}
The \emph{geodesics, exponential} and \emph{logarithmic maps} follow from the canonical Levi-Civita connection on~$M$.%
\footnote{
    It might seem strange to compute geodesics and feature transporters based on potentially different connections.
    When the transporter connection differs from Levi-Civita, this is usually due to the Levi-Civita connection not being $G$-compatible with the chosen $G$-structure when~$G<\O{d}$.
    Some examples are given in the paragraph on $G$-compatible connections below.
}
The \emph{isometry group} $\IsomGM$ w.r.t. which the network is equivariant follows from the metric and the $G$-structure.
All \emph{kernel spaces} $\KG$ are determined by the group representations of the feature spaces between which they map.
\emph{Weight sharing} is performed by placing a $G$-steerable template kernel relative to an arbitrary $G$-frame in $\GpM$ for each point $p\in M$.

In practice, the user is faced with additional design questions, for instance concerning the discretization of the geometry, the encoding of feature fields, numerical algorithms for computing geodesics and transporters,~etc.
This section gives a high level overview of all relevant design choices.
More specific details are found in the following Sections~\ref{sec:instantiations_euclidean}, \ref{sec:instantiations_euclidean_polar}, \ref{sec:instantiations_spherical} and~\ref{sec:instantiations_mesh}.



\paragraph{Discretizations of manifolds and feature fields:}
The implementations differ in their representation of the manifolds and sampling of the feature fields.

Euclidean spaces $\Euc_d$ admit regular pixels grids, for instance $\Z^d$ or the hexagonal grid~\cite{Hoogeboom2018-HEX}.
More generally, locally regular grids are suitable for locally flat manifolds like the M\"obius strip and the icosahedron; see Figs.~\ref{fig:mobius_conv_numerical} and~\ref{fig:ico_cutting}.
Feature fields on Euclidean spaces may furthermore be sampled on a non-regular point cloud.
This is for instance useful when processing atomic environments, where the atom positions serve as sampling locations~\cite{Thomas2018-TFN}.

An important difference between the two approaches is that regular pixel grids are not equivariant w.r.t. continuous translations in $\Trans_d = (\R^d,+)$, but only w.r.t. the subgroup of discrete translations which preserves the grid, for instance~$(\Z^d,+)$.
CNNs on regular grids are furthermore usually applying spatial pooling operations which reduce the models' equivariance even further.
Specifically, given that the pooling operation has a stride of $n$~pixels, it is equivariant w.r.t. translations in $(n\Z^d,+)$.
After $L$ pooling layers in a convolutional network, this implies that the model as a whole is only equivariant w.r.t. translations in $(n^L\Z^d,+)$
-- this issue was empirically investigated in~\cite{azulay2018shift}.
\citet{zhang2019CNNsShiftInvariant} propose to remedy this issue by replacing stride $n$ pooling layers with stride 1 pooling layers (with the same pooling window size), a low-pass filtering, and an $n$-pixel subsampling.
The additional low-pass filtering between the pooling and subsampling operations prevents aliasing effects, which is shown to make the networks sufficiently more stable under translations which are not elements of $(n\Z^d,+)$.


Curved spaces like the 2-sphere $S^2$ do in general not admit regular sampling grids.
A seemingly obvious discretization is in terms of a regular sampling grid in spherical coordinates (Eq.~\eqref{eq:spherical_coords} and Fig.~\ref{fig:spherical_equirectangular_1}), however, as these coordinates are non-isometric, they oversample the signal towards the poles~\cite{zhao2018distortion,tateno2018distortion}.
Approximately uniform sampling grids on $S^2$ are the ``generalized spiral set''~\cite{coors2018spherenet} or the icospherical grid~\cite{jiang2019spherical,kicanaoglu2019gaugeSphere}.
Alternatively, feature fields may be discretized in the spectral domain.
For the sphere, this is done via an expansion in terms of spherical harmonics for scalar fields, spin-weighted spherical harmonics for irrep fields or Wigner D-matrices for general feature fields~\cite{esteves2018zonalSpherical,esteves2020spinweighted,Cohen2018-S2CNN,kondor2018ClebschGordan}.

General surfaces are most commonly represented by triangle meshes; see Section~\ref{sec:surfaces_geom_mesh}.
Feature fields can then be sampled on the mesh vertices, edges or faces~\cite{deGoes2016VectorFieldProcessing}.
A higher resolution of the feature fields can be achieved by encoding them via texture maps~\cite{li2019crossAtlas,huang2019texturenet}.
Alternatively, surfaces may be represented as point clouds~\cite{tatarchenko2018tangent,jin2019NPTCnet}.






\paragraph{\textit{G}-structures \textit{GM} and structure groups \textit{G}:}
The specific choice of $G$-structure to be respected by the network depends on the learning task and the topology of~$M$ (if continuity or smoothness of the convolution is demanded).
In general, $M$ comes equipped with an $\O{d}$-structure, i.e. a bundle of orthonormal reference frames with respect to the given Riemannian metric.
A \emph{lift} to structure groups $G$ with ${\O{d} < G \leq \GL{d}}$ is uniquely determined by $G$-valued gauge transformations of orthonormal frames.
\emph{Reductions} of the structure group to $G < \O{d}$ are, in contrast, not necessarily unique, and encode additional geometric information.
For instance, a reduction to $G=\SO{d}$ requires an orientation on the manifold.%
\footnote{
    For a single, connected manifold, this choice is arbitrary as long as the kernel initialization is symmetric w.r.t. both orientations.
    In this case the network will simply learn reflected kernels for different orientations.
    When considering a dataset consisting of multiple manifolds, their relative orientation is relevant for a correct generalization.
}
The following sections discuss further (mostly implicitly made) choices of $G$-structures found in the literature; see for instance
Figs.~\ref{fig:G_structures_R2_main},
\ref{fig:G_structures_R2_no_origin},
\ref{fig:G_structure_R2_no_origin_O2},
\ref{fig:G_structure_R2_no_origin_logpolar},
\ref{fig:G_structure_R3_no_origin}
\ref{fig:G_structures_S2_main},
or~\ref{fig:G_structures_ico}.
They are either determined by a demand for the equivariance under the isometry group $\IsomGM$, canonically given on the manifold or, specifically for $\{e\}$-structures, algorithmically fixed via some heuristic.
Recall that $\{e\}$-structures are on non-parallelizable manifolds (by definition) necessarily discontinuous.



The most commonly encountered structure groups in the literature are the following:
\begin{itemize}
    \item[{\rule[2.0pt]{2pt}{2pt}}]
        \emph{trivial} group $\{e\}$, corresponding to non-coordinate independent CNNs with unconstrained kernels
    \item[{\rule[2.0pt]{2pt}{2pt}}]
        \emph{reflection} group $\Flip \cong \Z/2\Z$, flipping the first frame axis
    \item[{\rule[2.0pt]{2pt}{2pt}}]
        \emph{special orthogonal} groups $\SO{d}$
        \ \ (continuous rotations)
    \item[{\rule[2.0pt]{2pt}{2pt}}]
        \emph{orthogonal} groups $\O{d}$
        \ \ (continuous rotations and reflections)
    \item[{\rule[2.0pt]{2pt}{2pt}}]
        \emph{scaling} group $\Scale \cong (\R^+,*)$,
\end{itemize}
Since the last three groups are continuous Lie groups, they are in numerical implementations sometimes approximated by finite subgroups.
For instance, $\SO2$ and $\O2$ are often modeled by \emph{cyclic} groups $\CN$ or \emph{dihedral} groups~$\DN$, while three-dimensional rotations and reflections in $\O3$ can be approximated by \emph{polyhedral} groups (symmetry groups of Platonic solids, e.g. the icosahedron).
To reduce the complexity of the classification of models in Table~\ref{tab:network_instantiations} we chose to not distinguish between the continuous symmetries and their approximations by finite subgroups.
We will, however, state such approximations in our detailed discussion of the models in the following sections.





\paragraph{\textit{G}-compatible connections:}
All of the models consider either the canonical \emph{Levi-Civita} connection on $M$ or the unique \emph{trivial connection} which is induced by an $\{e\}$-structure.
The choice of connection becomes irrelevant (thus unspecified) for networks which operate solely on \emph{scalar fields}, whose transport is always trivial.

More specifically, all Euclidean CNNs from Section~\ref{sec:instantiations_euclidean} use Levi-Civita transporters, which transport vectors such that they remain parallel in the usual sense on Euclidean spaces $\Euc_d$; see Fig.~\ref{fig:transport_flat}.
This is possible since the Levi-Civita connection is $G$-compatible with the models' $G$-structures (defined in Eq.~\eqref{eq:G_lifted_G_structure_Rd} and visualized in Fig.~\ref{fig:G_structures_R2_main}).%
\footnote{
    In contrast, the Euclidean $\{e\}$-structure in Fig.~\ref{fig:frame_field_automorphism_2} would be incompatible with the Levi-Civita connection on~$\Euc_2$.
}

The models on the punctured Euclidean spaces $\Euc_d \backslash \{0\}$ from Section~\ref{sec:instantiations_euclidean_polar} are either based on $\{e\}$-structures and/or consider scalar fields.
They utilize therefore trivial connections which differ from the canonical Levi-Civita connection on~$\Euc_d \backslash \{0\}$.

All spherical CNNs that rely on the $\SO2$-structure in Fig.~\ref{fig:G_structure_S2_1} (reviewed in Section~\ref{sec:spherical_CNNs_fully_equivariant}) transport features according to the Levi-Civita connection on $S^2$ (Fig.~\ref{fig:transport_sphere}).
Those which operate on the $\{e\}$-structure in Fig.~\ref{fig:G_structure_S2_2} (reviewed in Section~\ref{sec:spherical_CNNs_azimuthal_equivariant}) are again considering a trivial connection since the spherical Levi-Civita connection is incompatible with this $\{e\}$-structure.
The icosahedral CNN with $\C6$-structure, Fig.~\ref{fig:G_structure_ico_3}, transports features according to the $\C6$-compatible icosahedral Levi-Civita connection.%
\footnote{
    Discrete Levi-Civita connections on meshes are discussed in Section~\ref{sec:surfaces_geom_mesh} and~\cite{craneDiscreteDifferentialGeometry2014,craneTrivialConnectionsDiscrete2010}.
}

All CNNs on general surfaces that are listed in rows~(37-39) of Table~\ref{tab:network_instantiations}
assume oriented surfaces that are equipped with an $\SO2$-structure.
They transport features with the $\SO2$-compatible Levi-Civita connection of the surfaces.
The other surface CNNs are based on $\{e\}$-structures and/or operate on scalar fields -- their feature transport is therefore trivial.

Our M\"obius strip convolutions transport features via the Levi-Civita connection, which is compatible with the assumed $\Flip$-structure.

Recall that the Levi-Civita connection is uniquely determined by the metric, and is therefore generally isometry invariant; cf. footnote~\ref{footnote:LeviCivita_isometry_invariance} in Section~\ref{sec:isom_expmap_transport}.
As trivial $\{e\}$-compatible connections are uniquely specified by the $\{e\}$-structure they share its symmetries, that is, they are invariant under the action of~$\IsomeM$.
This implies by Theorem~\ref{thm:isom_equiv_GM_conv} that the $\GM$-convolutions, which are based on these connections, are $\IsomGM$-equivariant.




\paragraph{Transporter pullbacks and alternative projections to $\boldsymbol{\TpM}$:}
The transporter pullback $\Expspf$, defined in Def.~\ref{dfn:Expf_pullback_field} and Eq.~\eqref{eq:transporter_pullback_in_coords}, represents a feature field $f$ in a geodesic parametrization on the tangent space~$\TpM$.
The transportation part of the operation is determined by the $G$-compatible connection.
Geodesics -- and therefore exponential maps $\exp_p: \TpM\to M$ -- have closed form expressions on Euclidean spaces $\Euc_d$ and the sphere~$S^2$.
Specifically, the exponential maps on $\Euc_d$ reduce in Cartesian coordinates to the vector summation in Eq.~\eqref{eq:exp_map_euclidean}, such that Euclidean $\GM$-convolutions reduce to conventional convolutions on~$\R^d$; see Theorem~\ref{thm:Euclidean_GM_conv_is_conventional_conv}.
Geodesics on $S^2$ are well known to be given by the great circles of the sphere.
If the sphere is viewed as being embedded in $\R^3$, the exponential map is explicitly given by Eq.~\eqref{eq:sphere_expmap_explicit}.
The geodesics on general surface meshes are not described by closed form solutions but are computed numerically; see Section~\ref{sec:surfaces_geom_mesh}.
In contrast to the smooth setting, one needs to distinguishes between ``shortest'' and ``straightest'' geodesics on meshes~\cite{polthier1998straightest}.


The pullback of feature fields into geodesic normal coordinates is not the only way of representing feature fields on the tangent spaces.
In the literature on spherical CNNs it is rather common to use gnomonic projections, which are visualized in Fig.~\ref{fig:gnomonic_proj}.
Theorem~\ref{thm:gnomonic} shows that this projection can be viewed as a special case of our more general geodesic parameterization after applying a radial warp to the kernels.
The corresponding models are therefore exactly identified as $\GM$-convolutions.
Surfaces which are embedded in an ambient space like $\R^3$ might furthermore rely on various projections in the embedding space; see for instance the last three models that are discussed in Section~\ref{sec:e_surface_conv}.
Note that these approaches are truly different from ours, i.e. these three models are not exactly $\GM$-convolutions.







\paragraph{\textit{G}-representations and nonlinearities:}
Almost all models consider either of the trivial representation, irreducible representations or regular representations as field types.
Exceptions are quotient representations, more general induced representations, tensor product representations and, specifically for $G=\SO3$, the quaternion representation.
Infinite-dimensional representations, in particular regular and quotient representations of Lie groups, are in implementations discretized.
This can either happen via Monte Carlo sampling or by falling back to the corresponding representations of finite subgroups as discussed above.

The nonlinearities are required to be equivariant w.r.t. the action of the chosen $G$-representations.
Since scalar fields are $G$-invariant, they are acted on by usual nonlinearities like ReLU.
Feature fields that transform according to permutation representations, most importantly regular representations, are acted on channel-wise.
All other field types require custom-tailored nonlinearities -- we refer to~\cite{Weiler2019_E2CNN} for a discussion of specific choices.




\paragraph{\textit{G}-steerable kernel spaces:}
$\GM$-convolutions map input fields of type $\rhoin$ to output fields of type~$\rhoout$ by convolving them with $G$-steerable kernels~$K\in\KG$.
Since the space $\KG$ of $G$-steerable kernels, Def.~\ref{dfn:G-steerable_kernel_def_43}, is a vector space, it is usually parameterized in terms of a basis $\{K_1,\,\dots,\,K_N\}$ of~$\KG$.
\mbox{Before} computing the convolution, the learned kernel $K = \sum_{i=1}^N w_i K_i$ is expanded in this basis, where $\{w_1,\dots,w_N\}$ are real-valued weights to be optimized.
Provably complete kernel spaces for the groups $G\leq\O2$ were implemented in~\cite{Weiler2019_E2CNN}.%
\footnote{\url{https://quva-lab.github.io/e2cnn/api/e2cnn.kernels.html}}
A generalization of the Wigner-Eckart theorem characterizes the kernel space bases for general compact structure groups~$G$~\cite{lang2020WignerEckart}.


In practice, the majority of authors does not use a representation theoretic formulation of feature fields and steerable kernels, but formulate them based on intuition.
Specifically, most authors assume a given input field type and propose various convolution operations which are engineered such that the resulting output field transforms in an equivariant (or coordinate independent) manner.%
\footnote{
    This is opposed to our approach, which fixes the input and output fields and subsequently asks for the resulting constraint on convolution kernels.
}
While these approaches propose certain $G$-steerable kernels that map between $\rhoin$-fields and $\rhoout$-fields, these kernels do sometimes not span the complete space of possible kernels.
This applies for instance to the \emph{MDGCNNs} and \emph{PFCNNs}, which are discussed in Section~\ref{sec:so2_surface_conv}.
