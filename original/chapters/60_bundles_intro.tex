%!TEX root=../GaugeCNNTheory.tex


\section{Associated bundles and coordinate free feature fields}
\label{sec:bundles_fields}

Fields of geometric quantities on manifolds are formalized as sections of fiber bundles (Eq.~\eqref{cd:section_proj_idM}).
Any smooth manifold is naturally endowed with its tangent bundle and frame bundle.
A choice of $G$-structure, which is a $G$-bundle of reference frames, allows to define $G$-associated feature vector bundles.
The feature spaces of our coordinate independent neural networks are spaces of feature fields, i.e. sections of these feature vector bundles.

Fiber bundles in general are reviewed in Section~\ref{sec:fiber_bundles_general}.
Section~\ref{sec:GL_associated_bundles} discusses the tangent bundle $\TM$ and the frame bundle $\FM$.
$G$-structures $\GM$, which are subsets of reference frames which are distinguished by the given geometric structure on the manifold, are introduced in Section~\ref{sec:G_associated_bundles}.
Associated $G$-bundles, including the feature vector bundles $\A$, are constructed from the $G$-structure.
Section~\ref{sec:bundle_trivializations} gives details on the local trivializations (gauges) of $\TM$, $\FM$, $\GM$ and $\A$, which reintroduces coordinates and recovers the formulation in Section~\ref{sec:gauge_cnns_intro_local}.
The mutual transformation of the trivialized feature fields with trivialized tangent vector coefficients and reference frames follows thereby from the coordinate free formulation via associated $G$-bundles.
Section~\ref{sec:bundle_transport} discusses parallel transporters on the associated bundles, in particular how they induce each other.


All concepts presented here are well established in differential geometry and can easily be found in the literature~\cite{schullerGeometricalAnatomy2016,nakahara2003geometry,husemollerFibreBundles1994a,steenrodTopologyFibreBundles,shoshichikobayashiFoundationsDifferentialGeometry1963,marshGaugeTheoriesFiber2016,wendlLectureNotesBundles2008,sternberg1999lectures,piccione2006theory,crainic2013GStructuresExamples}.
Our contribution is to give a comprehensive exposition which bridges between the mathematical theory and its application in geometric deep learning.
